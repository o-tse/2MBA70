  

\section{Convergence of integrals}

One of the motivations for developing a new theory of integration using measurable functions instead of continuous ones was that we would be able to change limits and integrals more often. We have already seen an example of such a result in the monotone convergence theorem, Theorem~\ref{th:monotone-convergence-I}. However, this required that the sequence $f_n$ of functions was monotone (i.e. non-decreasing) everywhere, which sounds a bit restrictive. That is why in this section we will use the monotone convergence theorem to derive other convergence results with less restrictive conditions.

\subsection{Monotone convergence (continued)}

Theorem~\ref{th:monotone-convergence-I} states that if we have a sequence of measurable functions $(f_n)_{n \in \bbN}$ from some measure space $(\Omega, \cF, \mu)$ to $[0,+\infty]$ such that $f_n \le f_{n+1}$, then we could interchange limits and integration so that
\[
	\lim_{n \to \infty} \int_\Omega f_n \, \dd \mu = \int_\Omega \lim_{n \to \infty} f_n \, \dd \mu.
\]
It should be noted that the monotone properties requires that $f_n(\omega) \le f_{n+1}(\omega)$ for all $\omega \in \Omega$. However, from the definition of the Lebesgue integral we see that it is not affected by sets measure zero. Hence, we would expect that we can relax the monotone property to hold $\mu$-almost everywhere, i.e. the set where it does not hold has measure zero. This turns out to be the case, providing a slightly more general version of the monotone convergence theorem.

\begin{theorem}[Monotone convergence II]\label{thm:monotone_convergence_ii}
Let $(\Omega, \cF, \mu)$ be a measure space. Let $(f_n)_{n \ge 1}$ be a sequence of non-negative, measurable functions and let $f$ be a non-negative measurable functions such that the following holds $\mu$-almost everywhere
\begin{enumerate}[label={(\arabic*)}]
	\item $f_n \le f_{n+1}$ for all $n \in \bbN$, and
	\item $\lim_{n \to \infty} f_n = f$.
\end{enumerate}
Then
\[
	\lim_{n \to \infty} \int_\Omega f_n \,\dd \mu = \int_\Omega f \,\dd \mu.
\]
\end{theorem}

\begin{proof}
As you might expect, the proof will utilize the first monotone convergence theorem. For this, we first note that by assumption, there exists a $\mu$-null set $N\subset \Omega$ such that properties (1) \& (2) hold in $\Omega\backslash N$. By the definition of null sets, we find a measurable set $A \supset N$ for which $\mu(A)=0$. In particular, the properties (1) \& (2) hold in $\Omega\backslash A$.

Now define the function $g_n(\omega) = \max_{1 \le k \le n} f_k(\omega)$ and $g(\omega) := \lim_{n \to \infty} g_n(\omega)$. Then $g_n(\omega) \le g_{n+1}(\omega)$ holds for \emph{all} $\omega \in \Omega$. Here comes the key observation. For every $\omega \in \Omega\backslash A$ it holds that $g_n(\omega) = f_n(\omega)$ and $g(\omega) = f(\omega)$. Moreover, since $\mu(A) = 0$ we have that
\[
	\int_\Omega f_n \, \dd \mu = \int_\Omega g_n \, \dd \mu \qquad \text{and} \qquad
	\int_\Omega f \, \dd \mu = \int_\Omega g \, \dd \mu.
\]
The result then follows by applying Theorem~\ref{th:monotone-convergence-I} to the functions $g_n$ and $g$.
\end{proof}

\subsection{Fatou's Lemma}

\begin{theorem}[Fatou's lemma]\label{thm:fatou}
Let $(\Omega, \cF, \mu)$ be a measure space. Let $(f_n)_{n \ge 1}$ be a sequence of non-negative, measurable functions and define
\[
	f := \liminf_{n\to \infty} f_n = \lim_{n \to \infty} \inf_{k \ge n} f_k.
\]
Then
\[
	\int_\Omega f\, \dd \mu \le \liminf_{n\to \infty} \int_\Omega f_n \,\dd \mu.
\]
\end{theorem}

\begin{proof}
Our proof will use the monotone convergence theorem. There are however other proofs, based on first principles. 

Define the function $g_n(\omega) := \inf_{k \ge n} f_k(\omega)$ and note that by Lemma~\ref{lem:limit_operations_measurable_functions} $g_n$ are measurable. Moreover, $g_n(\omega) \le g_{n+1}(\omega)$ for all $\omega \in \Omega$ and $\lim_{n \to \infty} g_n(\omega) = f(\omega)$. Hence, Theorem~\ref{thm:monotone_convergence_ii} implies that
\[
	\lim_{n \to \infty} \int_\Omega g_n \, \dd \mu = \int_\Omega \lim_{n \to \infty} g_n \, \dd \mu
	= \int f \, \dd \mu. 
\]
Finally we observe that by definition $g_n \le f_n$ holds for all $n \in \bbN$ so that
\begin{align*}
	\int f \, \dd \mu = \lim_{n \to \infty} \int_\Omega g_n \, \dd \mu
	&= \lim_{n \to \infty} \int_\Omega \inf_{k \ge n} f_k \, \dd \mu \\
	&\le \lim_{n \to \infty} \inf_{\ell \ge n} \int_\Omega  f_\ell \, \dd \mu
	= \liminf_{n \to \infty} \int_\Omega  f_\ell \, \dd \mu.
\end{align*}
Here, we used that $\inf_{k \ge n} f_k \le f_\ell$ for all $\ell \ge n$ and the monotonicity property of the integral (see Proposition~\ref{prop:properties-integral}).
\end{proof}

\begin{example}[Strict inequality in Fatou]\label{ex:Fatou}
	Consider the sequence $f_n = \frac{n}{2}\mathbf{1}_{(-\frac{1}{n},\frac{1}{n})}$ for $n\ge 1$. Clearly, $f_n$ is Borel measurable with
	\[
		\int_\bbR f_n\,\dd\lambda = \frac{n}{2}\lambda\bigl(\bigl(-\tfrac{1}{n},\tfrac{1}{n}\bigr)\bigr) = \qquad\text{for all $n\ge 1$}.
	\]
	On the other hand, we have that $f_n(x)\to 0$ for every $x\ne 0$. In particular, $f_n\to 0$ pointwise $\lambda$-almost everywhere. Hence,
	\[
		\int_\bbR \Bigl(\lim_{n\to\infty} f_n(x)\Bigr)\,\mu(\dd x) = 0 < 1 = \liminf_{n\to\infty} \int_\bbR f_n\,\dd\lambda.
	\]
\end{example}

This previous example shows that, in general, one cannot expect to interchange integrals and limits unless the sequences' monotonicity is assumed. Morally, the failure of the convergence here comes from the \emph{loss of mass} for the integral (cf.\ Example).

\subsection{Dominated Convergence}

Armed with Fatou's lemma, we can now prove one of the most useful convergence results for Lebesgue integrals.

\begin{theorem}[Dominated convergence]\label{thm:dominated_convergence}
Let $(\Omega, \cF, \mu)$ be a measure space. Let $(f_n)_{n \ge 1}$ be a sequence of measurable functions and let $f$ be a measurable function such that $\lim_{n\to\infty}f_n = f$ $\mu$-almost everywhere. Moreover, assume there exists a non-negative $\mu$-integrable function $g : \Omega \to [0,\infty]$ such that $|f_n| \le g$ $\mu$-almost everywhere. Then
\[
	\lim_{n \to \infty} \int_\Omega f_n \,\dd \mu = \int_\Omega f \,\dd \mu.
\]
\end{theorem}

\begin{proof}
We will first prove the result for the case that both $|f_n| \le g$ and $\lim_{n\to\infty} f_n =f$ hold everywhere.  

Consider the functions $f_n+g$ and note that $|f_n| \le g$ implies that these are non-negative. Fatou's lemma (Theorem~\ref{thm:fatou}) now implies that
\[
	\int_\Omega f + g \, \dd \mu \le \liminf_{n \to \infty} \int_\Omega f_n + g \, \dd \mu.
\] 
Using the additive property of the integral we get
\[
	\int_\Omega f \, \dd \mu + \int_\Omega g \, \dd \mu 
	\le \liminf_{n \to \infty} \int_\Omega f_n \, \dd \mu + \int_\Omega g \, \dd \mu.
\]
Since $\int_\Omega g \, \dd \mu < \infty$ this implies that
\[
	\int_\Omega f \, \dd \mu \le \liminf_{n \to \infty} \int_\Omega f_n \, \dd \mu.
\]
On the other hand, the condition $|f_n| \le g$ also implies that the functions $g - f_n$ are non-negative. Applying Fatou's lemma here yields
\[
	\int_\Omega g - f \, \dd \mu \le \liminf_{n \to \infty} \int_\Omega g - f_n \, \dd \mu.
\]
The additive property of integral now yields
\[
	\int_\Omega g\,\dd \mu - \int_\Omega f \, \dd \mu \le \int_\Omega g \, \dd \mu 
	+ \liminf_{n \to \infty} \int_\Omega - f_n \, \dd \mu,
\]
which implies that 
\[
	\int_\Omega f \, \dd \mu \ge - \liminf_{n \to \infty} \int_\Omega - f_n \, \dd \mu 
	= \limsup_{n \to \infty} \int_\Omega f_n \, \dd \mu.
\]
We thus conclude that
\[
	\lim_{n \to \infty} \int_\Omega f_n \, \dd \mu = \int_\Omega f \, \dd \mu.
\]

Now, let us consider the general case. Then there exists a $A \in \cF$ such that $\mu(A) = 0$ and both $|f_n| \le g$ and $f_n \to f$ hold for every $\omega \in \Omega \backslash A$. Let us now define the following functions
\[
	\hat{f}_n := f_n \mathbf{1}_{\Omega\backslash A},\qquad
	\hat{f} := f \mathbf{1}_{\Omega\backslash A},\qquad \hat{g} := g \mathbf{1}_{\Omega\backslash A}.
\]
Then,
\[
	\int_\Omega \hat{f}_n \, \dd \mu = \int_\Omega f_n \, \dd \mu \quad \text{and} \quad
	\int_\Omega \hat{f} \, \dd \mu = \int_\Omega f \, \dd \mu
\]
Moreover, $\hat{f}_n \le \hat{g}$ and $\hat{f}_n \to \hat{f}$ hold \emph{everywhere}. So using the first part of the proof we have that
\[
	\lim_{n \to \infty} \int_\Omega f_n \, \dd \mu = \lim_{n \to \infty} \int_\Omega \hat{f}_n \, \dd \mu
	= \lim_{n \to \infty} \int_\Omega \hat{f} \, \dd \mu = \lim_{n \to \infty} \int_\Omega f \, \dd \mu.\qedhere
\]
\end{proof}

\begin{example}
Consider the sequence of functions $f_n(x) = \frac{n \sin(x/n)}{x(x^2+1)}$. We will use dominated convergence to determine $\lim_{n \to \infty} \int_\bbR f_n \, \dd \lambda$. Define $g(x) = \frac{1}{x^2 +1}$ and note that 
\[
	f_n(x) = \frac{\sin(x/n)}{x/n} g(x).
\]
Note that $|\sin(y)| \le |y|$ holds for all $y > 0$ and that for every $x$ we have that $\lim_{n \to \infty} \frac{\sin(x/n)}{x/n} = 1$. We thus conclude that $|f_n(x)| \le g(x)$ and $f_n \to g(x)$ holds for all $x \in \bbR \backslash \{0\}$. Since the set $\{0\}$ has Lebesgue measure zero, all the conditions of Theorem~\ref{thm:dominated_convergence} are satisfied. Hence (see Example~\ref{ex:computation_lebesgue_integral})
\[
	\lim_{n \to \infty} \int_\bbR \frac{n \sin(x/n)}{x(x^2+1)} \lambda(\dd x) 
	= \int_\bbR \frac{1}{x^2 +1} \lambda(\dd x) = \pi.
\]
\end{example}

\section{Convergence of finite measures}\label{sec:weak_convergence_finite_measures}

Until now we have been mainly concerned with convergence of integrals for a sequence of functions $(f_n)_{n \ge 1}$ and fixed measure $\mu$. But what if instead we have a sequence of measures $(\mu_n)_{n \ge 1}$ on a given measurable space. When does this sequence convergence to a limit measure $\mu$? And what does that actually mean?

These are the questions we will address in this section. To properly address them we need to restrict ourselves to finite measures, and thus we will without loss of generality consider probability measures. Moreover, while the concepts we will introduce can be generalized to any topological space with the corresponding Borel \sigalg/, for the sake of clarity we will restrict our attention to $(\bbR, \cB_\bbR)$. 

We start by defining what convergence of probability measures means. We will use $C_b(\bbR)$ to denote the class of continuous bounded functions on $\bbR$.

\begin{definition}
Let $(\mu_n)_{n \ge 1}$ and $\mu$ be probability measures on $(\bbR, \cB_\bbR)$. We say that $\mu_n$ \emph{converges weakly} (or \emph{narrowly}) to $\mu$ if for every continuous bounded function $h\in C_b(\bbR)$ it holds that
\[
	\int_\bbR h \, \dd \mu_n \to \int_\bbR h \, \dd \mu.
\]
If this is the case we write $\mu_n \Rightarrow \mu$.
\end{definition}

The definition of weak convergence asks us to verify the convergence of the $\mu_n$ integral of $h$ for any $h\in C_b(\bbR)$. In some cases that can be a cumbersome task. Hence it would be helpful if we would have some equivalent conditions for weak convergence. The beauty here is that there are many equivalent definitions. They are often summarized in what is known as the Portmanteau theorem (or lemma). We provide one version of it below.

%We will first prove an important technical lemma, needed for the proof of this theorem. For any function $h\colon \bbR \to \bbR$ we denote by $\cC_h \subset \bbR$ the set of continuity points of $h$, i.e. the set of all points $x\in \bbR$ at which $h$ is continuous. 

%\medskip
%
%We first mention an important technical lemma needed for the proof of this theorem. For any function $h\colon \bbR \to \bbR$, we denote by $\cC_h \subset \bbR$ the set of continuity points of $h$, i.e., the set of all points $x\in \bbR$ at which $h$ is continuous. The following technical lemma, which can be deduced from Lusin's theorem (cf.~Theorem~\ref{thm:lusin}), allows us to approximate measurable functions by continuous ones, with arbitrary precision in terms of the integrals. 
%
%\begin{lemma}
%Consider a probability measure $\mu$ on $(\bbR, \cB_\bbR)$ and bounded measurable function $h\in B_b(\bbR)$ with $\mu(\cC_h) = 1$. Then for every $\varepsilon > 0$, there exist continuous bounded functions $h^-_\varepsilon$ and $h^+_\varepsilon$ such that
%\begin{enumerate}[label={(\arabic*)}]
%\item $h^-_\varepsilon \le h \le h^+_\varepsilon$ and
%\item $\int_\bbR h^+_\varepsilon \, \dd \mu - \int_\bbR h^-_\varepsilon \, \dd \mu < \varepsilon$.
%\end{enumerate} 
%\end{lemma}
%\begin{proof}
%For $k \in \bbN$ define the functions
%\[
%	g_k := \inf_{y \in \bbR} h(y) + k\|x-y\| \quad G_k := \sup_{y \in \bbR} h(y) - k\|x-y\|.
%\]
%We then observe that for any $k \in \bbN$, $g_k \le g_{k + 1}$, $G_{k+1} \le G_k$ and $g_k \le h \le G_k$. Hence
%\[
%	g_1 \le g_2 \le \dots \le h \le \dots \le G_2 \le G_1.
%\]
%Moreover, since for any fixed $x \in \bbR$ the sequences $(g_k(x))_{k \ge 1}$ and $(G_k(x))_{k \ge 1}$ are bounded we get that their limits as $k \to \infty$ exist and
%\[
%	\lim_{k \to \infty} g_k(x) \le h(x) \le \lim_{k \to \infty} G_k(x).
%\]
%
%We now claim that for every $k \ge 1$ the functions $g_k$ and $M_k$ are continuous and bounded. The last part follows from directly from the definitions. For the continuity we note that
%\begin{align*}
%	g_k(x) = \inf_{y \in \bbR} h(y) + k \|x-y\|
%	\le \inf_{y \in \bbR} h(y) + k \|z-y\| +k \|x-z\|
%	= g_k(z) + k\|x-z\|,
%\end{align*}
%which implies that $\|g_k(x) - g_k(y)\| \le k \|x-y\|$. A similar argument works for $G_k$.
%
%Now, let $x \in \cC_h$, i.e. $x$ is a continuity point of $h$, and fix $\varepsilon > 0$. Then there exist a $\delta > 0$ such that $\|x-y\| < \delta$ implies that $\|h(x) - h(y)\| < \varepsilon$. If we then define
%\[
%	r =  \left \lceil \frac{h(x) - \inf_{z \in \bbR} h(z)}{\delta} \right \rceil,
%\]
%then
%\begin{align*}
%	\lim_{k \to \infty} g_k(x) &\ge g_r(x)\\
%	&= \min\{ \inf_{\|x-y\| \ge \delta} h(y) + r\|x-y\| + \inf_{\|x-y\| < \delta} h(y) + r\|x-y\|\}\\
%	&\ge \min\{h(x) - \varepsilon, \, \inf_{z \in \bbR} h(z) + (h(x) - \inf_{z \in \bbR} h(z))\delta/\delta\} \\
%	&= h(x) - \varepsilon.
%\end{align*}
%Similarly, we get that $\lim_{k \to \infty} G_K(x) \le h(x) + \varepsilon$. Since $\mu(\cC_h) = 1$ this then implies that
%\[
%	\int_\bbR \lim_{k \to \infty} g_k \, \dd \mu  = \int_\bbR h \, \dd \mu 
%	= \int_\bbR \lim_{k \to \infty} G_k \, \dd \mu. 
%\]
%
%Applying Theorem~\ref{thm:monotone_convergence_ii} to $g_k$ and to $-G_k$ we get that
%\[
%	\lim_{k \to \infty} \int_\bbR g_k \, \dd \mu = \int_\bbR h \, \dd \mu
%	= \lim_{k \to \infty} \int_\bbR G_k \, \dd \mu.
%\]
%Finally, since $g_k$ is non-decreasing and $G_k$ is non-increasing, for every $\varepsilon$ there must exist an $K$ such that for all $k \ge K$
%\[
%	\int_\bbR (G_k - g_k) \, \dd \mu = \int_\bbR (G_k - h) \, \dd \mu + \int_\bbR (h - g_k) \, \dd \mu \le \varepsilon. 
%\]
%So we can take
%\[
%	h^-_\varepsilon := g_K \quad \text{and} \quad h^+_\varepsilon := G_K.\qedhere
%\]
%\end{proof}

%Recall that a set $A \subset \bbR^d$ is open if for every $x \in A$ there exists an $r > 0$ such that $B_x(r) \subset A$. In addition, a set $B\subset \bbR^d$ is called \emph{closed} if $B = \bbR^d \backslash A$ for some open set $A$.

For a set $A \subset \bbR$ denote by $\bar{A}$ the smallest closed set that contains $A$ and by $A^\circ$ the largest open set that is contained in $A$. The sets $\bar{A}$ and $A^\circ$ are called the \emph{closure} and \emph{interior} of $A$, respectively. We now define the \emph{boundary} of $A$ as $\partial A := \bar{A} \backslash A^\circ$. Given a measure $\mu$ on $(\bbR, \cB_\bbR)$, a set $A$ is called a \emph{$\mu$-continuity set} if $\mu(\partial A) = 0$.

\begin{definition}
	\begin{itemize}
		\item Let $A\in\cB_\bbR$. Given a measure $\mu$ on $(\bbR, \cB_\bbR)$, a set $A$ is called a \emph{$\mu$-continuity set} if $\mu(\partial A) = 0$.
		\item Let $h$ be a bounded measurable function. The \emph{continuity set of $h$} is the set
		\[
			\cC_h := \{ x\in \bbR : \text{$h$ is continuous in $x$}\}.
		\]
		\item The \emph{support} of a function $h$ is the (closed) set $\text{supp}(h) := \overline{\{x\in\bbR : h(x)\ne 0\}}.$
	\end{itemize}


\end{definition}

We can now state a list of equivalent definitions for weak convergence of probability measures.

\begin{theorem}[Portmanteau Theorem]\label{thm:portmanteau}
Let $(\mu_n)_{n \ge 1}$, $\mu$ be probability measures on $(\bbR, \cB_\bbR)$. Then the following are equivalent:
\begin{enumerate}[label={(\arabic*)}]
\item $\mu_n \Rightarrow \mu$.
\item $\int_\bbR h \, \dd \mu_n \to \int_\bbR h \, \dd \mu$ for all bounded measurable functions $h$ with $\mu(\cC_h) = 1$.
\item $\int_\bbR g \, \dd \mu_n \to \int_\bbR g \, \dd \mu$ for all continuous functions $g$ with compact support.
\item $\limsup_{n \to \infty} \mu_n(B) \le \mu(B)$ for all closed sets $B \subset \bbR$.
\item $\liminf_{n \to \infty} \mu_n(A) \ge \mu(A)$ for all open sets $A \subset \bbR$.
\item $\lim_{n \to \infty} \mu(C) = \mu(C)$ for all $\mu$-continuity sets $C$.
\end{enumerate}
\end{theorem} 

\begin{proof}
The proof of this theorem is mostly technical and, for the most part, does not provide any interesting insights. The implication \textbf{1} $\iff$ \textbf{3} is dealt with in Problem~\ref{prb:portmanteau}. For the full proof, we refer to Appendix~\ref{chapter:appendix-portmanteau}. 
\end{proof}

\begin{example}
	Piggybacking on Example~\ref{ex:Fatou}, we consider the sequence of measures $(\mu_n)_n$ defined by (cf.\ Problem~\ref{prb:measure})
	\[
		\mu_n(A):= \int_A f_n\,\dd\lambda,\qquad A\in\cB_\bbR.
	\]
	We claim that $\mu\Rightarrow \mu:=\delta_0$ as $n\to\infty$.
	
	\smallskip
	\noindent We prove this claim by making use of the equivalence (1)$\Leftrightarrow$(5) of Portmanteau's Theorem.
	
	\smallskip
	\noindent Let $A\subset\bbR$ be an arbitrary open set \emph{not} containing $0$. Since $0\ne A$, we have that $f_n(x)\to 0$ for every $x\in A$. Therefore, Fatou's lemma yields
	\[
		\mu(A) = 0 \le \liminf_{n\to\infty} \int_A f_n\,\dd\lambda. 
	\]
	Now let $A$ be an arbitrary set containing $0$. Since $A$ is open, there exists some $\varepsilon>0$ such that $B_\varepsilon(0)\subset A$. Moreover, we have that 
	\[
		\int_A f_n\,\dd\lambda \ge \int_{B_\varepsilon(0)} f_n\,\dd\lambda = 1\qquad\text{for every $n>1/\varepsilon$}.
	\]
	Taking the liminf then yields
	\[
		\liminf_{n\to\infty} \int_A f_n\,\dd\lambda \ge 1 = \mu(A).
	\]
	Together, we find that (5) is satisfied for all open sets $A\subset\bbR$, thereby proving the claim.
\end{example}

\section{Littlewood's principles}

In this section, we will discuss three principles---called Littlewood's principles---that provide practical ways of viewing measurable sets, almost everywhere convergence, and measurable functions. These principles hold for measures that are both inner and outer regular as defined in the following.

\begin{definition}[Inner/Outer regularity]
A measure $\mu$ on $(\bbR^d, \cB_{\bbR^d})$ is \emph{inner regular} if for all $A \in \cB_{\bbR^d}$ 
\[
	\mu(A) = \sup\bigl\{ \mu(K) : K\subset A\;\text{compact}\bigr\}.
\]
We say that a measure $\mu$ on $(\bbR^d, \cB_{\bbR^d})$ is \emph{outer regular} if for all $A \in \cB_{\bbR^d}$ 
\[
	\mu(A) = \inf \bigl\{ \mu(O): A\subset O\;\text{open}\bigr\}.
\]
\end{definition}

Surprisingly, finite measures on $(\bbR^d, \cB_{\bbR^d})$ are always both inner and outer regular.

\begin{theorem}\label{th:regularity-measure}
	Every finite measure $\mu$ on $(\bbR^d, \cB_{\bbR^d})$ is both inner and outer regular.	
\end{theorem}

%The third principle, named after Lusin, provides a nice and practical way of seeing measurable functions. In particular, it allows us to show that 


\paragraph{Littlwood's first principle:} 
\begin{quotation}
	\emph{Every measurable set is ``practically open"}.
\end{quotation}

This principle allows us to approximate arbitrary Borel measurable sets in $\bbR^d$ with a finite union of open rectangles.

\begin{theorem}
Let $\mu$ be a finite measure on $(\bbR^d, \cB_{\bbR^d})$.
Let $A \in \cB_{\bbR^d}$ be a Borel measurable set. Then for any $\epsilon > 0$, there exists a set $O$ of a finite union of open rectangles in $\bbR^d$, such that the measure of the \emph{symmetric difference} $A \Delta O := (A \backslash O) \cup (O \backslash A)$ is smaller than $\epsilon$, that is
\[
\mu( A \Delta O  ) = \mu( A \backslash O ) + \mu( O \backslash A) < \epsilon.
\]
\end{theorem}

\begin{proof}
	We make use of Theorem \ref{th:regularity-measure} for the proof.
Let $\epsilon > 0$ be arbitrary. Then the inner regularity of $\mu$ provides a compact set $K \subset A$ such that
\[
\mu(K) > \mu(A) - \epsilon/2.
\]	
Moreover, the outer regularity of $\mu$ provides a family of open rectangles $(O_i)_{i\in\bbN}$ such that
\[
	K \subset \bigcup_{i=1}^\infty O_i\qquad\text{and}\qquad \sum_{i=1}^\infty \mu(O_i) \leq \mu(K) + \epsilon / 2.
\]
However, $K$ is compact and therefore there exists an $N \in \bbN$ such that
\[
K \subset \bigcup_{i=1}^N O_i =: O.
\]
Clearly,
\[
\sum_{i=1}^N \mu(O_i) \leq \sum_{i=1}^\infty \mu(O_i) \leq \mu(K) + \epsilon / 2,
\]
and hence
\[
\mu(A \Delta O) 
= \mu(A \backslash O) + \mu(O \backslash A)
\leq \mu(A \backslash K) + \mu(O \backslash K)
 \leq  \epsilon /2 + \epsilon /2=\epsilon.\qedhere
\]
\end{proof}


\paragraph{Littlewood's second principle:} 
\begin{quotation}
	\emph{Pointwise almost everywhere convergence is ``practically uniform convergence"}.
\end{quotation}

In other words, one can think of almost everywhere convergence as uniform convergence on a `smaller' set that can be chosen arbitrarily `close' to the full set. 

\begin{theorem}[Egorov's Theorem]
	Let $\mu$ be a finite measure on $(\bbR^d,\cB_{\bbR^d})$ and $(f_n)_{n\in\bbN}$ be a sequence of $\cB_{\bbR^d}$-measurable functions that converges $\mu$-almost everywhere to a function $f$. 
	Then for all $\epsilon > 0$ there exists a set $E\in \cB_{\bbR^d}$ such that
	$\mu(\bbR^d\backslash E) \le\epsilon$ and $f_n \to f$ uniformly on $E$.
\end{theorem}
\begin{proof}
	Since $f_n\to f$\, $\mu$-almost everywhere, where exists a $\mu$-null set $N\subset\Omega$, i.e., $\mu(N)=0$, for which $f_n(x)\to f(x)$ for every $x\in \Omega:=\bbR^d\setminus N$. Consider the sets
	\[
		E_{\ell,n} := \bigcap_{k\ge n} \left\{ \omega\in \Omega\,:\, |f_k(\omega)-f(\omega)| < \frac{1}{\ell}\right\},\qquad n,\ell\ge 1
	\]
	It is not difficult to check that $E_{\ell,n}$ is measurable for every $\ell,n\ge 1$ and that $E_{\ell,n}\subset E_{\ell,m}$ for $n\le m$. Moreover, for each $\ell\ge 1$, we have that $\Omega = \cup_{n\ge 1} E_{\ell,n}$. Hence, by the continuity-from-below property of $\mu$, we obtain
	\[
		\mu(\Omega) = \mu\Bigl(\bigcup_{n\ge 1} E_{\ell,n}\Bigr) = \lim_{n\to\infty} \mu(E_{\ell,n}).
	\]
	Now choose $n_\ell\ge 1$ such that $\mu(\Omega\backslash E_{\ell,n_\ell}) \le \varepsilon\,2^{-\ell}$ and define the measurable set
	\[
		E := \bigcap_{\ell\ge 1} E_{\ell,n_\ell}.
	\]
	Then, by the subadditivity of $\mu$, we obtain
	\[
		\mu(\Omega\backslash E) = \mu\left(\bigcup_{\ell\ge 1}\bigl(\Omega\backslash E_{\ell,n_\ell}\bigr)\right) \le \sum_{\ell\ge 1} \mu\bigl(\Omega\backslash E_{\ell,n_\ell}\bigr) = \epsilon.
	\]
	Moreover, for every $k \in \bbN$ and every $x \in E$,
	\[
		|f_k(\omega) - f(\omega)| \leq \frac{1}{\ell}\qquad\text{for all $k\ge n_k$},
	\]
	thus implying that $f_n \to f$ uniformly on $E$.
\end{proof}

\begin{remark}
	Given the inner regularity of $\mu$, one may choose $E$ compact in Egorov's Theorem. 
\end{remark}

\paragraph{Littlewood's third principle:} 
\begin{quotation}
	\emph{Every Borel measurable function is ``practically continuous"}.	
\end{quotation}

\begin{theorem}[Lusin's Theorem]\label{thm:lusin}
    Let $\mu$ be a finite measure on the measurable space $(\bbR^d, \cB_{\bbR^d})$ and $f: \bbR^d \to \bbR$ be Borel measurable. 
	Then for every $\epsilon > 0$ there exists a compact set $K \subset \bbR^d$ and a continuous function $g: \bbR^d \to \bbR$ such that $\mu(\bbR^d \backslash K) < \epsilon$ and $f \equiv g$ on $K$.
\end{theorem}

\begin{proof}
	Let $\epsilon > 0$.
	Define for $n \in \bbN$ and $k \in \bbZ$ the measurable sets
	\[
	A^n_k := \Bigl\{ \omega \in \bbR^d : \  (k-1) 2^{-n} < f(\omega) \leq k 2^{-n} \Bigr\}.
	\]
	Now there exist open sets $U^n_k \supset A^n_k$ and compact sets $K^n_k \subset A^n_k$ such that
	\[
		\mu(U_k^n \backslash A_k^n ) < \frac{1}{n 2^{|k|}} \qquad \mu(A_k^n \backslash K_k^n ) < \frac{1}{n 2^{|k|}}.
	\]
	We define the continuous functions $\varphi_k^n: \bbR^d \to \bbR$ such that $\varphi_k^n$ is compactly supported in $U_k^n$, satisfying $0 \leq \varphi_k^n \leq 1$ and $\varphi_k^n(\omega) = 1$ for $\omega \in K_k^n$. We set
	\[
		\varphi^n := \sum_{k = -2^n}^{2^n} k 2^{-n} \varphi_k^n.
	\]
	which is continuous for all $n\ge 1$. Since the functions $\varphi^n\to f$\, $\mu$-almost everywhere, by Egorov's Theorem, there is a compact set $K$ such that $\varphi^n$ converge to $f$ uniformly on $K$. Since uniform convergence preserves continuity, $f|_K$ is uniformly continuous on $K$.
	
	We now construct $g: \bbR^d \to \bbR$. Since $f|_K$ is uniformly continuous on $K$, there is a continuous increasing function $\eta: [0,\infty) \to [0,\infty)$ with $\eta(0)=0$ (also called the \emph{modulus of continuity}) such that
	\[
		|f(\omega) - f(\sigma)| \leq \eta( |\omega-\sigma|)\qquad \omega,\sigma\in K.
	\]
	Setting
	\[
		g(\omega) := \sup_{\sigma \in K} \Bigl\{f(\sigma) - \eta(|\omega-\sigma|)\Bigr\}.
	\]
	Note that $g$ is continuous on $\bbR^d$ and coincides with $f$ on $K$.
\end{proof}

The final result of this chapter is an important application of Lusin's theorem, which allows us to approximate any integrable function with continuous and bounded functions whenever $\mu$ is a finite measure. In other words, the following statement shows that the space of continuous and bounded functions $C_b(\bbR^d)$ is \emph{dense} in $L^1(\bbR^d,\mu)$. This fact is widely used in, e.g., Approximation Theory, Functional Analysis, Partial Differential Equations, and Stochastic Analysis.

\begin{theorem}[Approximation by continuous functions]\label{thm:L1-approximation}
	Let $\mu$ be a finite measure on $(\bbR^d,\cB_{\bbR^d})$ and $f:\bbR^d\to\bbR$ be $\mu$-integrable. Then for any $\varepsilon>0$, there is a bounded, continuous, $\mu$-integrable function $g:\bbR^d\to\bbR$ such that
	\[
		\int_\Omega |f-g|\,d\mu <\varepsilon.
	\]
\end{theorem}
\begin{proof}
	Let $E_n:=\{\omega\in\bbR^d\,:\, |f(\omega)|\ge n\}$. Since $\mathbf{1}_{E_n}f \to 0$ as $n\to\infty$, and $\mathbf{1}_{E_n}|f|\le |f|$ for every $n\ge 1$, we can apply DCT to conclude that
	\[
		\int_{E_n} |f|\,\dd\mu = \int_{\bbR^d} \mathbf{1}_{E_n}|f|\,\dd\mu\;\longrightarrow\;0\quad\text{as $n\to\infty$}.
	\]
	Now pick some $n\ge 1$ such that $\int_{E_n} |f|\,\dd\mu <\varepsilon/3$ and define
	\[
		f_n(\omega):= \max\{-n,\min\{f(\omega),n\}\}, \qquad\omega\in\bbR^d,
	\]
	i.e., $f_n$ is a truncation of $f$. From Lusin's theorem, we find a continuous function $g$ such that $f_n\equiv g$ on a compact set $K\subset\bbR^d$ with $\mu(\bbR^d\backslash K)<(2\varepsilon)/(3n)$. We assume w.l.o.g.\ that $|g|\le n$, since otherwise, we can consider a truncation of $g$. Altogether, this yields
	\begin{align*}
		\int_{\bbR^d} |f-g|\,\dd\mu &= \int_{\bbR^d} |f-f_n|\,\dd\mu + \int_{\bbR^d} |f_n-g|\,\dd\mu \\
		&= \int_{E_n} |f|\,\dd\mu + \int_{\bbR^d\backslash K} |f_n-g|\,\dd\mu \\
		&\le \frac{\varepsilon}{3} + 2n\,\mu(\bbR^d\backslash K) \le  \varepsilon.
	\end{align*}
	Finally, the $\mu$-integrability of $g$ holds simply due to the triangle inequality.
\end{proof}

\begin{remark}
	All three Littlewood principles can be generalized to inner and outer regular measures $\mu$ that are \emph{locally finite} on any measurable space $(\Omega,\cF)$, i.e., a measure for which every point $\omega\in\Omega$ has a neighborhood $N_\omega\in\cF$ such that $\mu(N_\omega)<+\infty$. 
	
	In particular, the Littlewood principles hold also for the Lebesgue measure $\lambda$ on $(\bbR^d,\cB_{\bbR^d})$ since $\lambda(B_\omega(r))<+\infty$ for every $\omega\in\bbR^d$ and any $r>0$.
\end{remark}


\section{Problems}

\begin{problem}\label{prb:reverse_fatou}
Prove the reverse Fatou lemma:

\smallskip
\textit{
Let $(\Omega, \cF, \mu)$ be a measure space. Let $(f_n)_{n \ge 1}$ and $f$ be non-negative, measurable functions such that $f_n \le f$ and $\int_\Omega f \, \dd \mu <\infty$. Then
\[
	\limsup_{n\to \infty} \int_\Omega f_n \dd \mu \le \int_\Omega \limsup_{n\to \infty} f_n \dd \mu.
\]
}
\end{problem}

\begin{problem}\label{pb:DCT-parametric-function}
	Let $(\Omega,\mathcal{F},\mu)$ be a measure space. Assume that $f:\Omega\times(a,b)\to\bbR$ and that $\omega\mapsto f(\omega,t)$ is integrable with respect to $\mu$ for all fixed $t\in(a,b)$. Suppose there exists a non-negative $\mu$-integrable function $g$ such that $|f(\omega,t)|\le g(\omega)$ for all $t\in(a,b)$ and all $\omega\in\Omega$. 
	\begin{enumerate}[label={(\alph*)}]
		\item Fix $t_0\in(a,b)$. Show that if $\lim_{t\to t_0} f(\omega,t)=f(\omega,t_0)$ for all $\omega\in\Omega$, then
		\[
			\lim_{t\to t_0}\int_\Omega f(\omega,t)\,\mu(d\omega) = \int_\Omega f(\omega,t_0)\,\mu(d\omega).
		\]
		\item Deduce from (a) that if $f(\omega,\cdot)$ is continuous for all $\omega$, then the map
		\[
			(a,b)\ni t\mapsto F(t):=\int_\Omega f(\omega,t)\,\mu(d\omega)\quad\text{is continuous.}
		\]
	\end{enumerate}
\end{problem}

\begin{problem}
		Let $(\Omega,\mathcal{F},\mu)$ be a measure space. Assume that $f:\Omega\times(a,b)\to\bbR$ and that $\omega\mapsto f(\omega,t)$ is integrable with respect to $\mu$ for all fixed $t\in(a,b)$. Suppose $\partial f/\partial t$ exists on $(a,b)$ for all $\omega\in \Omega$, i.e.\ for every fixed $\omega\in\Omega$,
		\[
			\frac{\partial f}{\partial t}(\omega, t_0):=\lim_{t\to t_0} \frac{f(\omega,t)-f(\omega,t_0)}{t-t_0} \quad\text{exists for all $t_0\in(a,b)$}.
		\]
		Furthermore, suppose that there is a non-negative $\mu$-integrable function $g$ such that $|\partial f/\partial t|(\omega,t)\le g(\omega)$ for all $t\in(a,b)$ and all $\omega\in\Omega$. 
		
		The goal of this exercise is to show that the following equality holds:
		\begin{equation}\label{eq:prb_diff_main}
			\frac{d}{dt}\int_\Omega f(\omega,t)\,\mu(d\omega) = \int_\Omega \frac{\partial f}{\partial t}(\omega,t)\,\mu(d\omega).
		\end{equation}
		
		We will do this in several steps.
		\begin{enumerate}[label={(\alph*)}]
			\item Show that $\omega \mapsto (\partial f/\partial t)(\omega,t)$ is measurable and integrable for all $t\in(a,b)$.
			\item Show that for $t_0\in(a,b)$ arbitrary
			\[
				\left|\frac{f(\omega,t)-f(\omega,t_0)}{t-t_0}\right| \le g(\omega)\quad\text{for any $t\in(a,b)$, $t\ne t_0$ and all $\omega\in\Omega$.}
			\]
			[\textbf{Hint:} Recall the Mean Value Theorem.]
			\item Consider the function
			\[
				F(t) = \int_\Omega f(\omega,t)\,\mu(d\omega).
			\] 
			Use Problem~\ref{pb:DCT-parametric-function} to show that $F(t)$ is differentiable on $(a,b)$ and conclude that~\eqref{eq:prb_diff_main} holds.
		\end{enumerate}		
		
		\smallskip
		
		\noindent\textbf{Hint:} Make the following steps:
		\begin{enumerate}[label={(\arabic*)}]
			\item Show that $(\partial f/\partial t)(\cdot,t)$ is measurable and integrable for all $t\in(a,b)$.
			\item Show that for $t_0\in(a,b)$ arbitrary
			\[
				\left|\frac{f(\omega,t)-f(\omega,t_0)}{t-t_0}\right| \le g(\omega)\quad\text{for any $t\in(a,b)$, $t\ne t_0$ and all $\omega\in\Omega$.}
			\]
			\item Conclude with the help of Problem~\ref{pb:DCT-parametric-function}.
		\end{enumerate}
\end{problem}

\begin{problem}
	Compute the following limits and justify the computation
	\begin{align*}
		& \lim_{n\to\infty} \int_0^1 \frac{1 + n x^2}{(1 + x^2)^n}\,\dd x\,,\qquad\lim_{n\to\infty} \int_{(0,+\infty)} \frac{x^{n-2}}{1+ x^n}\cos\left(\frac{\pi x}{n}\right) \lambda(\dd x) \,.
	\end{align*}
\end{problem}

\begin{problem}[Generalized DCT] Prove the following generalization of DCT:

\textit{
	Let $(\Omega,\mathcal{F},\mu)$ be a measure space. Assume that $f_n$, $g_n$, $f$ and $g$ are $\mu$-integrable functions satisfying
	\begin{enumerate}[label={(\roman*)}]
		\item $f_n\to f$ and $g_n\to g$ $\mu$-almost everywhere,
		\item $|f_n|\le g_n$ for all $n\in\bbN$ and $\int_\Omega g_n\,\dd\mu \to \int_\Omega g\,\dd\mu$ as $n\to \infty$.
	\end{enumerate}
	Then also $\int_\Omega f_n\,\dd\mu \to \int_\Omega f\,\dd\mu$ as $n\to\infty$.	
}
\end{problem}

\begin{problem}\label{prb:portmanteau}
Here we will prove the implication \textbf{1$\iff$3} from Theorem~\ref{thm:portmanteau}. 

\begin{enumerate}[label={(\alph*)}]
\item Prove that \textbf{1} implies \textbf{3}.
\end{enumerate}

The remainder of this problem is dedicate to proving \textbf{3} $\Rightarrow$ \textbf{1}. So we can assume that $\int_\bbR g \, \dd \mu_n \to \int_\bbR g \, \dd \mu$ holds for all continuous bounded functions with compact support.

Now let $f\colon \bbR \to \bbR$ be a continuous bounded function with $|f(x)| \le M$ for all $x \in \bbR$.
\begin{enumerate}[label={(\alph*)}]
\setcounter{enumi}{1}
\item Suppose that for any such $f$ and for any $\varepsilon > 0$ the following holds
\begin{equation}\label{eq:portmanteau_problem_main}
	\limsup_{n \to \infty} \left|\int_\bbR f \, \dd \mu_n - \int_\bbR f \, \dd \mu\right| \le \varepsilon.
\end{equation}
Prove the \textbf{1} holds.
\end{enumerate}

It is clear that we have to prove that~\eqref{eq:portmanteau_problem_main} holds for any continuous bounded function $f$. So from now on let $\varepsilon > 0$ be fixed.
\begin{enumerate}[label={(\alph*)}]
\setcounter{enumi}{2}
\item Prove that there exists an $\alpha > 0$ such that $\mu(\bbR \backslash [-\alpha, \alpha]) < \varepsilon/(2M)$.
\item Show that we can define a non-negative continuous function $g$ such that $g = 1$ on $[-\alpha, \alpha]$ and $g = 0$ on $\bbR \backslash (-(\alpha+1),\alpha+1)$.
\end{enumerate}

We now write 
\begin{align*}
	\left|\int_\bbR f \, \dd \mu_n - \int_\bbR f \, \dd \mu\right| &\le \left|\int_\bbR f \, \dd \mu_n - \int_\bbR fg \, \dd \mu_n\right| + \left|\int_\bbR f \, \dd \mu - \int_\bbR fg \, \dd \mu\right|\\ &\hspace{10pt}+ \left|\int_\bbR fg \, \dd \mu_n - \int_\bbR fg \, \dd \mu\right|
\end{align*}

\begin{enumerate}[label={(\alph*)}]
\setcounter{enumi}{4}
\item Show that
\[
	\left|\int_\bbR f \, \dd \mu - \int_\bbR fg \, \dd \mu\right| < \frac{\varepsilon}{2}.
\]
\item Show that
\[
	\limsup_{n \to \infty} \left|\int_\bbR f \, \dd \mu_n - \int_\bbR fg \, \dd \mu_n\right| < \frac{\varepsilon}{2}.
\]
\item Conclude that~\eqref{eq:portmanteau_problem_main} holds.
\end{enumerate}

%Also observe that we can define a non-negative continuous function $g$ such that $g = 1$ on $[-\alpha, \alpha]$ and $g = 0$ on $\bbR \backslash (-(\alpha+1),\alpha+1)$. Observe that $g$ is a non-negative continuous bounded function that is zero outside the interval $[-(\alpha+1), \alpha+1]$, and thus we can apply (3). We now have that
%\begin{align*}
%	\left|\int_\bbR f \, \dd \mu_n - \int_\bbR fg \, \dd \mu_n\right|
%	&= \left|\int_\bbR f (1-g) \, \dd \mu_n\right| \le M \int_\bbR (1-g) \, \dd \mu_n \\
%	&\le M \int_\bbR (1-g) \, \dd \mu_n = M\left(1-\int_\bbR g \, \dd \mu_n\right)
%\end{align*}
%Since the later term converges to $\int_\bbR g \, \dd \mu$ by our assumption we get that
%\begin{align*}
%	\limsup_{n \to \infty} \left|\int_\bbR f \, \dd \mu_n - \int_\bbR fg \, \dd \mu_n\right|
%	&\le M \int_\bbR (1-g) \, \dd \mu \le M \mu(\bbR \backslash [-\alpha,\alpha]) < \frac{\varepsilon}{2}.
%\end{align*}
%The same conclusion holds true for $\left|\int_\bbR f \, \dd \mu - \int_\bbR fg \, \dd \mu\right|$.
%
%If we now write
%\begin{align*}
%	\left|\int_\bbR f \, \dd \mu_n - \int_\bbR f \, \dd \mu\right| &\le \left|\int_\bbR f \, \dd \mu_n - \int_\bbR fg \, \dd \mu_n\right| + \left|\int_\bbR f \, \dd \mu - \int_\bbR fg \, \dd \mu\right|\\ &\hspace{10pt}+ \left|\int_\bbR fg \, \dd \mu_n - \int_\bbR fg \, \dd \mu\right|
%\end{align*}
%we see that the first two terms converge to $\varepsilon/2$ (by the computation above) while the term on the second line converges to zero by our assumption since $fg$ is also a continuous bounded function that is zero outside the interval $[-(\alpha+1),\alpha+1]$. 
\end{problem}

