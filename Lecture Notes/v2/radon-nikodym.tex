\section{A General Radon-Nikodym Theorem}


\begin{lemma}\label{lem:partial-radon-nikodym}
	Let $\mu$, $\nu$ be finite measures on $(\Omega,\cF)$ satisfying $\nu\le \mu$ on $\cF$, i.e.\ $\nu(A)\le \mu(A)$ for every $A\in\cF$. Then there exists an $(\cF,\cB_\bbR)$-measurable function $f_0$ with $0\le f_0\le 1$ such that
	\[
		\nu(E) = \int_E f_0\,\dd\mu\qquad\text{for all $E\in\cF$}.
	\]
\end{lemma}
\begin{proof}
	Let
	\[
		H := \left\{\text{$f$ measurable}\,:\, 0\le f\le 1,\; \int_E f\,\dd\mu \le \nu(E)\;\;\text{for all $E\in\cF$}\right\}.
	\]
	Note that $H\ne \emptyset$ since $0$ belongs to $H$. Moreover, when $f_1$, $f_2\in H$, also $\max\{f_1,f_2\}\in H$. Indeed, if $A=\{x\in\Omega\,:\, f_1(x)\ge f_2(x)\}$, then
	\begin{align*}
		\int_E \max\{f_1,f_2\}\,\dd\mu &= \int_{E\cap A} \max\{f_1,f_2\}\,\dd\mu + \int_{E\cap A^c} \max\{f_1,f_2\}\,\dd\mu \\
		&= \int_{E\cap A} f_1\,\dd\mu + \int_{E\cap A^c} f_2\,\dd\mu \le \nu(E\cap A) + \nu(E\cap A^c) = \nu(E).
	\end{align*}
	
	Now let $M=\sup\{\int_\Omega f\,\dd\mu\,:\,f\in H\}$. Then, $0\le M<+\infty$ and we find from the previous argument a sequence of measurable functions $(f_n)_{n\in\bbN}\subset H$ with $0\le f_1\le \cdots\le 1$ such that
	\[
		\int_\Omega f_n\,\dd\mu > M - \frac{1}{n}.
	\] 
	Define $f_0:= \lim_{n\to\infty} f_n$. Then $f_0$ is measurable. By the Monotone Convergence Theorem, $f_0\in H$ and $\int_\Omega f_0\,\dd\mu \ge M$. Hence, $\int_\Omega f_0\,\dd\mu = M$.
	
	To complete the proof, we show that $\nu(E)=\int_E f_0\,\dd\mu$ for all $E\in\cF$. Suppose otherwise, i.e., there is a set $E\in\cF$ for which $\nu(E)>\int_E f_0\,\dd\mu$. Then we can write $E=E_0\cup E_1$, where $E_1:= \{ \omega\in\Omega\,:\, f_0(\omega)=1\}$ and $E_0:= E\backslash E_1$. Since
	\begin{align*}
		\nu(E_0) + \nu(E_1) = \nu(E) > \int_E f_0\,\dd\mu = \int_{E_0} f_0\,\dd\mu + \mu(E_1) \ge \int_{E_0} f_0\,\dd\mu + \nu(E_1),
	\end{align*}
	it follows that $\nu(E_0)>\int_{E_0} f_0\,\dd\mu$. Let $F_n:= \{f_0<1-n^{-1}\}\cap E_0$, which gives a sequence of increasing measurable sets. Due to the continuity from below of $\nu$, we obtain
	\[
		\lim_{n\to\infty}\nu(F_n) = \nu\Biggl(\bigcup_{n\ge 1} F_n\Biggr) = \nu(E_0)>\int_{E_0} f_0\,\dd\mu.
	\]
	In particular, there exists some $n_0$ such that
	\begin{align*}
		\nu(F_{n_0}) &> \int_{E_0} f_0\,\dd\mu = \int_{F_{n_0}}f_0\,\dd\mu + \int_\Omega f_0\,\mathbf{1}_{\{f_0\ge 1-n_0^{-1}\}\cap E_0}\,\dd\mu \\
		&\ge \int_{F_{n_0}}f_0\, \dd\mu + \bigl(1 + n_0^{-1}\bigr)\mu(\{f_0\ge 1-n_0^{-1}\}\cap E_0) \\
		&= \int_{F_{n_0}}f_0 + \varepsilon_0\mathbf{1}_{F_{n_0}}\, \dd\mu,
	\end{align*}
	with $\varepsilon_0 := \bigl(1 + n_0^{-1}\bigr)\mu(F_0^c\cap E_0)/\mu(F_{n_0}) >0$. 
	
	Based on this fact, we claim the existence of a measurable set $F\subset F_{n_0}$ with $\mu(F)>0$ such that $f_0 + \varepsilon_0\mathbf{1}_F\in H$. Otherwise, every measurable set $F\subset F_{n_0}$ with $\mu(F)>0$ would contain a measurable subset $G\subset F$ with $\int_G f_0 + \varepsilon_0 \mathbf{1}_F\,\dd\mu > \nu(G)$. By an exhaustion argument, we can construct a disjoint partition $\bigcup_{m\ge 1} G_m = F_{n_0}$ of $F_{n_0}$ such that $\int_{G_m} f_0 + \varepsilon_0 \mathbf{1}_F\,\dd\mu > \nu(G_m)$ for all $m\ge 1$. Consequently,
\[
	\nu(F_{n_0}) > \int_{F_{n_0}}f_0 + \varepsilon_0\mathbf{1}_{F}\, \dd\mu = \sum_{m\ge 1} \int_{G_m} f_0 + \varepsilon_0 \mathbf{1}_F\,d\mu > \sum_{m\ge 1} \nu(G_m) = \nu(F_{n_0}),
\]
which is a contradiction, i.e., such a measurable set $F\subset F_{n_0}$ must exist. 

However, since $\int_\Omega f_0 + \varepsilon_0\mathbf{1}_F\,\dd\mu = M + \varepsilon_0\mu(F) > M$, this leads to another contradiction, and hence $\nu(E)=\int_E f_0\,\dd\mu$ for all $E\in\cF$ as desired.
\end{proof}


\begin{theorem}[Radon-Nikodym Theorem]\label{thm:radon-nikodym}
	Let $\mu$, $\nu$ be finite measures on $(\Omega,\cF)$. Then there exists a $\mu$-null set $D\in\cF$ and a nonnegative $\mu$-integrable function $f_0$ such that
	\[
		\nu(E) = \nu(E\cap D) + \int_E f_0\,\dd\mu\qquad\text{for all $E\in\cF$}.
	\]
\end{theorem}
\begin{proof}
	Let $\lambda=\mu+\nu$. Then $0\le \nu\le \lambda$, so by Lemma~\ref{lem:partial-radon-nikodym}, there exists a measurable function $g$ with $0\le g\le 1$ such that $\nu(E)=\int_E g\,\dd\lambda$ for all $E\in\cF$. It follows that
	$\mu(E) = \int_E (1-g)\,\dd\lambda$ for all $E\in\cF$. Let $D=\{g=1\}$. Then $\mu(D) = 0$. 
	
	Moreover, since $\nu(E)=\int_E g\,\dd\nu + \int_E g\,\dd\mu$, we have $\int_E (1-g)\,\dd\nu = \int_E g\,\dd\mu$ for all $E\in\cF$. In particular, $\int_\Omega (1-g)f\,\dd\nu = \int_\Omega gf\,\dd\mu$ for all nonnegative measurable functions $f$. Taking $f=(1+g +\cdots g^n)\mathbf{1}_E$, we learn that
	\[
		\int_E (1-g^{n+1})\,\dd\nu = \int_E g(1+g\cdots+g^n)\,\dd\mu\qquad\text{for all $E\in\cF$ and $n\ge 1$}.
	\]
	Now since $0\le g <1$ on $D^c$, the Monotone Convergence Theorem yields
	\begin{align*}
		\nu(E\cap D^c) &= \lim_{n\to\infty} \int_{E\cap D^c} (1-g^{n+1})\,\dd\nu = \lim_{n\to\infty}\int_{E\cap D^c} g(1+g\cdots+g^n)\,\dd\mu \\
		&=\int_{E\cap D^c} g(1-g)^{-1}\,\dd\mu = \int_E f_0\,d\mu,
	\end{align*}
	where $f_0:= g(1-g)^{-1}\mathbf{1}_{D^c}$, thus concluding the proof.
\end{proof}

\begin{definition}[Absolute continuity of measures]
	Let $\mu$, $\nu$ be two measures on a measurable space $(\Omega, \cF)$.
	We say that $\nu$ is \emph{absolutely continuous w.r.t.\ $\mu$}, if for every $E\in\cF$ with $\mu(E)=0$, we also have that $\nu(E)=0$, i.e.\ $\mu$-null sets are $\nu$-null sets. In this case, we write $\nu\ll\mu$.
\end{definition}

\begin{theorem}[Radon-Nikodym II]\label{thm:radon-nikodym-2}
	Let $\mu$, $\nu$ be finite measures on $(\Omega,\cF)$ such that $\nu\ll \mu$. Then there exists a unique (up to $\mu$-null sets) nonnegative $\mu$-integrable function $\dd\nu/\dd\mu$, called the \emph{Radon-Nikodym derivative of $\nu$ w.r.t.\ $\mu$} such that
	\[
		\nu(E) = \int_E \frac{\dd\nu}{\dd\mu}\,\dd\mu\qquad\text{for all $E\in\cF$}.
	\]
	The function $\dd\nu/\dd\mu$ is also often called the $\mu$-density of $\nu$.
\end{theorem}
\begin{proof}
	We apply Theorem~\ref{thm:radon-nikodym} to obtain a $\mu$-measurable function $f_0$ such that
	\[
		\nu(E) = \nu(E\cap D) + \int_E f_0\,\dd\mu \qquad\text{for all $E\in\cF$},
	\]
	where $D$ is a $\mu$-null set. In particular, $E\cap D$ is a $\mu$-null set. Since $\nu\ll \mu$, we have also that $\nu(E\cap D)=0$. Setting $\dd\nu/\dd\mu:= f_0$, we then obtain the assertion.
\end{proof}

\begin{theorem}[Radon-Nikodym III] 
	Let $\bbP$, $\bbQ$ be probability measures on $(\Omega,\cF)$ such that $\bbQ\ll \bbP$. Then there exists a unique (up to $\bbP$-null sets) nonnegative $\bbP$-integrable function $\frac{\dd\bbQ}{\dd\bbP}\in L^1(\bbP)$ such that
	\[
		\bbQ(E) = \int_E \frac{\dd\bbQ}{\dd\bbP}\,\dd\bbP\qquad\text{for all $E\in\cF$}.
	\]
\end{theorem}


\begin{definition}[Conditional expectation]\label{def:conditional_expectation}
	Let $(\Omega,\cF,\bbP)$ be a probability space, and let $X$ be an integrable random variable, i.e., $X\in L^1(\bbP)$. Let $\cG\subset\cF$ be a sub-$\sigma$-algebra. Then the \emph{conditional expectation of $X$ given $\cG$} is the unique (up to $\bbP$-null sets) $\cG$-measurable function $Z\in L^1(\bbP)$ such that
	\[
		\int_A X\,\dd\bbP = \int_A Z\,\dd\bbP\qquad\forall\,A\in\cG.
	\]
	Since $Z$ is unique in $L^1(\bbP)$, we denote it by $Z=\bbE[X|\cG]$.
\end{definition}

The existence of the conditional expectation follows from the Radon-Nikodym Theorem II (cf.\ Theorem~\ref{thm:radon-nikodym-2}). Indeed, we begin by defining finite measures on $(\Omega,\cG)$ given by
\[
	\mu_X^+(A) := \int_A X^+\,\dd\bbP,\qquad \mu_X^-(A) := \int_A X^-\,\dd\bbP\qquad\forall\,A\in\cG.
\]
Clearly, these measures satisfies $\mu_X^\pm\ll \bbP|_\cG$ since, by construction, $\mu_X^\pm(A)=0$ for every $A\in\cG$ with $\bbP|_{\cG}(A)=0$. Recall that $\bbP|_{\cG}$ is the restriction of $\bbP$ on the sub-$\sigma$-algebra $\cG$. By Theorem~\ref{thm:radon-nikodym-2}, we then find unique (up to $\bbP$-null sets) measurable functions $Z^\pm\in L^1(\bbP)$ such that
\[
	\mu_X^\pm = \int_A Z^\pm\,\dd\bbP\qquad\forall\,A\in\cG.
\]
Setting $Z:= Z^+-Z^-\in L^1(\bbP)$, we then find that
\[
	\int_A X\,\dd\bbP = \int_A Z^+\,\dd\bbP - \int_A Z^-\,\dd\bbP = \int_A Z\,\dd\bbP\qquad \forall\,A\in\cG,
\]
which is what was claimed.


\section{Problems}

\begin{problem}
	Let $\mu$, $\nu$ be two measures on $(\Omega, \cF)$ such that $\nu$ is absolutely continuous with respect to $\mu$. Show that for every $\epsilon > 0$ there exists a $\delta_\epsilon > 0$ such that for every $A \in \cF$:
	\[
		\mu(A) < \delta_\epsilon\;\;\Longrightarrow\;\; \nu(A) < \epsilon.
	\]
	\textbf{Hint:} Prove by contradiction.
\end{problem}

%\begin{proof}
%	Suppose otherwise. Then, there exists an $\epsilon > 0$ and a sequence of measurable sets $A_i \in \cF$ such that $\mu(A_i) < 2^{-i}$ but $\nu(A_i) > \epsilon$. Setting 
%	\[
%	B_i := \bigcup_{j \geq i} A_j,
%	\]
%	we obtain a decreasing sequence $(B_i)_{i\in\bbN}$ of measurable sets. Now set 
%	\[
%	B:= \bigcap_{i=1}^\infty B_i.
%	\]
%	Then, $\lim_{i \to \infty } \mu(B_i ) = 0$ and $\lim_{i \to \infty } \nu(B_i) \ge \varepsilon$, which leads to a contradiction with the assumption that $\nu$ is absolutely continuous with respect to $\mu$.
%\end{proof}

\begin{problem}
Let $(\Omega, \cF, \bbP)$ be a probability space and $\cH$ be a sub-\sigalg/ of $\cF$. Let $f, g : \Omega \to \bbR$ be $\cH$-measurable functions such that
\[
	\int_B f \, \dd \bbP = \int_B g \, \dd \bbP\qquad\text{for all $B\in \cH$}.
\]
Prove that $f=g$ $\bbP$-almost everywhere, i.e.
\[
	\int_\Omega |f - g| \, \dd \bbP = 0.
\]
\end{problem}


