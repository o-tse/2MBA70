\documentclass{lecturenotes}

\usepackage{lecture_notes}

% Reset mathcal back to previous style (not curly)
\DeclareMathAlphabet{\mathcal}{OMS}{cmsy}{m}{n}



%%%%%%%%%%%%%%%%%%%%%%%%%%%%%%%%%%%%%%%%%%%%%%%%%%%%%%%%%%%%%%%%%%%%%%%%%%%%%%%%%%%%%%%%%%%%%%%%%%%
%										For leaving comments									  %
%%%%%%%%%%%%%%%%%%%%%%%%%%%%%%%%%%%%%%%%%%%%%%%%%%%%%%%%%%%%%%%%%%%%%%%%%%%%%%%%%%%%%%%%%%%%%%%%%%%

%Name:			XXX
%Description:	Adds a piece of text in blue, surrounded by [] brackets. 
%				#1: the person the comment is addressed to
%				#2: the person the comment is from
%				#3: the comment
%Usage:			Write \XXX{collaborator}{myName}{myComment} to write [myComment] addressed to [collaborator]
\newcommand{\XXX}[3]{{\color{blue} \textbf{ [#1:  #3 \textit{ -#2-} ]}}}

%%%%%%%%%%%%%%%%%%%%%%%%%%%%%%%%%%%%%%%%%%%%%%%%%%%%%%%%%%%%%%%%%%%%%%%%%%%%%%%%%%%%%%%%%%%%%%%%%%%
%									     BibTeX commands 									      %
%%%%%%%%%%%%%%%%%%%%%%%%%%%%%%%%%%%%%%%%%%%%%%%%%%%%%%%%%%%%%%%%%%%%%%%%%%%%%%%%%%%%%%%%%%%%%%%%%%%

%Name:			Swap
%Description:	Command for properly typesetting "van der" and related expressions in Dutch and German names in the
%				bibliography. It swaps [van der] and [Lastname] in [van der Lastname] so that ordering will be performed on 
%				[Lastname] instead of [van der].
%Usage:			Write \swap{Lastname}{~van~der~}, Firstname instead of [van der Lastname, Firstname] in the author header
%				of the bibtex entry.

\newcommand*{\swap}[2]{\hspace{-0.5ex}#2#1}


%%%%%%%%%%%%%%%%%%%%%%%%%%%%%%%%%%%%%%%%%%%%%%%%%%%%%%%%%%%%%%%%%%%%%%%%%%%%%%%%%%%%%%%%%%%%%%%%%%%
%											Document											  %
%%%%%%%%%%%%%%%%%%%%%%%%%%%%%%%%%%%%%%%%%%%%%%%%%%%%%%%%%%%%%%%%%%%%%%%%%%%%%%%%%%%%%%%%%%%%%%%%%%%

\setlength\parindent{0pt}

\begin{document}

\textbf{Problem 7.1}

Similar to the proof of Fatou's lemma, we define $g_n = \sup_{k \ge n} f_n$ which are measurable due to Proposition 3.13. Moreover, we have that $\limsup_{n \to \infty} f_n = \lim_{n \to \infty} g_n$.

Next we note that $g_n \ge f_\ell$ for all $\ell \ge n$. Thus, by monotonicity of the integral, we have that
\[
	\int_\Omega g_n \, \dd \mu \ge \int_\Omega f_\ell \, \dd \mu,
\]
holds for all $\ell \ge n$, which implies that
\[
	\int_\Omega g_n \, \dd \mu \ge \sup_{k \ge n} \int_\Omega f_n \, \dd \mu.
\]

In addition, since $g_n < f$ with $f$ being non-negative and integrable we can apply Dominated Convergence to conclude that
\[
	\int_\Omega \lim_{n \to \infty} g_n \, \dd \mu = \lim_{n \to \infty} \int_\Omega g_n \, \dd \mu.
\]

Putting all this together we get
\[
	\int_\Omega \limsup_{n \to \infty} f_n \, \dd \mu
	= \int_\Omega \lim_{n \to \infty} g_n \, \dd \mu = \lim_{n \to \infty} \int_\Omega g_n \, \dd \mu \ge \lim_{n \to \infty} \sup_{k \ge n} \int_\Omega f_n \, \dd \mu.
\]

\bigskip

\textbf{Problem 7.2}

\begin{enumerate}[label={(\alph*)}]
	\item Let $t_0\in (a,b)$ be fixed. It suffices to check the continuity result for arbitrary sequences $(t_n)_{n\ge 1} \subset (a, b)$ such that $t_n\to t_0$ as $n\to\infty$. Fix such a sequence and define $g_n(\omega):= f(\omega,t_n)$ for all $\omega\in\Omega$ and $n\ge 1$. Since $\lim_{t\to t_0}f(\omega,t)=f(\omega,t_0)$ for all $\omega\in\Omega$, we deduce that $\lim_{n\to\infty} g_n(\omega) = f(\omega,t_0)$ for every $\omega\in\Omega$. Moreover, by assumption $|g_n| \le g$ for all $n \ge 1$ and $g$ is integrable. By the Dominated Convergence Theorem
\[
	\lim_{n\to\infty} \int_\Omega g_n(\omega)\,\mu(\dd\omega) = \int_\Omega f(\omega,t_0)\,\mu(\dd\omega).
\]
As the chosen sequence was arbitrary, we deduce that $\lim_{t\to t_0} F(t) = F(t_0)$.

	\item If $t\mapsto f(\omega,t)$ is continuous on $(a, b)$ for all $\omega\in\Omega$ then $\lim_{t\to t_0}f(\omega,t)=f(\omega,t_0)$ at every $t_0\in(a,b)$ for all $\omega\in\Omega$. In particular, (a) applies, showing that $\lim_{t\to t_0} F(t) = F(t_0)$ for every $t_0\in (a,b)$, i.e., $F$ is continuous on $(a, b)$.
\end{enumerate}

\end{document}