
\textbf{Problem 4.2}




\bigskip

\textbf{Problem 4.3}

\begin{enumerate}[label={(\alph*)}]
\item The fact that the sets are disjoint is immediate from the definition. Measurability follows from Lemma 3.11
\item Let us fix a $\omega \in \Omega$. Then if $f(\omega) = +\infty$ we get that $f_n(\omega) = 2^n$ holds for all $n \ge 1$ and hence $\lim_{n \to \infty} f_n(\omega) = +\infty = f(\omega)$. So assume that $f(\omega) < +\infty$. Then there exists an $M \in \bbN$ such that $f(\omega) < M$. Hence, for all $n \ge M$ we have that 
\[
	\|f_n(\omega) - f(\omega)\| = f(\omega) - f_n(\omega) \le 2^{-n},
\]
which implies that $\lim_{n \to \infty} f_n(\omega) = f(\omega)$.
\item Fix $n \ge 1$ and $\omega \in \Omega$. Clearly, if $f(\omega) = +\infty$ then $f_n(\omega) = 2^n < +\infty = f(\omega)$. 

\item Fix $\omega \in \Omega$ such that $f(\omega) < +\infty$ and $\omega \in A_k^n$ for some $0 \le k < N_n = n 2^n$. 

Note that $k 2^{-n} \le f(\omega) < (k+1) 2^{-n}$ holds and this interval can be split into two intervals as follows:
\[
	[k 2^{-n}, (k+1) 2^{-n}) = [(2k) 2^{-(n+1)}, (2k +1)2^{-(n+1)}) \cup [(2k +1)2^{-(n+1)}, (2k + 2)2^{-(n+1)}).
\] 
Hence, we conclude that either $\omega \in A_{2k}^{n+1}$ or $\omega \in A_{2k+1}^{n+1}$. In both case we get that 
\[
	f_n(\omega) = k2^{-n} = 2k n^{-(n+1)} \le f_{n+1}(\omega).
\]

\item Now let us consider the case where $\omega \in A_k^n$ with $k = n 2^n$, so that $n \le f(\omega) < +\infty$. Then, if $f(\omega) \ge n + 1$ it follows that $f_n(\omega) = n < n + 1 = f_{n+1}(\omega)$. If, on the other hand, $n \le f(\omega) < n + 1$ there exists an $2n \, 2^n \le \ell \le (2n+2) \, 2^n$ such that $\omega \in A_\ell^{n+1}$, which then implies that 
\[
	f_n(\omega) = n = (2n 2^{n}) \, 2^{-(n+1)} \le f_{n+1}(\omega).
\]
\end{enumerate}

\bigskip

\textbf{Problem 4.5}
\begin{enumerate}[label={(\alph*)}]
\item First suppose $f = \sum_{i = 1}^N a_i \mathbbm{1}_{A_i}$ is a simple function. Then $f \mathbbm{1}_B = \sum_{i = 1}^N a_i \mathbbm{1}_{A_i \cap B}$ is also a simple function and thus
\[
	\int_B f \, \dd \mu = \int_\Omega f \mathbbm{1}_B \, \dd \mu = \sum_{i = 1}^N a_i \mu(A_i \cap B) \le \mu(B) \sum_{i = 1}^N a_i \mu(A_i) = 0.
\]

Now let $f$ be a non-negative function and $g \le f$ be a simple function. Then $g \mathbbm{1}_B \le f \mathbbm{1}_B$ and thus by Definition 4.7
\begin{align*}
	\int_B f \, \dd \mu = \int_\Omega f \mathbbm{1}_B \, \dd \mu \ge \int_\Omega g \mathbbm{1}_B \, \dd \mu = 0,
\end{align*}
which implies the result.

\item Suppose $f \le g$ are non-negative functions and observe that if $h$ is a simple function such that $h \le f$ then also $h \le g$. Therefore we get
\[
	\int_\Omega f \, \dd \mu = \sup_{h \le f}\left\{\int_\Omega h \, \dd \mu\right\}
	\le \sup_{h \le g}\left\{\int_\Omega h \, \dd \mu\right\} = \int_\Omega g \, \dd \mu.
\]
\item Suppose that $h$ is a simple function. Then $\alpha h$ is also simple and it immediately follows that $\int_\Omega (\alpha h) \, \dd \mu = \alpha \int_\Omega h \, \dd \mu$. Now let $f$ be non-negative. Then $h \le f \iff \alpha h \le \alpha f$ and $h \le \alpha f \iff \alpha^{-1} h \le f$. Thus by Definition 4.7 we have
\begin{align*}
	\alpha \int_\Omega f \, \dd \mu &= \alpha \sup_{h \le f}\left\{\int_\Omega h \, \dd \mu\right\}\\
	&= \sup_{h \le f} \alpha \left\{\int_\Omega  h \, \dd \mu\right\}\\
	&= \sup_{h \le f} \left\{\int_\Omega (\alpha h) \, \dd \mu\right\}\\
	&= \sup_{\alpha^{-1} h \le f} \left\{\int_\Omega h) \, \dd \mu\right\}\\
	&= \sup_{h \le \alpha f} \left\{\int_\Omega (\alpha h) \, \dd \mu\right\} = \int_\Omega (\alpha f) \, \dd \mu.
\end{align*}
\end{enumerate}

\bigskip

\textbf{Problem 4.7}

\begin{enumerate}[label={(\alph*)}]
\item (2pts)
\textbf{1 pt}
Note that $(f \mathbbm{1}_{B})^+ = f^+ \mathbbm{1}_B$ and $(f \mathbbm{1}_B)^- = f^- \mathbbm{1}_B$. 
\textbf{1 pt}
Then we get, using Lemma 4.8,
\[
	\int_{B} f \, \dd \mu = \int_\Omega (f \mathbbm{1}_{B})^+ \, \dd \mu - \int_\Omega (f \mathbbm{1}_{B})^- \, \dd \mu
	= \int_\Omega f^+ \mathbbm{1}_B \, \dd \mu - \int_\Omega f^- \mathbbm{1}_B \, \dd \mu = 0 + 0 = 0.
\]
\item (2pts)

\textbf{1 pt}
Here we note that if $f \le g$ then $f^+ \le g^+$, while $f^- \ge g^-$. 

\textbf{1 pt}
Hence, using Lemma 4.8 again, 
\[
	\int_\Omega f \, \dd \mu = \int_\Omega f^+ \, \dd \mu - \int_\Omega f^- \, \dd \mu
	\le \int_\Omega g^+ \, \dd \mu - \int_\Omega g^- \, \dd \mu = \int_\Omega g \, \dd \mu.
\]
\item (3 pt)

\textbf{1 pt}
Assume first that $\alpha \ge  0$. Then it follows from Lemma 4.8 that
\[
	\alpha \int_\Omega f^\pm \, \dd \mu = \int_\Omega (\alpha f)^\pm \, \dd \mu,
\]
which implies the result.

\textbf{1 pt}
Now suppose that $\alpha < 0$ so that $\beta := -\alpha > 0$. Note that for any function $f$ we have
\[
	(-f)^+ := \max\{-f,0\} = \min\{f,0\} = f^-
\]
and similarly $(-f)^- = f^+$.

\textbf{1 pt}
We then get that
\[
	- \int_\Omega f \dd \mu = -\int_\Omega f^+ \, \dd \mu + \int_\Omega (f)^- \, \dd \mu
	= -\int_\Omega (-f)^- \, \dd \mu + \int_\Omega (-f)^+ \, \dd \mu = \int_\Omega (-f) \, \dd \mu. 
\]
The result then follows since
\[
	\alpha \int_\Omega f \, \dd \mu = - \beta \int_\Omega f\, \dd \mu
	= - \int_\Omega (\beta f) \, \dd \mu = \int_\Omega (-\beta f) \, \dd \mu
	= \int_\Omega (\alpha f) \, \dd \mu.
\]
\item (2 pt)
This result follows immediately from Lemma 4.8 and the observation that $(f+g)^\pm = f^\pm + g^\pm$.
\end{enumerate}

\bigskip

\textbf{Problem 4.8}

\begin{enumerate}[label={(\alph*)}]
	\item By definition, we have that $\nu_f(\Omega) = \int_\Omega f\,\dd\mu = 1$. Now let $(A_n)_{n\in\bbN}$ be a family of mutually disjoint measurable sets. Then we have that the sequence 
	\[
		g_n:= \sum_{i=1}^n f\,\mathbf{1}_{A_i} = f\,\mathbf{1}_{\bigcup_{i=1}^n\! A_i}\;\longrightarrow\; g:= f\,\mathbf{1}_{\bigcup_{i\in\bbN}\! A_i}\qquad\text{pointwise monotonically}.
	\]
	By MCT, we then have that
	\[
		\nu_f\left(\bigcup_{i\in\bbN} A_i\right) = \int_{\bigcup_{i\in\bbN}\! A_i} f\,\dd\mu = \lim_{n\to\infty} \int_{\bigcup_{i=1}^n\! A_i} f\,\dd\mu = \lim_{n\to\infty} \sum_{i=1}^n \int_{A_i} f\,\dd\mu = \sum_{i\in\bbN}\nu_f(A_i),
	\]
	thus showing that $\nu_f$ is a probability measure on $(\Omega,\cF)$.
	
	\item Following the hint, we start by considering nonnegative simple functions $g$. Suppose $g=\sum_{i=1}^n a_i \mathbf{1}_{A_i}$ for $a_i\in\bbR$ and $A_i\in\cF$ mutually disjoint. Then,
	\[
		\int_\Omega g\,d\nu_f = \sum_{i=1}^n a_i =\,\nu_f(A_i) = \sum_{i=1}^n a_i\int_{A_i} f\,\dd\mu = \int_\Omega gf\,\dd\mu.
	\]
	Now let $g$ be a nonnegative measurable function and $[g]_n$ be a sequence of nonnegative simple functions that converge pointwise monotonically to $g$. Then MCT yields
	\[
		\int_\Omega g\,\dd\nu_f = \lim_{n\to\infty}\int_\Omega [g]_n\,\dd\nu_f = \lim_{n\to\infty} \int_\Omega [g]_n f\,\dd\mu = \int_\Omega gf\,\dd\mu,
	\]
	where we used the fact that $[g]_n f$ converges pointwise monotonically to $gf$.
	
	\item Let $g$ be measurable. Then $g=g^+-g^-$, where $g^\pm$ are nonnegative measurable functions. Since $f$ is nonnegative, we have that $(fg)^\pm = f g^\pm$. Due to (b), we deduce
	\[
		\int_\Omega g^\pm\,\dd\nu_f = \int_\Omega g^\pm f\,\dd\mu = \int_\Omega (gf)^\pm\,\dd\mu.
	\]
	Hence, $g^\pm$ is $\nu_f$-integrable if and only if $(gf)^\pm$ is $\mu$-integrable. Consequently, $g$ is $\nu_f$-integrable if and only if $gf$ is $\mu$-integrable, since
	\[
		\int_\Omega |g|\,\dd\nu_f = \int_\Omega g^+\,\dd\nu_f + \int_\Omega g^-\,\dd\nu_f = \int_\Omega g^+f\,\dd\mu + \int_\Omega g^-f\,\dd\mu = \int_\Omega |gf|\,\dd\mu.
	\]
\end{enumerate}

\bigskip
\textbf{Problem 4.9}

\paragraph{($\Rightarrow$)} (4pts) 
\textbf{2 pts}
Let $f$ be $\mu$-integrable. Then both $|f|\mathbf{1}_{\{|f|<n\}}$ and $|f|\mathbf{1}_{\{|f|\ge n\}}$ are integrable, due to the monotonicity of the integral. By linearity of the integral,
\[
	\int_\Omega |f|\mathbf{1}_{\{|f|\ge n\}}\,\dd\mu = \int_\Omega |f|\,\dd\mu - \int_\Omega |f|\mathbf{1}_{\{|f|< n\}}\,\dd\mu.
\]

\textbf{2 pts}
Since the sequence $g_n:= |f|\mathbf{1}_{\{|f|< n\}}\ge 0$ converges pointwise monotonically to $|f|$, we can apply MCT to obtain
\[
	\lim_{n\to\infty} \int_\Omega |f|\mathbf{1}_{\{|f|< n\}}\,dd\mu = \int_\Omega |f|\,\dd\mu.
\]
Hence,
\[	
	\lim_{n\to\infty}\int_\Omega |f|\mathbf{1}_{\{|f|\ge n\}}\,\dd\mu = \int_\Omega |f|\,\dd\mu - \lim_{n\to\infty}\int_\Omega |f|\mathbf{1}_{\{|f|< n\}}\,\dd\mu = 0.
\]

\paragraph{($\Leftarrow$)} (3 pts) 

\textbf{1 pt}
By assumption, there is some $N\ge 1$ such that
\[
	\int_\Omega |f|\mathbf{1}_{\{|f|\ge N\}}\,\dd\mu \le 1.
\]

\textbf{2 pts}
By linearity of the integral,
\[
	\int_\Omega |f|\,\dd\mu = \int_\Omega |f|\mathbf{1}_{\{|f|< N\}}\,\dd\mu +\int_\Omega |f|\mathbf{1}_{\{|f|\ge N\}}\,\dd\mu \le N \mu\bigl(\{|f|< N\}\bigr) + 1.
\]
Since $\mu$ is a finite measure, the right-hand side is finite, implying that $f$ is $\mu$-integrable.

\bigskip
\textbf{Problem 4.10}

Observe that $\Omega = \bigcup_{n \in \bbN} \{|f| > n\}$. 

We then get that
\[
	\sum_{n = 1}^\infty \int_{\{|f| > n\}} |f| \, \dd \mu = \int_\Omega |f| \, \dd \mu < \infty.
\]

This implies that for some $N$ and all $n \ge N$: $\int_{\{|f| > n\}} |f| \, \dd \mu < 1/n$ or else the sum cannot be finite.

Now let $\varepsilon > 0$, take $M > \max\{N, 2/\varepsilon\}$ and $\delta = \varepsilon/(2M)$. Then
\begin{align*}
	\int_A |f| \, \dd \mu &= \int_A |f| \mathbf{1}_{|f|\le M} \, \dd \mu + \int_A |f| \mathbf{1}_{|f|> M} \, \dd \mu\\
	&\le M \mu(A) + \frac{1}{M} \le M\delta + \frac{1}{M} < \varepsilon.
\end{align*}

\bigskip
%
%\textbf{Problem 6.??}
%
%\begin{enumerate}[label=(\alph*)]
%\item Let $t \in \bbR$ and consider the set $A_t = (-\infty, t]$. Then by definition of the probability density function
%\[
%	\nu(A_t) = \int_{-\infty}^t \rho \, \dd \lambda = (X_\# \bbP)((-\infty, t]).
%\]
%We thus conclude that $\nu$ and $X_\# \bbP$ coincide on the family of set $A_t$ and since these generate $\cB$ Theorem 2.2.17 implies that $\nu = X_\# \bbP$.
%\item Since $g$ is a simple function, there exist an $N \in \bbN$, constants $(a_n)_{1 \le n \le N}$ and measurable sets $(A_n)_{1 \le n \le N}$ such that
%\[
%	g = \sum_{n = 1}^N a_n \mathbf{1}_{A_n}.
%\]
%
%Now, by first applying Proposition 4.8.11 and then part (a), we get that
%\begin{align*}
%	\bbE[g(X)] &= \int_{\Omega} g(X) \, \dd \bbP
%		= \int_{\Omega} g \, \dd X_\# \bbP 
%		= \int_\Omega g \, \dd \nu \\
%	&= \int_\Omega \sum_{n = 1}^N a_n \mathbf{1}_{A_n} \, \dd \nu 
%		= \sum_{n = 1}^N a_n \nu(A_n) 
%		= \sum_{n = 1}^N a_n \int_{A_n} \rho \, \dd \lambda \\
%	&= \int_\bbR \sum_{n = 1}^N a_n \mathbf{1}_{A_n} \rho \, \dd \lambda = \int_\bbR g \rho \, \dd \lambda
%\end{align*}
%
%\item First note that by part (b) we have that
%\[
%	\int_\Omega [h]_n(X) \, \dd \bbP = \int_\bbR [h_n] \rho \, \dd \lambda.
%\]
%Now we split the function $[h_n] \rho$ into its positive and negative part and note that
%\[
%	([h_n] \rho)^+ = [h]_n^+ \rho^+ + [h]_n^- \rho^- \quad \text{and} \quad
%	([h_n] \rho)^- = [h]_n^+ \rho^- + [h]_n^- \rho^+,
%\] 
%where $[h]_n^\pm$ and $\rho^\pm$ denote the positive and negative parts of $[h]_n$ and $\rho$.
%
%We will show that 
%\[
%	\int_\Omega h^+(X) \, \dd \bbP = \int_\bbR h^+ \rho \, \dd \lambda. 
%\]
%The proof for the negative part is similar.
%
%\begin{align*}
%	\int_\bbR h^+ \, \dd \nu &= \lim_{n \to \infty} \int_\bbR [h]_n^+ \, \dd \nu &&\text{by Theorem 4.3.4}\\
%	&= \lim_{n \to \infty} \int_\bbR [h]_n^+ \rho \, \dd \lambda &&\text{by part (b)} \\
%	&= \lim_{n \to \infty} \int_\bbR [h]_n^+ \rho^+ \, \dd \lambda
%		- \lim_{n \to \infty} \int_\bbR [h]_n^+ \rho^- \, \dd \lambda &&\text{by linearity of integration} \\
%	&= \int_\bbR h+ \rho^+ \, \dd \lambda - \int_\bbR h+ \rho^- \, \dd \lambda &&\text{by Theorem 4.3.4} \\
%	&= \int_\bbR h^+ \rho \, \dd \lambda &&\text{by linearity of integration}
%\end{align*}
%\item 
%\begin{align*}
%	\bbE[h(X)] &= \int_\Omega h(X) \, \dd \bbP \\
%	&= \int_\bbR h \, \dd X_\# \bbP &&\text{by Proposition 4.8.11}\\
%	&= \int_\bbR h \, \dd \nu &&\text{by part (a)}\\
%	&= \int_\bbR h \rho \, \dd \lambda &&\text{by part (c)}.
%\end{align*}
%\end{enumerate}   
%
%\bigskip
%\textbf{Problem 4.7}
%This follows from the following inequalities:
%\[
%	\int_\bbR |f|^p \, \dd \mu \ge \int_{\{|f| \ge t\}} |f|^p \, \dd \mu
%	\ge t^p \mu(\{|f| \ge t\}).
%\]