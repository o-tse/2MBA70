   
\textbf{Problem 7.1}

Similar to the proof of Fatou's lemma, we define $g_n = \sup_{k \ge n} f_n$ which are measurable due to Proposition 3.13. Moreover, we have that $\limsup_{n \to \infty} f_n = \lim_{n \to \infty} g_n$.

Next we note that $g_n \ge f_\ell$ for all $\ell \ge n$. Thus, by monotonicity of the integral, we have that
\[
	\int_\Omega g_n \, \dd \mu \ge \int_\Omega f_\ell \, \dd \mu,
\]
holds for all $\ell \ge n$, which implies that
\[
	\int_\Omega g_n \, \dd \mu \ge \sup_{k \ge n} \int_\Omega f_n \, \dd \mu.
\]

In addition, since $g_n < f$ with $f$ being non-negative and integrable we can apply Dominated Convergence to conclude that
\[
	\int_\Omega \lim_{n \to \infty} g_n \, \dd \mu = \lim_{n \to \infty} \int_\Omega g_n \, \dd \mu.
\]

Putting all this together we get
\[
	\int_\Omega \limsup_{n \to \infty} f_n \, \dd \mu
	= \int_\Omega \lim_{n \to \infty} g_n \, \dd \mu = \lim_{n \to \infty} \int_\Omega g_n \, \dd \mu \ge \lim_{n \to \infty} \sup_{k \ge n} \int_\Omega f_n \, \dd \mu.
\]

\bigskip

\textbf{Problem 7.2}

\begin{enumerate}[label={(\alph*)}]
	\item Let $t_0\in (a,b)$ be fixed. It suffices to check the continuity result for arbitrary sequences $(t_n)_{n\ge 1} \subset (a, b)$ such that $t_n\to t_0$ as $n\to\infty$. Fix such a sequence and define $g_n(\omega):= f(\omega,t_n)$ for all $\omega\in\Omega$ and $n\ge 1$. Since $\lim_{t\to t_0}f(\omega,t)=f(\omega,t_0)$ for all $\omega\in\Omega$, we deduce that $\lim_{n\to\infty} g_n(\omega) = f(\omega,t_0)$ for every $\omega\in\Omega$. Moreover, by assumption $|g_n| \le g$ for all $n \ge 1$ and $g$ is integrable. By the Dominated Convergence Theorem
\[
	\lim_{n\to\infty} \int_\Omega g_n(\omega)\,\mu(\dd\omega) = \int_\Omega f(\omega,t_0)\,\mu(\dd\omega).
\]
As the chosen sequence was arbitrary, we deduce that $\lim_{t\to t_0} F(t) = F(t_0)$.

	\item If $t\mapsto f(\omega,t)$ is continuous on $(a, b)$ for all $\omega\in\Omega$ then $\lim_{t\to t_0}f(\omega,t)=f(\omega,t_0)$ at every $t_0\in(a,b)$ for all $\omega\in\Omega$. In particular, (a) applies, showing that $\lim_{t\to t_0} F(t) = F(t_0)$ for every $t_0\in (a,b)$, i.e., $F$ is continuous on $(a, b)$.
\end{enumerate}

\bigskip

\textbf{Problem 7.3}

\begin{enumerate}[label={(\alph*)}]
	\item (4 pts) We start by showing that $(\partial f/\partial t)(\cdot,t)$ is measurable. 
	
	\textbf{2 pts}
	Let $(t_n)_{n\ge 1}\subset(a,b)$ be an arbitrary sequence with $t_n\ne t$ and $t_n\to t$ for $n\to\infty$. We set
	\[
		h_n(\omega) = \frac{f(\omega,t_n)-f(\omega,t)}{t_n-t}.
	\]
	Clearly, $h_n$ is measurable for every $n\ge 1$. 
	
	\textbf{1 pt}
	Moreover, $\lim_{n\to\infty} h_n(\omega) = (\partial f/\partial t)(\omega,t)$ by the definition of the derivative. Since $(\partial f/\partial t)(\cdot,t)$ is the pointwise limit of a sequence of measurable functions, it is also measurable. 
	
	\textbf{1 pt} Finally, $(\partial f/\partial t)(\cdot,t)$ is integrable since
	\[
		\int_\Omega |(\partial f/\partial t)(\omega,t)|\,\mu(\dd\omega) \le \int_\Omega g\,\dd\mu <+\infty.
	\]
	
	\item (3 pts)
	
	\textbf{2 pts} Let $t_0\in (a,b)$ and suppose w.l.o.g.\ $t_0<t$. Since $t\mapsto f(\omega,t)$ is differentiable on $(a,b)$ for all $\omega\in\Omega$, the Mean Value Theorem gives
	\[
		\frac{f(\omega,t)-f(\omega,t_0)}{t-t_0} = (\partial f/\partial t)(\omega,\tau)\qquad\text{	for some $\tau\in(t_0,t)$.}
	\]
	
	\textbf{1 pt}
	Taking the modulus on both sides, we obtain
	\[
		\left|\frac{f(\omega,t)-f(\omega,t_0)}{t-t_0}\right| \le |(\partial f/\partial t)(\omega,\tau)|\le g(\omega)\qquad\text{for all $\omega\in\Omega$}.
	\]
	\item (5 pts)
	
	\textbf{2 pts} Take $t_0 \in (a,b)$ and let $(t_n)_{n \ge 1}$ be a sequence in $(a,b)$ such that $t_n \to t_0$ and define
	\[
		h(\omega, t) := \frac{f(\omega, t) - f(\omega, t_0)}{t - t_0}.
	\] 
	Then by (b) and the conditions in this exercise, $h$ satisfies all the conditions listed in Problem 4.2.
	
	\textbf{1 pt}
	Next, we note that by linearity of the integral we have that
	\[
		\frac{F(t_n) - F(t_0)}{t_n - t_0} = \int_\Omega \frac{f(\omega, t_n) - f(\omega, t_0)}{t_n - t_0} \, \mu(d \omega)
		= \int_\Omega h(\omega, t_n) \, \mu(\dd \omega).
	\]
	
	\textbf{1 pt}
	Now by Problem 4.2 it holds that 
	\[
		\lim_{t_n \to t_0} \int_\Omega h(\omega, t_n) \, \mu(\dd \omega) 
		= \int_\Omega \lim_{t_n \to t_0} h(\omega, t_n) \, \mu(\dd \omega) = \int_\Omega \frac{\partial f}{\partial t}(\omega, t_0) \, \mu(\dd \omega).
	\]
	
	\textbf{1 pt}
	Since $t_0 \in(a,b)$ and the sequence $(t_n)_{n\ge 1}$ were arbitrary, we conclude that $F$ is indeed differentiable on $(a,b)$ with 
	\[
		\frac{\partial F}{\partial t}(t_0) := \lim_{t_n \to t_0} \frac{F(t_n) - F(t_0)}{t_n - t_0} 
		= \int_\Omega \frac{\partial f}{\partial t}(\omega,t_0)\,\mu(\dd\omega).
	\]
\end{enumerate}

\bigskip

\textbf{Problem 7.4}

\begin{enumerate}[label={(\alph*)}]
	\item Note that the integrand $f_n(x)=\frac{1+n x^2}{(1+x^2)^n}$ is continuous on $[0, 1]$ and non-negative. Hence, the Riemann integral and Lebesgue integral coincide, i.e.,
	\[
		\int_0^1 f_n(x)\,\dd x = \int_{[0,1]} f_n\,\dd\lambda.
	\] 
	Observe that we have the following pointwise limit
	\[
		\lim_{n\to\infty} f_n(x) = \begin{cases}
			1 &\text{if  $x=0$}, \\
			0 &\text{if $x\in (0,1]$},
		\end{cases}
	\]
	i.e., $\lim_{n\to\infty} f_n = 0$ $\lambda$-almost everywhere. Moreover, $f_n(x) \le 1$ for every $x\in[0,1]$ and $n\ge 1$. Since the constant function $g\equiv 1$ is $\lambda$-integrable on $[0,1]$, it is a valid dominator. Hence, the DCT gives
	\[
		\lim_{n\to\infty} \int_0^1 f_n(x)\,\dd x = \lim_{n\to\infty} \int_{[0,1]} f_n\,\dd \lambda = \int_{[0,1]} \lim_{n\to\infty} f_n\,\dd\lambda = 0
	\]
	
	\item For the purpose of convergence, we consider $n\ge 3$. Note that the integrand $f_n(x)=\frac{x^{n-2}}{1+x^n}\cos\left(\frac{\pi x}{n}\right)$ is continuous on $(0, +\infty)$ with pointwise limit
	\[
		\lim_{n\to\infty} f_n(x) = \begin{cases}
			0 & \text{if $x\in(0,1)$}, \\
			1/2 & \text{if $x=1$}, \\
			1/x^2 & \text{if $x>1$},
		\end{cases}
	\]
	Setting the function
	\[
		g(x) = \begin{cases}
			1 &\text{for $x\in(0,1)$},\\
			\frac{1}{x^2} &\text{for $x\ge 1$},
		\end{cases}
	\]
	we see that $f_n\le g$ $\lambda$-almost everywhere in $(0,+\infty)$ and for all $n\ge 3$. Indeed, for $x\ge 1$, we obtain
	\[
		|f_n(x)| \le \left| \frac{x^{n-2}}{1+x^n}\cos\left(\frac{\pi x}{n}\right)\right| \le \frac{x^{n-2}}{1+x^n} \le \frac{x^{n-2}}{x^n} =\frac{1}{x^2},
	\]
	while for $x\in(0,1)$, we have
	\[
		|f_n(x)| \le \left| \frac{x^{n-2}}{1+x^n}\cos\left(\frac{\pi x}{n}\right)\right| \le \frac{x^{n-2}}{1+x^n} \le 1.
	\]
	Notice that $g$ is non-negative and $\lambda$-integrable on $(0,+\infty)$. Indeed, using the MCT,
	\begin{align*}
		\int_{(0,+\infty)} g\,\dd\lambda &= \int_{(0,1)} g\,\dd\lambda + \int_{(1,+\infty)} g\,\dd\lambda = 1 + \lim_{n\to\infty} \int_{(1,n)} g\,\dd\lambda \\
		&= 1 + \lim_{n\to\infty} \int_1^n \frac{1}{x^2}\,\dd x = 1 + \lim_{n\to\infty} \Bigl(1-\frac{1}{n} \Bigr) = 2 < +\infty.
	\end{align*}
	To conclude, we apply DCT to deduce that the limit is $1$.
	
\end{enumerate}

\bigskip

\textbf{Problem 7.5}

\bigskip

\textbf{Problem 7.6}

\begin{enumerate}[label={(\alph*)}]
\item 
\item By definition
\[
	\lim_{n\to\infty} \left|\int_\bbR f \, \dd \mu_n - \int_\bbR f \, \dd \mu\right| \le \varepsilon,
\]
implies that for any $\delta > 0$
\[
	\left|\int_\bbR f \, \dd \mu_n - \int_\bbR f \, \dd \mu\right| < \varepsilon + \delta,
\]
holds for large enough $n$. Note that this holds for any $\varepsilon, \delta > 0$.

Now pick $\eta > 0$ and set $\varepsilon = \eta/2 = \delta$, then the above inequality implies that
\[
	\lim_{n\to\infty} \left|\int_\bbR f \, \dd \mu_n - \int_\bbR f \, \dd \mu\right| = 0.
\]

\item Consider the sequence of sets $A_n = \bbR\setminus [-n,n]$. Then $A_n \supset A_{n + 1}$ and $A_n \downarrow \emptyset$. Hence, it follows from Proposition 2.12 2) that $\lim_{n \to \infty} \mu(A_n) = 0$. Thus, there exists a $N$ such that $\mu(A_n) < \varepsilon/(2M)$ holds for all $n \ge N$. We can then take any $\alpha > N$.
\item The function
\[
	g(x) = \mathbbm{1}_{[-\alpha,\alpha]}(x) + \mathbbm{1}_{(-(\alpha+1), -\alpha)}(x)\left(x + (\alpha + 1)\right) + \mathbbm{1}_{(\alpha,\alpha+1)}(x)\left(-x + \alpha + 1\right)
\]
does the trick. This is simply a linear increase from zero to one from $-(\alpha + 1)$ to $-\alpha$ and from $\alpha + 1$ to $\alpha$.
\item Observe that $g$ is a non-negative continuous bounded function that is zero outside the interval $[-(\alpha+1), \alpha+1]$, and thus we can apply (3).
Using linearity of the integral, the fact that $|f| \le M$ and the definition of $g$, we get
\begin{align*}
	\left|\int_\bbR f \, \dd \mu - \int_\bbR fg \, \dd \mu\right|
	&= \left|\int_\bbR f (1-g) \, \dd \mu\right| \le M \int_\bbR (1-g) \, \dd \mu \\
	&\le M \int_\bbR (1-g) \, \dd \mu \\
	&= M\left(1-\int_\bbR g \, \dd \mu\right) \\
	&\le  M \mu(\bbR \backslash [-\alpha,\alpha]) < \frac{\varepsilon}{2}.
\end{align*}
\item Again, using linearity of the integral and the fact that $|f| \le M$ we get
\begin{align*}
	\left|\int_\bbR f \, \dd \mu_n - \int_\bbR fg \, \dd \mu_n\right|
	&= \left|\int_\bbR f (1-g) \, \dd \mu_n\right| \le M \int_\bbR (1-g) \, \dd \mu_n \\
	&\le M \int_\bbR (1-g) \, \dd \mu_n = M\left(1-\int_\bbR g \, \dd \mu_n\right)
\end{align*}

Now observe that the integral in the last term converges to $\int_\bbR g \, \dd \mu$ by (3). Thus, we obtain
\begin{align*}
	\limsup_{n \to \infty} \left|\int_\bbR f \, \dd \mu_n - \int_\bbR fg \, \dd \mu_n\right|
	&\le M \int_\bbR (1-g) \, \dd \mu \le M \mu(\bbR \backslash [-\alpha,\alpha]) < \frac{\varepsilon}{2}.
\end{align*}
\item 
Recall that
\begin{align*}
	\left|\int_\bbR f \, \dd \mu_n - \int_\bbR f \, \dd \mu\right| &\le \left|\int_\bbR f \, \dd \mu_n - \int_\bbR fg \, \dd \mu_n\right| + \left|\int_\bbR f \, \dd \mu - \int_\bbR fg \, \dd \mu\right|\\ &\hspace{10pt}+ \left|\int_\bbR fg \, \dd \mu_n - \int_\bbR fg \, \dd \mu\right|.
\end{align*}
For the first two terms, the (e) and (f) imply that the $\limsup_{n \to \infty}$ is bounded by $\varepsilon/2$. For the third term we not that $fg$ is a continuous bounded function and hence this term converges to zero by our assumption that (3) holds.

Together we then have that
\[
	\limsup_{n \to \infty} \left|\int_\bbR f \, \dd \mu_n - \int_\bbR f \, \dd \mu\right| < \varepsilon,
\]
which implies the result.
\end{enumerate}