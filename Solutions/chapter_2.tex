

\textbf{Problem 2.6}

First note that if $\mu(A \cap B) = \infty$ then by property 2 we have that also $\mu(A), \mu(B)$ and $\mu(A \cup B) = \infty$ and hence the result holds trivially. So assume now that $\mu(A \cap B) < \infty$. Since 
\[
	A \cup B = (A \setminus (A\cap B)) \cup (B \setminus (A \cap B)) \cup (A \cap B),
\] 
it follows from property 1 that
\[
	\mu(A \cup B) = \mu(A \setminus (A \cap B))) + \mu(A \cap B) + \mu(B \setminus (A \cap B)).
\] 
Adding $\mu(A \cap B) < \infty$ to both side we get
\begin{align*}
	\mu(A \cup B) + \mu(A \cap B) 
	&= \mu(A \setminus (A \cap B))) + \mu(A \cap B) + \mu(B \setminus (A \cap B)) + \mu(A \cap B)\\
	&= \mu(A) + \mu(B), 
\end{align*}
where the last line follows from applying property 3 twice.  
 
\bigskip 

\textbf{Problem 2.7}

The idea is to construct a family of disjoint sets $(E_i)_{i \in \bbN}$ with the following properties:
\begin{enumerate}
\item $E_i \subset A_i$, and
\item $\bigcup_{i \in \bbN} E_i = \bigcup_{i \in \bbN} A_i$.
\end{enumerate}

If such a sequence exists then we have
\begin{align*}
	\mu(\bigcup_{i \in \bbN} A_i) &= \mu(\bigcup_{i \in \bbN} E_i) &&\text{by 2}\\
	&= \sum_{i = 1}^\infty \mu(A_i) &&\text{because $E_i$ are disjoint and $\mu$ is $\sigma$-additive}\\
	&\le \sum_{i = 1}^\infty \mu(A_i) &&\text{by 1 and monotone property of $\mu$}.
\end{align*}

So we are left to construct the required family of sets $(E_i)_{i \in \bbN}$. The following set will do:
\[
	E_1 = A_1 \quad E_i = A_i \setminus \bigcup_{k < i}^i A_k \text{ for all } i > 1.
\]
Note that by definition the set $E_i$ are pair-wise disjoint and property1 holds. Finally, property 2 holds since $\bigcup_{i = 1}^k E_i = \bigcup_{i = 1}^k A_i$ holds for all $k \ge 1$.

\bigskip

\textbf{Problem 2.9} (23 points)
Let $\cO$ denote the open sets in $\bbR$.
\begin{enumerate}
\item (2 points) Note that the interval $(a,b)$ is open for any $a < b \in \bbR$. Hence $\cA_1 \subset \cA_1^\prime \subset \cO$ and thus by Lemma 2.1.5 we have that $\sigma(\cA_1) \subset \sigma(\cA_1^\prime) \subset \sigma(\cO) = \cB_\bbR$.
\item (2 points) The inclusion $\supset$ is trivial. So assume that $x \in O$. Then by definition there exist an $r > 0$ such that the ball $B_x(r) \subset O$. But $B_x(r) = (x-r, x+r) \in \cA_1$ so $x \in \bigcup_{I \in \cA, I \subset O} I$.
\item (3 points) Take $O \in \cO$. If we can show that $O \in \sigma(\cA)$ then $\cB_\bbR = \sigma(\cO) \subset \sigma(\cA)$. The result then follows from 1. 

From 2 it follows that $O$ is a union over a subset collection of interval $(a,b)$ where $a,b \in \bbQ$. Since $\bbQ$ is countable, the collection $\{(a,b) \, : \, a < b \in \bbQ\}$ is also countable and hence $O = \bigcup_{I \in \cA, I \subset O} I \in \sigma(\cA)$, from which it follows that $\cB_\bbR \subset \sigma(\cA)$.
\item (1 point) This follows immediately from 1 and 3 since these imply that $\cB_\bbR = \sigma(\cA_1) \subset \sigma(\cA_1^\prime) \subset \cB_\bbR$.
\item (3 points) The inclusion $\subset$ is trivial, since $(a,b] \subset (a + b +1/j)$ for any $j \in \bbN$. For the other inclusion we argue by contradiction. Suppose that $x \in \bigcap_{j \in \bbN} (a,b+1/j)$ but $x \notin (a,b]$. Then $x > b$ and hence there exists a $j \in \bbN$ such that $(b-x) > 1/j$. But this implies that $x \notin (a,b+1/j)$ which is a contradiction. So we conclude that $(a,b] \supset \bigcap_{j \in \bbN} (a,b+1/j)$.
\item (3 points) This time the inclusion $\supset$ is trivial since $(a,b-1/j] \subset (a,b)$ for every $j \in \bbN$. For the other inclusion suppose that $x \in (a,b)$. Then there exists a $r > 0$ such that the interval $(x-r,x+r) \subset (a,b)$. In particular, this implies that $b - (x+r) > 0$. Now take any $j \in \bbN$ such that $j >1/(b - (x+r))$. Then $b - x > r + 1/j$ which implies that $(x-r,x+r) \subset (x-r,b-1/j]$ and hence $x \in \bigcup_{j \in \bbN} (a,b-1/j]$.
\item (4 points) It is clear that $\cA_2 \subset \cA_2^\prime$. By 5 it follows that any interval $(a,b]$ can be obtained as a countable intersection of intervals of the form $(a,b+1/j)$. By 4 $\cB_\bbR = \sigma(\cA_1^\prime)$ which by Lemma 2.1.2 contains $\bigcap_{j \in \bbN} (a,b+1/j) = (a,b]$. So we conclude that any interval $(a,b] \in \cB_\bbR$ from which it now follows that
\[
	\sigma(\cA_2) \subset \sigma(\cA_2^\prime) \subset \sigma(\cA_1^\prime) = \cB_\bbR.
\]

For the other inclusion we consider a set $(a,b)$ with $a,b \in \bbQ$. Then by 6 we have that $(a,b) = \bigcup_{j \in \bbN} (a,b-1/j]$ where the later is a countable union of sets $(c,d]$ with $c,d \in \bbQ$ which must be in $\sigma(\cA_2)$ by definition of a \sigalg/. Hence, any interval $(a,b) \in \sigma(\cA_2)$ and we thus conclude, using 3, that
\[
	\cB_\bbR = \sigma(\cA_1) \subset \sigma(\cA_2) \subset \sigma(\cA_2^\prime) \subset \sigma(\cA_1^\prime) = \cB_\bbR,
\]
which implies the result.
\item (2 points) Step 1 is to show that any interval $[a,b)$ can be obtained as a countable intersection of intervals $(a-1/j,b)$. From this we can conclude that any set $[a,b)$ must be in $\cB_\bbR$ proving inclusions $\subset$.

For the other inclusions we have to show that any interval $(a,b)$ can be obtained as a countable union of intervals $[a+1/j, b)$, which implies that $(a,b)$ must be in the \sigalg/ generated by $[a,b)$. 
\item (3 points) The main tool is to show that each of the intervals $(-\infty, a],$ $(-\infty ,a), (a,\infty)$ and $[a,\infty)$ can be obtained by taking any allowed set operation for \sigalgs/, i.e. countable unions/intersections and finite complements. This will help use prove the $\subset$ inclusions. 

Then we show that any set of the form $(a,b), [a,b)$ or $(a,b]$ can also be obtained through countable unions/intersections and finite complements of intervals of the forms $(-\infty, a]$, $(-\infty ,a), (a,\infty)$ and $[a,\infty)$. These will then yield the $\supset$ inclusions and finish the proof.
\end{enumerate}