Let $(\Omega,\cF,\mu)$ be a measure space and $p\in[1,+\infty)$. Throughout this chapter, we denote
\[
	\|f\|_p := \left(\int_\Omega |f|^p\,\dd\mu\right)^{1/p}\qquad \text{for any measurable function $f:\Omega\to\bbR$}.
\]
For $p=+\infty$, we set
\[
\|f\|_{\infty} := \esssup \{ |f(\omega)| : \ \omega \in \Omega \} = \inf\{t \in [0,\infty) : \ \mu(\{ |f| > t\}) = 0 \}.
\]

\section{The H\"older inequality}

\begin{proposition}[H\"older's inequality] Let $(\Omega,\cF,\mu)$ be a measure space and $p,q\in[1,+\infty]$ be conjugate exponents, i.e.\ $\frac{1}{p}+\frac{1}{q}=1$. Then,
\[
	\|f g \|_1 \leq \|f\|_p\|g\|_q\qquad\text{for all measurable functions $f,g:\Omega\to\bbR$.}
\]
\end{proposition}

\begin{proof}
If the right-hand side is $+\infty$, there is nothing to prove. 

Now we will see a very important trick in proving inequalities like this. We note that it is enough to show the inequality for the case in which
\[
\int_\Omega |f|^p \,\dd \mu = \int_\Omega |g|^q \,\dd \mu = 1.
\]
By Young's inequality for conjugate exponents $p,q\in(1,+\infty)$,
\[
	ab \le \frac{1}{p}a^p + \frac{1}{q} b^q\qquad\text{for any $a,b\in[0,+\infty)$,}
\] 
we have for every $\omega \in \Omega$, that
\[
|f(\omega) g(\omega)| \leq \frac{1}{p}|f(\omega)|^p  + \frac{1}{q} |g(\omega)|^q.
\]
Hence
\[
\int_\Omega |fg|\,\dd \mu \leq \frac{1}{p} \int_\Omega |f|^p \,\dd \mu + \frac{1}{q} \int_\Omega |g|^q \,\dd \mu = 1.
\]
For the case $p=1$, $q=+\infty$, we easily get
\[
	\int_\Omega |fg|\,\dd\mu \le \int_\Omega |f|\|g\|_\infty\,\dd\mu = \|f\|_1\|g\|_\infty.\qedhere
\]
\end{proof}

\section{The Minkowski inequality}

\begin{proposition}
	Let $(\Omega,\cF,\mu)$ be a measure space and $p\in[1,+\infty]$ be conjugate exponents. Then the `triangle inequality' holds:
\[
	\|f + g\|_p \leq \|f\|_p + \|g \|_p\qquad \text{for all measurable functions $f,g:\Omega\to\bbR$}.
\]
\end{proposition}
\begin{proof}
As before, if the right-hand side is $+\infty$, then there is nothing to prove. Now suppose that $\|f\|_p,\|g\|_p<+\infty$. Then from the binomial formula for $p
\in[1,+\infty)$
\[
	|a+b|^p \le 2^{p-1}\bigl(|a|^p + |b|^p\bigr),
\]
we have that
\[
	\int_\Omega |f+g|^p\,\dd\mu \le 2^{p-1}\left(\int_\Omega |f|^p\,\dd\mu + \int_\Omega |g|^p\,\dd\mu\right),
\]
and hence $\|f+g\|_p<+\infty$. Next,
\[
\begin{split}
\|f + g\|^p_{p}
&= \int_\Omega |f + g|^p \,\dd \mu \\
&= \int_\Omega |f + g| |f + g|^{p-1} \,\dd \mu \\
&\leq \int_\Omega (|f| + |g|) |f + g|^{p-1} \,\dd \mu \\
&= \int_\Omega |f| |f+g|^{p-1} \,\dd \mu + \int_\Omega |g | |f+g|^{p-1} \,\dd \mu.
\end{split}
\]
Now we apply H\"older's inequality (with exponents $p$ and $p/(p-1)$) on both terms to obtain
\[
\|f+g\|^p_{L^p} \le \left(\int_\Omega |f|^p \dd \mu\right)^{1/p}  \|f+ g\|_p^{{p-1}} + \left(\int_\Omega |g|^p \dd \mu\right)^{1/p}  \|f+ g\|_p^{{p-1}}.
\]
Finally, we divide both sides by $\|f + g\|_p^{p-1}$ and find
\[
\|f+g\|_p \leq \|f\|_p + \|g\|_p.
\]
As for the case $p=+\infty$, we use the triangle inequality to obtain $|f+g|\le |f| + |g|$, and hence,
\[
	|f(\omega)+g(\omega)| \le \|f\|_\infty + \|g\|_\infty\qquad\text{for $\mu$-almost every $\omega\in\Omega$}.
\]
Taking the essential supremum then yields the required inequality.
\end{proof}

\section{Normed and semi-normed vector spaces}
\label{se:normed-and-seminormed-spaces}

Recall that a norm $\| \cdot \|$ on a vector space $V$ is a function $V \to [0,\infty)$ such that
\begin{enumerate}
\item $\|v\|= 0 \Leftrightarrow v = 0$ for all $v \in V$
\item $\|\lambda v \| = |\lambda| \|v\|$ for all $\lambda \in \mathbb{R}$ and $v \in V$
\item $\|v + w\| \leq \|v\| + \|w \|$ for all $v, w \in V$.
\end{enumerate}

If only the last two properties hold, $\|. \|$ is instead called a \emph{seminorm}.

Let $(V, \|\cdot \|)$ be a semi-normed space. We say that a sequence $(v_n)_{n\in\bbN} \subset V$ is a Cauchy sequence if for every $\epsilon > 0$ there exists an $N \in \mathbb{N}$ such that for all $m, n \geq N$,
\[
\| v_m - v_n \| < \epsilon.
\]
We say that a semi-normed space is \emph{complete}, if and only if every Cauchy sequence converges, that is, for every Cauchy sequence $(v_n)_{n\in\bbN} \subset V$ there exists a $v \in V$ such that
\[
\lim_{n \to \infty} \|v_n - v\| = 0.
\]

To every semi-normed space $(V, \|\cdot\|)$ one can associate a normed linear space in a standard way. 
One defines the equivalence relation $\sim$ by $v \sim w$ if and only if $\|v - w\| = 0$. Denote by $W$ the linear space of equivalence classes. One defines a norm on equivalence classes $[v]$ and $[w]$ in $W$ by $\|[w]-[v]\| = \|w - v\|$. If $(V, \|\cdot\|)$ is a complete semi-normed space, then $W$ is a \emph{Banach space}, which is a complete normed linear space.

\section{Semi-normed $\bbL^p$ spaces}

We have seen in Section~\ref{ch:integrable} that the set of $\mu$-integrable functions form a vector space (over $\mathbb{R}$). For $p \in [0,+\infty)$, we define the vector space $\bbL^p$ of integrable functions $f$ for which 
\[
	\| f \|_{p} < +\infty.
\]
By the Minkowski inequality, $\|\cdot\|_{p}$ is a seminorm on $\bbL^p$ for every $p \in [1,\infty]$.

Clearly, the seminorm $\|. \|_p$ is not a norm on $\bbL^p$: indeed $\|f - g\|_p=0$ if and only if $f(\omega) = g(\omega)$, for $\mu$-almost every $\omega \in \Omega$. 
We follow the standard construction described in Section \ref{se:normed-and-seminormed-spaces} to create an associated normed linear space. We say that $f \sim g$ if and only if $f$ is equal to $g$ $\mu$-almost everywhere. 
We denote by $L^p$ the vector space of equivalence classes
\[
L^p := \bbL^p/\sim.
\]

\section{Completeness of $L^p$-spaces}

\begin{theorem}[Completeness of $L^p$ spaces]
The normed linear space $L^p$ is complete, and is thus a Banach space, for every $p \in [1,+\infty]$.	
\end{theorem}

\begin{proof}
First let $p \in [1,\infty)$ and let $(f_n)_{n\in\bbN}$ be a Cauchy sequence. The trick is to select a subsequence $(f_{n_k})_{k\in\bbN}$ such that 
\[
\| f_{n_{k+1}} - f_{n_k} \|_{L^p(\Omega)} < 4^{-k-1}.
\]
For ease of notation we set $g_k := f_{n_{k}}$.
Note that by a telescoping argument, for all $\ell \geq k$,
\[
\| g_\ell - g_k \|_{L^p(\Omega)} < 4^{-k}.
\]
Then
\[
\mu \left(\left\{ x \in \Omega : \ |g_{k+1}(\omega) - g_k(\omega) | 
> 2^{-k}  \right\} \right) < \frac{1}{2^{-k p} }\| g_{k+1} - g_k \|_{L^p(\Omega)}^p <2^{-kp}.
\]
In particular, by the Borel-Cantelli Lemma, for $\mu$-a.e. $x \in \Omega$, there is an $N \in \mathbb{N}$ such that
\[
|g_{k+1}(\omega) - g_k(\omega)| \leq 2^{-k}\qquad\text{for all $k > N$}.
\]
For such $x$, the sequence $(g_k(\omega))_{k\in\bbN}$ is Cauchy. So by the completeness of $\mathbb{R}$, a limit exists, which we call $f(\omega)$.

By Fatou's Lemma,
\[
\| g_k - f\|_p \leq \liminf_{\ell \to \infty} \| g_k - g_\ell \|_p \leq 4^{-k}.
\]
To see that this implies that $f_n$ converges to $f$ in $L^p$, we take an arbitrary $\epsilon > 0$. 
Since $f_n$ is a Cauchy sequence, there exists an $N_1 \in \mathbb{N}$ such that for all $m, n \geq N_1$,
\[
\|f_{n} - f_m\|_{p} < \frac{\epsilon}{2}.
\]
Now there exists an $K \in \mathbb{N}$, with $K > N_1$ (and therefore $n_K > N_1$) such that for all $k \geq K$,
\[
\|f_{n_k} - f \|_{p} < \frac{\epsilon}{2}.
\]
Set $N_2 := \max( N_1, n_{K} )$. Then, for $n \geq N_2$, we find
\[
\| f_n - f \|_{p} 
\leq \| f_{n} - f_{n_{K}} \|_{p} + \|f_{n_K} - f\|_{L^p(\Omega)}  < \frac{\epsilon}{2} + \frac{\epsilon}{2} = \epsilon,
\]
which gives the required convergence.

The proof of completeness of $L^\infty(\Omega)$ follows similar lines but is in a way easier. Let again $(f_n)_{n\in\bbN}$ be a Cauchy sequence and select a subsequence $(f_{n_k})_{k\in\bbN}$ such that
\[
\| f_{n_{k+1}} - f_{n_k} \|_{L^\infty(\Omega)} < 4^{-k-1}.
\]
We define again $g_k = f_{n_k}$.
Then
\[
\mu\left( \left\{ x \in \Omega : \ |g_{k+1}(\omega) - g_k(\omega) | \geq 4^{-k-1} \right\} \right) = 0.
\]
So, $(g_k(\omega))_{k\in\bbN}$ is a Cauchy-sequence for almost every $\omega \in \Omega$.
For such $\omega$, the limit as $k \to \infty$ of $g_k(\omega)$ exists, and we denote it by $f(\omega)$.
Moreover, 
\[
\mu\left( \left\{ x \in \Omega : |g_k(\omega) - f(\omega)|  \geq 4^{-k} \right\} \right) = 0.
\]
It follows that $g_k$ converges to $f$ in $L^\infty(\Omega)$ as $k \to \infty$, and therefore that $f_n$ converges to $f$ in $L^\infty(\Omega)$ as $n \to \infty$ using the same argument as above.
\end{proof}


	
