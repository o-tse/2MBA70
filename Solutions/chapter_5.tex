
\textbf{Problem 5.2}
\begin{enumerate}[label={(\alph*)}]
	\item Let $t_0\in (a,b)$ be fixed. It suffices to check the continuity result for arbitrary sequences $(t_n)_{n\ge 1} \subset (a, b)$ such that $t_n\to t_0$ as $n\to\infty$. Fix such a sequence and define $g_n(\omega):= f(\omega,t_n)$ for all $\omega\in\Omega$ and $n\ge 1$. Since $\lim_{t\to t_0}f(\omega,t)=f(\omega,t_0)$ for all $\omega\in\Omega$, we deduce that $\lim_{n\to\infty} g_n(\omega) = f(\omega,t_0)$ for every $\omega\in\Omega$. Moreover, by assumption $|g_n| \le g$ for all $n \ge 1$ and $g$ is integrable. By the Dominated Convergence Theorem
\[
	\lim_{n\to\infty} \int_\Omega g_n(\omega)\,\mu(\dd\omega) = \int_\Omega f(\omega,t_0)\,\mu(\dd\omega).
\]
As the chosen sequence was arbitrary, we deduce that $\lim_{t\to t_0} F(t) = F(t_0)$.

	\item If $t\mapsto f(\omega,t)$ is continuous on $(a, b)$ for all $\omega\in\Omega$ then $\lim_{t\to t_0}f(\omega,t)=f(\omega,t_0)$ at every $t_0\in(a,b)$ for all $\omega\in\Omega$. In particular, (a) applies, showing that $\lim_{t\to t_0} F(t) = F(t_0)$ for every $t_0\in (a,b)$, i.e., $F$ is continuous on $(a, b)$.
\end{enumerate}
 

\bigskip
\textbf{Problem 5.3}

\begin{enumerate}[label={(\arabic*)}]
	\item We start by showing that $(\partial f/\partial t)(\cdot,t)$ is measurable. Let $(t_n)_{n\ge 1}\subset(a,b)$ be an arbitrary sequence with $t_n\ne t$ and $t_n\to t$ for $n\to\infty$. We set
	\[
		g_n(\omega) = \frac{f(\omega,t_n)-f(\omega,t)}{t_n-t}.
	\]
	Clearly, $g_n$ is measurable for every $n\ge 1$. Moreover, $\lim_{n\to\infty} g_n(\omega) = (\partial f/\partial t)(\omega,t)$ by the definition of the derivative. Since $(\partial f/\partial t)(\cdot,t)$ is the pointwise limit of a sequence of measurable functions, it is also measurable. Clearly, $(\partial f/\partial t)(\cdot,t)$ is integrable since
	\[
		\int_\Omega |(\partial f/\partial t)(\omega,t)|\,\mu(\dd\omega) \le \int_\Omega g\,\dd\mu <+\infty.
	\]
	
	\item Let $t_0\in (a,b)$ and suppose w.l.o.g.\ $t_0<t$. Since $t\mapsto f(\omega,t)$ is differentiable on $(a,b)$ for all $\omega\in\Omega$, the Mean Value Theorem gives
	\[
		\frac{f(\omega,t)-f(\omega,t_0)}{t-t_0} = (\partial f/\partial t)(\omega,\tau)\qquad\text{	for some $\tau\in(t_0,t)$.}
	\]
	Taking the modulus on both sides, we obtain
	\[
		\left|\frac{f(\omega,t)-f(\omega,t_0)}{t-t_0}\right| \le |(\partial f/\partial t)(\omega,\tau)|\le g(\omega)\qquad\text{for all $\omega\in\Omega$}.
	\]
	\item We now have all the ingredients needed to apply the DCT, which yields
	\[
		\lim_{n\to\infty} \frac{F(t_n)-F(t)}{t_n-t} = \lim_{n\to\infty} \int_\Omega g_n\,\dd\mu = \int_\Omega (\partial f/\partial t)(\omega,t)\,\mu(\dd\omega).
	\]
	Since $t\in(a,b)$ and the sequence $(t_n)_{n\ge 1}$ was arbitrary, we conclude that $F$ is differentiable on $(a,b)$ with 
	\[
		F'(t) = \int_\Omega (\partial f/\partial t)(\omega,t)\,\mu(\dd\omega).
	\]
\end{enumerate}

