
\textbf{Problem 6.2}


\bigskip
\textbf{Problem 6.4}

	Let $E_n:=\{\omega\in\bbR^d\,:\, |f(\omega)|\ge n\}$. Since $\mathbf{1}_{E_n}f \to 0$ as $n\to\infty$, and $\mathbf{1}_{E_n}|f|\le |f|$ for every $n\ge 1$, we can apply DCT to conclude that
	\[
		\int_{E_n} |f|\,\dd\mu = \int_{\bbR^d} \mathbf{1}_{E_n}|f|\,\dd\mu\;\longrightarrow\;0\quad\text{as $n\to\infty$}.
	\]
	Now pick some $n\ge 1$ such that $\int_{E_n} |f|\,\dd\mu <\varepsilon/3$ and define
	\[
		f_n(\omega):= \max\{-n,\min\{f(\omega),n\}\}, \qquad\omega\in\bbR^d,
	\]
	i.e., $f_n$ is a truncation of $f$. From Lusin's theorem, we find a continuous function $g$ such that $f_n\equiv g$ on a compact set $K\subset\bbR^d$ with $\mu(\bbR^d\backslash K)<(2\varepsilon)/(3n)$. We assume w.l.o.g.\ that $|g|\le n$, since otherwise, we can consider a truncation of $g$. Altogether, this yields
	\begin{align*}
		\int_{\bbR^d} |f-g|\,\dd\mu &= \int_{\bbR^d} |f-f_n|\,\dd\mu + \int_{\bbR^d} |f_n-g|\,\dd\mu \\
		&= \int_{E_n} |f|\,\dd\mu + \int_{\bbR^d\backslash K} |f_n-g|\,\dd\mu \\
		&\le \frac{\varepsilon}{3} + 2n\,\mu(\bbR^d\backslash K) \le  \varepsilon.
	\end{align*}
	Finally, $g\in L^1(\bbR^d,\mu)$ holds simply due to the triangle inequality.


