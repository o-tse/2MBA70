\documentclass[a4paper,10pt]{article}

\usepackage{lecture_notes}

% Reset mathcal back to previous style (not curly)
\DeclareMathAlphabet{\mathcal}{OMS}{cmsy}{m}{n}
\setlength\parindent{0pt}

\begin{document}

\subsection*{Solutions Week 1}

\textbf{Problem 2.6}

First note that if $\mu(A \cap B) = \infty$ then by property 2 we have that also $\mu(A), \mu(B)$ and $\mu(A \cup B) = \infty$ and hence the result holds trivially. So assume now that $\mu(A \cap B) < \infty$. Since 
\[
	A \cup B = (A \setminus (A\cap B)) \cup (B \setminus (A \cap B)) \cup (A \cap B),
\] 
it follows from property 1 that
\[
	\mu(A \cup B) = \mu(A \setminus (A \cap B))) + \mu(A \cap B) + \mu(B \setminus (A \cap B)).
\] 
Adding $\mu(A \cap B) < \infty$ to both side we get
\begin{align*}
	\mu(A \cup B) + \mu(A \cap B) 
	&= \mu(A \setminus (A \cap B))) + \mu(A \cap B) + \mu(B \setminus (A \cap B)) + \mu(A \cap B)\\
	&= \mu(A) + \mu(B), 
\end{align*}
where the last line follows from applying property 3 twice.  
 
\bigskip 

\textbf{Problem 2.7}

The idea is to construct a family of disjoint sets $(E_i)_{i \in \bbN}$ with the following properties:
\begin{enumerate}
\item $E_i \subset A_i$, and
\item $\bigcup_{i \in \bbN} E_i = \bigcup_{i \in \bbN} A_i$.
\end{enumerate}

If such a sequence exists then we have
\begin{align*}
	\mu(\bigcup_{i \in \bbN} A_i) &= \mu(\bigcup_{i \in \bbN} E_i) &&\text{by 2}\\
	&= \sum_{i = 1}^\infty \mu(A_i) &&\text{because $E_i$ are disjoint and $\mu$ is $\sigma$-additive}\\
	&\le \sum_{i = 1}^\infty \mu(A_i) &&\text{by 1 and monotone property of $\mu$}.
\end{align*}

So we are left to construct the required family of sets $(E_i)_{i \in \bbN}$. The following set will do:
\[
	E_1 = A_1 \quad E_i = A_i \setminus \bigcup_{k < i}^i A_k \text{ for all } i > 1.
\]
Note that by definition the set $E_i$ are pair-wise disjoint and property1 holds. Finally, property 2 holds since $\bigcup_{i = 1}^k E_i = \bigcup_{i = 1}^k A_i$ holds for all $k \ge 1$.

\end{document}
