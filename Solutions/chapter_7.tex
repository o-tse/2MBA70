   
\textbf{Problem 7.1}
\begin{enumerate}[label={(\alph*)}]
\item Note that $\cA_1 \times \cA_2 \subset \cF_1 \times \cF_2$, and hence
\[
	\sigma(\cA_1 \times \cA_2) \subset \cF_1 \otimes \cF_2.
\]
\item Let $B \in \cA_2$. Then we have that
\[
	\Omega_1 \times B = \bigcup_{n \ge 1} A_n \times B \in \sigma(\cA_1 \times \cA_2) 
\]
since $A_n \times B \in \sigma(\cA_1 \times \cA_2)$ for all $n \ge 1$. So $\Omega_1 \in \Sigma$

For the second property, let $C \in \Sigma$ and note that $C^c \times B = (\Omega_1 \times B) \setminus (C \times B)$.
Since both these sets are in $\sigma(\cA_1 \times \cA_2)$ it follows that $C^c \times B \in \sigma(\cA_1 \times \cA_2)$ and hence $C^c \in \Sigma$.

Finally consider a countable sequence $(C_n)_{n \ge 1}$ of sets in $\Sigma$. Then for any $B \in \cA_2$
\[
	\left(\bigcup_{n \ge 1} C_n \right) \times B = \bigcup_{n \ge 1} (C_n \times B) \in \sigma(\cA_1 \times \cA_2),
\]
since each $C_n \times B \in \sigma(\cA_1 \times \cA_2)$.
\item Note that $\cA_1 \subset \Sigma_1 \subset \cF_1$. From which it follows that $\Sigma_1 = \cF_1$. But then, from the definition of $\Sigma_1$ we have that $\cF_1 \times \cA_2 \subset \sigma(\cA_1 \times \cA_2)$.
\item We can show in a similar fashion that
\[
	\Sigma_2 := \{C \in \cF_2 \, : \, B \times C \in \sigma(\cA_1 \times \cA_2) \, \forall B \in \cA_1\}.
\]
is a \sigalg/ on $\Omega_2$, from which we conclude that $\cA_1 \times \cF_2 \subset \sigma(\cA_1 \times \cA_2)$.
\item take any $A \in \cF_1$ and $B \in \cF_2$. Then
\[
	A \times B = (A \times \Omega_2) \cap (\Omega_1 \times B) = \bigcup_{n,m \ge 1} (A \times B_m) \cap (A_n \times B) \in \sigma(\cA_1 \times \cA_2).
\]
From this we conclude that $\cF_1 \times \cF_2 \subset \sigma(\cA_1 \times \cA_2)$, which finishes the proof.
\end{enumerate}

\bigskip
\textbf{Problem 7.4}

One direction is easy. Assume that $X_1$ and $X_2$ are independent according to Definition 7.1.4. Now take any $a,b \in \bbR$ and note that $A_1 := X_1^{-1}((-\infty,a]) \in \sigma(X_1)$ and $A_2 := X_2^{-1}((-\infty, b]) \in \sigma(X_2)$. Then by the definition of independence we have that
\[
	\mathbb{P}(X_1 \le a, X_2 \le b) = \bbP(A_1 \cap A_2) = \bbP(A_1) \bbP(A_2) = \mathbb{P}(X_1 \le a) \mathbb{P}(X_2 \le b).
\]

So let us focus now on the other direction. Assume that for all $a,b \in \bbR$
\[
	\mathbb{P}(X_1 \le a, X_2 \le b) = \mathbb{P}(X_1 \le a) \mathbb{P}(X_2 \le b).
\]
We now have to show that $X_1$ and $X_2$ are independent according to Definition 7.1.4.

First note that since the family $(-\infty, a] \times (-\infty ,b]$ generate the 2-dimensional Borel \sigalg/ we have, using Theorem 2.2.17, that
\[
	\bbP(X_1 \in B_1, X_2 \in B_2) = \bbP(X_1 \in B_1) \bbP(X_2 \in B_2)
\]
for all $B_1, B_2 \in \cB_{\bbR}$.

Now fix a set $B_2 \in \cB_{\bbR}$, set $A_2 := X_2^{-1}(B_2) \in \sigma(X_2)$, and define the following two measures on the space $(\Omega, \sigma(X_1))$
\[
	\mu_1(A) = \bbP(A \cap A_2) \quad \text{and} \quad \mu_2(A) = \bbP(A)\bbP(A_2).
\]  

Let $a \in \bbR$ and consider the set $A_1 := X_1^{-1}((-\infty ,a]) \in \sigma(X_1)$. Then, by our assumption we have that
\[
	\mu_1(A_1) = \bbP(A_1 \cap A_2) = \bbP(X_1 \le a, X_2 \in B_2) = \bbP(X_1 \le a) \bbP(X_2 \in B_2) = \bbP(A_1) \bbP(A_2)
	= \mu_2(A_1).
\]
In other words, the measures $\mu_1, \mu_2$ coincide on the set $\{X_1^{-1}((-\infty, a]) \, : \, a \in \bbR\}$.

Since the set $(-\infty,a]$ generate $\cB_{\bbR}$ it follows that 
\[
	\sigma(X_1) = \sigma(\{X_1^{-1}((-\infty, a]) \, : \, a \in \bbR\}).
\]
In addition, this set satisfies the conditions of Theorem 2.2.17 and hence we conclude that $\mu_1(A) = \mu_2(A)$ for all $A \in \sigma(X_1)$.

We can repeat this argument for the two measures on $(\Omega, \sigma(X_2))$
\[
	\nu_1(A) = \bbP(A_1 \cap A) \quad \text{ and } \quad \nu_2(A) = \bbP(A_1) \bbP(A), 
\]
where $A_1 \in \{X_1^{-1}((-\infty, a]) \, : \, a \in \bbR\}$ is fixed. 

From this we conclude that for any $A_1 \in \sigma(X_1)$ and $A_2 \in \sigma(X_2)$
\[
	\bbP(A_1 \cap A_2) = \bbP(A_1)\bbP(A_2)
\]
and hence $X_1$ and $X_2$ are independent.