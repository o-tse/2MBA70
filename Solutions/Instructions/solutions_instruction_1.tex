\documentclass{lecturenotes}

\usepackage{lecture_notes}

% Reset mathcal back to previous style (not curly)
\DeclareMathAlphabet{\mathcal}{OMS}{cmsy}{m}{n}



%%%%%%%%%%%%%%%%%%%%%%%%%%%%%%%%%%%%%%%%%%%%%%%%%%%%%%%%%%%%%%%%%%%%%%%%%%%%%%%%%%%%%%%%%%%%%%%%%%%
%										For leaving comments									  %
%%%%%%%%%%%%%%%%%%%%%%%%%%%%%%%%%%%%%%%%%%%%%%%%%%%%%%%%%%%%%%%%%%%%%%%%%%%%%%%%%%%%%%%%%%%%%%%%%%%

%Name:			XXX
%Description:	Adds a piece of text in blue, surrounded by [] brackets. 
%				#1: the person the comment is addressed to
%				#2: the person the comment is from
%				#3: the comment
%Usage:			Write \XXX{collaborator}{myName}{myComment} to write [myComment] addressed to [collaborator]
\newcommand{\XXX}[3]{{\color{blue} \textbf{ [#1:  #3 \textit{ -#2-} ]}}}

%%%%%%%%%%%%%%%%%%%%%%%%%%%%%%%%%%%%%%%%%%%%%%%%%%%%%%%%%%%%%%%%%%%%%%%%%%%%%%%%%%%%%%%%%%%%%%%%%%%
%									     BibTeX commands 									      %
%%%%%%%%%%%%%%%%%%%%%%%%%%%%%%%%%%%%%%%%%%%%%%%%%%%%%%%%%%%%%%%%%%%%%%%%%%%%%%%%%%%%%%%%%%%%%%%%%%%

%Name:			Swap
%Description:	Command for properly typesetting "van der" and related expressions in Dutch and German names in the
%				bibliography. It swaps [van der] and [Lastname] in [van der Lastname] so that ordering will be performed on 
%				[Lastname] instead of [van der].
%Usage:			Write \swap{Lastname}{~van~der~}, Firstname instead of [van der Lastname, Firstname] in the author header
%				of the bibtex entry.

\newcommand*{\swap}[2]{\hspace{-0.5ex}#2#1}


%%%%%%%%%%%%%%%%%%%%%%%%%%%%%%%%%%%%%%%%%%%%%%%%%%%%%%%%%%%%%%%%%%%%%%%%%%%%%%%%%%%%%%%%%%%%%%%%%%%
%											Document											  %
%%%%%%%%%%%%%%%%%%%%%%%%%%%%%%%%%%%%%%%%%%%%%%%%%%%%%%%%%%%%%%%%%%%%%%%%%%%%%%%%%%%%%%%%%%%%%%%%%%%

\setlength\parindent{0pt}

\begin{document}

\textbf{Problem 2.6}
Let $\cO$ denote the open sets in $\bbR$.
\begin{enumerate}[label=(\alph*)]
\item Note that the interval $(a,b)$ is open for any $a < b \in \bbR$. Hence $\cA_1 \subset \cA_1^\prime \subset \cO$ and thus by Lemma 2.1.5 we have that $\sigma(\cA_1) \subset \sigma(\cA_1^\prime) \subset \sigma(\cO) = \cB_\bbR$.
\item The inclusion $\supset$ is trivial. So assume that $x \in O$. Then by definition there exist an $r > 0$ such that the ball $B_x(r) \subset O$. But $B_x(r) = (x-r, x+r) \in \cA_1$ so $x \in \bigcup_{I \in \cA, I \subset O} I$.
\item Take $O \in \cO$. If we can show that $O \in \sigma(\cA)$ then $\cB_\bbR = \sigma(\cO) \subset \sigma(\cA)$. The result then follows from 1. 

From 2 it follows that $O$ is a union over a subset collection of interval $(a,b)$ where $a,b \in \bbQ$. Since $\bbQ$ is countable, the collection $\{(a,b) \, : \, a < b \in \bbQ\}$ is also countable and hence $O = \bigcup_{I \in \cA, I \subset O} I \in \sigma(\cA)$, from which it follows that $\cB_\bbR \subset \sigma(\cA)$.
\item This follows immediately from 1 and 3 since these imply that $\cB_\bbR = \sigma(\cA_1) \subset \sigma(\cA_1^\prime) \subset \cB_\bbR$.
\end{enumerate}

\bigskip


\end{document}