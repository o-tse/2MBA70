\documentclass{lecturenotes}

\usepackage{lecture_notes}

% Reset mathcal back to previous style (not curly)
\DeclareMathAlphabet{\mathcal}{OMS}{cmsy}{m}{n}



%%%%%%%%%%%%%%%%%%%%%%%%%%%%%%%%%%%%%%%%%%%%%%%%%%%%%%%%%%%%%%%%%%%%%%%%%%%%%%%%%%%%%%%%%%%%%%%%%%%
%										For leaving comments									  %
%%%%%%%%%%%%%%%%%%%%%%%%%%%%%%%%%%%%%%%%%%%%%%%%%%%%%%%%%%%%%%%%%%%%%%%%%%%%%%%%%%%%%%%%%%%%%%%%%%%

%Name:			XXX
%Description:	Adds a piece of text in blue, surrounded by [] brackets. 
%				#1: the person the comment is addressed to
%				#2: the person the comment is from
%				#3: the comment
%Usage:			Write \XXX{collaborator}{myName}{myComment} to write [myComment] addressed to [collaborator]
\newcommand{\XXX}[3]{{\color{blue} \textbf{ [#1:  #3 \textit{ -#2-} ]}}}

%%%%%%%%%%%%%%%%%%%%%%%%%%%%%%%%%%%%%%%%%%%%%%%%%%%%%%%%%%%%%%%%%%%%%%%%%%%%%%%%%%%%%%%%%%%%%%%%%%%
%									     BibTeX commands 									      %
%%%%%%%%%%%%%%%%%%%%%%%%%%%%%%%%%%%%%%%%%%%%%%%%%%%%%%%%%%%%%%%%%%%%%%%%%%%%%%%%%%%%%%%%%%%%%%%%%%%

%Name:			Swap
%Description:	Command for properly typesetting "van der" and related expressions in Dutch and German names in the
%				bibliography. It swaps [van der] and [Lastname] in [van der Lastname] so that ordering will be performed on 
%				[Lastname] instead of [van der].
%Usage:			Write \swap{Lastname}{~van~der~}, Firstname instead of [van der Lastname, Firstname] in the author header
%				of the bibtex entry.

\newcommand*{\swap}[2]{\hspace{-0.5ex}#2#1}


%%%%%%%%%%%%%%%%%%%%%%%%%%%%%%%%%%%%%%%%%%%%%%%%%%%%%%%%%%%%%%%%%%%%%%%%%%%%%%%%%%%%%%%%%%%%%%%%%%%
%											Document											  %
%%%%%%%%%%%%%%%%%%%%%%%%%%%%%%%%%%%%%%%%%%%%%%%%%%%%%%%%%%%%%%%%%%%%%%%%%%%%%%%%%%%%%%%%%%%%%%%%%%%


\begin{document}
\begin{titlepage}
    \university{TU/e}
    \courseid{2MBA70}
    \title{Measure and Probability Theory}
    \author{Pim van der Hoorn and Oliver Tse}
    \version{Version 0.2 August 26 2024}
    \maketitle
\end{titlepage}

\begin{chapquote}[30pt]{Emile Borel}
The occurrence of any event where the chances are beyond one in ten followed by 50 zeros is an event that we can state with certainty will never happen, no matter how much time is allotted and no matter how many conceivable opportunities could exist for the event to take place.
\end{chapquote}
\vfill
\noindent\textbf{Disclaimer:}
%\vspace{0.5cm} 
These are lecture notes for the course \emph{Measure and Probability Theory}. They are by no means a replacement for the lectures, instructions, and/or the books. Nor are they intended to cover every aspect of the field of measure theory or probability theory.

Since these are lecture notes, they also include problems. Each chapter ends with a set of exercises that are designed to help you understand the contents of the chapter better and master the tools and concepts.

These notes are still in progress and they almost surely contain small typos. If you see any or if you think that the presentation of some concepts is not yet crystal clear and might enjoy some polishing feel free to drop a line. The most efficient way is to send an email to us, \href{mailto:w.l.f.v.d.hoorn@tue.nl}{w.l.f.v.d.hoorn@tue.nl} or \href{mailto:o.t.c.tse@tue.nl}{o.t.c.tse@tue.nl}. Comments and suggestions are greatly appreciated.

Finally, we hold the copyrights to this document and as such, their contents, either partially or as a whole, cannot be redistributed or used without our explicit consent.

\newpage

\tableofcontents

\chapter{Introduction}
\label{chapter:introduction}
\section{The need to measure}
This course, as the name suggests, is about measuring and probabilities. The field of measure theory is fundamental for many modern versions of mathematical disciplines, such as analysis and probability theory. These lecture notes are designed to introduce the foundations of measure theory and highlight its use in developing the modern theory of probabilities and expectations.  

At the core of measure theory is the notion of \emph{measuring}. We do this every day; whether we are measuring the length of a wall to determine if our bed fits, making sure everyone gets an equal size pizza slice, or counting down the minutes till the end of a boring lecture. 

The term measuring here refers to the act of assigning a (non-negative) number to every object in some collection. If that was the only requirement, this would be a very brief course. However, such an assignment should satisfy some properties if we want it to be useful. For example, if we want to compute the area of some complex shape, we often (without any thought) divide this shape into smaller more regular pieces for which we can compute the area and then combine these to get the area of the entire shape. If the shape is very complex, we might even approximate it by a collection of easier shapes (think of approximating with little squares for example). The actual area is then computed by taking a finer and finer approximation and considering the limit of these computed areas. In both these cases, we are (implicitly) making use of the fact that our notion of measuring respects these operations. The goal of measure theory is to provide clear mathematical definitions for the operations a proper way of measuring should respect and use these to develop new important concepts and theory. The most fundamental of which is integration. 

\section{A new theory of integration}

The ability to properly measure is instrumental for integration. If you think back to your analysis course, integrating a function $f : \bbR \to \bbR$ is basically computing the area under the curve. In other words, measuring the set $A \subset \bbR^2$ given by $A := \{(x,f(x)) \, :\, x \in \bbR\}$. In most cases, this area is actually very complex and we need to approximate it by looking at very small sections and then combine the outcomes to get the full answer.

Being able to integrate is fundamental in a wide variety of mathematical fields. For example, solving Partial Differential Equations and analyzing their solutions requires a powerful theory of integration, as this is the inverse operation to differentiation. Another example is Harmonic Analysis, eg. Fourier Analysis in $\bbR^d$, where functions are studied by transforming them using integrals. But also Functional Analysis, which plays a fundamental part in the foundation of modern quantum mechanics, requires integration to map functions to numbers or other objects. Finally,  the field of probability theory heavily relies on being able to measure sets and integrate them (more on this later). 

But we already know \emph{Riemann} integration, so why do we need another course on this? Unfortunately, Riemann integration has undesirable issues, highlighted in these two examples:
\begin{enumerate}
\item Consider a sequence $(f_n)_{n \ge 1}$ of functions that are each Riemann integrable. Suppose now these functions converge point-wise to a limit function $f$. Then we would like to say something about whether $f$ is Riemann integrable. Even nicer would be if $\lim_{n \to \infty} \int f_n = \int \lim_{n \to \infty} f_n = \int f$, i.e., if we can interchange limits and integration. The issue is that both these things or not generally possible, and the conditions for the interchange of limits and integration are very restrictive.
\item In general, it is even difficult to provide a practical characterization for when any function is Riemann integrable. This means that, in the worst case, you have to prove the convergence of the upper and lower Riemann sums for a function $f$ you want to integrate. 
\end{enumerate}

One of the main outcomes of this course is a new theory of integration called \emph{Lebesgue integration}. The beauty of this theory is that not only does it not suffer from any of the issues outlined above. We can easily characterize if a function is Riemann-integrable within this new theory. More importantly, any point-wise limit of Lebesgue-integrable functions is, under uniform bounds, again Lebesgue integrable and (most of the time) $\lim_{n \to \infty} \int f_n = \int \lim_{n \to \infty} f_n$. Finally, the theory of Lebesgue integration also generalizes Riemann integration. That is, if you know the Riemann integral $\int f$ of a function $f$ exists, its Lebesgue integral will have the same value. So $\int_0^1 x^2 dx$ is still equal to $1/3$, don't worry.

\section{Measure theory as the foundation of probability theory}

Aside from providing us with a new and powerful theory of integration, measure theory is the true foundation of modern probability theory. 

During the first course on probability theory, Probability and Modeling (2MBS10), the concept of probabilities was introduced. The idea here (in its simplest version) is that you have a space $\Omega$ of possible outcomes of an experiment, and you want to assign a value in $[0,1]$ to each set $A$ of potential outcomes. This value would then represent the \emph{probability} that the experiment will yield an outcome in this set $A$, and was denoted by $\prob{A}$. 

It turned out that to define these concepts, we needed to impose structure on both the space of events as well as on the probability measure. For example, if we had two sets $A, B$ of possible outcomes, would like to say something about the probability that the outcome is in either $A$ or $B$. This means that not only do we need to be able to compute $\prob{A \cup B}$, but we want that $A \cup B$ is also an event in our space $\Omega$. Another example concerned the probability of the outcome not being in $A$, which means computing the probability of the event $\Omega\setminus A$, requiring that this set should also be in $\Omega$. 

In the end, this prompted the definition of an \emph{event space} which was a collection $\cF$ of subsets of $\Omega$ satisfying a certain set of properties. In addition, the probability assignment $\bbP$ was defined as a map $\bbP : \cF \to [0,1]$ with some addition properties, such as $\prob{\Omega} = 1$.


With this setup, it was then possible to define what a \emph{random variable} is. Here a random variable $X$ was defined on a triple $(\Omega, \cF, \bbP)$, consisting of a space of outcomes, an event space and a probability on that space. Formally it is a mapping $X : \Omega \to \bbR$ such that for each $x \in \bbR$ the set $X^{-1}(-\infty,x):=\{\omega \in \Omega \, : \, X(\omega) \in (-\infty,x)\}$ is in $\cF$. This then allowed us to define the \emph{cumulative distribution function} as $F_X(x) := \prob{X^{-1}(-\infty,x)}$.

It is important to note here that already it was needed to make a distinction between how to define a discrete and a continuous random variable. In addition, a separate definition was required to define multivariate distribution functions. All of this limits the extent to which this theory can be applied. For example, let $U$ be distributed uniformly on $[0,1]$ and $Y$ be distributed uniformly on the set $\{1,2, \dots, 10\}$ and define the random variable $X$ to be equal to $U$ with probability $1/3$ and equal to $Y$ with probability $2/3$. How would you deal with this random variable, which is both discrete and continuous? 

The setting would become even more complex and fuzzy if we were not talking about random numbers in $\bbR$ but, say, random vectors of infinite length or random functions. Do these even exist? Many other things remain fuzzy or simply impossible in a theory of probability without measure theory. What are conditional probabilities/expectations? How do you define a continuous time Markov Process or a point process? What is a stochastic process? Or, does there exist such a thing as a random probability measure?

The solutions to all these issues come from a generalization of event spaces and probability measures introduced above. These go by the names \emph{sigma-algebra} and \emph{measure}, respectively, which are the core concepts in measure theory. With this, we can then define when any mapping between spaces is \emph{measurable} and use such mappings to define random objects in the space such a function maps to. Finally, armed with the theory of Lebesgue integration, measure theory provides the foundation to define expectations, convergence of random variables, and, most importantly, the notion of conditional probability/expectation. 

All of this is to say that a proper study of Probability Theory cannot happen without Measure Theory. By the end of these notes, we hope you will appreciate this and be inspired by the versatility and beauty of measures theory and Lebesgue integration.





\chapter{Measure spaces ($\sigma$-algebras and measures)}
\label{chapter:sigma_algebras}

%\section{Recalling basic probability theory}\label{sec:recalling_probability_theory}
%
%During the first course on probability theory, Probability and Modeling (2MBS10), the concept of probabilities were introduced. The idea here (in its simplest version) is that you have a space $\Omega$ of possible outcomes of an experiment, and you want to assign a value in $[0,1]$ to each set $A$ of potential outcomes that represent the \emph{probability} that the experiment will yield an outcome in this set $A$. This value was then denoted by $\prob{A}$. 
%
%It turned out that in order to properly define these concepts, we needed to impose structure on both the space of events as well as on the probability measure. For example, if we had two sets $A, B$ of possible outcomes, would like to say something about the probability that the outcome is in either $A$ or $B$. This means we not only do we need to be able compute $\prob{A \cup B}$, we actually want that $A \cup B$ is also an event in our space $\Omega$. Another example concerned the probability of the outcome not being in $A$, which means compute the probability of the event $\Omega\setminus A$, requiring that this set should also be in $\Omega$. In the end this prompted the definition of an \emph{event space} which was a collection $\cF$ of subsets of $\Omega$ such that
%\begin{enumerate}
%\item $\cF$ is non-empty;
%\item If $A \in \cF$, then $A^c := \Omega \setminus A \in \cF$;
%\item If $A_1, A_2, \dots \in \cF$, then $\bigcup_{i = 1}^\infty A_i \in \cF$.
%\end{enumerate}
%
%In addition, the probability assignment $\bbP$ was defined as a map $\bbP : \cF \to [0,1]$ such that
%\begin{enumerate}
%\item $\prob{\Omega} = 1$ and $\prob{\emptyset} = 0$, and
%\item for any collection $A_1, A_2, \dots$ of disjoint events in $\cF$ it holds that 
%\[
%	\prob{\bigcup_{i = 1}^\infty A_i} = \sum_{i = 1}^\infty \prob{A_i}.
%\]
%\end{enumerate}
%
%With this setup it was possible to formally define what a \emph{random variable} is. Here a random variable $X$ was defined on a triple $(\Omega, \cF, \bbP)$, consisting of a space of outcomes, an event space and a probability on that space. Formally it is a mapping $X : \Omega \to \bbR$ such that for each $x \in \bbR$ the set $X^{-1}(-\infty,x):=\{\omega \in \Omega \, : \, X(\omega) \in (-\infty,x)\}$ is in $\cF$. This then allowed us to define the \emph{cumulative distribution function} as $F_X(x) := \prob{X^{-1}(-\infty,x)}$.
%
%It is important to note here that already it was needed to make a distinction of how to define a discrete and a continuous random variable. In addition, a separate definition was required to defined multivariate distribution functions. This limits the extend to which this theory can be applied. For example let $U$ have the uniform distribution on $[0,1]$ and $Y$ have uniform distribution on the set $\{1,2, \dots, 10\}$ and define the random variable $X$ to be equal to $U$ with probability $1/2$ and equal to $Y$ with probability $1/2$. How would you deal with this random variable, which is both discrete and continuous? However, the setting would becomes even more complex if we are not talking about random numbers in $\bbR$ but, say, random vectors of infinite length or random functions. Do these even exist?
%
%The solutions to all these issues comes from a generalization of event spaces and probability measures introduced above. These go by the names sigma-algebra and measure, respectively. With this we can then define when any mapping between spaces is \emph{measurable} and use such mappings to define random objects in that space such a function maps to. The remainder of this chapter is dedicated to introduced all these concepts.

At the core of measure theory are two things: 1) the objects we want to measure and 2), a way to assign a measure (value) to these objects. The objects are subsets of some given space that satisfy certain properties, which we call \emph{\sigalgs/} (sigma-algebras). The structure of these \sigalgs/ allows us to define the notion of a measure on them, which is a map that assigns to each set a value in $[0,\infty)$. Of course, we will not consider any such map but impose a few properties which will imply many interesting properties of measures and allow us later on to define a general notion on integration. This chapter is concerned with the two basic notions, \sigalgs/ and measures. We will provide the definitions, important properties and some key examples that will be fundamental for the remainder of this course.


\section{Sigma-algebras}


% Definition sigma algebra

\subsection{Definition and examples}

We begin this section with introducing the general structure needed on a collection of sets to be able to assign a notion of measurement to them. Such a collection is called a sigma-algebra, often written as $\sigma$-algebra.

\begin{definition}[Sigma Algebra]\label{def:sigma_algebra}
A \emph{$\sigma$-algebra} $\cF$ on a set $\Omega$ is a collection of subsets of $\Omega$ with the following properties:
\begin{enumerate}
\item $\emptyset \in \cF$ and $\Omega \in \cF$;
\item For every $A \in \cF$, it holds that $A^c := \Omega \setminus A \in \cF$;
\item For every sequence $A_1, A_2, \dots \in \cF$, it holds that $\bigcup_{i = 1}^\infty A_i \in \cF$.
\end{enumerate}
A set $A \in \cF$ is called \emph{$\cF$-measurable}, or simply \emph{measurable} if it is clear which \sigalg/ is meant.
\end{definition}

This definition might look very familiar. In the course Probability and Modeling you have been introduced to the concept of an \emph{event space}. It turns out that these concepts are actually the same, see Problem~\ref{prb:event_space_are_sigma_algebras}.

Before we give some examples, we first provide a result that states that any \sigalg/ is closed under countable intersections. The proof is left as an exercise to the reader.

\begin{lemma}\label{lem:sigma_algebra_closed_intersections}
Let $\cF$ be a \sigalg/ on $\Omega$ and let $A_1, A_2, \dots \in \cF$. Then it holds that $\bigcap_{i = 1}^\infty A_i \in \cF$
\end{lemma}

\begin{proof}
See Problem~\ref{prb:sigma_algebra_closed_intersections}.
\end{proof}

We now give some examples and non-examples of \sigalgs/.

\begin{example}[(Non-)Examples of $\sigma$-algebras]\label{example:sigma_algebras}
\hfil
\begin{enumerate}
\item The collection $\cF = \{\emptyset, \Omega\}$ is a $\sigma$-algebra. This is called the \emph{trivial $\sigma$-algebra} or the \emph{minimal} $\sigma$-algebra on $\Omega$.
\item For any subset $A \subset \Omega$ we have that $\cF := \{\emptyset, A, \Omega\setminus A, \Omega\}$ is a $\sigma$-algebra.
\item The \emph{power set} $\cP(\Omega)$ (the collection of all possible subsets of $\Omega$) is a $\sigma$-algebra. This is sometimes called the \emph{maximal $\sigma$-algebra} on $\Omega$.
\item For any subset $A \subset \Omega$ such that $A \ne \emptyset, \Omega$, we have that $\cF := \{\emptyset, A, \Omega\}$ is \textbf{not} a $\sigma$-algebra.
\item Let $\Omega = [0,1]$ and $\cF$ be the collections of finite unions of intervals of the form $[a,b]$, $[a,b)$, $(a,b]$ and $(a,b)$ for $0 \le a < b \le 1$. Then $\cF$ is \textbf{not} a $\sigma$-algebra.
\item Let $f : \Omega \to \Omega^\prime$ and let $cF^\prime$ be a \sigalg/ on $\Omega^\prime$. Then the collection
\[
	\cF := f^{-1}(\cF^\prime) = \{f^{-1}(A^\prime) \, : \, A^\prime \in \cF^\prime\},
\]
is a \sigalg/. The converse to this is not always true, see Problem~\ref{prb:converse_preimage_sigma_algebra}.
\end{enumerate}
\end{example}

Proving these claims is left as an exercise, see Problem~\ref{prb:example_sigalgs}.

The idea of measure theory is that one can assign a notion of measure to each set in a $\sigma$-algebra. In line with this, a pair $(\Omega, \cF)$ where $\Omega$ is a set and $\cF$ a \sigalg/ on $\Omega$ is called a \emph{measurable space}. 

\subsection{Constructing \sigalgs/}\label{ssec:construction_sigalgs}

We now know what a \sigalg/ is and have seen some example and some non-examples. But the examples we have seen are still quite uninspiring. We will actually discuss a very important \sigalg/ in the next section. But for now, we will describe several ways to construct \sigalgs/. The first is restricting an existing \sigalg/ to a given set.

\begin{lemma}[Restriction of a \sigalg/]\label{lem:restriction_sigma_algebra}
Let $(\Omega, \cF)$ be a measurable space and $A \subset \Omega$. Then the collection defined by 
\[
	\cF_A := \{A \cap B \, : \, B \in \cF\},
\]
is a \sigalg/ on $A$, called the \emph{restriction of $\cF$ to $A$}.
\end{lemma}

\begin{proof}
We need to check all three properties.
\begin{enumerate}
\item Since $A \cap \Omega = A$ and $A \cap \emptyset = \emptyset$, it follows that $A, \emptyset \in \cF_A$.
\item Consider a set $C \in \cF_A$. Then by definition $C = A \cap B$ for some $B \in \cF$. Next, we note
\[
	A \setminus C = A \setminus (A \cap B) = A \cap (\Omega \setminus B).
\]
Since $\cF$ is a \sigalg/, it follows that $\Omega \setminus B \in \cF$ and so $A \setminus C \in \cF_A$.
\item Let $C_1, C_2, \dots$ be sets in $\cF_A$. Then there are $B_1, B_2, \dots \in \cF$ such that $C_i = A \cap B_i$. Hence
\[
	\bigcup_{i=1}^\infty C_i = \bigcup_{i=1}^\infty A \cap B_i = A \cap \bigcup_{i=1}^\infty B_i \in \cF_A,
\]
since $\bigcup_{i=1}^\infty B_i \in \cF$ because this is a \sigalg/.
\end{enumerate}
\end{proof}

While it is nice to be able to take a given \sigalg/ and create a possibly smaller one by restricting it to a given set, we might also want to start with a given collection of sets $\cA$ and then create a \sigalg/ that contains this collection. Of course, the powerset $\cP(\Omega)$ will always work. However, it is not always desirable to take this maximal \sigalg/. It would be much better if we could create the smallest \sigalg/ that contains $\cA$. It turns out that this can be done and the resulting \sigalg/ is said to be \emph{generated by $\cA$}. 

\begin{proposition}[Generated \sigalg/]\label{prop:generated_sigalg}
Let $\cA$ be a collection of subsets of $\Omega$ and denote by $\Sigma_\cA$ the collection of all \sigalgs/ on $\Omega$ that contain $\cA$. Then the collection defined by
\[
	\sigma(\cA) := \bigcap_{\cF \in \Sigma_\cA} \cF,
\]
is a \sigalg/. It is called the \emph{\sigalg/ generated by $\cA$}. Equivalently, $\cA$ is called the \emph{generator of $\sigma(\cA)$}.

Moreover, $\sigma(\cA)$ is the smallest \sigalg/ that contains $\cA$. 
\end{proposition}

\begin{proof}
If we can show that $\cF$ is a \sigalg/, then the claim about it being the smallest \sigalg/ that contains $\cA$ follows from its definition. So we will prove that $\cF$ is a \sigalg/.

Similar to Lemma~\ref{lem:restriction_sigma_algebra} we need to check all the requirements.
\begin{enumerate}
\item Since $\emptyset, \Omega \in \cF$ holds for every $\cF \in \Sigma_\cA$ it follows that $\emptyset, \Omega \in \sigma(\cA)$. In particular, we note that $\sigma(\cA)$ is not empty.
\item Take $A \in \sigma(\cA)$. Then $A \in \cF$ for each $\cF \in \Sigma_\cA$. Since $\cF$ is a \sigalg/ it holds that $\Omega \setminus A \in \cF$ for each $\cF \in \Sigma_\cA$. This then implies that $\Omega \setminus A \in \sigma(\cA)$.
\item Let $(A_i)_{i \in \bbN}$ be a sequence of sets in $\sigma(\cA)$. Then by definition $A_i \in \cF$ for each $\cF \in \Sigma_\cA$. Hence
\[
	\bigcup_{i \in \bbN} A_i \in \cF,
\] 
holds for each $\cF \in \Sigma_\cA$ and thus it follows that $\bigcup_{i \in \bbN} A_i \in \sigma(\cA)$.
\end{enumerate}
\end{proof}

If $\cF$ is a \sigalg/ on $\Omega$ and $\cA$ is a collection of subsets such that $\cF = \sigma(\cA)$, we call $\cA$ the \emph{generator of $\cF$}. 

The nice thing about this construction of \sigalgs/ is that it respects inclusions.

\begin{lemma}[Inclusion property of \sigalgs/]\label{lem:inclusion_sigalgs}
If $A \subset B \subset C$ are subset of $\Omega$, then also $\sigma(A) \subset \sigma(B) \subset \sigma(C)$.
\end{lemma}

\begin{proof}
See Problem~\ref{prb:inclusion_sigalgs}
\end{proof}

On example of a generated \sigalg/ is to construct products of measurable spaces.

\begin{definition}[Product \sigalg/]\label{def:product_sigalg}
Let $(\Omega, \cF)$ and $(\Omega^\prime, \cF^\prime)$ be two measurable spaces. Then we define $\cF \otimes \cF^\prime$ to be the \sigalg/ on $\Omega \times \Omega^\prime$ generated by sets of the form $A \times B$, with $A \in \cF$ and $B \in \cF^\prime$.
\end{definition}

However, there is a much more important \sigalg/ that is constructed from a generator set.

\subsection{Borel \sigalg/}

The Euclidean space $\bbR^d$ is omnipresent in mathematics and hence pops up in many bachelor courses as well. In particular, the concept of random variables, as given in the course Probability and Modeling, is mainly concerned with objects that have values in $\bbR$. Based on this, the need to impose a measurable structure on this space, by means of a \sigalg/, should not come as a surprise. It turns out that there is a canonical \sigalg/ which is called the \emph{Borel \sigalg/} and is named after the French mathematician \'{E}mile Borel, who was one of the pioneers of measure theory.

In order to define the Borel \sigalg/ we need the notion of an open set in $\bbR^d$. For any $x \in \bbR$ and $r >0$, we denote by $B_x(r) := \{y \in \bbR^d \, : \, \|x-y\|<r\}$ the open ball of radius $r$ around $x$. A set $U \subset \bbR^d$ is called \emph{open} if and only if for every $x \in U$, there exists an $r > 0$ such that $B_x(r) \subset U$.

\begin{definition}[Borel \sigalg/]
The \emph{Borel \sigalg/} on $\bbR^d$, denoted by $\cB_{\bbR^d}$, is the \sigalg/ generated by all open sets in $\bbR^d$. Elements of $\cB_{\bbR^d}$ are called \emph{Borel sets}.
\end{definition}

\begin{remark}
From the definition, it should be clear that one can actually define a \emph{Borel \sigalg/} on any metric space. Actually, we can define it on any topological space. However, this requires the notion of a topology which is beyond the scope of this course. [ADD REFERENCES]
\end{remark}

While this is a perfectly fine definition, it is often cumbersome to work with. It is therefore convenient that $\cB_{\bbR^d}$ is generated by other, more compact, collections of sets. At this point we state the result for the one-dimensional Borel \sigalg/. 

\begin{proposition}\label{prop:characterization_borel_sigalg_1d}
The Borel \sigalg/ on $\bbR$ is the \sigalg/ generated by any of the following family of sets, 
\begin{enumerate}
\item $\{(a_1, b_1)\}$,
\item $\{(a, b]\}$,
\item $\{[a, b)\}$,
\item $\{(-\infty,a]\}$,
\item $\{(-\infty,a)\}$,
\item $\{[a, \infty)\}$,
\item $\{(a,\infty)\}$,
\end{enumerate}
where $a, b \in \bbQ,$ or $a, b \in \bbR$
\end{proposition}

\begin{proof}
See Problem~\ref{prb:borel_sigalg_1d}.
\end{proof}

\section{Measures}

\subsection{Definition and examples}
In the previous section we have seen how we can define and construct collections of sets that we would like to be able to measure. It turned out that this collection should satisfy some properties. Likewise, when defining the notion of a \emph{measure} we also will require it to have certain properties.

The main property we require is called \emph{$\sigma$-additive}. Consider any collection $\cC$ of subsets of some set $\Omega$. Then a set function $\mu : \cC \to [0,\infty]$ is called \emph{$\sigma$-additive} if for any countable family $(A_i)_{i \in \bbN}$ of pairwise disjoint sets in $\cC$
\[
	\mu\left(\bigcup_{i \in \bbN} A_i\right) = \sum_{i = 1}^\infty \mu(A_i).
\]

\begin{definition}[Measure]\label{def:measure}
Let $(\Omega, \cF)$ be a measurable space. A set function $\mu : \cF \to [0,\infty]$ is called a \emph{measure on $(\Omega, \cF)$} if the following holds:
\begin{enumerate}
\item $\mu(\emptyset) = 0$ and,
\item $\mu$ is $\sigma$-additive.
\end{enumerate}
\end{definition}

A triple $(\Omega, \cF, \mu)$, consisting of a measurable space $(\Omega, \cF)$ and a measure $\mu$ on that space is called a \emph{measure space}. If the $\mu(\Omega) < \infty$ we say that $\mu$ is $\sigma$-finite and call the associated measure space a \emph{$\sigma$-finite measure space}. If $\mu(\Omega) = 1$ we call $\mu$ a \emph{probability measure} and the associated measure space a \emph{probability space}.

Let us give some simple examples of measures.

\begin{example}[Examples of measures]
\hfil
\begin{enumerate}
\item \textit{(Trivial measures)} Let $(\Omega, \cF)$ be a measurable space. Then the following two set functions are measures:
\[
	\mu(A) = \begin{cases}
		0 &\text{if } A = \emptyset, \\
		\infty &\text{otherwise.}
	\end{cases}
	\quad \text{and} \quad
	\mu(A) = 0 \quad \text{for all } A \in \cF.
\]
\item \textit{(Dirac measure)} Let $(\Omega, \cF)$ be a measurable space and $x \in \Omega$. Then the function
\[
	\delta_x(A) = \begin{cases}
		1 &\text{if } x \in A, \\
		0 &\text{otherwise},
	\end{cases}
\]
is a measure called the \emph{Dirac delta measure} or \emph{unit mass} at $x$.
\item \textit{(Counting measure)} Let $(\Omega, \cF)$ be a measurable space. Then the function defined as
\[
	\mu(A) = \begin{cases}
		|A| &\text{if $A$ is a finite set},\\
		\infty &\text{otherwise}, 
	\end{cases}
\]
is a measure called the \emph{counting measure}.
\item \textit{(Discrete measure)} Let $\Omega = \{\omega_1, \omega_2, \dots\}$ be a countable set and consider the measurable space $(\Omega, \cP(\Omega))$. Take any sequence of $(a_i)_{i \in \bbN}$ such that $\sum_{i = 1}^\infty a_i < \infty$. Then the function
\[
	\mu(A) = \sum_{j = 1}^\infty a_j \delta_{\omega_j}(A),
\]
is a measure called the \emph{discrete measure}. If the $a_i$ are such that $\sum_{i = 1}^\infty a_i = 1$ we call this the \emph{discrete probability measure}.
\end{enumerate}
\end{example}

However, there is a measure, not included above, that plays a fundamental role in measure theory and especially probability theory.

\subsection{Null sets, complete measure spaces and the Lebesgue measure}

It should be noted that, outside maybe the discrete measure, the examples listed above do not include any interesting measure. More specifically, consider the Borel space $(\bbR^d, \cB_{\bbR^d})$. Then how can we construct a non-trivial measure on this space? The problem is that the Borel \sigalg/ is only defined in terms of its generator. Hence if we want to define what $\mu(A)$ is for any $A \in \cB_{\bbR^d}$ we first have to get a better handle on the full \sigalg/. That might seem daunting, and it really is. The problem becomes even more challenging when we want the measure on $(\bbR^d, \cB_{\bbR^d})$ to have additional properties. For example, that the measure of any rectangle is simply its volume, which seems like a very natural property to ask for. 

Still, it turns out that such a measure exists. This fundamental measure is called the \emph{Lebesgue measure}, named after the French mathematician Henri Lebesgue who was architect of the modern notion of integration we will cover in this course. Moreover, in addition to the measure of any rectangle being equal to its volume, the Lebesgue measure has several other strong features. 

However, to formally state the theorem that we need to introduce the concept of \emph{null sets} and complete measure spaces.

\begin{definition}[Null set]
Let $(\Omega, \cF, \mu)$ be a measure space. A set $N \subset \Omega$ is called a \emph{null set} if there exists a $A \in \cF$ such that $N \subset A$ and $\mu(A) = 0$.
\end{definition}

It is important to note that a null set does not have to be measurable, i.e. be in $\cF$. We call a measure space $(\Omega, \cF, \mu)$ \emph{complete} if every null set $N \in \cF$. 

Let $(\Omega, \cF, \mu)$ be a measure space, not necessarily complete. Then we can construct a new measure space $(\Omega, \bar{\cF}, \bar{\mu})$ that is complete and such that the measure $\bar{\mu}$ is equal to $\mu$ on $\cF$, i.e. $\bar{\mu}|_\cF = \mu$. We refer to this construction an \emph{completing} the measure space $(\Omega, \cF, \mu)$. The details of this construction are not important. It basically entails adding all null sets to the \sigalg/. For more details see Problem~\ref{prb:completion_measure_space}.

We can now state the main result that proves the existence of the Lebesgue measure and states its important properties. 

\begin{theorem}[Lebesgue measure]\label{thm:lebesgue_measure}
There exists a \sigalg/ $\cL^d \supset \cB_{\bbR^d}$ on $\bbR^d$ and a unique measure $\lambda$ such that $(\bbR^d, \cL^d, \lambda)$ is complete and satisfies the the following properties, for any $B \in \cB_{\bbR^d}$:
\begin{enumerate}
\item For any half-open rectangle $I := [a_1, b_1) \times \dots \times [a_d, b_d)$ it holds that $\lambda(I) = \prod_{i = 1}^d (b_i-a_i)$;
\item For any $x \in \bbR^d$, $\lambda(B+x) = \lambda(B)$, where $B + x = \{y+x \, : \, y \in B\}$;
\item For any combination of translation, rotation and reflection $R$, $\lambda(R^{-1}(B)) = \lambda(B)$;
\item For any invertible matrix $M \in \bbR^{d \times d}$, $\lambda(M^{-1}(B)) = |\mathrm{det} M|^{-1} \lambda(B)$.
\end{enumerate}
\end{theorem}

The proof of this theorem is involved and relies on a more abstract approach to constructing measures. The interested student is referred to the Appendix, where we provide the full details.  

It follows from Theorem~\ref{thm:lebesgue_measure} that the Lebesgue measure formally defined on a larger \sigalg/ $\cL$ then the Borel \sigalg/. This \sigalg/ is called the \emph{Lebesgue \sigalg/} and functions that are $\cL^d$-measurable are called \emph{Lebesgue measurable}. The Lebesgue measure on $\cB_{\bbR^d}$ is now defined as the restriction of $\lambda$ to the Borel \sigalg/. 

\begin{remark}[Lebesgue vs Borel measurable]
It should be noted that $\cB_{\bbR^d} \subsetneq \cL^d$. That is, there are sets that are Lebesgue measurable but not Borel measurable.
\end{remark}


We end this section by looking at some important general properties of measures.


\subsection{Important properties}

Although the number of properties a measure needs to satisfy are very limited, they actually imply a great number of other important properties. We will start with the basic ones, which relate the measure of a set that is obtained from a given set operation on two sets $A, B$ to the measure of these sets.

\begin{proposition}[Basic properties of measures]\label{prop:basic_properties_measures}
Let $(\Omega, \cF, \mu)$ be a measure space and let $A, B \in \cF$. Then the following properties hold for $\mu$.
\begin{enumerate}
\item (finitely additive) If $A \cap B = \emptyset$, then $\mu(A \cup B) = \mu(A) + \mu(B)$.
\item (monotone) If $A \subseteq B$, then $\mu(A) \le \mu(B)$.
\item (exclusion) If in addition $\mu(A) < \infty$, then $\mu(B \setminus A) = \mu(B) - \mu(A)$.
\item (strongly additive) $\mu(A \cup B) + \mu(A \cap B) = \mu(A) + \mu(B)$.
\item (subadditive) $\mu(A \cup B) \le \mu(A) + \mu(B)$.
\end{enumerate}
\end{proposition}

\begin{proof}
\hfil
\begin{enumerate}
\item Let $A_1 = A$, $A_2 = B$ and $A_i = \emptyset$ for all $i \ge 3$. Then this property follows directly from the fact that $\mu$ is $\sigma$-additive.
\item Since $A \subseteq B$ we have that $B = A \cup (B \setminus A)$, with $A$ and $B \setminus A$ disjoint sets. It then follows from property 1 that $\mu(B) = \mu(A) + \mu(B \setminus A)$ and thus $\mu(A) \le \mu(B)$.
\item Since $\mu(A) < \infty$ we can subtract $\mu(A)$ from both sides of the equation $\mu(B) = \mu(A) + \mu(B \setminus A)$ to obtain the desired result.
\item First note that if $\mu(A \cap B) = \infty$ then by property 2 we have that also $\mu(A), \mu(B)$ and $\mu(A \cup B) = \infty$ and hence the result holds trivially. So assume now that $\mu(A \cap B) < \infty$. Since 
\[
	A \cup B = (A \setminus (A\cap B)) \cup (B \setminus (A \cap B)) \cup (A \cap B),
\] 
it follows from property 1 that
\[
	\mu(A \cup B) = \mu(A \setminus (A \cap B))) + \mu(A \cap B) + \mu(B \setminus (A \cap B)).
\] 
Adding $\mu(A \cap B) < \infty$ to both side we get
\begin{align*}
	\mu(A \cup B) + \mu(A \cap B) 
	&= \mu(A \setminus (A \cap B))) + \mu(A \cap B) + \mu(B \setminus (A \cap B)) + \mu(A \cap B)\\
	&= \mu(A) + \mu(B), 
\end{align*}
where the last line follows from applying property 3 twice.
\item Property 4 implies that $\mu(A) + \mu(B) = \mu(A \cup B) + \mu(A \cap B) \ge \mu(A \cup B)$.
\end{enumerate}
\end{proof}

The subadditive property can actually be extended to any countable family of sets.

\begin{lemma}[Measures are $\sigma$-subadditive]\label{lem:sigma_subadditive}
Let $(\Omega, \cF, \mu)$ be a measure space and let $(A_i)_{i \in \bbN}$ be a family of sets in $\cF$. Then
\[
	\mu\left(\bigcup_{i \in \bbN} A_i\right) \le \sum_{i = 1}^\infty \mu(A_i),
\]
and the measure $\mu$ is said to be \emph{$\sigma$-subadditive}.
\end{lemma}

\begin{proof}
See Problem~\ref{prb:sigma_subadditive}
\end{proof}

In addition to properties relating a measure $\mu$ to set operations, we also want to understand what happens if we take a limit of the measures of an infinite family of sets. Let $(A_i)_{i \in \bbN}$ be a family of measurable sets. We say this family is \emph{increasing} if $A_i \subset A_{i+1}$ holds for all $i \in \bbN$. Because a measure is monotone it follows that the sequence $(\mu(A_i))_{i \in \bbN}$ is a monotone sequence in $[0,\infty]$. So a natural question would be: what is the limit of this sequence? It turns out that this can be expressed as the measure of the union of all sets.

\begin{proposition}[Continuity from below]\label{prop:continuity_measure_below}
Let $(\Omega, \cF, \mu)$ be a measure space and let $(A_i)_{i \in \bbN}$ be an increasing family of measurable sets. Then
\[
	\lim_{i \to \infty} \mu(A_i) = \mu\left(\bigcup_{i \in \bbN} A_i\right).
\]
\end{proposition}

\begin{proof}
Define the sets $E_1 = A_1$ and $E_i = A_{i+1}\setminus A_i$, for all $i \ge 2$. Then $(E_i)_{i \in \bbN}$ is a family of mutually disjoint measurable sets with the following properties:
\[
	A = \bigcup_{i = 1}^\infty E_i \quad \text{and} \quad A_k = \bigcup_{i = 1}^k E_i.
\]
Therefore, using $\sigma$-additivity we get
\[
	\mu(A) = \sum_{i = 1}^\infty \mu(E_i) = \lim_{k \to \infty} \sum_{i = 1}^k \mu(E_k)
	= \lim_{k \to \infty} \mu(\bigcup_{i = 1}^k E_i) = \lim_{k \to \infty} \mu(A_k).
\]
\end{proof}

A similar property holds for any \emph{decreasing} family of sets. That is, a family $(A_i)_{i \in \bbN}$ of measurable sets such that $A_i \supset A_{i+1}$ holds for all $i \in \bbN$. Here we do have to make an assumption on the measure of the biggest set $A_1$.

\begin{proposition}[Continuity from above]\label{prop:continuity_measure_above}
Let $(\Omega, \cF, \mu)$ be a measure space and let $(A_i)_{i \in \bbN}$ be an decreasing family of measurable sets such that $\mu(A_1) < \infty$. Then
\[
	\lim_{i \to \infty} \mu(A_i) = \mu\left(\bigcap_{i \in \bbN} A_i\right).
\]
\end{proposition}

\begin{proof}
See Problem~\ref{prb:proof_continuity_above}.
\end{proof}

In addition to being useful in determining the limits of the measure of families of sets, these continuity properties are actually powerful enough to characterize a measure. 

\begin{theorem}[Alternative definition of a measure]
Let $(\Omega, \cF)$ be a measurable space. A set function $\mu : \cF \to [0,\infty]$ is a measure if, and only if,
\begin{enumerate}
\item $\mu(\emptyset) = 0$,
\item $\mu(A \cup B) = \mu(A) + \mu(B)$, for any two disjoint sets $A, B \in \cF$, and
\item for any increasing family $(A_i)_{i \in \bbN}$ of measurable sets such that $A_\infty := \bigcup_{i \in \bbN} A_i \in \cF$, it holds that
\[
	\mu(A_\infty) = \lim_{i \to \infty} \mu(A_i) \quad (= \sup_{i \in \bbN} \mu(A_i)).
\]
\end{enumerate}
\end{theorem}

\begin{proof}
The fact that any measure satisfies these three properties follows from the definition and Propositions~\ref{prop:basic_properties_measures} and~\ref{prop:continuity_measure_below}. So let us now assume that we have a set function $\mu$ that satisfies the three properties listed above. Then to show that $\mu$ is a measure we only have to prove that it is $\sigma$-additive.

To this end, let $(A_i)_{i \in \bbN}$ be a family of pairwise disjoint measurable sets. Now define $B_k = \bigcup_{i = 1}^k A_i$ and note that $B_k \in \cF$ for all $k \in \bbN$ and
\[
	B_\infty := \bigcup_{k \in \bbN} B_k = \bigcup_{i \in \bbN} A_i.
\]
Using property 2. we get that $\mu(B_k) = \sum_{i = 1}^k \mu(A_i)$ while property 3. now implies that
\[
	\mu(B_\infty) = \lim_{k \to \infty} \mu(B_k) = \lim_{k \to \infty} \sum_{i = 1}^k \mu(A_i) = \sum_{i = 1}^\infty \mu(A_i). 
\]
\end{proof}

Finally, let use discuss a uniqueness result for measures. In Section~\ref{ssec:construction_sigalgs} we discussed how to construct \sigalgs/ from a generator set $\cA$. Suppose now that we have two measures $\mu_1$ and $\mu_2$ agree on $\cA$, that is $\mu_1(A) = \mu_2(A)$ for all $A \in \cA$. Then we would intuitively expect that they should agree on the entire \sigalg/ $\sigma(\cA)$. This turns out to be true, under some small conditions on the generator set.

\begin{theorem}[Uniqueness of measures]\label{thm:uniqueness_measures}
Let $(\Omega ,\cF)$ be a measurable space where $\cF = \sigma(\cA)$ with $\cA$ satisfying the following properties:
\begin{enumerate}
\item for any $A, B \in \cA$, $A \cap B \in \cA$, and
\item there exists a sequence $(A_i)_{i \in \bbN}$ with $\Omega = \bigcup_{i \in \bbN} A_i$.
\end{enumerate}
Then any two measure $\mu_1$ and $\mu_2$ that are equal on $\cA$ and are finite on every element of the sequence $(A_i)_{i \in \bbN}$ are equal on the entire \sigalg/ $\cF = \sigma(\cA)$.
\end{theorem}

The proof of this theorem is covered in the Appendix, as it is requires another more technical result. What is important is the implication of Theorem~\ref{thm:uniqueness_measures}: to study a measure on $\sigma(\cA)$ it suffices to look at what it does on the generator $\cA$.



\section{Problems}

\begin{problem}\label{prb:event_space_are_sigma_algebras}
Recal that an \emph{event space} is a collection $\cF$ of subsets of $\Omega$ such that
\begin{enumerate}
\item $\cF$ is non-empty;
\item If $A \in \cF$, then $A^c := \Omega \setminus A \in \cF$;
\item If $A_1, A_2, \dots \in \cF$, then $\bigcup_{i = 1}^\infty A_i \in \cF$.
\end{enumerate}
Show that the definition of an \emph{event space} is equivalent to that of a $\sigma$-algebra as given in Definition~\ref{def:sigma_algebra}.
\end{problem}

\begin{problem}\label{prb:sigma_algebra_closed_intersections}
Prove Lemma~\ref{lem:sigma_algebra_closed_intersections}. [Hint: how are intersections related to the other operations used in the definition of a \sigalg/?]
\end{problem}

\begin{problem}\label{prb:example_sigalgs}
Prove the claims made in Example~\ref{example:sigma_algebras}.
\end{problem}

\begin{problem}\label{prb:converse_preimage_sigma_algebra}
Provide a counter example to the statement: if $(\Omega, \cF)$ is a measurable space and $f : \Omega \to \Omega^\prime$. Then $f(\cF)$ is a \sigalg/ on $\Omega^\prime$.
\end{problem}

\begin{problem}\label{prb:inclusion_sigalgs}
Prove Lemma~\ref{lem:inclusion_sigalgs}.
\end{problem}

\begin{problem}\label{prb:sigma_subadditive}
The goal of this exercise is to prove Lemma~\ref{lem:sigma_subadditive}. We will do this in two steps. 
\begin{enumerate}
\item Suppose $(B_i)_{i \in \bbN}$ is an increasing sequence, i.e. $B_i \subset B_{i +1}$. Show that $\mu(\bigcup_{i \in \bbN} B_i) \le \sum_{i = 1}^\infty \mu(B_i)$. [Hint: Construct a family of disjoint sets and use the properties finitely additive and monotone.]
\item Prove Lemma~\ref{lem:sigma_subadditive}. [Hint: can you construct a helpful increasing sequence?]
\end{enumerate}  
\end{problem}

\begin{problem}\label{prb:proof_continuity_above}
Prove Proposition~\ref{prop:continuity_measure_above}. [Hint: The proof is very similar to that of Proposition~\ref{prop:continuity_measure_below}.]
\end{problem}

\begin{problem}[Homework]\label{prb:borel_sigalg_1d}
The goal of this problem is to prove Proposition~\ref{prop:characterization_borel_sigalg_1d}. We will do this in several stages. First we will show point 1, that the Borel \sigalg/ $\cB_\bbR$ is generated by eitherh the family $\cA_1 := \{(a, b), a,b \in \bbQ\}$ or $\cA_1^\prime := \{(a, b), a,b \in \bbR\}$.

\begin{enumerate}
\item Prove that $\sigma(\cA_1) \subset \sigma(\cA_1^\prime) \subset \cB_\bbR$. [Hint: what is the relation between an interval $(a,b)$ and an open set?]
\item We will now focus on the intervals with rational endpoints. Show that for any open set $O \subset \bbR$
\[
	O = \bigcup_{I \in \cA_1, I \subset O} I
\]
\item Prove that $\sigma(\cA_1) = \cB_\bbR$. [Hint: You only need one inclusion, for which you can use 2 and the fact that $\bbQ$ is countable.]
\item Prove that $\cB_\bbR = \sigma(\cA_1^\prime)$.
\end{enumerate}

We now move to the other family of sets. By symmetry of 2 and 3 it suffices to prove only 2, the other proof will be almost identical.
\begin{enumerate}
\setcounter{enumi}{4}
\item Show that for any $a < b \in \bbR$
\[
	(a,b] = \bigcap_{j \in \bbN} (a, b+\frac{1}{j}).
\]
\item Show that for any $a < b \in \bbR$
\[
	(a,b) = \bigcup_{j \in \bbN} (a,b-\frac{1}{j}].
\]
\item Prove that $\cB_\bbR = \sigma(\cA_2^\prime) = \sigma(\cA_2)$, where $\cA_2 = \{(a,b] \, : \, a,b \in \bbQ\}$ and $\cA_2^\prime = \{(a,b] \, : \, a,b \in \bbR\}$.
\end{enumerate}

This basically covers the full set of ideas to prove the rest of Proposition~\ref{prop:characterization_borel_sigalg_1d}. We invite you to work these out yourself to practice with these kind of arguments. For this problem however we will ask you to explain the idea for the proofs.
\begin{enumerate}
\setcounter{enumi}{7}
\item Explain what changes in the proof of point 3 of Proposition~\ref{prop:characterization_borel_sigalg_1d} from the proof of point 2 of this proposition outlined above.
\item Describe the proof strategy to get points 4-8 of Proposition~\ref{prop:characterization_borel_sigalg_1d} using 1-4.
\end{enumerate}
\end{problem}

\begin{problem}\label{prb:completion_measure_space}
The goal of this problem is to complete a given measure space. To this end, let $(\Omega, \cF, \mu)$ be a measure space. Let $\cN$ be the family of null sets of $\mu$ and define the family of sets $\bar{\cF}$ as
\[
	\bar{\cF} := \{A \cup N \, : \, A \in \cF \text{ and } N \in \cN\}.
\]
\begin{enumerate}
\item Show that $\bar{\cF}$ is a \sigalg/ that contains $\cF$.
\end{enumerate}
Define the set function $\bar{\mu} : \bar{\cF} \to [0,\infty]$ as
\[
	\bar{\mu}(A \cup N) := \mu(A).
\]
\begin{enumerate}
\setcounter{enumi}{1}
\item Prove that $\bar{\mu}$ is a measure on $\bar{\cF}$.
\item Show that $\bar{\mu}|_\cF = \mu$.
\item Conclude that $(\Omega, \bar{\cF}, \bar{\mu})$ is a complete measure space.
\end{enumerate}
\end{problem}

%%%%%%%%%%%%%%%%%%%%%%%%%%%%%%%%%%%%%%%%%%%%%%%%%%%%%%%%%%%%%%%%%%%%%%%%%%%%%%%%%%%%%%%%%%%%%%%%%%%%%%%%%%%%%%%%%%%%%%%%%%%%%
%												Clippings																	%
%%%%%%%%%%%%%%%%%%%%%%%%%%%%%%%%%%%%%%%%%%%%%%%%%%%%%%%%%%%%%%%%%%%%%%%%%%%%%%%%%%%%%%%%%%%%%%%%%%%%%%%%%%%%%%%%%%%%%%%%%%%%%

%But in order to state and prove this result we need to introduce a few concepts as well as a powerful theorem, called the monotone class theorem.
%
%We start with the definition of an algebra.
%
%\begin{definition}[Algebra's of sets]
%A collection $\cA$ of subsets of $\Omega$ is called an \emph{algebra} if
%\begin{enumerate}
%\item $\emptyset \in \cA$,
%\item $\Omega \setminus A \in \cA$ for all $A \in \cA$, and
%\item $A \cup B \in \cA$ for every $A, B \in \cA$.
%\end{enumerate}
%\end{definition}
%
%Observe that, as the name suggests, every \sigalg/ is indeed and algebra. However, in addition to the properties of an algebra, \sigalgs/ where also closed under countable unions and intersections. We will actually take these properties on their own and define any collection of subsets that have these two properties a monotone class.
%
%\begin{definition}[Monotone classes]
%A collection $\cM$ of subsets of $\Omega$ is called a \emph{monotone class} if
%\begin{enumerate}
%\item $\bigcup_{i \in \bbN} A_i \in \cM$ holds for any increasing family of sets $(A_i)_{i \in \bbN}$ in $\cM$, and
%\item $\bigcap_{i \in \bbN} A_i \in \cM$ holds for any decreasing family of sets $(A_i)_{i \in \bbN}$ in $\cM$
%\end{enumerate}
%\end{definition}
%
%As we already remarked, any \sigalg/ is a monotone class. However, there are monotone classes that are not algebras and vise versa, there are algebras that are not monotone classes. However, suppose we start with an algebra $\cA$ and we want to turn this into a \sigalg/. Then we at least need to ensure it is also a monotone class. Similar to the construction of $\sigma(\cA)$ we can construct the smallest monotone class that contains $\cA$. Moreover, it turns out, maybe not surprisingly, that the resulting collection is \sigalg/. Even better, it is exactly $\sigma(\cA)$. This is the content of the monotone class theorem. 
%
%\begin{theorem}[Monotone class theorem]
%Let $\cA$ be an algebra on $\Omega$ and let $\Xi_\cA$ denote the collection of all monotone classes that contain $\cA$. Then 
%\begin{enumerate}
%\item the collection defined by
%\[
%	\cM(\cA) = \bigcup_{\cM \in \Xi_\cA} \cM,
%\]
%is a monotone class, and moreover
%\item $\cM(\cA)$ is the smallest \sigalg/ containing $\cA$, i.e. $\cM(\cA) = \sigma(\cA)$.
%\end{enumerate}
%\end{theorem}
%
%The proof of this theorem is left as an structured exercise, see Problem [REF].
%
%We will leverage the full power of this result to show that, in particular, any two measure that agree on the basis of a \sigalg/ are the same on the entire \sigalg/. The result we state is actually more general.
%
%\begin{theorem}[Uniqueness of measures]
%Let $(\Omega, \cF)$ be a measureable space and let $\cA \subset \cF$ be an algebra. In addition, let $\mu_1$ and $\mu_2$ be finite measures on $(\Omega, \cF)$ such that $\mu_1 = \mu_2$ on $\cA$, i.e. $\mu_1(A) = \mu_2(A)$ for every $A \in \cA$. Then $\mu_1 = \mu_2$ on $\sigma(\cA)$.
%\end{theorem}
%
%\begin{proof}
%Define the collection
%\[
%	\cM := \left\{A \in \cF \, : \, \mu_1(A) = \mu_2(A)\right\}.
%\]
%The goal of the proof is to show that this is a monotone class. If that is true then, since $\cA \subset \cM$, the monotone class theorem (Theorem [REF]) implies that $\sigma(\cA) = \cM(\cA) \subset \cM$ and hence $\mu_1 = \mu_2$ on $\sigma(\cA)$.
%
%To show that $\cM$ is a monotone class let $(A_i)_{i \in \bbN}$ be an increasing sequence in $\cM$. Since by definition, $\mu_1(A_i) = \mu_2(A_i)$ for all $i \in \bbN$, continuity from below (Proposition [REF]) implies that
%\[
%	\mu_1\left(\bigcup_{i \in \bbN} A_i\right) = \lim_{i \to \infty} \mu_1(A_i) 
%	= \lim_{i \to \infty} \mu_2(A_i) = \mu_2\left(\bigcup_{i \in \bbN} A_i\right),
%\]
%which implies that $\bigcup_{i \in \bbN} A_i \in \cM$.
%
%Similarly, now let $(A_i)_{i \in \bbN}$ be a decreasing sequence. Again, by definition $\mu_1(A_i) = \mu_2(A_i)$ for all $i \in \bbN$ and moreover $\mu_1(A_1) = \mu(A_1) < \infty$ since both measures are finite. Hence continuity from above (Proposition [REF]) implies that
%\[
%	\mu_1\left(\bigcap_{i \in \bbN} A_i\right) = \lim_{i \to \infty} \mu_1(A_i) 
%	= \lim_{i \to \infty} \mu_2(A_i) = \mu_2\left(\bigcap_{i \in \bbN} A_i\right).
%\]
%It then follows that $\bigcap_{i \in \bbN} A_i \in \cM$ which shows that $\cM$ is indeed a monotone class.
%\end{proof}


\chapter{Measurable functions and stochastic objects}
\label{chapter:measurable_functions}

\section{Measurable functions}

Now that we have defined measure spaces $(\Omega, \cF, \mu)$, through \sigalgs/ and measures and studied properties of both these objects, it is time to look at functions between such spaces. We will focus on functions that preserve the measurable structure of the spaces.

The main object in analysis were \emph{continuous} function $f : \bbR^d \to \bbR^m$. This property was important, as it allowed us to differentiate the function and perform integration. 

\subsection{Definition and properties}

We want to consider functions $f : \Omega \to E$ between measurable spaces $(\Omega, \cF)$ and $(E, \cG)$ that preserve the measurable structure, as imposed by the \sigalgs/. It turns out that it the best way to do this it to look at the preimage of measurable sets in $E$.

\begin{definition}[Measurable function]\label{def:measurable_function}
Let $(\Omega, \cF)$ and $(E, \cG)$ be two measurable spaces. A function $f: \Omega \to E$ is said to be \emph{$(\cF, \cG)$-measurable} is $f^{-1}(B) \in \cF$ for every $B \in \cG$.
\end{definition}

It is important to note that whether a function is measurable or not depends on the \sigalgs/ we consider in each of the measurable spaces. This means that a function $f : \Omega \to E$ might be $(\cF, \cG)$-measurable but not $(\cF^\prime, \cG)$-measurable for a different sigma algebra $\cF^\prime$ on $\Omega$. This is different from the notion of continuity of functions on $\bbR^d$. 

We will often omit the explicit reference to the \sigalgs/ in the definition of a measurable function if it is clear which \sigalgs/ are considered. That is, we will simply say that the function $f$ between the two measurable spaces $(\Omega, \cF)$ and $(E, \cG)$ is \emph{measurable}. We will sometimes make the choice of \sigalgs/ explicit by saying that $f: (\Omega, \cF) \to (E, \cG)$ is measurable.

We will provide an important example of measurable functions to $\bbR$, the indicator functions.

\begin{example}[Indicator functions are measurable]
Let $(\Omega,\cF)$ be a measurable space, $A \in \cF$ and $f : \Omega \to \bbR$ be defined as $f = \mathbf{1}_A$, that is 
\[
	f(\omega) = \begin{cases}
		1 &\text{if } \omega \in A,\\
		0 &\text{otherwise.}
	\end{cases}
\]
Then $f$ is measurable.

To see this, first note that $f^{-1}(\{1\}) = A \in \cF$ and $f^{-1}(\{0\}) = \Omega\setminus A \in \cF$. This implies that for any set $B \in \cB_\bbR$ we have that $f^{-1}(B \cap \{x\}) \in \cF$ with $x = 0, 1$. Hence
\[
	f^{-1}(B) = f^{-1}(B \cap \{0\}) \cup f^{-1}(B \cap \{1\}) \in \cF.
\]
\end{example}

The fact that measurability of $f$ depends on the \sigalgs/ involved mean we need to take a bit of care when considering operations on functions, as these might destroy the measurability. The most natural operation we should check first is composition, as we would like to be able to compose measurable functions into measurable functions. Luckily this is possible.

\begin{proposition}[Composition of measurable functions]
Let $(\Omega_i, \cF_i)$, for $i = 1,2,3$ be three measurable spaces and $f : \Omega_1 \to \Omega_2$, $g : \Omega_2 \to \Omega_3$ be two measurable functions. Then the composition $h := g \circ f : \Omega_1 \to \Omega_3$ is measurable.
\end{proposition}

\begin{proof}
By definition, we need to show that for every $A \in \cF_3$ the preimage $h^{-1}(A) \in \cF_1$. First note that
\begin{align*}
	h^{-1}(A) = (g\circ f)^{-1}(A) &= \{x \in \Omega \, : \, g(f(x)) \in A\} \\
	&= \{x \in \Omega \, : \, f(x) \in g^{-1}(A) \} = f^{-1}(g^{-1}(A)).
\end{align*}
Since $g$ is $(\cF_2, \cF_3)$-measurable, $g^{-1}(A) \in \cF_2$. Then, using that $f$ is $(\cF_1, \cF_2)$-measurable, we conclude that $h^{-1}(A) = f^{-1}(g^{-1}(A)) \in \cF_1$ as was required to show.
\end{proof}

The next result shows that we can also restrict a measurable function $f : \Omega \to E$ to a measurable subset $A \subset \Omega$, as long as we consider the appropriate (and natural) \sigalg/. The same holds for extensions. 

\begin{proposition}[Restriction and extension of measurable functions]
Let $f: (\Omega, \cF) \to (E, \cG)$ be a measurable function and let $A \in \cF$ be non-empty. Then the restriction map $f|_A : \Omega \to E$ is $(\cF_A, \cG)$-measurable.

Moreover, if $g_A : A \to E$ is $(\cF_A, \cG)$-measurable, and $p \in E$, then the extension
\[
	g(\omega) := \begin{cases}
		g_A(\omega) &\text{if } \omega \in A,\\
		p &\text{if } \omega \notin A,
	\end{cases}
\]
is $(\cF, \cG)$-measurable.
\end{proposition}

\begin{proof}
TODO
\end{proof}

At this stage these are the only general properties of measurable function we can consider. However, if the measurable space a function maps to has more structure we can see if this structure also respect the measurability. For example, we will see later in Section~\ref{sec:measurable_functions_real_line} that for measurable functions $f, g : \Omega \to \bbR$ their product and sum are also measurable, as well as many other operations.

\subsection{Checking for measurability}

Given any function $f : \Omega \to E$ between two measurable spaces $(\Omega, \cF)$ and $(E, \cG)$, when is this measurable? Definition~\ref{def:measurable_function} tells us that to answer this question we need to check that the preimage of any measurable set is again measurable. But this can be a cumbersome exercise. Or even impossible when we do not have an explicit description of the sigma algebra. This can happen, for example, when $\cG$ is generated by some collection of sets $\cA$, which is the case for the important Borel \sigalg/. 

Fortunately, the definition of measurability works very well with generated \sigalgs/. In particular, to show that a function is measurable, it suffices to only consider sets from the generator set $\cA$, instead of the entire \sigalg/ $\sigma(\cA)$.

\begin{lemma}\label{lem:measurable_condition_generator}
Let $(\Omega, \cF)$ and $(E, \cG)$ be two measurable spaces such that $\cG = \sigma(\cA)$. Let $f: \Omega \to E$ be a function such that $f^{-1}(A) \in \cF$ for all $A \in \cA$. Then $f$ is $(\cF, \cG)$-measurable.
\end{lemma}

\begin{proof}
Consider the following collection of subsets:
\[
	\cH := \{B \subset \cG \, : \, f^{-1}(B) \in \cF\}.
\]
We claim that $\cH$ is a \sigalg/ on $E$. Suppose this is indeed true. Then, since by construction $\cA \subseteq \cH$, it follows from Lemma~\ref{lem:inclusion_sigalgs} that $\cG = \sigma(\cA) \subseteq \cH$. But this then implies that $f^{-1}(B) \in \cF$ for each $B \in \cG$ which means that $f$ is $(\cF, \cG)$-measurable.

So let's prove that $\cH$ is a \sigalg/. First we note that $f^{-1}(\emptyset) = \emptyset \in \cF$ and $f^{-1}(E) = \Omega \in \cF$. So $\emptyset, E \in \cH$. 

Next, let $B \in \cH$. Then
\[
	f^{-1}(E\setminus B) = \Omega \setminus f^{-1}(B) \in \cF,
\]
since by definition $f^{-1}(B) \in \cF$. So $E\setminus B \in \cH$.

Finally, if $(B_i)_{i \in \bbN}$ is a sequence of sets in $\cH$, then
\[
	f^{-1}\left(\bigcup_{i = 1}^\infty B_i\right) = \bigcup_{i = 1}^\infty f^{-1}(B_i) \in \cF,
\]
which shows that $\bigcup_{i = 1}^\infty B_i \in \cH$, completing the proof that $\cH$ is a \sigalg/.
\end{proof}

We thus see that at least. But that still requires us to check if any given function is measurable. For example, is the function $f : \bbR \to \bbR$ given by $f(x) = e^x$, measurable? It would be be much better if we have a more familiar criteria that would imply measurability. Continuity does exactly this. 

\begin{proposition}
Every continuous map $f : \bbR^d \to \bbR^m$ is $(\cB_{\bbR^d}, \cB_{\bbR^m})$-measurable.
\end{proposition}

\begin{proof}
Recall from analysis that a map $f : \bbR^d \to \bbR^m$ is continuous if for every $x \in \bbR^d$ and $\varepsilon > 0$, there exists an $r = r(x,\varepsilon)$ such that 
\[
	\|f(x) - f(y)\| < \varepsilon \quad \text{for every } y \in B_x(r).
\]
The key step for this proof is to show that this is equivalent to the following condition\footnote{Actually, the definition we state here using open sets is the general definition for continuous functions in the mathematical field of topology.}:
\[
	\text{for every open set } O \subset \bbR^m \quad f^{-1}(O) \text{ is open}.
\]
If this is true then, since the Borel \sigalg/ is generated by the open sets, it follows that $f^{-1}(O) \in \cB_{\bbR^d}$ for each open set $O \subset \bbR^m$. Lemma~\ref{lem:measurable_condition_generator} then implies that $f$ is measurable.

So we are left to show the equivalence of the two conditions for continuity. First assume that $f$ is continuous and take an arbitrary open set $O \subset \bbR^m$. We need to show that $f^{-1}(O)$ is open, which means that for every $x \in f^{-1}(O)$ we should find an $r$ such that $B_x(r) \subset f^{-1}(O)$. Since $O$ is open, there exists a $\varepsilon > 0$ such that $B_{f(x)}(\varepsilon) \subset O$. Continuity of $f$ now implies the existence of an $r$ such that $\|f(x) - f(y)\| < \varepsilon$ for all $y \in B_x(r)$. But this simply means that $f(y) \in B_{f(x)}(\varepsilon) \subset O$ for every $y \in B_x(r)$, which implies that $B_x(r) \in f^{-1}(O)$.

Now assume that $f^{-1}(O)$ is open in $\bbR^d$, for every open set $O \in \bbR^m$ and take $x \in \bbR^d$ and $\varepsilon > 0$. Then the ball $B_{f(x)}(\varepsilon)$ is open in $\bbR^m$, so that by assumption $f^{-1}(B_{f(x)}(\varepsilon))$ is open in $\bbR^d$. Since $x \in f^{-1}(B_{f(x)}(\varepsilon))$ there now must exist an $r > 0$ such that $B_x(r) \subset f^{-1}(B_{f(x)}(\varepsilon))$. But this then implies that for every $y \in B_x(r)$, $f(y) \in B_{f(x)}(\varepsilon)$, which is equivalent to $\|f(x) - f(y)\| < \varepsilon$.
\end{proof}

With this result we have a vast world of measurable functions $f : \bbR^d \to \bbR^m$ at our disposal. It should also be noted that the space of measurable functions is larger than that of continuous functions. For example, the indicator functions are measurable but not continuous.

So on the Borel space $(\bbR^d, \cB_{\bbR^d})$ we have a large class of measurable functions. However, when dealing with functions that map to measurable spaces that are not the Borel space, we still need to carefully check if it is measurable. But what if we can simply construct a \sigalg/ such that it makes a function measurable?

\subsection{\sigalgs/ generated by measurable functions}

Suppose we have a function $f : \Omega \to E$ from a set $\Omega$ to some measurable space $E, \cG)$. If we want to study the function $f$ in the framework of measure theory, we need to turn $\Omega$ into a measurable space $(\Omega, \cF)$ and have $f$ be $(\cF, \cG)$-measurable. The good news is that we can construct a minimal \sigalg/ that does the job for us. It can even be done for multiple functions at the same time.

\begin{proposition}\label{prop:sigalg_generated_functions}
Let $(\Omega_i, \cF_i)$, for $i \in I$ be measurable spaces and $(f_i)_{i \in I}$ be a family of functions $f_i : \Omega \to \Omega_i$. Then the smallest \sigalg/ on $\Omega$ that makes all $f_i$ simultaneously measurable is
\[
	\sigma(f_i \, : \, i \in I):=\sigma\left(\bigcup_{i \in I} f_i^{-1}(\cF_i)\right).
\] 
\end{proposition}

\begin{proof}
First note that by Proposition~\ref{prop:generated_sigalg}, $\sigma(f_i \, : \, i \in I)$ is a \sigalg/. We will show that any \sigalg/ that makes each $f_i$ measurable much contain $\sigma(f_i \, : \, i \in I)$. So let $\cF$ be such a \sigalg/. Then in particular, for any $i\in I$ and $B \in \cF_i$ we have that $f_i^{-1}(B) \in \cF$. This implies that
\[
	\bigcup_{i \in I} f_i^{-1}(\cF_i) \subseteq \cF.
\]
Now since $\sigma(f_i \, : \, i \in I)$ is generated by the collection on the left hand side, Lemma~\ref{lem:inclusion_sigalgs} implies that 
\[
	\sigma(f_i \, : \, i \in I):=\sigma\left(\bigcup_{i \in I} f_i^{-1}(\cF_i)\right) \subset \sigma(\cF) = \cF.
\]
\end{proof}

Similar to Lemma~\ref{lem:measurable_condition_generator}, when $\cF_i = \sigma(\cA_i)$ it turns out that to construct $\sigma(f_i \, : \, i \in I)$ it suffices to consider only preimages of the generator sets $\cA_i$.

\begin{proposition}\label{prop:extension_measurable_function}
Let $(\Omega, \cF)$ and $(\Omega_i, \cF_i)$, for $i \in I$ be measurable spaces such that $\cF_i = \sigma(\cA_i)$. Let $(f_i)_{i \in I}$ be a family of functions $f_i : \Omega \to \Omega_i$. Then 
\[
	\sigma(f_i \, : \, i \in I) = \sigma\left(\bigcup_{i \in I} f_i^{-1}(\cA_i)\right).
\] 
\end{proposition}

\begin{proof}
Let us write $\cG_1 = \sigma(f_i \, : \, i \in I)$ and $\cG_2 = \sigma\left(\bigcup_{i \in I} f_i^{-1}(\cA_i)\right)$. From the definition it is clear that $\cG_2 \subseteq \cG_1$. Moreover, each $f_i$ is $(\cG_2, \cF_i)$-measurable by Lemma~\ref{lem:measurable_condition_generator}. But by Proposition~\ref{prop:sigalg_generated_functions} $\cG_1$ is the smallest \sigalg/ that makes all $f_i$ $(\cG_1, \cF_i)$-measurable and hence $\cG_1 \subseteq \cG_2$, which implies the result.
\end{proof}

We end this section by going back to the product \sigalg/ given in Definition~\ref{def:product_sigalg}. There is an alternative way to construct it using functions. Let $(\Omega_1, \cF_1)$ and $(\Omega_2, \cF_2)$ be two measurable spaces and consider the functions $\pi_i : \Omega_1 \times \Omega_2 \to \Omega_i$, defined by 
\[
	\pi_1(x,y) = x \quad \pi_2(x,y) = y.
\]
These are called the \emph{canonical projections}. Following Proposition~\ref{prop:sigalg_generated_functions} we can construct the \sigalg/ $\sigma(\pi_1, \pi_2)$ on $\Omega_1 \times \Omega_2$, which makes both canonical projections measurable. Then it follows that, see Problem~\ref{prb:product_sigalg_equivalence},
\begin{equation}\label{eq:product_sigalg_equivalence}
	\cF_1 \otimes \cF_2 = \sigma(\pi_1, \pi_2).
\end{equation}

\subsection{Push forward measure}

Given a measure space $(\Omega, \cF, \mu)$ and measurable function $f : \Omega \to E$ to a measurable space $(E,\cG)$ we can construct a measure on $(E,\cG)$ using $f$ and $\mu$. This measure is called the \emph{push-forward measure}, as it can be thought of a pushing $\mu$ to $\cG$ via the function $f$.

\begin{proposition}[Push-forward measure]\label{prop:push_forward_measure}
Let $(\Omega, \cF, \mu)$ be a measure space, $(E, \cG)$ a measurable space and $f : \Omega \to E$ a measurable function. Then the set function $f_\# \mu$ defined as
\[
	f_\# \mu (B) = \mu(f^{-1}(B)) \text{ for every } B \in \cG,
\]
is a measure on $(E, \cG)$ called the \emph{push-forward measure} of $\mu$ under $f$.

Moreover, if $\mu$ is a probability measure, so if $f_\# \mu$.
\end{proposition}

The proof of this result is elementary and left as an exercise, see Problem~\ref{prb:push_forward_measure}. 

Push-forward measures play an important role in measure theory, and especially in probability theory. For example, they come up for example when we apply a change of variables in integrals (see [REF]). More importantly, we will see in Section~\ref{sec:random_variables} that the cumulative distribution function of a random variable is actually defined as the push-forward measure of some probability measure $\bbP$ under a suitable measurable function.

\section{Measurable functions on the real line}\label{sec:measurable_functions_real_line}

When studying properties of measurable function we could only do a few things for general measurable spaces. So in this section we will focus on a specific measurable space: the real line $(\bbR, \cB_\bbR)$. We will see that most of the natural operations we can apply to function in a point-wise manner, such as addition and multiplication, preserve their measurability. But we will do even better. We will show that taking point-wise limit operations, such as taking a supremum of a family of measurable functions, preserves measurability as well. This makes the class of measurable functions much more powerful then that of continuous functions, as point-wise limits of continuous functions are not guaranteed to be continuous again. All thes properties will be useful when we introduce the concept of integration of measurable functions in Chapter [REF] and develop limit theorems for integrals in Chapter [REF].

To properly study limit operations on measurable functions, that could diverge, we need to have $\infty$ be a part of the real line (which it is not). So we first extend the real line to include both $\infty$ and $-\infty$.

\subsection{Extended real line}

We define $\bar{\bbR} := [-\infty, \infty]$ as the \emph{extended real line}. We impose the natural ordering on $\bar{\bbR}$, inherited from $\bbR$, with the addition that $-\infty < x$ and $x < \infty$ for all $x \in \bbR$. The extended real line also has the same operations of addition and multiplications, with are extend to include the two new elements $\pm \infty$:
\begin{enumerate}
\item for every $x\in \bbR$, $x + \infty = \infty + x = \infty$ and $x + (-\infty) = (-\infty) + x = -\infty$,
\item $\infty + \infty = \infty$ and $(-\infty) + (-\infty) = -\infty$,
\item for every $x \in (0,\infty]$, $\pm x (\infty) = (\infty) \pm x = \pm \infty$, $\pm x (-\infty) = (-\infty) \pm x = \mp \infty$,
\item $0 (\pm \infty) = (\pm \infty) 0 = 0$ and $1/\pm \infty = 0$.
\end{enumerate}

To turn $\bar{\bbR}$ into a measurable space we extend the Borel \sigalg/ to include the new elements $\pm \infty$.

\begin{definition}[Extended real line]
The Borel \sigalg/ $\bar{\cB}$ of the extended real line $\bar{\bbR}$ is defined by
\[
	\bar{\cB} := \{A \cup S \, :\, A \in \cB_\bbR \text{ and } S \in \{\emptyset, \{-\infty\}, \{\infty\}, \{-\infty, \infty\}\}
\]
\end{definition}

The following results, whose proof is left as an exercise, relates $\bar{\cB}$ to the original Borel \sigalg/.

\begin{lemma}\label{lem:characterization_extended_borel}
The extended Borel \sigalg/ $\bar{\cB}$ satisfies
\[
	\cB_\bbR = \bar{\cB} \cap \bbR.
\]
Moreover, it is generated by sets of the form $[a,\infty]$, with $a \in \bbQ$ (or $(a,\infty]$, $[-\infty.a)$, $[-\infty,a]$).
\end{lemma}

\begin{proof}
See Problem [REF]
\end{proof}

\subsection{Basic operations}

For the rest of this section, for any set $A$ we will write $\{f \in A\}$ as a shorthand notation for the set $\{\omega \in \Omega \, :,\ f(\omega) \in A\}$. In addition, we write $\{f \le a\}$ for the set $\{f \in (-\infty, a]\}$ and similarly for $<, \ge, >, =$ and $\ne$.

\begin{lemma}\label{lem:measurable_set_real_line}
Let $f : (\Omega, \cF) \to \bbR$ be measurable and take $a \in \bbR$. Then the following sets 
\[
	\{f < a\}, \{f \le a\}, \{f > a\}, \{f \ge a\}, \{f = a\} \text{ and } \{f \ne a\},
\]
are also measurable.
\end{lemma}

\begin{proof}
Since $f$ is measurable, it follows immediately from Proposition~\ref{prop:characterization_borel_sigalg} and Lemma~\ref{lem:measurable_condition_generator} that $\{f < a\}, \{f \le a\},\{f > a\}, \{f \ge a\} \in \cF$. This then implies the other two claims since $\{f = a\} = \{f \le a\} \setminus \{f < a\}$ and $\{f \ne a\} = \Omega \setminus \{f = a\}$.
\end{proof}

\begin{lemma}
Let $f, g : (\Omega, \cF) \to \bbR$ be measurable. Then the following functions (where operations are always taken point-wise) are measurable as well:
\begin{enumerate}
\item $f + g$,
\item $f \vee g := \max\{f,g\}$,
\item $f \wedge g := \min\{f,g\}$,
\item $f g$, and
\item $f/g$ if $g \ne 0$ on $\Omega$.
\end{enumerate} 
\end{lemma}

\begin{proof}
We will prove 2 and 4. The other parts are left as an exercise, see Problem [REF].

\paragraph{2} We first note that the sets $\{f \ge g\}$ and $\{g > f\}$ are measurable. This follows from Lemma~\ref{lem:measurable_set_real_line} and the fact that
\[
	\{f \ge g\} = \bigcup_{q \in \bbQ} \{f \ge q\} \cap \{g \le q\},
\]
while
\[
	\{g > f\} = \bigcup_{q \in \bbQ} \{g \ge q\} \cap \{f < q\}.
\]
Next we observe that for any set $A \subset \bbR$
\[
	(f \vee g)^{-1}(A) = \left(f^{-1}(A) \cap \{f \ge g\}\right) \cup \left(g^{-1}(A) \cap \{g > f\}\right),
\]
which implies that $(f \vee g)^{-1}(A) \in \cF$ for any $A \in \cB_\bbR$.

Lemma~\ref{lem:characterization_extended_borel} $\bar{\cB}$ is generated by the sets $[a,\infty]$, for $a \in \bbQ$. Hence, by Lemma~\ref{lem:measurable_condition_generator} it suffices to show that 
\[
	(fg)^{-1}([a,\infty]) = \{\omega \in \Omega \, : \, f(\omega) g(\omega) \in [a, \infty]\} \in \cF.
\]

\paragraph{4} This proof requires several steps, so please bare with us. We first write
\[
	\{fg \in (-\infty, t]\} = \{fg \in (-\infty, t \wedge 0)\} \cup \{fg = 0\} \cup \{fg \in (0,t \vee 0]\},
\]
were we will disregard the set $\{fg = 0\}$ if $t < 0$. Our goal is to show that each of these three sets is measurable which will then imply the result.

First note $\{fg = 0\} = \{f = 0\} \cup \{g = 0\} \in \cF$ by Lemma~\ref{lem:measurable_set_real_line}.

Now assume that $t > 0$ so that $\{fg \in (0,t \vee 0]\} \ne \emptyset$. Then
\[
	\{fg \in (0,t \vee 0]\} = \bigcup_{q \in \bbQ_{>0}} \{f \in (0,q]\} \cap \{g \in (0,t/q]\}.
\]
Since for any $x >0$, $(0,x) = (-\infty,x] \setminus (-\infty,0] \in \cB_\bbR$ and the union above is over a countable number of elements ($\bbQ$ is countable) it follows that $\{fg \in (0,t \vee 0]\} \in \cF$.

We are thus left to show that $\{fg \in (-\infty, t \wedge 0)\}$ is measurable. First we observe that
\[
	\{fg \in (-\infty, t \wedge 0)\} = \bigcup_{q \in \bbQ_{>0}} \{fg \in (-\infty,-q)\},
\]
and hence it suffices to show that $\{fg \in (-\infty,-q)\}$ is measurable for any $q \in \bbQ_{>0}$. To achieve this we further split this event as follows:
\[
	\left(\{fg \in (-\infty,-q)\} \cap \{f < 0\} \cap \{g > 0\}\right) 
	\cup \left(\{fg \in (-\infty,-q)\} \cap \{f > 0\} \cap \{g < 0\}\right),
\]
and observe that due to the symmetry on the right hand side, it is enough to show that $\{fg \in (-\infty,-q)\} \cap \{f < 0\} \cap \{g > 0\}$ is measurable. For this we note that
\[
	\{fg \in (-\infty,-q)\} \cap \{f < 0\} \cap \{g > 0\} 
	= \bigcup_{p \in \bbQ_{>0}} \{f \in (-\infty, -p)\} \cap \{g \in (0, q/p)\}.
\]
Since this is a countable union of measurable sets, it is indeed measurable and thus so is $\{fg \in (-\infty, t \wedge 0)\}$. This concludes the proof of 4.
\end{proof}

\subsection{Limit operations}

In addition to the fact that most of the obvious point-wise operations on measurable functions yields another measurable function, it turns out that this also holds for limit operations. 

\begin{lemma}\label{lem:limit_operations_measurable_functions}
Let $(f_n)_{n \ge 1}$ be a family of measurable functions from $(\Omega, \cF)$ to  $(\bar{\bbR}, \bar{\cB})$. Then the following functions are also measurable (where again operations are taken point wise):
\begin{enumerate}
\item $\sup_{n \ge 1} f_n$,
\item $\inf_{n \ge 1} f_n$,
\item $\limsup_{n \to \infty} f_n$, and
\item $\liminf_{n \to \infty} f_n$.
\end{enumerate}

Moreover, if the limit $\lim_{n \to \infty} f_n$ exists it is also measurable. 
\end{lemma}

\begin{proof}
We will prove 1 and leave the other parts as an exercise, see Problem [REF]. 

To this end, we will show that for any $x \in \bbR$
\begin{equation}\label{eq:limit_operations}
	\{\sup_{n \ge 1} f_n > x\} = \bigcup_{n \ge 1} \{f_n > a\}.
\end{equation}
Note that if this holds then $\{\sup_{n \ge 1} f_n > x\} \in \cF$ since each set $\{f_n > a\}$ is measurable by Lemma~\ref{lem:measurable_set_real_line} and hence $\{\sup_{n \ge 1} f_n > x\}$ (check this yourself, see Problem [REF]).

Since $a < f_n(\omega) \le \sup_{n \ge 1} f_n(\omega)$ holds for any $\omega$ we get the inclusion $\supset$ for the above two sets. For the other inclusion $\subset$ we will argue by contradiction. Suppose that $f_n(\omega) \le x$ for all $n \ge 1$, then also $\sup_{n \ge 1} f_n(\omega) \le x$. This implies that
\[
	\{\sup_{n \ge 1} f_n \le x\} \supset \bigcap_{n \ge 1} \{f_n \le a\},
\] 
where each side is the complement of the sets in~\eqref{eq:limit_operations}.

\end{proof}

Although the proof makes the content of Lemma~\ref{lem:limit_operations_measurable_functions} look rather trivial, it is actual very important. In particular is shows the power of the class of measurable functions. In contrast, the class of continuous functions is not stable under point-wise limit operations. 

\begin{example}[Point-wise limits of continuous functions are not continuous]
Consider the sequence of functions $(f_n)_{n \ge 1}$ defined by $f_n(x) = \arctan(xn)$. Each $f_n$ is clearly continuous. So let us consider the point-wise limit $f(x) = \lim_{n \to \infty} f_n(x)$. For any $x > 0$ we have that
\[
	f(-x) = \lim_{n \to \infty} \arctan(-x n) = -\frac{\pi}{2},
\]
while
\[
	f(x) = \lim_{n \to \infty} \arctan(x n) = \frac{\pi}{2}.
\]
Moreover, $f(0) = \arctan(0) = 0$. We thus conclude that the point-wise limit of $f_n$ is given by
\[
	f(x) = \begin{cases}
		-\frac{\pi}{2} &\text{if } x < 0,\\
		0 &\text{if } x = 0,\\
		\frac{\pi}{2} &\text{if } x >0,
	\end{cases}
\]
which is clearly not continuous. However, by Lemma~\ref{lem:limit_operations_measurable_functions} it is measurable.
\end{example}

The fact that point-wise limits of continuous functions are not necessary continuous is the reason why one has to be careful when, for example, exchanging limits and integration. Here the notion of uniform continuity is often needed. In contrast, as we will see later, this is not an issue for measurable functions and we once we have defined the notion of integration of these functions we obtain a powerful set of limit results for such integrals.

For now we will move from the general setting of measurable functions to their application in the field of probability theory, in particular the concept of random variables.

\section{Random variables and general stochastic objects}\label{sec:random_variables}

% Random variables   

\subsection{Definition}

In the course Probability and Modeling two types of random variables were defined: discrete and continuous. Recall that a random variable was defined as a function $X : \Omega \to \bbR$ for some probability space $(\Omega, \cF, \bbP)$ such that
\[
	\{\omega \in \Omega \, : \, X(\omega) \le x\} \in \cF \quad \text{for all } x \in \bbR.
\]
Let us make two observations here. The first is that the set above is simply the preimage $X^{-1}((-\infty,x])$. Secondly, the sets $(-\infty, x]$ generate the Borel \sigalg/. Thus it follows from Lemma~\ref{lem:measurable_condition_generator} that $X$ is a measurable function. This is actual the proper way to define a random variable.

\begin{definition}[Random variable]
A \emph{random variable} is a measurable function from some probability space $(\Omega,\cF, \bbP)$ to the (extended) real line.
\end{definition}

It is important to observe that the definition of a random variable does not make any specific claims on what the probability space should be. 

Let $X$ be a random variable and recall that its \emph{cumulative distribution function} $F_X : \bbR \to [0,1]$ is defined as
\[
	F_X(t) = \bbP(X \le t).
\] 
The fact that we use $\bbP$ here, which is the probability measure on the space $(\Omega, \cF)$ is not a coincidence.

The idea behind the cdf $F_X(t)$ is that it denotes the "probability" that $X \in (-\infty ,t]$. From the perspective of measure theory, this means we need to assign a measure to the preimage of $(-\infty, t]$ under the measurable function $X$. For this, the only things we have at our disposal is the probability measure $\bbP$ and the measurable function $X$. Now recall from Proposition~\ref{prop:push_forward_measure} that we can always construct a measure from this, the push-forward measure. That is exactly what the cummulative distribution is,
\[
	F_X(t) := X_\# \bbP((-\infty, t]) = \bbP(X^{-1}((-\infty,t])).
\]

In fact we can actually define, at a much more general level, random elements in any measurable space and put an associated probability measure on this space by a push-forward.

\begin{definition}[Random elements]
Let $(\Omega,\cF, \bbP)$ be a probability space and $(E,\cG)$ some measurable space. A \emph{random element} in $(E,\cG)$ is a measurable map $X : \Omega \to E$. It associated \emph{probability measure} is defined as the push forward of $\bbP$ under $X$, i.e.
\[
	\bbP(X \in A) := \bbP(X^{-1}(A)) \quad \text{for every } A \in \cG.
\]
\end{definition}

Sometimes we use the term \emph{stochastic} instead of \emph{random}. 

With this general definition we can now easily define random vectors, random matrices, random functions and so one. The only thing we need is to start with the appropriate space (vectors, matrices, functions) and turn it into a measurable space by endowing it with a suitable \sigalg/. 

\begin{example}[Random elements]
\hfill
\begin{enumerate}
\item A random vector in $\bbR^d$ is a random element in $(\bbR^d, \cB_{\bbR^d})$.
\item A random $n \times m$ matrix is a random element in $(\bbR^n \times \bbR^m, \cB_{\bbR^n} \otimes \cB_{\bbR^m})$.
\end{enumerate}
\end{example}

\subsection{Constructing random variables}

Now that we know what random variables are, there is one problem. In order to define an random variable we need to formally define a probability space $(\Omega, \cF, \bbP)$ and measurable function $X: \Omega \to \bbR$. This is different from how we are used to work with random variables. Here we simply present a cdf $F$ and say that $X$ is a random variable with $\bbP(X \le t) = F(t)$, without worrying about a probability space or the measurability if $X$ as a function. It turns out that this way of working with random variables is valid, as for any cdf $F$ we can construct a probability space $(\Omega, \cF, \bbP)$ and measurable function $X$ such that $X_\# \bbP = F$. We will start this construction for a specific random variable and then use it to construct a random variable with any cumulative distribution function.

One of the first random variables you encounter in any course in probability theory is the \emph{standard uniform random variable}. This is a random variable $U$ that takes values in $[0,1]$ such that its cdf satisfies $F(t) = t$ for all $0\le t \le 1$. In the course Probability and Modeling this description would be enough to work with. But now that we know what a random variable actually is, we need a bit more. More precisely, we have to construct a probability space $(\Omega,\cF, \bbP)$ and a measurable function $U : \Omega \to \bbR$ such that 
\begin{equation}\label{eq:cdf_uniform_rv}
	\bbP\left(U^{-1}((-\infty,t])\right) = \begin{cases}
		0 &\text{if } t < 0,\\
		t &\text{if } 0 \le t\le 1,\\
		1 &\text{if } t > 1.
	\end{cases}
\end{equation}

The following result shows that this is indeed possible. Moreover, in its proof we see a first nice usage of the Lebesgue measure.

\begin{proposition}[Uniform random variable]\label{prop:uniform_random_variable}
There exist a probability space $(\Omega,\cF, \bbP)$ and random variable $U$, such that $\bbP\left(U^{-1}((-\infty,t])\right)$ satisfies~\eqref{eq:cdf_uniform_rv}.
\end{proposition}

\begin{proof}
Consider the space $\Omega = [0,1]$ together with the restricted Borel \sigalg/ $\cF = \cB_\bbR|_{[0,1]}$ and as probability measure the restricted Lebesgue measure $\bbP := \lambda|_{[0,1]}$. Now consider the function $U(t) = \mathbf{1}_{(0,1]} \, t$. Then, it follows that
\[
	U^{-1}((-\infty,t]) = \begin{cases}
		\emptyset &\text{if } t \le 0,\\
		(0,t] &\text{if } 0 < t \le 1, \\
		[0,1] &\text{if } t > 1.
	\end{cases}
\]
Since by Theorem~\ref{thm:lebesgue_measure}
\[
	\lambda|_{[0,1]}((0,t]) = \lambda((0,t]) = t,
\]
for any $0 < t \le 1$ we have
\[
	\bbP\left(U^{-1}((-\infty,t])\right) := \lambda|_{[0,1]}\left(U^{-1}((-\infty,t])\right)
	= \begin{cases}
		0 &\text{if } t < 0,\\
		t &\text{if } 0 \le t\le 1,\\
		1 &\text{if } t > 1.
	\end{cases}
\]
\end{proof}

The standard uniform random variable is extremely important, as it is the base from which we can construct any other random variable. To illustrate this let us first consider the case of an \emph{exponential random variable} with rate $\lambda > 0$. This is a random variable $X$ with cdf
\[
	F_X(t) = \begin{cases}
		0 &\text{if } t \le 0,\\
		1-e^{-\lambda t} &\text{if } t > 0.
	\end{cases}
\]

For $u \in (0,1)$, write $H(u) := F_X^{-1}(u)$ and note that
\[
	H(u) = \frac{1}{\lambda} \log\left(\frac{1}{1-u}\right).
\]
Now let $U$ be the standard normal random variable and consider the composition $H \circ U : [0,1] \to \bbR$. First we note that since cdf $F_X(x)$ is strictly monotonic increase, so is $H$. In particular it follows that for any $t > 0$,
\[
	H^{-1}((-\infty,t]) = (-\infty, H^{-1}(t)] = (-\infty, F_X(t)].
\]
While $H^{-1}((-\infty,t]) = \emptyset$ if $t \le 0$.

Hence we get
\begin{align*}
	(H \circ U)^{-1}((-\infty, t]) = U^{-1}(H^{-1}((-\infty, t]))
	&= \begin{cases}
		U^{-1}(\emptyset) &\text{if } t \le 0,\\
		U^{-1}((-\infty, F_X(t)]) &\text{if } t > 0.
	\end{cases}
\end{align*}
From this it follows that 
\[
	\bbP\left((H \circ U)^{-1}((-\infty, t])\right) = \begin{cases}
			0 &\text{if } t \le 0,\\
			1-e^{-\lambda t} &\text{if } t > 0,
		\end{cases}
\]
from which we conclude that $H \circ U$ is a way to construct an exponential random variable with rate $\lambda$.

The main point of the construction above is to consider the inverse of the cdf $F^{-1}$ and evaluate this on a standard uniform random variable. However, when extending this to the more general case we have to deal with the fact that not every cdf has an inverse. For example, consider a Bernoulli random variable with success probability $0 < p < 1$. Then
\[
	F(t) = \begin{cases}
		0 &\text{if } t < 0, \\
		1-p &\text{if } 0 \le t < 1, \\
		1 &\text{if } t \ge 1,
	\end{cases}
\] 
which does not have an inverse as for any $y \in (0,1-p)$ there is no $t$ such that $F(t) = y$.

Nevertheless, if does hold than any cdf $F$ is non-decreasing and right continuous. For these type of functions there exists the notion of a \emph{generalized inverse}, defined as
\begin{equation}
	\overleftarrow{F}(u) := \inf\{ x \in \bbR \, : \, F(x) \ge y\}. 
\end{equation}
The construction we used for the exponential random variable can now be generalize by using $\overleftarrow{F}$ instead of $F^{-1}$. This results in the following theorem on the existence of random variables with a given cdf.

\begin{theorem}[Constructing random variables]\label{thm:construction_random_variable}
Let $F : \bbR \to [0,1]$ be a right continuous, non-decreasing function with 
\[
	\lim_{x \to -\infty} F(x) = 0 \quad \text{and} \quad \lim_{x \to \infty} F(x) = 1.
\]
Then there exists a probability space  $(\Omega,\cF, \bbP)$ and random variable $X$, such that 
\[
	\bbP\left(X \in (-\infty,t]\right) := \bbP\left(X^{-1}((-\infty,t])\right) = F(t).
\]
In other words, $X$ is a random variable with cdf $F$.

Moreover, $(\Omega,\cF, \bbP)$ can be chosen as $([0,1], \cB_{[0,1]}, \lambda|_{[0,1]})$ and $X = \overleftarrow{F}\circ U$, where $U$ is the standard uniform random variable.
\end{theorem}

\begin{proof}
We start with the following important observation: 
\[
	\overleftarrow{F}(u) \le x \iff F(x) \ge u.
\]
The implication from right to left is by definition of $\overleftarrow{F}$ and the fact that $F$ is non-decreasing. The implication from left to right is because $F$ is right continuous.


Now let $(\Omega,\cF, \bbP)$ be a probability space and $U$ a standard normal random variable. We will show that $X = \overleftarrow{F} \circ U$ is a random variable with the right probability measure. Since we can construct a standard uniform random variable on the probability $([0,1], \cB_{[0,1]}, \lambda|_{[0,1]})$ this also implies the last part. 

Consider the preimage of $(-\infty, t]$ under $X$. Then, using the above observation, we have
\begin{align*}
	X^{-1}((-\infty,t]) &= \{\omega \in \Omega \, : \, \overleftarrow{F}(U(\omega)) \in (-\infty,t]\}\\
	&= \{\omega \in \Omega \, :\ , U(\omega) \in (-\infty, F(t)]\} = U^{-1}((-\infty, F(t)]) \in \cB_{[0,1]}.
\end{align*}
Hence, $X$ is measurable. Finally, the above computation, together with Proposition~\ref{prop:uniform_random_variable}, also implies that
\[
	\bbP\left(X^{-1}((-\infty,t])\right) = \bbP\left(U^{-1}((-\infty, F(t)])\right) = F(t),
\]
which finished the proof.
\end{proof}

We end this section with an important remark for working with random variables, and random objects in general.

\begin{remark}[Probability spaces are implicit!]
It is important to note that even though we used a very explicit probability space to construct a standard uniform random variable and the random variable $X$ in the proof of Theorem~\ref{thm:construction_random_variable}, in general the probability space will often be \emph{implicit}. That is, if we say that $X$ is a random variable, we assume there is some probability space $(\Omega,\cF, \bbP)$ that makes $X$ into a measurable function with the right cdf. Theorem~\ref{thm:construction_random_variable} actually says that this is okay as we can always construct an appropriate probability space and measurable function to achieve the needed cdf.

Actually, when considering general random objects in $(E, \cG)$ we often also do not explicitly state or define the probability space. Since the relevant measure is defined through the push-forward we often only have to worry about taking the right measurable space $(E, \cG)$.

There are, however, some cases where one should be cautious about the probability space that is used. For example when considering the notion of \emph{convergence in probability} or \emph{almost sure convergence}. Or when constructing joint distributions of random variables. 
\end{remark}

%\subsection{Joint distributions and couplings}

\section{Problems}

\begin{problem}[Equivalence of product \sigalg/]\label{prb:product_sigalg_equivalence}
Prove equation~\eqref{eq:product_sigalg_equivalence}.
\end{problem}

\begin{problem}[Push-forward measure]\label{prb:push_forward_measure}
Prove Proposition~\ref{prop:push_forward_measure}.
\end{problem}


\chapter{The Lebesgue Integral}
\label{chapter:integration}
We now arrive at one of our main characters in Measure Theory: The Lebesgue integral. Unlike the Riemann integral, the Lebesgue integral can be constructed on any measure space $(\Omega,\cF,\mu)$. The construction will be done in multiple steps, starting with simple functions.

\section{The integral of a simple function}\label{sec:integral_simple_functions}

\begin{definition}
	A function $f: \Omega \to \bbR$ is called \emph{simple} if it takes the form
	\[
	f = \sum_{i=1}^N a_i \mathbf{1}_{A_i}
	\]
	for some positive integer $N \in \bbN$, disjoint measurable sets $A_1, \ldots, A_N \in \cF$ and constants $a_1, \dots, a_N  \in \bbR$. We define the \emph{$\mu$-integral} of a simple function $f$ by
	\[
		\int_\Omega f\, \dd \mu = \int_\Omega f(\omega)\,\mu(\dd\omega) := \sum_{i = 1}^N a_i \mu(A_i).
	\]
\end{definition}

A priori there could be different representations of the same simple function, so we should check that the integral of a simple function is well-defined. 
This follows, however, because $f$ actually has a unique representation 
\[
	f = \sum_{i=1}^M b_i \mathbf{1}_{B_i},\qquad \text{for which $b_i < b_{i+1}$}.
\]
By the finite additivity of the measure $\mu$,
\[
\sum_{i=1}^N a_i \mu(A_i) = \sum_{i=1}^M b_i \mu(B_i).
\]

\begin{remark}
	In case $(\Omega, \cF, \bbP)$ is a probability space, and $X$ is a simple, real-valued random variable on $\Omega$ having the representation
\[
	X = \sum_{i=1}^N a_i \mathbf{1}_{A_i},
\]
with mutually disjoint $A_i \in \cF$ and $a_i \in \bbR$, the integral is usually called the \emph{expectation} value of $X$ and is written as
\[
	\ex{X} := \int_\Omega X(\omega)\, \bbP(\dd\omega) = \sum_{i=1}^N a_i \prob{A_i}.
\]

\end{remark}

\section{The Lebesgue integral of nonnegative functions}

We now extend the $\mu$-integral from non-negative simple functions to arbitrary non-negative $\cF$-measurable functions.

\begin{definition}
	The \emph{$\mu$-integral} of a $(\cF,\cB_{[0,+\infty]})$-measurable function $f:\Omega\to[0,+\infty]$ is defined by
\[
	\int_\Omega f\, \dd \mu = \int_\Omega f(\omega)\,\mu(\dd\omega) := \sup\left\{ \int_\Omega g\, \dd \mu : \ g \text{ simple},\; 0 \leq g  \leq f \right\}.
\]
The function $f$ is said to be \emph{$\mu$-integrable} if its $\mu$-integral is finite.
\end{definition}

\begin{remark}
If $X$ is a nonnegative random variable on a probability space $(\Omega, \cF, \bbP)$, we call the integral the expectation value of $X$ and often write instead
\[
	\ex{X} := \int_\Omega X \dd \bbP.
\]
\end{remark}

For a measurable set $A \in \cF$, we use the following notation and definition for integration of $f$ over the set $A$
\[
	\int_A f\, \dd \mu := \int_\Omega \mathbf{1}_A\, f\, \dd \mu.
\]
If we denote by $f_A$ the restriction of $f$ to $A$, and by $\mu_A$ the restriction of $\mu$ to $\cF_A$, then 
\[
\int_A f_A\,\dd \mu_A = \int_A f\, \dd \mu.
\]
Similarly, if $f_A:(A, \cF_A) \to ([0,+\infty],\cB_{[0,+\infty]})$ is measurable, and $f$ is a measurable extension of $f_A$ to the whole of $\Omega$, then
\[
	\int_A f\,\dd \mu = \int_A f_A\, d \mu_A.
\]

\begin{lemma}[Properties of the Lebesgue integral of nonnegative functions]
	\label{pr:properties-integral-nonneg}
	Let $f$, $g$ be two nonnegative, measurable functions and $\alpha \geq 0$ be a constant.
	\begin{enumerate}
		\item (absolute continuity) If $B \in \cF$ satisfies $\mu(B) = 0$, then
		\[
		\int_{B} f\, \dd \mu = 0. 
		\]
		\item (monotonicity) If $f \leq g$, then
		\[
		\int_\Omega f\, \dd \mu \leq \int_\Omega g\, \dd \mu.
		\]
		\item (homogeneity)
		\[
			\alpha \int_\Omega f\, \dd \mu = \int_\Omega (\alpha f )\,\dd \mu.
		\]
	\end{enumerate}
\end{lemma}


\section{The monotone convergence theorem}
\label{sec:monotone_convergence}

We now arrive at the first convergence result telling us that the point-wise limit of monotone sequences of $\mu$-integrable functions is again $\mu$-integrable, highlighting the difference with Riemann integration.


\begin{theorem}[Monotone convergence theorem I]
	\label{th:monotone-convergence-I}
	Let $(\Omega, \cF, \mu)$ be a measure space. Let $f_n\colon\Omega \to [0,+\infty]$, $n \in \bbN$, be a sequence of nonnegative $(\cF,\cB_{[0,+\infty]})$-measurable functions, such that $f_n(\omega) \leq f_{n+1}(\omega)$ for all $\omega \in \Omega$ and $n \in \bbN$. Define the function 
	\[
		f(\omega) := \lim_{n \to \infty} f_n(\omega),\qquad \omega\in\Omega.
	\]
	Then $f$ is $(\cF,\cB_{[0,+\infty]})$-measurable and
	\[
	\lim_{n \to \infty} \int_\Omega f_n\, \dd \mu = \int_\Omega f\, \dd \mu.
	\]
\end{theorem}

\begin{proof}
	From the monotonicity of the integral, we immediately conclude that
	\[
	\limsup_{n \to \infty} \int_\Omega f_n\, \dd \mu \leq \int_\Omega f\, \dd \mu.
	\]
	Hence, we are left to show that
	\[
	\liminf_{n \to \infty } \int_\Omega f_n\, \dd \mu \geq \int_\Omega f\, \dd \mu.
	\]
	This is obvious if $\int_\Omega f\, \dd \mu = 0$, so we assume that $\int_\Omega f\, \dd \mu >0$.
	
	By the definition of the integral, for every $0<\varepsilon<L$, there exists a nonnegative simple function $g: \Omega \to \bbR$ such that $0\le g \leq f$ on $\Omega$ and
	\[
		\int_\Omega g\, \dd \mu > \int_\Omega f\, \dd \mu - \varepsilon.
	\]
	Because $g$ is simple, there exist an $N \in \bbN$, nonnegative constants $a_i \in (0,\infty)$ and disjoint, measurable sets $A_i \in \cF$ such that
	\[
		g = \sum_{i=1}^N a_i \mathbf{1}_{A_i}.
	\]
	Moreover, we find some $\delta>0$, such that
	\[
		g_\delta:= \sum_{i=1}^N (a_i-\delta)\mathbf{1}_{A_i}, 
	\]
	satisfies
	\[
		\int_\Omega g_\delta\,\dd \mu = \sum_{i=1}^N(a_i-\delta)\,\mu(A_i) \ge \int_\Omega f\,\dd\mu - \varepsilon. 
	\]
	
	Now define for $i \in \{ 1, \dots, N\}$ and $n \in \bbN$ the measurable set
	\[
	G_n^i := \Bigl\{ x \in A_i : \ f_n(x) \geq a_i - \delta \Bigr\}.
	\]
	Then, because $f_n \leq f_{n+1}$, we have $G_n^i \subset G_{n+1}^i$ for all $n \in \bbN$ and by the pointwise convergence of $f_n$ to $f$, we have
	\[
	\bigcup_{n=1}^\infty G_n^i = A_i,\qquad i=1,\ldots,N.
	\]
	Hence, by the continuity from below of measures
	\[
		\lim_{n \to \infty}\mu(G_n^i) = \mu(A_i).
	\]
	Since for every $n\in\bbN$,
		\[
		\int_\Omega f_n\,\dd\mu \ge \sum_{i=1}^N \int_{A_i} f_n\,\dd \mu \ge \sum_{i=1}^N \int_{A_i} f_n\,\dd \mu \ge \sum_{i=1}^N \int_{G_n^i} f_n\,\dd \mu \ge \sum_{i=1}^N (a_i-\delta)\,\mu(G_n^i),
	\]
	we find that
	\[
		\liminf_{n \to \infty} \int_\Omega f_n\, \dd \mu \ge  \liminf_{n \to \infty } \sum_{i=1}^N (a_i - \delta) \mu(G_n^i) = \int_\Omega g_\delta\,\dd\mu \ge \int_\Omega f\,\dd\mu - \varepsilon. 
	\]
	Because $\varepsilon>0$ was arbitrary, it follows that
	\[
	\liminf_{n \to \infty} \int_\Omega f_n \dd \mu \geq \int_\Omega f \dd \mu.\qedhere
	\]	
\end{proof}

\section{Intermezzo: Approximation by simple functions}
\label{sec:simple-approximation}

In this section, we will give a few explicit approximations to arbitrary measurable functions. 
First consider a nonnegative measurable function $f : (\Omega, \cF) \to ([0,\infty], \cB_{[0,\infty]})$. We define the function $(f_n)_{n\in\bbN}$ by setting $f_n(\omega) = 0$ if $f(\omega) = 0$,
\[
	f_n(\omega) := k\, 2^{-n}\qquad \text{if}\quad f(\omega)\in [k\,2^{-n},(k+1)\,2^{-n}),
\]
for some $k \in \bbN\cup\{0\}$ and setting $f_n(\omega) = +\infty$ if $f(\omega) = +\infty$. Note that we can write
\[
	f_n = +\infty \mathbf{1}_{\{f=+\infty\}} + \sum_{k=0}^\infty k\, 2^{-n} \mathbf{1}_{\{k\,2^{-n}\le f < (k+1)\,2^{-n}\}} ,\qquad n\in\bbN
\]
and easily deduce that $f_n$ is measurable for every $n\in\bbN$.

The advantage of the approximation $f_n$ to $f$ is most clearly seen when $f(\omega) < +\infty$ for all $\omega \in \Omega$. In this case, $f_n$ converges to $f$ uniformly: In fact
\[
|f_n(\omega) - f(\omega)| \leq 2^{-n}
\]
for all $n \in \bbN$ and all $\omega \in \Omega$.

The disadvantage of the approximation $f_n$ is that if $f$ is unbounded, the approximation $f_n$ is not simple. To remedy this, we truncate $f_n$ to get the approximation
\[
	[f]_n := \min( 2^n, f_n ).
\]
The function $[f]_n$ is indeed simple.

Both the approximations $f_n$ and $[f]_n$ are nondecreasing in $n$. Moreover, they are pointwise approximations of the functions $f$. In particular, the function $f_n$ converges uniformly to $f$ on the set where $f$ is finite, and the functions $[f]_n$ converge uniformly to $f$ on any set on which $f$ is bounded.


\section{Additivity of the Lebesgue integral of nonnegative functions}

A fundamental property that we need for a good notion of an integral is linearity.

\begin{lemma}[Additivity of the Lebesgue integral of nonnegative functions]
	\label{pr:additivity-integral-nonneg}
	Let $f$, $g$ be two nonnegative $(\cF,\cB_{[0,+\infty]})$-measurable functions. Then,
	\[
		\int_\Omega (f + g)\, \dd \mu = \int_\Omega f\, \dd \mu + \int_\Omega g\, \dd \mu.
	\]	
\end{lemma}
\begin{proof}
	For simple functions, the additivity of the integral is easy to check. Therefore, 
	\[
	\int_\Omega ([f]_n + [g]_n)\, \dd \mu = \int_\Omega [f]_n\, \dd \mu + \int_\Omega [g]_n\, \dd \mu\qquad\text{for every $n \in \bbN$.}
	\]
	We now take the limit on both sides of the equation. On one hand, the functions $[f]_n + [g]_n$ are increasing in $n$, and converge pointwise to $(f + g)$. By the monotone convergence theorem,
	\[
	\lim_{n \to \infty} \int_\Omega ([f]_n + [g]_n)\, \dd \mu = 
	\int_\Omega ( f + g)\, \dd \mu.
	\]
	On the other hand, by a limit theorem and Problem~\ref{prb:simple-approx-integral}, we know that 
	\[
	\lim_{n \to \infty} \left( \int_\Omega [f]_n\, \dd \mu + \int_\Omega [g]_n\, \dd \mu \right) = \int_\Omega f\, \dd \mu + \int_\Omega g\, \dd \mu.
	\]
	Therefore,
	\[
	\int_\Omega (f + g)\, \dd \mu = \int_\Omega f\, \dd \mu + \int_\Omega g\, \dd \mu.\qedhere
	\]
\end{proof}


\section{Integrable functions}\label{sec:integrable}

The next goal is to define the integral of functions $f$ that are not necessarily nonnegative. We can only do this if the integral of $|f|$ is finite.

\begin{definition}
	A $(\cF,\cB_\bbR)$-measurable function $f: \Omega \to \bbR$ is \emph{$\mu$-integrable} if 
	\[
	\int_\Omega |f|\, \dd \mu < +\infty.	
	\]
	For any function $f: \Omega \to \overline{\bbR}$, we define its \emph{positive part} $f^+$ and \emph{negative part} $f^-$ as
\[
	f^+(\omega) := \max( f(\omega), 0 ),\qquad 
	f^-(\omega) := - \min( f(\omega), 0 ) 
\]
It follows that $f = f^+ - f^-$ and $|f| = f^+ + f^-$.

	The \emph{Lebesgue integral} of a $\mu$-integrable function $f: \Omega \to \bbR$ is
	\[
		\int_\Omega f\,\dd \mu := \int_\Omega f^+\, \dd \mu - \int_\Omega f^-\, \dd \mu.
	\]
\end{definition}

As in the case of non-negative measurable functions, we have the following properties for $\mu$-integrable functions.

\begin{proposition}\label{prop:properties-integral}
	Let $f$, $g$ be two $\mu$-integrable functions and $\alpha \in \bbR$ be a constant.
	\begin{enumerate}
		\item (absolute continuity) If $B \in \cF$ satisfies $\mu(B) = 0$, then
		\[
		\int_{B} f\, \dd \mu = 0. 
		\]
		\item (monotonicity) If $f \leq g$ $\mu$-a.e., then
		\[
		\int_\Omega f \,\dd \mu \leq \int_\Omega g \,\dd \mu.
		\]
		\item (homogeneity) 
		\[
		\alpha \int_\Omega f \,\dd \mu = \int_\Omega (\alpha f )\,\dd \mu.
		\]
		\item (additivity)
		\[
		\int_\Omega (f + g)\, \dd \mu = \int_\Omega f 
		\,\dd \mu + \int_\Omega g \,\dd \mu.
		\]	
	\end{enumerate}
\end{proposition}

\begin{definition}
We say that a measurable function $f\colon(\Omega, \cF) \to (\bbR, \cB_\bbR)$ is integrable on a set $A \in \cF$ if the function $\mathbf{1}_A f$ is integrable on $\Omega$. Equivalently, we say that $f$ is integrable on $A$ if the restriction $f|_A$ is integrable on the measure space $(A, \cF_A, \mu|_A)$. 
\end{definition}

\section{Riemann vs Lebesgue integration}

A fundamental fact about the Lebesgue integral is its relationship with the Riemann integral, which allows us to make use of the integration techniques we know from Calculus and Analysis to compute the Lebesgue integral of a Lebesgue integrable function.

We state an important result, which we will not prove, but will be essential for computing integrals (cf.\ Appendix~\ref{chapter:appendix-2}). The first part of the result provides a full characterization of Riemann-integrable functions, while the second provides the means to compute Lebesgue integrals.

\begin{theorem}[Riemann vs Lebesgue]\label{thm:riem-leb}
	A bounded function $f\colon [a,b] \to \bbR$ on a compact set $[a,b]\subset\bbR$ is Riemann integrable if and only if it is continuous $\lambda$-almost everywhere.
	
	If a bounded function $f\colon [a,b] \to \bbR$ is Riemann integrable, then $f$ is $\cL$-measurable and $\lambda$-integrable. Moreover,
	\[
		\int_a^b f(x) \,\dd x = \int_{[a,b]} f \,\dd \lambda,
	\]
	where the left-hand side denotes the Riemann integral of $f$.
\end{theorem}

\begin{example}\label{ex:computation_lebesgue_integral}
	Let us determine the value $\displaystyle \int_\bbR \frac{1}{x^2+1}\,\lambda(\dd x)$.
	
	\noindent To do so, we set $g(x):= \frac{1}{x^2+1}\ge 0$ and let $g_n:= g\mathbf{1}_{[-n,n]}$. Then clearly, $g_n\to g$ point-wise monotonically. By the MCT, we have that
\[
	\lim_{n\to\infty} \int_\bbR g_n\,\dd\lambda = \int_\bbR g\,\dd\lambda.
\]
On the other hand, for every $n\ge 1$,
\[
	\int_\bbR g_n\,\dd\lambda = \int_{[-n,n]} g\,\dd\lambda = \int_{-n}^n g\,\dd x = \int_{-n}^n \frac{1}{1+x^2}\,\dd x = \arctan(n) - \arctan(-n),
\]
where the second equality follows from the fact that $g$ is continuous on the compact set $[-n,n]$ and from Theorem~\ref{thm:riem-leb}. Hence,
\[
	\int_\bbR g\,\dd\lambda=\lim_{n\to\infty} \int_\bbR g_n\,\dd\lambda = \frac{\pi}{2} + \frac{\pi}{2} = \pi,
\]
thus implying that $g$ is $\lambda$-integrable.
\end{example}

\section{Change of variables formula}
\label{sec:change-of-variables}

\begin{proposition}
Let $(\Omega, \cF, \mu)$ be a measure space and $(E,\cG)$ be a measurable space. Further, let $f\colon \Omega \to E$ and $h\colon E \to [0,+\infty]$ be $(\cF,\cG)$- and $(\cG,\cB_{[0,+\infty]})$-measurable maps respectively. Then,
\[
\int_\Omega h \circ f\,\dd \mu = \int_E h \, \dd (f_\# \mu).
\]
\end{proposition}

\begin{proof}
We first show the statement when $h$ is simple and nonnegative, i.e.,
\[
h = \sum_{i=1}^N a_i \mathbf{1}_{A_i}
\]
for some $N \in \bbN$, $a_i \in (0,\infty)$, and $A_i \in \cF$ mutually disjoint. Then 
\[
	h \circ f = \sum_{i=1}^N a_i  \mathbf{1}_{f^{-1}(A_i)}.
\]
It follows that
\[
\begin{split}
\int_\Omega h \circ f\, \dd \mu 
= \sum_{i=1}^N a_i \, \mu(f^{-1}(A_i)) 
= \sum_{i=1}^N a_i \, (f_\# \mu)(A_i) 
= \int_E h \, \dd (f_\# \mu),
\end{split}
\]
which shows the proposition in the case when $h$ is simple and nonnegative. 

We now turn to the case in which $h$ is a general, nonnegative measurable function. Note that $[h]_n \circ f$ is a nondecreasing sequence of functions, which converges pointwise to $h \circ f$. By the monotone convergence theorem,
\[
\int_\Omega h \circ f\, \dd \mu 
= \lim_{n \to \infty} \int_\Omega [h]_n \circ f\, \dd \mu 
= \lim_{n \to \infty} \int_E [h]_n \, \dd (f_\# \mu) 
= \int_E h \, \dd (f_\# \mu).\qedhere
\]
\end{proof}

\bigskip

As a direct consequence, we have the following proposition.
\begin{proposition}
Let $(\Omega, \cF, \mu)$ be a measure space and $(E,\cG)$ be a measurable space. Further, let $f\colon \Omega \to E$ and $h\colon E \to \bbR$ be $(\cF,\cG)$- and $(\cG,\cB_\bbR)$-measurable maps respectively. Then $h \circ f$ is integrable with respect to $\mu$ if and only if $h$ is integrable with respect to $f_\# \mu$, in which case,
\[
\int_\Omega h \circ f\, \dd \mu = \int_E h \, \dd (f_\# \mu).
\]
\end{proposition}




\section{Problems}

\begin{problem}
	Consider the measure space $(\bbN,2^{bbN},\mu)$, where $\mu$ is the counting measure on $\bbN$. Show that for any function $f:\bbN\to[0,+\infty]$,
	\[
		\int_{\bbN} f\,d\mu = \sum_{n\ge 1} f(n).
	\]
\end{problem}

\begin{problem}
	\label{prb:simple-approx-integral}
 Let $(\Omega, \cF, \mu)$ be a measure space and let $f\colon (\Omega,\cF) \to ([0,+\infty), \cB_{[0,+\infty)})$ be a nonnegative measurable function. Show that
\[
\int_\Omega f\, \dd \mu = \lim_{n \to \infty} \int_\Omega [f]_n\, \dd \mu.
\]	
\end{problem}

\begin{problem}\label{prb:measure}
	Let $(\Omega,\cF,\mu)$ be a measure space and suppose that $f$ is a non-negative $(\cF,\cB)$-measurable function such that $\int_\Omega f\,d\mu=1$. Define the set function $\nu_f\colon\cF\to[0,+\infty]$ by
	\[
		\nu_f(A):=\int_A f\,d\mu,\qquad \forall A\in\cF.
	\]
	\begin{enumerate}[label=(\alph*)]
		\item Show that $\nu_f$ is a probability measure on $(\Omega,\cF)$.
		\item Show that for all nonnegative $(\cF,\cB_{[0,+\infty]})$-measurable functions $g\colon\Omega\to [0,+\infty]$,
		\[
			\int_\Omega g\, d\nu_f = \int_\Omega g f\,d\mu.
		\]
		\textbf{Hint:} Start with simple functions and then approximate.
		\item Show that $g$ is $\nu_f$-integrable if and only if $g f$ is $\mu$-integrable, in which case
		\[
			\int_\Omega g\,d\nu_f = \int_\Omega g f\,d\mu.
		\]
	\end{enumerate}
\end{problem}


\begin{problem}
	Let $(\Omega,\cF,\mu)$ be a measure space and $\mu$ be a finite measure. Show that an $(\cF,\cB_\bbR)$-measurable function $f\colon\Omega\to \bbR$ is integrable if and only if
	\[
		\lim_{n\to\infty}\int_\Omega |f|\,\mathbbm{1}_{\{|f|\ge n\}}\,d\mu = 0.
	\]
\end{problem}


\begin{problem}[Continuity property of the integral]
	Let $(\Omega,\cF,\mu)$ be a measure space and $f$ be $\mu$-integrable. Show that for all $\varepsilon>0$ there exists $\delta>0$ such that
	\[
		\int_A |f|\,d\mu\le \varepsilon\quad \text{for all}\quad A\in\cF\quad \text{with}\quad \mu(A)<\delta.
	\]
	
\smallskip
	
	\noindent\textbf{Hint:} If $f$ is bounded, things are easy, so consider the set where $|f|$ is larger than some value and where $|f|$ is smaller than such value.
\end{problem}





\chapter{Convergence of integrals and functions}
\label{chapter:convergence}
  

\section{Convergence Theorems}

One of the motivations for developing a new theory of integration using measurable functions instead of continuous ones, was that we would be able to change limits and integrals more often. We have already seen an example of such a result in the monotone convergence theorem, Theorem~\ref{th:monotone-convergence-I}. However, this required that the sequence $f_n$ of function was monotone (i.e. non-decreasing) everywhere, which sounds a bit restrictive. That is why in this section we will use the monotone convergence theorem to derive other convergence results that have less restrictive conditions.

\subsection{Monotone convergence (continued)}

Theorem~\ref{th:monotone-convergence-I} states that if we have a sequence of measurable functions $(f_n)_{n \in \bbN}$ from some measure space $(\Omega, \cF, \mu)$ to $[0,+\infty]$ such that $f_n \le f_{n+1}$, then we could interchange limits and integration so that
\[
	\lim_{n \to \infty} \int_\Omega f_n \, \dd \mu = \int_\Omega \lim_{n \to \infty} f_n \, \dd \mu.
\]
It should be noted that the monotone properties requires that $f_n(\omega) \le f_{n+1}(\omega)$ for all $\omega \in \Omega$. However, from the definition of the Lebesgue integral we see that it is not affected by sets measure zero. Hence, we would expect that we can relax the monotone property to hold $\mu$-almost everywhere, i.e. the set where it does not hold has measure zero. This turns out to be the case, providing a slightly more general version of the monotone convergence theorem.

\begin{theorem}[Monotone convergence II]\label{thm:monotone_convergence_ii}
Let $(\Omega, \cF, \mu)$ be a measure space. Let $(f_n)_{n \ge 1}$ be a sequence of non-negative, measurable functions and let $f$ be a non-negative measurable functions such that the following holds $\mu$-almost everywhere
\begin{enumerate}
\item $f_n \le f_{n+1}$ for all $n \in \bbN$, and
\item $\lim_{n \to \infty} f_n = f$.
\end{enumerate}
Then
\[
	\lim_{n \to \infty} \int_\Omega f_n \,\dd \mu = \int_\Omega f \,\dd \mu.
\]
\end{theorem}

\begin{proof}
As you might have expected, the proof will use the first monotone convergence theorem. For this we first note that by assumption there exists a $N \in \cF$ with $\mu(N) = 0$ such that properties 1 and 2 from theorem statement hold for all $\omega \in \Omega \setminus N$. Now define the function $g_n(\omega) = \max_{1 \le k \le n} f_k(\omega)$. Then $g_n(\omega) \le g_{n+1}(\omega)$ holds for \emph{all} $\omega \in \Omega$. We further define $g(\omega) := \lim_{n \to \infty} g_n(\omega)$. Here comes the key observation. For every $\omega \in \Omega\setminus N$ it holds that $g_n(\omega) = f_n(\omega)$ and $g(\omega) = f(\omega)$. Moreover, since $\mu(N) = 0$ we have that
\[
	\int_\Omega f_n \, \dd \mu = \int_\Omega g_n \, \dd \mu \qquad \text{and} \qquad
	\int_\Omega f \, \dd \mu = \int_\Omega g \, \dd \mu.
\]
The result then follows by applying Theorem~\ref{th:monotone-convergence-I} to the functions $g_n$ and $g$.
\end{proof}

\subsection{Fatou's Lemma}

\begin{theorem}[Fatou's lemma]\label{thm:fatou}
Let $(\Omega, \cF, \mu)$ be a measure space. Let $(f_n)_{n \ge 1}$ be a sequence of non-negative, measurable functions and define
\[
	f := \liminf_{n\to \infty} f_n = \lim_{n \to \infty} \inf_{k \ge n} f_k.
\]
Then
\[
	\int_\Omega f\, \dd \mu \le \liminf_{n\to \infty} \int_\Omega f_n \,\dd \mu
\]
\end{theorem}

\begin{proof}
Our proof will use the monotone convergence theorem. There are however other proofs, based on first principles. 

Define the function $g_n(\omega) := \inf_{k \ge n} f_k(\omega)$ and note that by Lemma~\ref{lem:limit_operations_measurable_functions} $g_n$ are measurable. Moreover, $g_n(\omega) \le g_{n+1}(\omega)$ for all $\omega \in \Omega$ and $\lim_{n \to \infty} g_n(\omega) = f(\omega)$. Hence, Theorem~\ref{thm:monotone_convergence_ii} implies that
\[
	\lim_{n \to \infty} \int_\Omega g_n \, \dd \mu = \int_\Omega \lim_{n \to \infty} g_n \, \dd \mu
	= \int f \, \dd \mu. 
\]
Finally we observe that by definition $g_n \le f_n$ holds for all $n \in \bbN$ so that
\begin{align*}
	\int f \, \dd \mu &= \lim_{n \to \infty} \int_\Omega g_n \, \dd \mu \\
	&= \lim_{n \to \infty} \int_\Omega \inf_{k \ge n} f_k \, \dd \mu \\
	&\le \lim_{n \to \infty} \inf_{\ell \ge n} \int_\Omega  f_\ell \, \dd \mu\\
	&= \liminf_{n \to \infty} \int_\Omega  f_\ell \, \dd \mu.
\end{align*}
Here we used that $\inf_{k \ge n} f_k \le f_\ell$ for all $\ell \ge k$ and monotonicity of the integral (see Proposition~\ref{prop:properties-integral}).
\end{proof}

\subsection{Dominated Convergence}

Armed with Fatou's lemma we can now prove one of the most useful convergence results for Lebesgue integrals.

\begin{theorem}[Dominated convergence]\label{thm:dominated_convergence}
Let $(\Omega, \cF, \mu)$ be a measure space. Let $(f_n)_{n \ge 1}$ be a sequence of non-negative, measurable functions and let $f$ be a non-negative measurable function such that $f_n \to f$ point-wise $\mu$-almost everywhere. Moreover, assume there exists a non-negative $\mu$-integrable function $g : \Omega \to [0,\infty]$ such that $|f_n| \le g$ $\mu$-almost everywhere. Then
\[
	\lim_{n \to \infty} \int_\Omega f_n \,\dd \mu = \int_\Omega f \,\dd \mu.
\]
\end{theorem}

\begin{proof}
We will first prove the result for the case that both $|f_n| \le g$ and $f_n \to f$ hold everywhere.  

Consider the functions $f_n+g$, and note that $|f_n| \le g$ implies that these are non-negative. Fatou's lemma (Theorem~\ref{thm:fatou}) now implies that
\[
	\int_\Omega f + g \, \dd \mu \le \liminf_{n \to \infty} \int_\Omega f_n + g \, \dd \mu.
\] 
Using the additive property of the integral we get
\[
	\int_\Omega f \, \dd \mu + \int_\Omega g \, \dd \mu 
	\le \liminf_{n \to \infty} \int_\Omega f_n \, \dd \mu + \int_\Omega g \, \dd \mu.
\]
Since $\int_\Omega g \, \dd \mu < \infty$ this implies that
\[
	\int_\Omega f \, \dd \mu \le \liminf_{n \to \infty} \int_\Omega f_n \, \dd \mu.
\]
On the other hand, the condition $|f_n| \le g$ also implies that the functions $g - f_n$ are non-negative. Applying Fatou's lemma here yields
\[
	\int_\Omega g - f \, \dd \mu \le \liminf_{n \to \infty} \int_\Omega g - f_n \, \dd \mu.
\]
The additive property of integral now yields
\[
	\int_\Omega g \dd \mu - \int_\Omega f \, \dd \mu \le \int_\Omega g \, \dd \mu 
	+ \liminf_{n \to \infty} \int_\Omega - f_n \, \dd \mu,
\]
which implies that 
\[
	\int_\Omega f \, \dd \mu \ge - \liminf_{n \to \infty} \int_\Omega - f_n \, \dd \mu 
	= \limsup_{n \to \infty} \int_\Omega f_n \, \dd \mu.
\]
We thus conclude that
\[
	\lim_{n \to \infty} \int_\Omega f_n \, \dd \mu = \int_\Omega f \, \dd \mu.
\]

Now let us consider the general case. Then there exists a $N \in \cF$ such that $\mu(N) = 0$ and both $|f_n| \le g$ and $f_n \to f$ hold for every $\omega \in \Omega \setminus N$. Let us now define the following functions
\[
	\hat{f}_n(\omega) = \begin{cases}
		f_n(\omega) &\text{if } \omega \in \Omega \setminus N,\\
		0 &\text{else,}
	\end{cases}
	\quad 
	\hat{f}(\omega) = \begin{cases}
		f(\omega) &\text{if } \omega \in \Omega \setminus N,\\
		0 &\text{else,}
	\end{cases}
\]
and
\[
	\hat{g}(\omega) = \begin{cases}
		g(\omega) &\text{if } \omega \in \Omega \setminus N,\\
		0 &\text{else.}
	\end{cases}
\]

Then 
\[
	\int_\Omega \hat{f}_n \, \dd \mu = \int_\Omega f_n \, \dd \mu \quad \text{and} \quad
	\int_\Omega \hat{f} \, \dd \mu = \int_\Omega f \, \dd \mu
\]
Moreover, $\hat{f}_n \le \hat{g}$ and $\hat{f}_n \to \hat{f}$ hold \emph{everywhere}. So using the first part of the proof we have that
\[
	\lim_{n \to \infty} \int_\Omega f_n \, \dd \mu = \lim_{n \to \infty} \int_\Omega \hat{f}_n \, \dd \mu
	= \lim_{n \to \infty} \int_\Omega \hat{f} \, \dd \mu = \lim_{n \to \infty} \int_\Omega f \, \dd \mu.
\]
\end{proof}

\begin{example}
Consider the sequence of functions $f_n(x) = \frac{n \sin(x/n)}{x(x^2+1)}$. We will use dominated convergence to determine $\lim_{n \to \infty} \int_\bbR f_n \, \dd \lambda$. Define $g(x) = \frac{1}{x^2 +1}$ and note that 
\[
	f_n(x) = \frac{\sin(x/n)}{x/n} g(x).
\]
Note that $|\sin(y)| \le |y|$ holds for all $y > 0$ and that for every $x$ we have that $\lim_{n \to \infty} \frac{\sin(x/n)}{x/n} = 1$. We thus conclude that $|f_n(x)| \le g(x)$ and $f_n \to g(x)$ holds for all $x \in \bbR \setminus \{0\}$. Since the set $\{0\}$ has Lebesgue measure zero, all the conditions of Theorem~\ref{thm:dominated_convergence} are satisfied. Hence (see Example~\ref{ex:computation_lebesgue_integral})
\[
	\lim_{n \to \infty} \int_\bbR \frac{n \sin(x/n)}{x(x^2+1)} \lambda(\dd x) 
	= \int_\bbR \frac{1}{x^2 +1} \lambda(\dd x) = \pi.
\]
\end{example}

\section{Convergence of random variables}

Consider a sequence $(X_n)_{n \ge 1}$ of random variables. Similar to the setting of real numbers, we would like to have a notion of convergence of this sequence. In other words, we would like to say that $X_n \to X$ where $X$ is a different random variable. It turns out that there are different ways to define the concept of convergence of random variables. In this section, we will discuss three of them: convergence in distribution, convergence in probability, and almost sure convergence. While the last one has a more straightforward definition (see ??) the other two rely on a notion of convergence of probability measures, which we will discuss first.

\subsection{Weak convergence of probability measures}

\begin{definition}
Let $(\mu_n)_{n \ge 1}$ and $\mu$ be probability measures on $(\bbR, \cB_\bbR)$. We say that $\mu_n$ \emph{converges weakly} to $\mu$ if for every continuous bounded function $f : \bbR \to \bbR$ it holds that
\[
	\int_\bbR f \, \dd \mu_n \to \int_\bbR f \, \dd \mu
\]
If this is the case we write $\mu_n \Rightarrow \mu$.
\end{definition}

The definition of weak convergence ask us to verify the convergence of the $\mu_n$ integral of $h$ for any continuous bounded function $h$. In some cases that can be cumbersome task. Hence it would be helpful if we would have some equivalent conditions for weak convergence. The beauty here is that there are many equivalent definitions. They are often summarized in what is known as the Portmanteau lemma (or theorem). We provide one version of it below.

We will first prove an important technical lemma, needed for the proof of this theorem. For any function $h : \bbR \to \bbR$ we denote by $\cC_h \subset \bbR$ the set of continuity points of $h$, i.e. the set of all points $x\in \bbR$ at which $h$ is continuous. 

The following technical lemma allows us to approximates measurable functions by continuous ones, with arbitrary precision in terms of the integrals.

\begin{lemma}
Let $\mu$ be a probability measure on $(\bbR, \cB_\bbR)$ and $h : \bbR \to \bbR$ be a bounded measurable function with $\mu(\cC_h) = 1$. Then for every $\varepsilon > 0$, there exist continuous bounded functions $h^-_\varepsilon$ and $h^+_\varepsilon$ such that
\begin{enumerate}
\item $h^-_\varepsilon \le h \le h^+_\varepsilon$ and
\item $\int_\bbR h^+_\varepsilon \, \dd \mu - \int_\bbR h^-_\varepsilon \, \dd \mu < \varepsilon$.
\end{enumerate} 
\end{lemma}

\begin{proof}
For $k \in \bbN$ define the functions
\[
	g_k := \inf_{y \in \bbR} h(y) + k\|x-y\| \quad G_k := \sup_{y \in \bbR} h(y) - k\|x-y\|.
\]
We then observe that for any $k \in \bbN$, $g_k \le g_{k + 1}$, $G_{k+1} \le G_k$ and $g_k \le h \le G_k$. Hence
\[
	g_1 \le g_2 \le \dots \le h \le \dots \le G_2 \le G_1.
\]
Moreover, since for any fixed $x \in \bbR$ the sequences $(g_k(x))_{k \ge 1}$ and $(G_k(x))_{k \ge 1}$ are bounded we get that their limits as $k \to \infty$ exist and
\[
	\lim_{k \to \infty} g_k(x) \le h(x) \le \lim_{k \to \infty} G_k(x).
\]

We now claim that for every $k \ge 1$ the functions $g_k$ and $M_k$ are continuous and bounded. The last part follows from directly from the definitions. For the continuity we note that
\begin{align*}
	g_k(x) = \inf_{y \in \bbR} h(y) + k \|x-y\|
	\le \inf_{y \in \bbR} h(y) + k \|z-y\| +k \|x-z\|
	= g_k(z) + k\|x-z\|,
\end{align*}
which implies that $\|g_k(x) - g_k(y)\| \le k \|x-y\|$. A similar argument works for $G_k$.

Now, let $x \in \cC_h$, i.e. $x$ is a continuity point of $h$, and fix $\varepsilon > 0$. Then there exist a $\delta > 0$ such that $\|x-y\| < \delta$ implies that $\|h(x) - h(y)\| < \varepsilon$. If we then define
\[
	r =  \left \lceil \frac{h(x) - \inf_{z \in \bbR} h(z)}{\delta} \right \rceil,
\]
then
\begin{align*}
	\lim_{k \to \infty} g_k(x) &\ge g_r(x)\\
	&= \min\{ \inf_{\|x-y\| \ge \delta} h(y) + r\|x-y\| + \inf_{\|x-y\| < \delta} h(y) + r\|x-y\|\}\\
	&\ge \min\{h(x) - \varepsilon, \, \inf_{z \in \bbR} h(z) + (h(x) - \inf_{z \in \bbR} h(z))\delta/\delta\} \\
	&= h(x) - \varepsilon.
\end{align*}
Similarly, we get that $\lim_{k \to \infty} G_K(x) \le h(x) + \varepsilon$. Since $\mu(\cC_h) = 1$ this then implies that
\[
	\int_\bbR \lim_{k \to \infty} g_k \, \dd \mu  = \int_\bbR h \, \dd \mu 
	= \int_\bbR \lim_{k \to \infty} G_k \, \dd \mu. 
\]

Applying Theorem~\ref{thm:monotone_convergence_ii} to $g_k$ and to $-G_k$ we get that
\[
	\lim_{k \to \infty} \int_\bbR g_k \, \dd \mu = \int_\bbR h \, \dd \mu
	= \lim_{k \to \infty} \int_\bbR G_k \, \dd \mu.
\]
Finally, since $g_k$ is non-decreasing and $G_k$ is non-increasing, for every $\varepsilon$ there must exist an $K$ such that for all $k \ge K$
\[
	\int_\bbR (G_k - g_k) \, \dd \mu = \int_\bbR (G_k - h) \, \dd \mu + \int_\bbR (h - g_k) \, \dd \mu \le \varepsilon. 
\]
So we can take
\[
	h^-_\varepsilon := g_K \quad \text{and} \quad h^+_\varepsilon := G_K.
\]
\end{proof}

Recall that a set $A \subset \bbR^d$ is open if for every $x \in A$ there exists an $r > 0$ such that $B_x(r) \subset A$. In addition, a set $B\subset \bbR^d$ is called \emph{closed} if $B = \bbR^d \setminus A$ for some open set $A$.

For a set $A \subset \bbR$ denote by $\bar{A}$ the smallest closed set that contains $A$ and by $A^\circ$ the largest open set that is contained in $A$. The sets $\bar{A}$ and $A^\circ$ are called the \emph{closure} and \emph{interior} of $A$, respectively. We now define the \emph{boundary} of $A$ as $\partial A := \bar{A} \setminus A^\circ$. Given a measure $\mu$ on $(\bbR, \cB_\bbR)$, a set $A$ is called a \emph{$\mu$-continuity set} if $\mu(\delta A) = 0$.

\pagebreak

We can now state a list of equivalent definition for weak convergence of probability measures.

\begin{theorem}[Portmanteau Theorem]\label{thm:portmanteau}
Let $(\mu_n)_{n \ge 1}$ and $\mu$ be probability measures on $(\bbR, \cB_\bbR)$. Then the following statements are equivalent:
\begin{enumerate}
\item $\mu_n \Rightarrow \mu$.
\item $\int_\bbR h \, \dd \mu_n \to \int_\bbR h \, \dd \mu$ for all bounded measurable functions $h: \bbR \to \bbR$ with $\mu(\cC_h) = 1$.
\item $\int_\bbR g \, \dd \mu_n \to \int_\bbR g \, \dd \mu$ for all continuous bounded function $g : \bbR \to \bbR$ that are zero outside an interval $[-K,K]$ for some $K > 0$.
\item $\limsup_{n \to \infty} \mu_n(B) \le \mu(B)$ for all closed sets $B \subset \bbR$.
\item $\liminf_{n \to \infty} \mu_n(A) \ge \mu(A)$ for all open sets $A \subset \bbR$.
\item $\lim_{n \to \infty} \mu(C) = \mu(C)$ for all $\mu$-continuity sets $C$.
\end{enumerate}
\end{theorem} 

\begin{proof}
We will prove the following implication chain: 5 $\iff$ 4 $\Rightarrow$ 1 $\Rightarrow$ 2 $\Rightarrow$ 6 $\Rightarrow$ 4 and 1 $\iff$ 3. 

\textbf{5 $\iff$ 4:} This follows directly since every closed set $B$ is the complement of an open set $A$, i.e. $B = \bbR \setminus A$ and thus
\[
	\limsup_{n \to \infty} \mu_n(B) = \limsup_{n \to \infty} 1 - \mu_n(A) = 1 - \liminf_{n \to \infty} \mu_n(A).
\]

\textbf{4 $\Rightarrow$ 1:} Let $h$ be a continuous bounded function. Then, without loss of generality we may assume that $0 \ge h < 1$. Now fix some $k \in \bbN$ and define the following sets:
\[
	B_j := \{x \in \bbR \, : \, \frac{j}{k} \le f(x)\} \quad \text{for } j = 0, 1, \dots, k.
\]
Note that since $f$ is continuous these are closed sets. Also note that $\mu(B_0) = 1$ and $\mu(B_k) = 0$.

We further observe that $f(x) = \sum_{j = 1}^k f(x) \mathbf{1}_{B_{j-1} \cap B_j^c}$, where $B_j^c = \bbR \setminus B_j$. Hence we can now bound the integral $\mu \ast h$ from above and below as follows:
\begin{equation}\label{eq:portmanteau_1}
	\sum_{j = 1}^k \frac{j-1}{k} \mu(B_{j - 1} \cap B_j^c) \le \int_\bbR h \, \dd \mu \le \sum_{j = 1}^k \frac{j}{k} \mu(B_{j - 1} \cap B_j^c).
\end{equation}

Using that $B_{j-1} \supset B_{j}$ we get
\[
	\mu(B_{j - 1}) = \mu(B_{j-1} \cap B_j^c) + \mu(B_{j-1} \cap B_j) = \mu(B_{j-1} \cap B_j^c) + \mu(B_j)
\]
so that
\[
	\mu(B_{j-1} \cap B_j^c) = \mu(B_{j-1}) - \mu(B_j)
\]

Plugging this into the sum on the right hand side in Equation~\eqref{eq:portmanteau_1} we get
\begin{align*}
	\sum_{j = 1}^k \frac{j}{k} \mu(B_{j - 1} \cap B_j^c) &= \sum_{j = 1}^k \frac{j}{k} (\mu(B_{j-1}) - \mu(B_j))\\
	&= \frac{1}{k} \left(\mu(B_0) + \sum_{j = 1}^{k-1} (j+1) \mu(B_j) - \sum_{j = 1}^k \mu(B_j)\right) \\
	&= \frac{1}{k} \left(1 + \sum_{j = 1}^{k-1} \mu(B_j) - k \mu(b_k)\right)\\
	&= \frac{1}{k} + \frac{1}{k} \sum_{j = 2}^{k} \mu(B_j),
\end{align*}
where we used that $\mu(B_0) = 1$ and $\mu(B_k) = 0$.

In a similar fashion, the sum on the left hand side in Equation~\eqref{eq:portmanteau_1} equals
\[
	\frac{1}{k} \sum_{j = 1}^k \mu(B_j).
\]

We thus conclude that for any $k \ge 1$,
\begin{equation}
	\frac{1}{k} \sum_{j = 1}^k \mu(B_j) \le \int_\bbR h \, \dd \mu \le
	\frac{1}{k} + \frac{1}{k} \sum_{j = 2}^{k} \mu(B_j).
\end{equation}
Moreover, the same inequalities hold for the measures $\mu_n$.

Applying 4 we then get
\begin{align*}
	\limsup_{n \to \infty} \int_\bbR h \, \dd \mu_n
	&\le  \limsup_{n \to \infty} \left(\frac{1}{k} + \frac{1}{k} \sum_{j = 1}^k \mu_n(B_j) \right)\\
	&\le \frac{1}{k} + \frac{1}{k} \sum_{j = 1}^k \limsup_{n \to \infty} \mu_n(B_j)\\
	&\le \frac{1}{k} + \frac{1}{k} \sum_{j = 1}^k \mu(B_j) \\
	&\le \frac{1}{k} + \int_\bbR h \, \dd \mu.
\end{align*}
So that by taking $k \to \infty$ we obtain
\[
	\limsup_{n \to \infty} \int_\bbR h \, \dd \mu_n \le \int_\bbR h \, \dd \mu.
\]

Apply this conclusion to the function $-h$, which is also continuous and bounded, we get
\[
	\int_\bbR h \, \dd \mu \le \liminf_{n \to \infty} \int_\bbR h \, \dd \mu_n,
\]
from which it follows that $\lim_{n \to \infty} \int_\bbR h \, \dd \mu_n = \int_\bbR h \, \dd \mu$ for any bounded continuous function.

\textbf{1 $\Rightarrow$ 2:} Fix $\varepsilon > 0$ and let $h^-_\varepsilon$ and $h^+_\varepsilon$ be the function from Lemma [REF]. Then
\[
	\int_\bbR h \, \dd \mu \le \int_\bbR h^+_\varepsilon \, \dd \mu 
	= \int_\bbR h^+_\varepsilon \, \dd \mu - \int_\bbR h^-_\varepsilon \, \dd \mu
	+ \int_\bbR h^-_\varepsilon \, \dd \mu,
\]
which implies that
\[
	 \int_\bbR h \, \dd \mu -\varepsilon \le \int_\bbR h^-_\varepsilon \, \dd \mu.
\]

In a similar way we obtain that
\[
	\int_\bbR h^+_\varepsilon \, \dd \mu \le \int_\bbR h \, \dd \mu +\varepsilon.
\]

Now we employ condition 1 for the functions $h^-_\varepsilon$ and $h^+_\varepsilon$ to get
\begin{align*}
	\int_\bbR h \, \dd \mu -\varepsilon &\le \int_\bbR h^-_\varepsilon \, \dd \mu\\
	&= \lim_{n \to \infty} \int_\bbR h^-_\varepsilon \, \dd \mu_n\\
	&\le \liminf_{n \to \infty} \int_\bbR h \, \dd \mu_n \\
	&\le \limsup_{n \to \infty} \int_\bbR h \, \dd \mu_n\\
	&\le \int_\bbR h^+_\varepsilon \, \dd \mu_n\\
	&= \int_\bbR h^+_\varepsilon \, \dd \mu \le \int_\bbR h \, \dd \mu +\varepsilon.
\end{align*}
From this it follows that
\[
	\int_\bbR h \, \dd \mu -\varepsilon \le \liminf_{n \to \infty} \int_\bbR h \, \dd \mu_n
	\le \limsup_{n \to \infty} \int_\bbR h \, \dd \mu_n \le \int_\bbR h \, \dd \mu +\varepsilon.
\]
And since $\varepsilon > 0$ was arbitrary we conclude that
\[
	\liminf_{n \to \infty} \int_\bbR h \, \dd \mu_n = \limsup_{n \to \infty} \int_\bbR h \, \dd \mu_n,
\]
which then implies that $\int_\bbR h \, \dd \mu_n \to \int_\bbR h \, \dd \mu$.

\textbf{2 $\Rightarrow$ 6:} Let $C$ be a $\mu$-continuity set and consider the function $h(x) = \mathbf{1}_{C}$. Then clearly $h$ is measurable and bounded. Moreover, the function $h$ is discontinuous precisely on the boundary $\partial C$ and hence
\[
	\mu(\cC_h) =\mu(\bbR \setminus \partial C) = 1 - \mu(\partial C) = 1-0 = 1.
\]
Hence the function $h$ satisfies the conditions of 2 and thus
\[
	\mu_n(C) = \int_\bbR h \, \dd \mu_n \to \int_\bbR h \, \dd \mu = \mu(C).
\]


\textbf{6 $\Rightarrow$ 4:} Let $B$ be a closed set, take $\delta > 0$ and consider the sets
\[
	A_\delta = \{x \in \bbR \, : \, \|x - B\| < \delta\},
\]
where $\|x - B\| = \inf_{y \in B} \|x - y\|$ denotes the distance from $x$ to the set $B$. Note that $A_\delta$ is an open set in $\bbR$, and hence $A_\delta^\circ = A_\delta$.

Next we observe that $A_\delta \subset \{x \in \bbR \, : \, \|x-B\| \le \delta\}$ where the latter sets are closed. It then follows that
\[
	\partial A_\delta = \bar{A_\delta} \setminus A_\delta \subset \{x \in \bbR \, : \, \|x-B\| \le \delta\} \setminus A_\delta = \{x \in \bbR \, : \, \|x-B\| = \delta\}.
\]
It then follows that $\partial A_\delta \cap \partial A_{\delta^\prime} = \emptyset$ for all $\delta \ne \delta^\prime$. Since $\mu$ is a probability measure, there can be only a countable number of disjoint sets with positive measure. From this we conclude that there exists a sequence $(\delta_k)_{k \ge 1}$ with $\delta_k \to 0$ such that $\mu(\partial A_{\delta_k}) = 0$ for all $k \ge 1$. Let us write $B_k := A_{\delta_k}$. Then each $B_k$ is a $\mu$-continuity set, $B_k \supset B_{k + 1}$ and $B_k \downarrow B$ because $B$ is closed.

We then have that
\[
	\limsup_{n \to \infty} \mu_n(B) \le \limsup_{n \to \infty} \mu_n(B_k) = \mu(B_k),
\]
where the last equality is due to 6, which implies that $\mu_n(B_k) \to \mu(B_k)$.

Taking $k \to \infty$ now yields 4.

\textbf{1 $\iff$ 3:} The implication $1 \Rightarrow 3$ is trivial. So assume that $\int_\bbR g \, \dd \mu_n \to \int_\bbR g \, \dd \mu$ holds for all continuous bounded function that are zero outside some compact interval $[-K,K]$ and let $f : \bbR \to \bbR$ be a continuous bounded function with $\|f(x)\| \le M$ for all $x \in \bbR$. We will show that for any $\varepsilon > 0$
\[
	\|\int_\bbR f \, \dd \mu_n - \int_\bbR f \, \dd \mu\| \to \varepsilon,
\]
which then implies the result.

So let $\varepsilon > 0$ be fixed and observe that there exists an $\alpha > 0$ such that $\mu(\bbR \setminus [-\alpha, \alpha]) < \varepsilon/(2M)$. Also observe that we can define a non-negative continuous function $g$ such that $g = 1$ on $[-\alpha, \alpha]$ and $g = 0$ on $\bbR \setminus (-(\alpha+1),\alpha+1)$. Observer that $g$ is a non-negative continuous bounded functions that is zero outside the interval $[-(\alpha+1), \alpha+1]$ and thus we can apply 3.

We now have that
\begin{align*}
	\|\int_\bbR f \, \dd \mu_n - \int_\bbR fg \, \dd \mu_n\|
	&= \|\int_\bbR f (1-g) \, \dd \mu_n\| \le M \int_\bbR (1-g) \, \dd \mu_n \\
	&\le M \int_\bbR (1-g) \, \dd \mu_n = M(1-\int_\bbR g \, \dd \mu_n)
\end{align*}
Since the later term converges to $\int_\bbR g \, \dd \mu$ by our assumption we get that
\begin{align*}
	\limsup_{n \to \infty} \|\int_\bbR f \, \dd \mu_n - \int_\bbR fg \, \dd \mu_n\|
	&\le M \int_\bbR (1-g) \, \dd \mu \le M \mu(\bbR \setminus [-\alpha,\alpha]) < \frac{\varepsilon}{2}.
\end{align*}
The same conclusion holds true for $\|\int_\bbR f \, \dd \mu - \int_\bbR fg \, \dd \mu\|$.

If we now write
\begin{align*}
	\|\int_\bbR f \, \dd \mu_n - \int_\bbR f \, \dd \mu\| &\le \|\int_\bbR f \, \dd \mu_n - \int_\bbR fg \, \dd \mu_n\| + \|\int_\bbR f \, \dd \mu - \int_\bbR fg \, \dd \mu\|\\ &\hspace{10pt}+ \|\int_\bbR fg \, \dd \mu_n - \int_\bbR fg \, \dd \mu\|
\end{align*}
we see that the first two terms converge to $\varepsilon/2$ (by the computation above) while the term on the second line converges to zero by our assumption, since $fg$ is also a continuous bounded function that is zero outside the interval $[-(\alpha+1),\alpha+1]$. 
\end{proof}

\subsection{Convergence in distribution}

Convergence in distribution of a sequence $(X_n)_{n \ge 1}$ is defined as weak convergence of the corresponding probability measures.

\begin{definition}[Convergence in distribution]\label{def:convergence_distribution}
Let $(X_n)_{n \ge 1}$ and $X$ be random variables, possibly defined on different probability spaces with probability measures $\bbP_n$ and $\bbP$, respectively. We say that $X_n$ \emph{converges in distribution} to $X$ if
\[
	(X_n)_\# \bbP_n \Rightarrow X_\# \bbP.
\]
If this is the case write we $X_n \stackrel{d}{\rightarrow} X$.
\end{definition}

Note that convergence in distribution of random variables is simply defined as weak convergence of their push-forward measures on $(\bbR, \cB_\bbR)$. This might seem strange to those who have encountered the \emph{more standard} definition used in courses on Probability Theory. There $X_n \stackrel{d}{\rightarrow} X$ if and only if
\[
	\lim_{n \to \infty} F_n(t) = F(t)
\]
holds for all continuity points $t$ of $F$, with $F_n$ and $F$ denoting the cdfs of $X_n$ and $X$ respectively.

But fear not. It turns out that this definition is yet another equivalent statement for Definition~\ref{def:convergence_distribution}. 


\begin{lemma}\label{lem:convergence_distribution_cdfs}
Let $(X_n)_{n \ge 1}$ and $X$ be random variables and denote by, respectively, $F_n$ and $F$ their associated cdfs. Then
$X_n \stackrel{d}{\rightarrow} X$ if and only if
\[
	\lim_{n \to \infty} F_n(t) = F(t)
\]
holds for all continuity points $t$ of $F$.
\end{lemma}

\begin{proof}
See Problem~\ref{prb:convergence_probability_classic}.
\end{proof}



\subsection{Convergence in probability}

\begin{definition}[Convergence in probability]\label{def:convergence_probability}
Let $(X_n)_{n \ge 1}$ and $X$ be random variables that are defined on the \emph{same} probability space $(\Omega, \cF, \bbP)$ and define the random variable $Y_n := \|X_n - X\|$. We say that $X_n$ \emph{converges in probability} to $X$ if
\[
	(Y_n)_\# \bbP \Rightarrow 0_\# \bbP,
\] 
where $0$ denotes the constant function $\omega \mapsto 0$.

If this is the case we write $X_n \stackrel{\bbP}{\rightarrow} X$.
\end{definition}

\begin{remark}
Note that unlike convergence in distribution, the concept of convergence in probability requires all random variables to be defined on the same probability space. This requirement can be relaxed a bit by having each $X_n$ be defined on a different probability space $(\Omega_n, \cF_n. \bbP_n)$ but have $X$ be defined on each of these spaces. From this perspective we see that if we talk about convergence in probability to a constant $X_n \plim a \in \bbR$, then this is always true as constant random variables can be defined on any probability space.
\end{remark}

Recall that for a random variable $X$ on a probability space $(\Omega, \cF, \bbP)$ we wrote $\bbP(X \le t)$ as a short hand notation for $X_\# \bbP ((-\infty,t])$, i.e. the cdf of $X$ at $t$, and $\bbP(X > t)$ for $X_\# \bbP ((t,\infty))$, i.e. the ccdf of $X$ at $t$.

The following result relates the definition of convergence in probability to a version that is presented in most probability courses.

\begin{lemma}\label{lem:convergence_probability_classical}
Let $(X_n)_{n \ge 1}$ and $X$ be random variables define on the same probability space. Then $X_n \stackrel{\bbP}{\rightarrow} X$ if and only if
\[
	\lim_{n \to \infty} \bbP(\|X_n - X\| > \varepsilon) = 0 \quad \text{for every } \varepsilon > 0.
\]
\end{lemma}

\begin{proof}
See Problem~\ref{prb:convergence_probability_classic}
\end{proof}


The notion of convergence in probability is a stronger condition than convergence in distribution. In particular, the first statement implies the second. 

\begin{lemma}
Let $(X_n)_{n \ge 1}$ and $X$ be random variables such that $X_n \plim X$. Then $X_n \dlim X$.
\end{lemma}

\begin{proof}
We will use the form of from Lemma~\ref{lem:convergence_probability_classical}. Denote by $F_n$ and $F$ the cdfs of $X_n$ and $X$, respectively. We will also write $\bbP(X > t)$ Let $t$ be a continuity point of $F_X$ and fix some $\varepsilon > 0$. First we note that if $X_n \le t$ and $|X-X_n|\le \varepsilon$ then $X \le t + \varepsilon$. We thus obtain
\begin{align*}
	\bbP(X_n \le t)
	&= \bbP(\{X_n \le t\} \cap \{|X_n - X| \le \varepsilon\}) + \bbP(\{X_n \le t\} \cap \{|X_n - X|>\varepsilon\})\\
	&\le \bbP(X \le t + \varepsilon) + \bbP(|X_n - X| > \varepsilon).
\end{align*}
Taking the limsup on both sides yields
\[
	\limsup_{n \to \infty} \bbP(X_n \le t)
	\le \bbP(X \le t + \varepsilon) + \limsup_{n \to \infty} \bbP(\|X_n - X| > \varepsilon)
	= \bbP(X \le t + \varepsilon),
\]
since $X_n \plim X$ implies that
\[
	\limsup_{n \to \infty} \bbP(|X_n - X| > \varepsilon)
	= \lim_{n \to \infty} \bbP(|X_n - X| > \varepsilon) = 0.
\]
Since $t$ is a continuity point of $F_X$ it follows that 
\[
	\lim_{\varepsilon \downarrow 0} \bbP(X \le t + \varepsilon) = \bbP(X \le t),
\]
which implies that $\limsup_{n \to \infty} \bbP(X_n \le t) \le \bbP(X \le t)$.

To prove the result, it now suffices to show that $\bbP(X \le t) \le \liminf_{n \to \infty} \bbP(X_n \le t)$.
We shall do this in a way that is similar to the case with the limsup. First observe that $X \le t - \varepsilon$ and $|X_n - X| \le \varepsilon$ implies that $X_n \le t$. This way we get
\begin{align*}
	\bbP(X \le t - \varepsilon)
	&= \bbP(\{X \le t - \varepsilon\} \cap \{|X_n -X|\le \varepsilon\}) + \bbP(\{X \le t - \varepsilon\} \cap \{|X_n - X|>\varepsilon\})\\
	&\le \bbP(X_n \le t) + \bbP(|X_n - X| > \varepsilon).
\end{align*}
Taking the liminf on both sides gives
\[
	\bbP(X \le t - \varepsilon)
	\le \liminf_{n \to \infty} \bbP(X_n \le t),
\]
and using that $t$ is a continuity point of $F_X$ we conclude that 
\[
	\bbP(X \le t) \le \liminf_{n \to \infty} \bbP(X_n \le t).
\]
\end{proof}

While convergence in probability implies convergence in distribution, the other implication is not true in general (see Problem~\ref{prb:dlim_not_plim}). However, if $X$ is constant (deterministic instead of random) then both notions of convergence are equivalent.

\begin{lemma}{}\label{lem:dlim_constant_plim}
Let $(X_n)_{n \ge 1}$ be a sequence of random variables such that $X_n \dlim a$ for some constant $a \in \mathbb{R}$. Then we also have that $X_n \plim a$.
\end{lemma}

\begin{proof}
We will use the form of from Lemma~\ref{lem:convergence_probability_classical}. Fix some $\varepsilon > 0$. We have to show that
\[
	\lim_{n \to \infty} \bbP(|X_n-a|>\varepsilon) = 0.
\]
Let $B_\varepsilon(a)$ denote the open ball of radius $\varepsilon$ around $a$ and consider the compliment $B_\varepsilon(a)^c := \mathbb{R} \setminus B_\varepsilon(a)$, which is a closed set. We then have
\[
	\bbP(|X_n-a|>\varepsilon) \le \bbP(|X_n-a|\ge\varepsilon) = \bbP(X_n \in B_\varepsilon(a)^c). 
\]
In particular we have
\[
	\lim_{n \to \infty} \bbP(|X_n-a|>\varepsilon) \le \limsup_{n \to \infty} \bbP(|X_n-a|>\varepsilon)
	\le \limsup_{n \to \infty} \bbP(X_n \in B_\varepsilon(a)^c).
\]
Since $X_n \dlim a$, statement 3 in Theorem~\ref{thm:portmanteau} implies that 
\[
	\limsup_{n \to \infty} \bbP(X_n \in B_\varepsilon(a)^c)
	\le \limsup_{n \to \infty} \bbP(a \in B_\varepsilon(a)^c) = 0,
\]
because obviously $a \notin B_\varepsilon(a)^c$. Therefore we conclude that
\[
	\lim_{n \to \infty} \bbP(|X_n-a|>\varepsilon)
	\le \limsup_{n \to \infty} \bbP(a \in B_\varepsilon(a)^c) = 0
\]
which implies that $\lim_{n \to \infty} \bbP(|X_n-a|>\varepsilon) = 0$.
\end{proof}


\subsection{Almost-sure convergence}

The final notion of convergence we will discuss is \emph{almost-sure convergence}, which looks much more natural than the previous two notions and requires less notation.

\begin{definition}[Almost-sure convergence]\label{def:almost_sure_convergence}
Let $(X_n)_{n \ge 1}$ and $X$ be random variables defined on the same probability space $(\Omega, \cF, \bbP)$. We say that $X_n$ \emph{converges almost-surely} to $X$ if
\[
	\bbP(\{\omega \in \Omega \, : \, \lim_{n \to \infty} X_n(\omega) = X(\omega)\}) = 1.
\] 
In this case we write $X_n \aslim X$.
\end{definition} 

The definition of almost-sure convergence says that the probability that the set for which $X$ is \emph{not} the limit of $X_n$ must have probability zero. This is why it is also often referred to as \emph{convergence with probability $1$}.

There is a different way to characterize \emph{almost-sure} convergence which is often useful. This requires the concept of \emph{infinitely often}.

\begin{definition}[Infinitely often]
Let $(\Omega, \cF, \bbP)$ be a probability space and consider a sequence $(A_n)_{n \ge 1}$ of measurable sets. We then define the event $\{A_n \text{ i.o.}\}$ ($A_n$ happens infinitely often) as
\[
	\{A_n \text{ i.o.}\} := \bigcap_{k = 1}^\infty \bigcup_{n \ge k} A_n.
\]
\end{definition}

\begin{lemma}\label{lem:almost_sure_alternative}
Let $(X_n)_{n \ge 1}$ and $X$ be random variables defined on the same probability space $(\Omega, \cF, \bbP)$. Then
\[
	X_n \aslim X \iff \bbP(\|X_n - X\| > \varepsilon \text{ i.o.}) = 0 \quad \text{for all } \varepsilon > 0.
\]
\end{lemma}

\begin{proof}
Write $A_n(\varepsilon) :=  \{\|X_n - X\| > \varepsilon\}$ and $A = \{\omega \in \Omega \, : \, \lim_{n \to \infty} X_n(\omega) = X(\omega)\}$. We first observe that
\[
	\Omega\setminus A = \bigcup_{m \in \bbN} \{A_n(1/m) \text{ i.o.}\}.
\]

Now suppose that $X_n \aslim X$ and let $\varepsilon > 0$. Then $\bbP(A) = 1$ and there exist a $m \in \bbN$ such that $\{A_n(\varepsilon) \text{ i.o.}\} \subset \{A_n(1/m) \text{ i.o}\}$. Thus
\[
	\bbP(\{A_n(\varepsilon) \text{ i.o.}\}) \le \bbP(\{A_n(1/m) \text{ i.o.}\}) \le \bbP(\bigcup_{m \in \bbN} \{A_n(1/m) \text{ i.o.}\}) = \bbP(\Omega \setminus A) = 0.
\]

For the other implication suppose that $\bbP(\{A_n(\varepsilon) \text{ i.o.}\}) = 0$ for all $\varepsilon > 0$. Then clearly the same holds for all $\varepsilon = 1/m$ with $m \in \bbN$. Hence
\[
	\bbP(\Omega\setminus A) = \bbP(\bigcup_{m \in \bbN} \{A_n(1/m) \text{ i.o.}\}) \le \sum_{m \in \bbN} \bbP(\{A_n(1/m) \text{ i.o.}\}) = 0,
\]
which implies that $\bbP(A) = 1$.
\end{proof}

While this notion of convergence looks very natural, it turn out it is the strongest of the three notions. In practice proving almost-sure convergence is often much harder than proving convergence in probability or distribution.

\begin{lemma}\label{lem:almost_sure_implies_probability}
Let $(X_n)_{n \ge 1}$ and $X$ be random variables such that $X_n \aslim X$. Then $X_n \plim X$.
\end{lemma}

\begin{proof}
See Problem~\ref{prb:almost_sure_implies_probability}.
\end{proof}

\section{Problems}

\begin{problem}\label{prb:reverse_fatou}
Prove the reverse Fatou lemma:

\textit{
Let $(\Omega, \cF, \mu)$ be a measure space. Let $(f_n)_{n \ge 1}$ and $f$ be non-negative, measurable functions such that $f_n \le f$ and $\int_\Omega f \, \dd \mu <\infty$. Then
\[
	\limsup_{n\to \infty} \int_\Omega f_n \dd \mu \le \int_\Omega \limsup_{n\to \infty} f_n \dd \mu.
\]
}
\end{problem}

\begin{problem}\label{pb:DCT-parametric-function}
	Let $(\Omega,\mathcal{F},\mu)$ be a measure space. Assume that $f:\Omega\times(a,b)\to\bbR$ and that $\omega\mapsto f(\omega,t)$ is integrable with respect to $\mu$ for all fixed $t\in(a,b)$. Suppose there exists a $\mu$-integrable function $g$ such that $|f(\omega,t)|\le g(\omega)$ for all $t\in(a,b)$ and all $\omega\in\Omega$. 
	\begin{enumerate}[label={(\alph*)}]
		\item Fix $t_0\in(a,b)$. Show that if $\lim_{t\to t_0} f(\omega,t)=f(\omega,t_0)$ for all $\omega\in\Omega$, then
		\[
			\lim_{t\to t_0}\int_\Omega f(\omega,t)\,\mu(d\omega) = \int_\Omega f(\omega,t_0)\,\mu(d\omega).
		\]
		\item Deduce from (a) that if $f(\omega,\cdot)$ is continuous for all $\omega$, then the map
		\[
			(a,b)\ni t\mapsto F(t):=\int_\Omega f(\omega,t)\,\mu(d\omega)\quad\text{is continuous.}
		\]
	\end{enumerate}
\end{problem}

\begin{problem}
		Let $(\Omega,\mathcal{F},\mu)$ be a measure space. Assume that $f:\Omega\times(a,b)\to\bbR$ and that $\omega\mapsto f(\omega,t)$ is integrable with respect to $\mu$ for all fixed $t\in(a,b)$. Suppose $\partial f/\partial t$ exists on $(a,b)$ for all $\omega\in \Omega$, i.e.\ for every fixed $\omega\in\Omega$,
		\[
			\frac{\partial f}{\partial t}(\omega, t_0):=\lim_{t\to t_0} \frac{f(\omega,t)-f(\omega,t_0)}{t-t_0} \quad\text{exists for all $t_0\in(a,b)$}.
		\]
		Furthermore, suppose that there is a $\mu$-integrable function $g$ such that $|\partial f/\partial t|(\omega,t)\le g(\omega)$ for all $t\in(a,b)$ and all $\omega\in\Omega$. Show that the map
		\[
			(a,b)\ni t\mapsto F(t) = \int_\Omega f(\omega,t)\,\mu(d\omega)\quad\text{is differentiable},
		\]
		and the following equality holds:
		\[
			\frac{d}{dt}\int_\Omega f(\omega,t)\,\mu(d\omega) = \int_\Omega \frac{\partial f}{\partial t}(\omega,t)\,\mu(d\omega).
		\]
		
		\smallskip
		
		\noindent\textbf{Hint:} Make the following steps:
		\begin{enumerate}[label={(\arabic*)}]
			\item Show that $(\partial f/\partial t)(\cdot,t)$ is $(\mathcal{F},\mathcal{B})$-measurable and integrable for all $t\in(a,b)$.
			\item Show that for $t_0\in(a,b)$ arbitrary
			\[
				\left|\frac{f(\omega,t)-f(\omega,t_0)}{t-t_0}\right| \le g(\omega)\quad\text{for any $t\in(a,b)$, $t\ne t_0$ and all $\omega\in\Omega$.}
			\]
			\item Conclude with the help of Problem~\ref{pb:DCT-parametric-function}.
		\end{enumerate}
\end{problem}

\begin{problem}
	Compute the following limits and justify the computation
	\begin{align*}
		& \lim_{n\to\infty} \int_0^1 \frac{1 + n x^2}{(1 + x^2)^n}\,\dd x\,,\qquad\lim_{n\to\infty} \int_0^\infty \frac{x^{n-2}}{1+ x^n}\cos\left(\frac{\pi x}{n}\right) \dd x\,.
	\end{align*}
\end{problem}

\begin{problem}[Generalized DCT] Prove the following generalization of DCT:

\textit{
	Let $(\Omega,\mathcal{F},\mu)$ be a measure space. Assume that $f_n$, $g_n$, $f$ and $g$ are $\mu$-integrable functions satisfying
	\begin{enumerate}[label={(\roman*)}]
		\item $f_n\to f$ and $g_n\to g$ $\mu$-almost everywhere,
		\item $|f_n|\le g_n$ for all $n\in\bbN$ and $\int_\Omega g_n\,\dd\mu \to \int_\Omega g\,\dd\mu$ as $n\to \infty$.
	\end{enumerate}
	Then also $\int_\Omega f_n\,\dd\mu \to \int_\Omega f\,\dd\mu$ as $n\to\infty$.	
}
\end{problem}


\begin{problem}\label{prb:dlim_not_plim}
Give an example (with proof) of a sequence of random variables that converge in distribution but not in probability.
\end{problem}

\begin{problem}\label{prb:convergence_probability_classic}
Prove Lemma~\ref{lem:convergence_probability_classical}.
\end{problem}

\begin{problem}\label{prb:convergence_distribution}
The goal of this problem is to prove Lemma~\ref{lem:convergence_distribution_cdfs}. That is
\[
	X_n \stackrel{d}{\rightarrow} X \iff F_n(t) \to F(t) \quad \text{for all } t \in \cC_F,
\]
where $F_n$ and $F$ denote the cdfs of the random variables $X_n$ and $X$, respectively.

Write $\mu_n = (X_n)_\# \bbP_n$ and $\mu = X_\# \bbP$ and note that for any measurable function $f$, $\bbE[f(X_n)] = \int f \, \dd \mu_n$ and $\bbE[f(X)] = \int f \, \dd \mu$.

We will first prove the $\Rightarrow$ implication. 
\begin{enumerate}[label={(\alph*)}]
\item Let $t \in \bbR$. Find a measurable function $h_t$, such that $F_n(t) = \mu_n \ast h_t$ and $F(t) = \mu \ast h_t$.
\item Show that $\mu(\cC_{h_t}) = 1$.
\item Prove the $\Rightarrow$ implication. 
\end{enumerate}

For the other implication $\Leftarrow$ we will use the Portmanteau Theorem. Let $g$ be a bounded continuous function that is zero outside some interval $[-K,K]$, for some $K > 0$. You may use the fact that any continuous function that is non-zero on a bounded and closed set is uniformly continuous, i.e. for every $\varepsilon > 0$ there exist a $\delta > 0$ such that $\|x-y\| < \delta$ implies that $\|g(x) - g(y)\| < \varepsilon$. 
\begin{enumerate}[label={(\alph*)}]
\setcounter{enumi}{3}
\item Construct a partition of $C$ into $T$ intervals $I_i = (a_i, b_i]$ such that for each $i \le T$ and $x, y \in I_i$ it holds that $\|x - y\| < \delta$.
\end{enumerate}

We will now define an approximate function
\[
	\hat{g}(x) = \sum_{i = 1}^T \eta_i \mathbf{1}_{I_i},
\]
where $\eta_i = h(b_i)$.

\begin{enumerate}[label={(\alph*)}]
\setcounter{enumi}{4}
\item Show that there exists sequences $(\gamma_i)_{1 \le i \le m}$ and $(t_i)_{1 \le i \le m}$ such that
\[
	\hat{g}(x) = \sum_{i = 1}^m \gamma_i \mathbf{1}_{(-\infty, t_i]}.
\]
\item Prove that $\lim_{n \to \infty} \mu_n \ast \hat{g} = \mu \ast \hat{g}$. [Hint: Use the assumption $F_n(t) \to F(t)$ for all $t \in \cC_F$.]
\item Prove that $\mu_n \ast g \to \mu \ast g$. [Hint: First use the previous result to show that $\|\mu_n \ast g(A) - \mu \ast g(A)\| \to 2\varepsilon$ by adding and subtracting $\mu_n \ast \hat{g}$ and $\mu \ast \hat{g}$.]
\item Conclude that $X_n \dlim X$.
\end{enumerate}
\end{problem}

\begin{problem}\label{prb:almost_sure_implies_probability}
Prove Lemma~\ref{lem:almost_sure_implies_probability}. [Hint: use the alternative definition of Lemma~\ref{lem:almost_sure_alternative} and reverse Fatou from Problem~\ref{prb:reverse_fatou}.]
\end{problem}


\chapter{$L^p$-spaces}
\label{chapter:lp_spaces}
Let $(\Omega,\cF,\mu)$ be a measure space and $p\in[1,+\infty)$. Throughout this chapter, we denote
\[
	\|f\|_p := \left(\int_\Omega |f|^p\,\dd\mu\right)^{1/p}\qquad \text{for any measurable function $f:\Omega\to\bbR$}.
\]
For $p=+\infty$, we set
\[
\|f\|_{\infty} := \esssup \bigl\{ |f(\omega)| : \ \omega \in \Omega \bigr\} = \inf\bigl\{t \in [0,\infty) : \ \mu(\{ |f| > t\}) = 0 \bigr\}.
\]

\section{The H\"older inequality}

\begin{proposition}[H\"older's inequality] Let $(\Omega,\cF,\mu)$ be a measure space and $p,q\in[1,+\infty]$ be conjugate exponents, i.e.\ $\frac{1}{p}+\frac{1}{q}=1$. Then,
\[
	\|f g \|_1 \leq \|f\|_p\|g\|_q\qquad\text{for all measurable functions $f,g:\Omega\to\bbR$.}
\]
\end{proposition}

\begin{proof}
If the right-hand side is $+\infty$, there is nothing to prove. 

Now we will see a very important trick in proving inequalities like this. We note that it is enough to show the inequality for the case in which
\[
\int_\Omega |f|^p \,\dd \mu = \int_\Omega |g|^q \,\dd \mu = 1.
\]
By Young's inequality for conjugate exponents $p,q\in(1,+\infty)$,
\[
	ab \le \frac{1}{p}a^p + \frac{1}{q} b^q\qquad\text{for any $a,b\in[0,+\infty)$,}
\] 
we have for every $\omega \in \Omega$, that
\[
|f(\omega) g(\omega)| \leq \frac{1}{p}|f(\omega)|^p  + \frac{1}{q} |g(\omega)|^q.
\]
Hence
\[
\int_\Omega |fg|\,\dd \mu \leq \frac{1}{p} \int_\Omega |f|^p \,\dd \mu + \frac{1}{q} \int_\Omega |g|^q \,\dd \mu = 1.
\]
For the case $p=1$, $q=+\infty$, we easily get
\[
	\int_\Omega |fg|\,\dd\mu \le \int_\Omega |f|\|g\|_\infty\,\dd\mu = \|f\|_1\|g\|_\infty.\qedhere
\]
\end{proof}

\section{The Minkowski inequality}

\begin{proposition}
	Let $(\Omega,\cF,\mu)$ be a measure space and $p\in[1,+\infty]$ be conjugate exponents. Then the `triangle inequality' holds:
\[
	\|f + g\|_p \leq \|f\|_p + \|g \|_p\qquad \text{for all measurable functions $f,g:\Omega\to\bbR$}.
\]
\end{proposition}
\begin{proof}
As before, if the right-hand side is $+\infty$, then there is nothing to prove. Now suppose that $\|f\|_p,\|g\|_p<+\infty$. Then from the binomial formula for $p
\in[1,+\infty)$
\[
	|a+b|^p \le 2^{p-1}\bigl(|a|^p + |b|^p\bigr),
\]
we have that
\[
	\int_\Omega |f+g|^p\,\dd\mu \le 2^{p-1}\left(\int_\Omega |f|^p\,\dd\mu + \int_\Omega |g|^p\,\dd\mu\right),
\]
and hence $\|f+g\|_p<+\infty$. Next,
\[
\begin{split}
\|f + g\|^p_{p}
&= \int_\Omega |f + g|^p \,\dd \mu \\
&= \int_\Omega |f + g| |f + g|^{p-1} \,\dd \mu \\
&\leq \int_\Omega (|f| + |g|) |f + g|^{p-1} \,\dd \mu \\
&= \int_\Omega |f| |f+g|^{p-1} \,\dd \mu + \int_\Omega |g | |f+g|^{p-1} \,\dd \mu.
\end{split}
\]
Now we apply H\"older's inequality (with exponents $p$ and $p/(p-1)$) on both terms to obtain
\[
\|f+g\|^p_{L^p} \le \left(\int_\Omega |f|^p \dd \mu\right)^{1/p}  \|f+ g\|_p^{{p-1}} + \left(\int_\Omega |g|^p \dd \mu\right)^{1/p}  \|f+ g\|_p^{{p-1}}.
\]
Finally, we divide both sides by $\|f + g\|_p^{p-1}$ and find
\[
\|f+g\|_p \leq \|f\|_p + \|g\|_p.
\]
As for the case $p=+\infty$, we use the triangle inequality to obtain $|f+g|\le |f| + |g|$, and hence,
\[
	|f(\omega)+g(\omega)| \le \|f\|_\infty + \|g\|_\infty\qquad\text{for $\mu$-almost every $\omega\in\Omega$}.
\]
Taking the essential supremum then yields the required inequality.
\end{proof}

\section{Normed and semi-normed vector spaces}
\label{se:normed-and-seminormed-spaces}

Recall that a norm $\| \cdot \|$ on a vector space $V$ is a function $V \to [0,\infty)$ such that
\begin{enumerate}
\item $\|v\|= 0 \Leftrightarrow v = 0$ for all $v \in V$
\item $\|\lambda v \| = |\lambda| \|v\|$ for all $\lambda \in \bbR$ and $v \in V$
\item $\|v + w\| \leq \|v\| + \|w \|$ for all $v, w \in V$.
\end{enumerate}
If only the last two properties hold, $\|. \|$ is instead called a \emph{seminorm}.

\medskip

Let $(V, \|\cdot \|)$ be a semi-normed space. We say that a sequence $(v_n)_{n\in\bbN} \subset V$ is a Cauchy sequence if for every $\epsilon > 0$ there exists an $N \in \bbN$ such that for all $m, n \geq N$,
\[
\| v_m - v_n \| < \epsilon.
\]
We say that a semi-normed space is \emph{complete}, if and only if every Cauchy sequence converges, that is, for every Cauchy sequence $(v_n)_{n\in\bbN} \subset V$ there exists a $v \in V$ such that
\[
\lim_{n \to \infty} \|v_n - v\| = 0.
\]

To every semi-normed space $(V, \|\cdot\|)$ one can associate a normed linear space in a standard way. 
One defines the equivalence relation $\sim$ by $v \sim w$ if and only if $\|v - w\| = 0$. Denote by $W$ the linear space of equivalence classes. One defines a norm on equivalence classes $[v]$ and $[w]$ in $W$ by $\|[w]-[v]\| = \|w - v\|$. If $(V, \|\cdot\|)$ is a complete semi-normed space, then $W$ is a \emph{Banach space}, which is a complete normed linear space.

\medskip

We have seen in Section~\ref{chapter:integration} that the set of $\mu$-integrable functions form a vector space (over $\bbR$). For $p \in [0,+\infty)$, we define the vector space $\bbL^p$ of integrable functions $f$ for which 
\[
	\| f \|_{p} < +\infty.
\]
By the Minkowski inequality, $\|\cdot\|_{p}$ is a seminorm on $\bbL^p$ for every $p \in [1,\infty]$.

Clearly, the seminorm $\|. \|_p$ is not a norm on $\bbL^p$: indeed $\|f - g\|_p=0$ if and only if $f(\omega) = g(\omega)$, for $\mu$-almost every $\omega \in \Omega$. 
We follow the standard construction described in Section \ref{se:normed-and-seminormed-spaces} to create an associated normed linear space. We say that $f \sim g$ if and only if $f$ is equal to $g$ $\mu$-almost everywhere. 
We denote by $L^p$ the vector space of equivalence classes
\[
L^p := \bbL^p/\sim.
\]

\section{Completeness of $L^p$-spaces}

\begin{theorem}[Completeness of $L^p$ spaces]
The normed linear space $L^p$ is complete, and is thus a Banach space, for every $p \in [1,+\infty]$.	
\end{theorem}

\begin{proof}
First let $p \in [1,\infty)$ and let $(f_n)_{n\in\bbN}$ be a Cauchy sequence. The trick is to select a subsequence $(f_{n_k})_{k\in\bbN}$ such that 
\[
\| f_{n_{k+1}} - f_{n_k} \|_{L^p(\Omega)} < 4^{-k-1}.
\]
For ease of notation we set $g_k := f_{n_{k}}$.
Note that by a telescoping argument, for all $\ell \geq k$,
\[
\| g_\ell - g_k \|_{L^p(\Omega)} < 4^{-k}.
\]
Then
\[
\mu \left(\left\{ \omega \in \Omega : \ |g_{k+1}(\omega) - g_k(\omega) | 
> 2^{-k}  \right\} \right) < \frac{1}{2^{-k p} }\| g_{k+1} - g_k \|_{L^p(\Omega)}^p <2^{-kp}.
\]
In particular, by the Borel-Cantelli Lemma (cf.\ Lemma~\ref{lem:Borel-Cantelli}), for $\mu$-a.e. $\omega \in \Omega$, there is an $N_\omega \in \bbN$ such that
\[
|g_{k+1}(\omega) - g_k(\omega)| \leq 2^{-k}\qquad\text{for all $k > N_\omega$}.
\]
For such $\omega\in\Omega$, the sequence $(g_k(\omega))_{k\in\bbN}$ is Cauchy. So by the completeness of $\bbR$, a limit exists, which we call $f(\omega)$.

By Fatou's Lemma,
\[
\| g_k - f\|_p \leq \liminf_{\ell \to \infty} \| g_k - g_\ell \|_p \leq 4^{-k}.
\]
To see that this implies that $f_n$ converges to $f$ in $L^p$, we take an arbitrary $\epsilon > 0$. 
Since $f_n$ is a Cauchy sequence, there exists an $N \in \bbN$ such that for all $m, n \geq N$,
\[
\|f_{n} - f_m\|_{p} < \frac{\epsilon}{2}.
\]
Now there exists an $K \in \bbN$ with $n_K > N$ such that for all $k \geq K$,
\[
\|f_{n_k} - f \|_{p} < \frac{\epsilon}{2}.
\]
Then, for $n \geq n_K$, we find
\[
\| f_n - f \|_{p} 
\leq \| f_{n} - f_{n_{K}} \|_{p} + \|f_{n_K} - f\|_{L^p(\Omega)}  < \frac{\epsilon}{2} + \frac{\epsilon}{2} = \epsilon,
\]
which gives the required convergence.

The proof of completeness of $L^\infty(\Omega)$ follows similar lines but is in a way easier. Let again $(f_n)_{n\in\bbN}$ be a Cauchy sequence and select a subsequence $(f_{n_k})_{k\in\bbN}$ such that
\[
\| f_{n_{k+1}} - f_{n_k} \|_{L^\infty(\Omega)} < 4^{-k-1}.
\]
We define again $g_k = f_{n_k}$.
Then
\[
\mu\left( \left\{ x \in \Omega : \ |g_{k+1}(\omega) - g_k(\omega) | \geq 4^{-k-1} \right\} \right) = 0.
\]
So, $(g_k(\omega))_{k\in\bbN}$ is a Cauchy-sequence for almost every $\omega \in \Omega$.
For such $\omega$, the limit as $k \to \infty$ of $g_k(\omega)$ exists, and we denote it by $f(\omega)$.
Moreover, 
\[
\mu\left( \left\{ x \in \Omega : |g_k(\omega) - f(\omega)|  \geq 4^{-k} \right\} \right) = 0.
\]
It follows that $g_k$ converges to $f$ in $L^\infty(\Omega)$ as $k \to \infty$, and therefore that $f_n$ converges to $f$ in $L^\infty(\Omega)$ as $n \to \infty$ using the same argument as above.
\end{proof}

%\section{Littlewood's principles}
%
%In this section, we will discuss 3 principles---called Littlewood's principles---that provide practical ways of seeing measurable sets, almost everywhere convergence, and measurable functions. These principles hold for measures that are both inner and outer regular as defined in the following.
%
%\begin{definition}[Inner/Outer regularity]
%A measure $\mu$ on $(\bbR^d, \cB_{\bbR^d})$ is \emph{inner regular} if for all $A \in \cB_{\bbR^d}$ 
%\[
%	\mu(A) = \sup\bigl\{ \mu(K) : K\subset A\;\text{compact}\bigr\}.
%\]
%We say that a measure $\mu$ on $(\bbR^d, \cB_{\bbR^d})$ is \emph{outer regular} if for all $A \in \cB_{\bbR^d}$ 
%\[
%	\mu(A) = \inf \bigl\{ \mu(O): A\subset O\;\text{open}\bigr\}.
%\]
%\end{definition}
%
%Surprisingly, finite measures on $(\bbR^d, \cB_{\bbR^d})$ are always both inner and outer regular.
%
%\begin{theorem}\label{th:regularity-measure}
%	Every finite measure $\mu$ on $(\bbR^d, \cB_{\bbR^d})$ is both inner and outer regular.	
%\end{theorem}
%
%%The third principle, named after Lusin, provides a nice and practical way of seeing measurable functions. In particular, it allows us to show that 
%
%
%\paragraph{Littlwood's first principle 1:} 
%\begin{quotation}
%	\emph{Every measurable set is ``practically open"}.
%\end{quotation}
%
%This principle allows us to approximate arbitrary Borel measurable sets in $\bbR^d$ with a finite union of open rectangles.
%
%\begin{theorem}
%Let $\mu$ be a finite measure on $(\bbR^d, \cB_{\bbR^d})$.
%Let $A \in \cB_{\bbR^d}$ be a Borel measurable set. Then for any $\epsilon > 0$, there exists a set $O$ of a finite union of open rectangles in $\bbR^d$, such that the measure of the \emph{symmetric difference} $A \Delta O := (A \backslash O) \cup (O \backslash A)$ is smaller than $\epsilon$, that is
%\[
%\mu( A \Delta O  ) = \mu( A \backslash O ) + \mu( O \backslash A) < \epsilon.
%\]
%\end{theorem}
%
%\begin{proof}
%	We make use of Theorem \ref{th:regularity-measure} for the proof.
%Let $\epsilon > 0$ be arbitrary. Then the inner regularity of $\mu$ provides a compact set $K \subset A$ such that
%\[
%\mu(K) > \mu(A) - \epsilon/2.
%\]	
%Moreover, the outer regularity of $\mu$ provides a family of open rectangles $(O_i)_{i\in\bbN}$ such that
%\[
%	K \subset \bigcup_{i=1}^\infty O_i\qquad\text{and}\qquad \sum_{i=1}^\infty \mu(O_i) \leq \mu(K) + \epsilon / 2.
%\]
%However, $K$ is compact and therefore there exists an $N \in \bbN$ such that
%\[
%K \subset \bigcup_{i=1}^N O_i =: O.
%\]
%Clearly,
%\[
%\sum_{i=1}^N \mu(O_i) \leq \sum_{i=1}^\infty \mu(O_i) \leq \mu(K) + \epsilon / 2,
%\]
%and hence
%\[
%\mu(A \Delta O) 
%= \mu(A \backslash O) + \mu(O \backslash A)
%\leq \mu(A \backslash K) + \mu(O \backslash K)
% \leq  \epsilon /2 + \epsilon /2=\epsilon.\qedhere
%\]
%\end{proof}
%
%
%\paragraph{Littlewood's second principle:} 
%\begin{quotation}
%	\emph{Pointwise almost everywhere convergence is ``practically uniform convergence"}.
%\end{quotation}
%
%In other words, one can think of almost everywhere convergence as uniform convergence on a `smaller' set that can be chosen arbitrarily `close' to the full set. 
%
%\begin{theorem}[Egorov's Theorem]
%	Let $\mu$ be a finite measure on $(\bbR^d,\cB_{\bbR^d})$ and $(f_n)_{n\in\bbN}$ be a sequence of $\cB_{\bbR^d}$-measurable functions that converges $\mu$-almost everywhere to a function $f$. 
%	Then for all $\epsilon > 0$ there exists a set $E\in \cB_{\bbR^d}$ such that
%	$\mu(\bbR^d\backslash E) \le\epsilon$ and $f_n \to f$ uniformly on $E$.
%\end{theorem}
%
%\begin{proof}
%	Since $f_n\to f$\, $\mu$-almost everywhere, where exists a $\mu$-null set $N\subset\Omega$, i.e., $\mu(N)=0$, for which $f_n(x)\to f(x)$ for every $x\in \Omega:=\bbR^d\setminus N$. Consider the sets
%	\[
%		E_{\ell,n} := \bigcap_{k\ge n} \left\{ \omega\in \Omega\,:\, |f_k(\omega)-f(\omega)| < \frac{1}{\ell}\right\},\qquad n,\ell\ge 1
%	\]
%	It is not difficult to check that $E_{\ell,n}$ is measurable for every $\ell,n\ge 1$ and that $E_{\ell,n}\subset E_{\ell,m}$ for $n\le m$. Moreover, for each $\ell\ge 1$, we have that $\Omega = \cup_{n\ge 1} E_{\ell,n}$. Hence, by the continuity-from-below property of $\mu$, we obtain
%	\[
%		\mu(\Omega) = \mu\Bigl(\bigcup_{n\ge 1} E_{\ell,n}\Bigr) = \lim_{n\to\infty} \mu(E_{\ell,n}).
%	\]
%	Now choose $n_\ell\ge 1$ such that $\mu(\Omega\backslash E_{\ell,n_\ell}) \le \varepsilon\,2^{-\ell}$ and define the measurable set
%	\[
%		E := \bigcap_{\ell\ge 1} E_{\ell,n_\ell}.
%	\]
%	Then, by the subadditivity of $\mu$, we obtain
%	\[
%		\mu(\Omega\backslash E) = \mu\left(\bigcup_{\ell\ge 1}\bigl(\Omega\backslash E_{\ell,n_\ell}\bigr)\right) \le \sum_{\ell\ge 1} \mu\bigl(\Omega\backslash E_{\ell,n_\ell}\bigr) = \epsilon.
%	\]
%	Moreover, for every $k \in \bbN$ and every $x \in E$,
%	\[
%		|f_k(\omega) - f(\omega)| \leq \frac{1}{\ell}\qquad\text{for all $k\ge n_k$},
%	\]
%	thus implying that $f_n \to f$ uniformly on $E$.
%\end{proof}
%
%\begin{remark}
%	Given the inner regularity of $\mu$, one may choose $E$ compact in Egorov's Theorem. 
%\end{remark}
%
%\paragraph{Littlewood's third principle:} 
%\begin{quotation}
%	\emph{Every Borel measurable function is ``practically continuous"}.	
%\end{quotation}
%
%\begin{theorem}[Lusin's Theorem]\label{thm:lusin}
%    Let $\mu$ be a finite measure on the measurable space $(\bbR^d, \cB_{\bbR^d})$ and $f: \bbR^d \to \bbR$ be Borel measurable. 
%	Then for every $\epsilon > 0$ there exists a compact set $K \subset \bbR^d$ and a continuous function $g: \bbR^d \to \bbR$ such that $\mu(\bbR^d \backslash K) < \epsilon$ and $f \equiv g$ on $K$.
%\end{theorem}
%
%\begin{proof}
%	Let $\epsilon > 0$.
%	Define for $n \in \bbN$ and $k \in \bbZ$ the measurable sets
%	\[
%	A^n_k := \Bigl\{ \omega \in \bbR^d : \  (k-1) 2^{-n} < f(\omega) \leq k 2^{-n} \Bigr\}.
%	\]
%	Now there exist open sets $U^n_k \supset A^n_k$ and compact sets $K^n_k \subset A^n_k$ such that
%	\[
%		\mu(U_k^n \backslash A_k^n ) < \frac{1}{n 2^{|k|}} \qquad \mu(A_k^n \backslash K_k^n ) < \frac{1}{n 2^{|k|}}.
%	\]
%	We define the continuous functions $\varphi_k^n: \bbR^d \to \bbR$ such that $\varphi_k^n$ is compactly supported in $U_k^n$, satisfying $0 \leq \varphi_k^n \leq 1$ and $\varphi_k^n(\omega) = 1$ for $\omega \in K_k^n$. We set
%	\[
%		\varphi^n := \sum_{k = -2^n}^{2^n} k 2^{-n} \varphi_k^n.
%	\]
%	which is continuous for all $n\ge 1$. Since the functions $\varphi^n\to f$\, $\mu$-almost everywhere, by Egorov's Theorem, there is a compact set $K$ such that $\varphi^n$ converge to $f$ uniformly on $K$. Since uniform convergence preserves continuity, $f|_K$ is uniformly continuous on $K$.
%	
%	We now construct $g: \bbR^d \to \bbR$. Since $f|_K$ is uniformly continuous on $K$, there is a continuous increasing function $\eta: [0,\infty) \to [0,\infty)$ with $\eta(0)=0$ (also called the \emph{modulus of continuity}) such that
%	\[
%		|f(\omega) - f(\sigma)| \leq \eta( |\omega-\sigma|)\qquad \omega,\sigma\in K.
%	\]
%	Setting
%	\[
%		g(\omega) := \sup_{\sigma \in K} \Bigl\{f(\sigma) - \eta(|\omega-\sigma|)\Bigr\}.
%	\]
%	Note that $g$ is continuous on $\bbR^d$ and coincides with $f$ on $K$.
%\end{proof}
%
%The final result of this chapter is an important application of Lusin's theorem, which allows us to approximate any integrable function with continuous and bounded functions whenever $\mu$ is a finite measure. In other words, the following statement shows that the space of continuous and bounded functions $C_b(\bbR^d)$ is \emph{dense} in $L^1(\bbR^d,\mu)$. This fact is widely used in, e.g., Approximation Theory, Functional Analysis, Partial Differential Equations, and Stochastic Analysis.
%
%\begin{theorem}[Approximation in $L^1$]\label{thm:L1-approximation}
%	Let $\mu$ be a finite measure on $(\bbR^d,\cB_{\bbR^d})$ and $f\in L^1(\bbR^d,\mu)$. Then for any $\varepsilon>0$, there is a bounded continuous function $g\in L^1(\bbR^d,\mu)$ such that $\|f-g\|_1<\varepsilon$.
%\end{theorem}
%\begin{proof}
%	Let $E_n:=\{\omega\in\bbR^d\,:\, |f(\omega)|\ge n\}$. Since $\mathbf{1}_{E_n}f \to 0$ as $n\to\infty$, and $\mathbf{1}_{E_n}|f|\le |f|$ for every $n\ge 1$, we can apply DCT to conclude that
%	\[
%		\int_{E_n} |f|\,\dd\mu = \int_{\bbR^d} \mathbf{1}_{E_n}|f|\,\dd\mu\;\longrightarrow\;0\quad\text{as $n\to\infty$}.
%	\]
%	Now pick some $n\ge 1$ such that $\int_{E_n} |f|\,\dd\mu <\varepsilon/3$ and define
%	\[
%		f_n(\omega):= \max\{-n,\min\{f(\omega),n\}\}, \qquad\omega\in\bbR^d,
%	\]
%	i.e., $f_n$ is a truncation of $f$. From Lusin's theorem, we find a continuous function $g$ such that $f_n\equiv g$ on a compact set $K\subset\bbR^d$ with $\mu(\bbR^d\backslash K)<(2\varepsilon)/(3n)$. We assume w.l.o.g.\ that $|g|\le n$, since otherwise, we can consider a truncation of $g$. Altogether, this yields
%	\begin{align*}
%		\int_{\bbR^d} |f-g|\,\dd\mu &= \int_{\bbR^d} |f-f_n|\,\dd\mu + \int_{\bbR^d} |f_n-g|\,\dd\mu \\
%		&= \int_{E_n} |f|\,\dd\mu + \int_{\bbR^d\backslash K} |f_n-g|\,\dd\mu \\
%		&\le \frac{\varepsilon}{3} + 2n\,\mu(\bbR^d\backslash K) \le  \varepsilon.
%	\end{align*}
%	Finally, $g\in L^1(\bbR^d,\mu)$ holds simply due to the triangle inequality.
%\end{proof}
%
%\begin{remark}
%	All three Littlewood principles can be generalized to inner and outer regular measures $\mu$ that are \emph{locally finite} on any measurable space $(\Omega,\cF)$, i.e., a measure for which every point $\omega\in\Omega$ has a neighborhood $N_\omega\in\cF$ such that $\mu(N_\omega)<+\infty$. 
%	
%	In particular, the Littlewood principles hold also for the Lebesgue measure $\lambda$ on $(\bbR^d,\cB_{\bbR^d})$ since $\lambda(B_\omega(r))<+\infty$ for every $\omega\in\bbR^d$ and any $r>0$.
%\end{remark}

\section{Problems}

\begin{problem}
		Let $(\Omega,\mathcal{F},\mu)$ be a measure space and $f\colon\Omega\to\bbR$ be an $(\mathcal{F},\mathcal{B}_\bbR)$-measurable function. Let $p_0,p_1\in[1,
		\infty)$ be such that $p_0<p_1$ and let $\theta\in(0,1)$. Define $p_\theta$ by
		\[
			\frac{1}{p_\theta} = \frac{\theta}{p_0} + \frac{1-\theta}{p_1}.
		\]
		\begin{enumerate}[label={(\alph*)}]
			\item Show that $\|f\|_{p_\theta}\le \|f\|_{p_0}^\theta\|f\|_{p_1}^{1-\theta}$.
			\item Show that for all $p\in(p_0,p_1)$, there exists $\theta\in(0,1)$ such that $p=p_\theta$. Deduce from this that if $f\in L^{p_0}(\mu)\cap L^{p_1}(\mu)$, then also $f\in L^p(\mu)$ for all $p\in(p_0,p_1)$.
		\end{enumerate}
\end{problem}

\begin{problem}
Let $(\mathcal{X},\mathcal{F},\mathbb{P})$ be a probability space and $X$ be a real-valued random variable.
	\begin{enumerate}[label={(\alph*)}]
		\item Show that if $X\in L^\infty(\mathbb{P})$, then $X\in L^p(\mathbb{P})$ for all $p\ge 1$.
		\item Let $X$ be a Gaussian random variable with mean $0$ and variance $1$. Show that $X\in L^p(\mathbb{P})$ for all $p\ge 1$, but $X\notin L^\infty(\mathbb{P})$.
	\end{enumerate}
\end{problem}

\begin{problem}
	Let $(\Omega,\mathcal{F},\mu)$ be a finite measure space. For any $p\in[1,\infty)$ and $(\mathcal{F},\mathcal{B}_\mathbb{R})$-measurable function $f:\Omega\to\bbR$, let 
	\[
		\Phi_p(f) := \begin{cases}\displaystyle
			\left(\frac{1}{\mu(\Omega)}\int_\Omega |f(\omega)|^p \mu(d\omega)\right)^{1/p} &\text{if $f\in L^p(\mu)$},\\
			+\infty &\text{otherwise}.
		\end{cases}
	\]
	\begin{enumerate}[label={(\alph*)}]
		\item Show that $p\mapsto \Phi_p(f)$ is monotonically nondecreasing.
		\item Show that if $f\in L^\infty(\mu)$, then
		\[
			\lim_{p\to\infty} \Phi_p(f) = \|f\|_\infty.
		\]
			\textbf{Hint:} Make use of Markov's inequality.
	\end{enumerate}
\end{problem}

\begin{problem}
	In this problem, we would like to refine Theorem~\ref{thm:L1-approximation}:
	
	Show that for any $\varepsilon>0$, we can find a bounded continuous function $g\in L^1(\bbR^d,\mu)$ with compact support, i.e., $g\in C_c(\bbR^d)$, such that the conclusion of Theorem~\ref{thm:L1-approximation} remains true.
\end{problem}






\appendix
\chapter{Appendix}
\label{chapter:appendix}

\section{Uniqueness of measures}

In this section will provide the proof of Theorem~\ref{thm:uniqueness_measures}. For this weneed to introduce a few concepts as well as a powerful theorem, called the monotone class theorem.

We start with the definition of an algebra.

\begin{definition}[Algebra's of sets]
A collection $\cA$ of subsets of $\Omega$ is called an \emph{algebra} if
\begin{enumerate}
\item $\emptyset \in \cA$,
\item $\Omega \setminus A \in \cA$ for all $A \in \cA$, and
\item $A \cup B \in \cA$ for every $A, B \in \cA$.
\end{enumerate}
\end{definition}

Observe that, as the name suggests, every \sigalg/ is indeed and algebra. However, in addition to the properties of an algebra, \sigalgs/ where also closed under countable unions and intersections. We will actually take these properties on their own and define any collection of subsets that have these two properties a monotone class.

\begin{definition}[Monotone classes]
A collection $\cM$ of subsets of $\Omega$ is called a \emph{monotone class} if
\begin{enumerate}
\item $\bigcup_{i \in \bbN} A_i \in \cM$ holds for any increasing family of sets $(A_i)_{i \in \bbN}$ in $\cM$, and
\item $\bigcap_{i \in \bbN} A_i \in \cM$ holds for any decreasing family of sets $(A_i)_{i \in \bbN}$ in $\cM$
\end{enumerate}
\end{definition}

As we already remarked, any \sigalg/ is a monotone class. However, there are monotone classes that are not algebras and vise versa, there are algebras that are not monotone classes. However, suppose we start with an algebra $\cA$ and we want to turn this into a \sigalg/. Then we at least need to ensure it is also a monotone class. Similar to the construction of $\sigma(\cA)$ we can construct the smallest monotone class that contains $\cA$. Moreover, it turns out, maybe not surprisingly, that the resulting collection is \sigalg/. Even better, it is exactly $\sigma(\cA)$. This is the content of the monotone class theorem. 

\begin{theorem}[Monotone class theorem]
Let $\cA$ be an algebra on $\Omega$ and let $\Xi_\cA$ denote the collection of all monotone classes that contain $\cA$. Then 
\begin{enumerate}
\item the collection defined by
\[
	\cM(\cA) = \bigcup_{\cM \in \Xi_\cA} \cM,
\]
is a monotone class, and moreover
\item $\cM(\cA)$ is the smallest \sigalg/ containing $\cA$, i.e. $\cM(\cA) = \sigma(\cA)$.
\end{enumerate}
\end{theorem}

\begin{proof}
TODO
\end{proof}

With the monotone class theorem at hand we can prove the uniqueness theorem for measures.

\begin{proof}[Proof of Theorem~\ref{thm:uniqueness_measures}]
Define the collection
\[
	\cM := \left\{A \in \cF \, : \, \mu_1(A) = \mu_2(A)\right\}.
\]
The goal of the proof is to show that this is a monotone class. If that is true then, since $\cA \subset \cM$, the monotone class theorem (Theorem [REF]) implies that $\sigma(\cA) = \cM(\cA) \subset \cM$ and hence $\mu_1 = \mu_2$ on $\sigma(\cA)$.

To show that $\cM$ is a monotone class let $(A_i)_{i \in \bbN}$ be an increasing sequence in $\cM$. Since by definition, $\mu_1(A_i) = \mu_2(A_i)$ for all $i \in \bbN$, continuity from below (Proposition [REF]) implies that
\[
	\mu_1\left(\bigcup_{i \in \bbN} A_i\right) = \lim_{i \to \infty} \mu_1(A_i) 
	= \lim_{i \to \infty} \mu_2(A_i) = \mu_2\left(\bigcup_{i \in \bbN} A_i\right),
\]
which implies that $\bigcup_{i \in \bbN} A_i \in \cM$.

Similarly, now let $(A_i)_{i \in \bbN}$ be a decreasing sequence. Again, by definition $\mu_1(A_i) = \mu_2(A_i)$ for all $i \in \bbN$ and moreover $\mu_1(A_1) = \mu(A_1) < \infty$ since both measures are finite. Hence continuity from above (Proposition [REF]) implies that
\[
	\mu_1\left(\bigcap_{i \in \bbN} A_i\right) = \lim_{i \to \infty} \mu_1(A_i) 
	= \lim_{i \to \infty} \mu_2(A_i) = \mu_2\left(\bigcap_{i \in \bbN} A_i\right).
\]
It then follows that $\bigcap_{i \in \bbN} A_i \in \cM$ which shows that $\cM$ is indeed a monotone class.
\end{proof}

\section{Construction of the Lebesgue measure}   

 But how can we construct a measure on this set? In particular, is it possible to start with a set function that does not satisfy all the properties of a measure? We will address these questions next. But in order to do so we need to introduce the notion of an \emph{algebra}.

\begin{definition}[Algebra's of sets]
A collection $\cA$ of subsets of $\Omega$ is called an \emph{algebra} if
\begin{enumerate}
\item $\emptyset \in \cA$,
\item $\Omega \setminus A \in \cA$ for all $A \in \cA$, and
\item $A \cup B \in \cA$ for every $A, B \in \cA$.
\end{enumerate}
\end{definition}

Note that every \sigalg/ is an algebra. The idea is that is we start with a set function on an algebra, we can extend this to all the way to a measure on \sigalg/. To ensure this extension is possible, we need to start with set functions that have some structure, suspiciously called premeasures. 

\begin{definition}[Premeasures]
Let $\cA$ be an algebra on $\Omega$. A set function $\mu_o : \cA \to [0,\infty]$ is called a \emph{premeasure} if
\begin{enumerate}
\item $\mu_o(\emptyset) = 0$, and
\item $\mu_o$ is $\sigma$-additive.
\end{enumerate}
\end{definition}

If we start with a premeasure $\mu_o$ on an algebra $\cA$ we can construct a new set function on the entire collection of subsets of $\Omega$.

\begin{definition}[Outer measure]
Let $\mu_o$ be a premeasure on an algebra $\cA$ on $\Omega$. Then the set function $\mu^\ast$ defined by
\[
	\mu^\ast(A) := \inf \left\{\sum_{i = 1}^\infty \mu_o(A) \, : \, A \subset \bigcup_{i \in \bbN} A_i, \, A_i \in \cA\right\},
\]
is called the \emph{outer measure induced by $\mu_o$}. 
\end{definition}

The idea is that the outer measure $\mu^\ast$ is almost a measure. This is captured by the following set of properties it has.

\begin{proposition}
Let $\mu_o$ be a premeasure on an algebra $\cA$ on $\Omega$ and $\mu^\ast$ be the outer measure induced by $\mu_o$. Then $\mu^\ast$ satisfies the following properties:
\begin{enumerate}
\item $\mu^\ast(A) = \mu_o(A)$ for all $A \in \cA$,
\item $\mu^\ast(\emptyset) = 0$ and $\mu^\ast(A) \ge 0$ for all $A \subset \Omega$,
\item $\mu^\ast$ is monotone, and
\item $\mu^\ast$ is $\sigma$-subadditive.
\end{enumerate}
\end{proposition}

\begin{proof}
TODO
\end{proof}

Observe that indeed, $\mu^\ast$ is almost a measure. The only property missing is full $\sigma$-additivity. Then next fundamental result, due to the Greek mathematician Constantin Carath\'{e}odory, provides a way to construct a \sigalg/ from a given algebra such that $\mu^\ast$ can be extended to a true measure on it. We state a partial version here, without proof.

\begin{theorem}[Carath\'{e}odory's extension theorem (partial)]\label{thm:Caratheorody_extenstion}
Let $\cA$ be an algebra on $\Omega$. Let $\mu_0$ be a pre-measure on $\cA$ and denote by $\mu^\ast$ the outer measure induced by $\mu_0$. Then the collection defined by
\[
	\cA_{\mu^\ast} := \left\{B \subset \Omega \, : \, \mu^\ast(A) \ge \mu^\ast(A \cap B) + \mu^\ast(A \setminus B) \, \forall A \in \cA\right\},
\] 
is a \sigalg/ on $\Omega$. Moreover, the restriction $\bar{\mu} := \mu^\ast|\cA_{\mu^\ast}$ of $\mu^\ast$ to $\cA_{\mu^\ast}$ is a measure on $\cA_{\mu^\ast}$ called the \emph{Carath\'{e}odory extension of $\mu_o$}.
\end{theorem}

At this point we should take some time to fully appreciate what Theorem~\ref{thm:Caratheorody_extenstion} gives us. In order to construct a measure all we need is an algebra on $\Omega$ and some premeasure.  

\begin{remark}
The statement in Theorem~\ref{thm:Caratheorody_extenstion} only covers part of the original theorem. It actually turns out that the \sigalg/ constructed has some very nice properties and the measure space $(\Omega, \cA_{\mu^\ast}, \bar{\mu})$ is \emph{complete}. However, in order to properly define these notions we needed to introduce additional concepts going beyond the goal of this section. The interested reader is referred to the Appendix for the full statement and details, including the proof of this theorem. 
\end{remark}

Let us now utilize the Carath\'{e}odory extension to obtain a measure on the Borel space $(\bbR^d, \cB_{\bbR^d})$. 

%a \emph{semi-algebra}.
%
%\begin{definition}[Semi-algebra]
%A collection $\cS$ of subsets of $\Omega$ is called a \emph{semi-algebra} if
%\begin{enumerate}
%\item $\emptyset, \Omega \in \cS$,
%\item $A \cap B \in \cS$ for every $A, B \in \cS$, and
%\item for every $A \in \cS$ such that $\Omega \setminus A \notin \cS$, there exist a family $(A_i)_{i \in \bbN}$ of pairwise disjoint sets such that $\Omega\setminus A = \bigcup_{i \in \bbN} A_i$.
%\end{enumerate}
%\end{definition}
%
%As the name suggests, any algebra is a semi-algebra. Moreover, as was the case for \sigalgs/ we can construct a minimal algebra that contains a given semi-algebra $\cS$.
%
%\begin{proposition}
%Let $\cS$ be a semi-algebra on $\Omega$. Then the collection
%\[
%	\cA(\cS) := \left\{A \subset \Omega \, : \, \exists n\in \bbN, \, A = \bigcup_{i =1}^n A_i, \, A_i \in \cS \text{ pairwise disjoint} \right\},
%\]
%is the smallest algebra containing $\cS$ and is called \emph{the algebra generated by $\cS$}.
%\end{proposition}
%
%This result is useful, as it allows us to extend any set function $\mu$ on a semi-algebra $\cS$ to a set function on the algebra $\cA(\cS)$ by simply defining
%\[
%	\mu(A) = \sum_{i = 1}^n \mu(A_i),
%\]
%where the $A_i$ come from the definition of $\cA(\cS)$.
\chapter{Appendix: From Riemann to Lebesgue}
\label{chapter:appendix-2}
\section{Recalling Riemann integration}

\begin{definition}
A partition $P = (x_0, \dots, x_n)$ of $[a,b]$ is an $(n+1)$-tuple of real numbers $x_i$ such that $a = x_0 < x_1 < \cdots < x_n = b$, and we denote by $\Delta x_i = x_i-x_{i-1}$ the length of the interval $[x_{i-1},x_i]$, $i=1,\ldots,n$. Furthermore, we say that a partition $Q = (y_0, y_1, \dots y_m)$ of $[a,b]$ is a refinement of $P$ if $\{x_0,\dots, x_n\} \subset \{y_0, \dots, y_m\}$.
\end{definition}

Recall that given a partition $P = (x_0, \dots, x_n)$, the upper and lower sum of a bounded function $f:[a,b] \to \mathbb{R}$ with respect to $P$ are defined as
\[
U(P,f) := \sum_{i=1}^n \sup\bigl\{f(x) : \ x \in [x_{i-1},x_i] \bigr\} \Delta x_i
\]
and
\[
L(P,f) := \sum_{i=1}^n \inf\bigl\{f(x) : \ x \in [x_{i-1},x_i] \bigr\} \Delta x_i
\]
Note that if a partition $Q$ is a refinement of a partition $P$, then
\[
L(Q, f) \geq L(P,f) \quad \text{ and } U(Q,f) \leq U(P,f).
\]
Finally, if $P$ and $R$ are two partitions of $[a,b]$, there exists a partition $Q$ of $[a,b]$ such that $Q$ is both a refinement of $P$ and a refinement of $R$.

The upper and lower Riemann integral of $f$ are respectively defined as
\[
\upRiemint f(x) \,\dd x := \inf \bigl\{ U(P,f) : \ P \text{ partition of } [a,b] \bigr\}
\]
and
\[
\lowRiemint f(x) \,\dd x:= \sup \bigl\{L(P,f) : \ P \text{ partition of } [a,b]  \bigr\}.
\]
\begin{definition}
Recall that a bounded function $f:[a,b] \to \mathbb{R}$ is said to be \emph{Riemann integrable} if 
\[
\upRiemint f(x) \,\dd x = \lowRiemint f(x) \,\dd x.
\]
If $f$ is Riemann integrable, the Riemann integral of $f$ is defined as
\[
\Riemint f(x) \,\dd x := \sup \bigl\{  U(P, f) : \ P \text{ partition of } [a,b] \bigr\}.
\]
\end{definition}

\section{Riemann vs Lebesgue integration}

\begin{theorem}
If a bounded function $f: [a,b] \to \mathbb{R}$ is Riemann integrable, then $f$ is \emph{Lebesgue}-measurable and integrable. Moreover
\[
\int_a^b f(x) \,\dd x = \int_{[a,b]} f \,\dd \lambda.
\]
\end{theorem}

\begin{proof}
	Let $f:[a,b] \to \mathbb{R}$ be Riemann-integrable. 
	We can then find a sequence of partitions $(P_n)$, $P_n = (x^n_1, \dots, x^n_{N_n})$, such that for every $n \in \mathbb{N}$, $P_{n+1}$ is a refinement of $P_n$ and such that
	\begin{equation}\label{eq:Convergence-Darboux}
	\lim_{n \to \infty} U(P_n, f) = \lim_{n \to \infty} L(P_n, f) = \Riemint f(x) \,\dd x.
	\end{equation}
	The details on how to find such $P_n$ are as follows: By the definition of the upper and lower Riemann integral and by the assumption that $f$ is bounded and Riemann integrable, we know that there exist partitions $Q_1$ and $R_1$ such that
	\[
	\Riemint f(x) \,\dd x - 1 < L(R_1, f) \quad \text{ and } \quad   U(Q_1, f) < \Riemint f(x) \,\dd x + 1 .
	\]
	We then choose the partition $P_1$ as a common refinement of the partitions $Q_1$ and $R_1$. Hence,
	\[
	\begin{split}
	\Riemint f(x) \,\dd x - 1 &< L(R_1, f) 
	\leq L(P_1, f) \\ &\leq U(P_1,f)
	\leq U(Q_1,f) < \Riemint f(x) \,\dd x +1.
	\end{split}
	\]
	
	Now suppose the partition $P_k$ has been defined for some $k \in \mathbb{N}$.
	Again by the definition of the upper and lower Riemann integral and by the assumption that $f$ is bounded and Riemann integrable, there exist partitions $Q_{k+1}$ and $R_{k+1}$ such that
	\[
	\begin{split}
	 \Riemint f(x) \,\dd x - \frac{1}{k+1} &< L(R_{k+1}, f)  \qquad \text{ and } \qquad U(Q_{k+1}, f) < \Riemint f(x) \,\dd x + \frac{1}{k+1}.
	 \end{split}
	\]
	Now define $P_{k+1}$ as a common refinement of $P_k$, $Q_{k+1}$ and $R_{k+1}$. Then
	\[
	\begin{split}
	\Riemint f(x) \,\dd x - \frac{1}{k+1} &< L(R_{k+1}, f) \leq L(P_{k+1}, f) \\
	& \leq U(P_{k+1},f) \leq U(Q_{k+1},f) < \Riemint f(x) \,\dd x + \frac{1}{k+1}.
	\end{split}
	\]
	It follows that for every $n \in \mathbb{N}$, the partition $P_{n+1}$ is a refinement of the partition $P_n$ and 
	\[
	\begin{split}
	L(P_n, f) &\leq \Riemint f(x) \,\dd x < L(P_n, f) + \frac{1}{n} \\
	U(P_n,f)-\frac{1}{n} &< \Riemint f(x) \,\dd x \leq U(P_n,f)
	\end{split}
	\]
	from which the limits (\ref{eq:Convergence-Darboux}) follow.
	
	Now define the functions
	\begin{align*}
	u_n &:= \sum_{i=1}^{N_n} \sup\bigl\{ f(x)  :\ x \in [x^n_{i-1},x^n_i] \bigr\} \mathbf{1}_{(x^n_{i-1},x^n_i]}, \\
	\ell_n &:= \sum_{i=1}^{N_n} \inf\bigl\{ f(x) :\ x \in [x^n_{i-1},x^n_i]\bigr\} \mathbf{1}_{(x^n_{i-1},x^n_i]}.
	\end{align*}
	Because $P_{n+1}$ is a refinement of $P_n$, we find that $\ell_n\le \ell_{n+1}$, $u_{n+1} \leq u_n$, and therefore
	\[
		\ell(x) := \lim_{n \to \infty} \ell_n(x)\quad\text{exists},\qquad u(x) := \lim_{n \to \infty} u_n(x)\quad\text{exists}.
	\]
	The functions $\ell,u:[a,b]\to \mathbb{R}$ are clearly Borel-measurable. Moreover, $\ell \leq f \leq u$. Note that
	\[
	U(P_n, f) = \int_{[a,b]} u_n \,\dd \lambda \qquad L(P_n,f) = \int_{[a,b]} \ell_n \,\dd \lambda.
	\]
	By the dominated convergence theorem (recall that $f$ is bounded),
	\[
	U(P_n, f) \to \int_{[a,b]} u \,\dd \lambda  \qquad L(P_n, f) \to \int_{[a,b]} \ell \,\dd \lambda\qquad\text{as $n\to\infty$}.
	\]
	However, since $f$ is Riemann integrable, both the upper and lower sums also converge to the Riemann integral of $f$, so 
	\[
	\int_{[a,b]} u \,\dd \lambda = \Riemint f(x) \,\dd x = \int_{[a,b]} \ell \,\dd \lambda.
	\]
	Now since $\ell\le f\le u$, we then obtain, by linearity of the integral,
	\[
		0\le \int_{[a',b']} \bigl(u-\ell\bigr) \,\dd \lambda = 0\qquad\Longrightarrow\qquad \ell\equiv u\quad\text{$\lambda$-almost everywhere},
	\]
from which we also obtain $\ell\equiv f\equiv \ell$ $\lambda$-almost everywhere. Moreover, since $u$ and $\ell$ are both Borel-measurable, $f$ is Borel-measurable, and particularly Lebesgue-measurable.
\end{proof}

The following theorem provides a full characterization of Riemann-integrable functions.

\begin{theorem}
	A bounded function $f:[a,b] \to \mathbb{R}$ is Riemann integrable if and only if it is continuous $\lambda$-almost everywhere.
\end{theorem}





%\bibliography{references}

\end{document}