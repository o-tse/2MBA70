We have now arrived at arguably the core aspect of Measure Theory: The Lebesgue integral. Unlike the Riemann integral, the Lebesgue integral can be constructed on any measure space $(\Omega,\cF,\mu)$ and facilitates many powerful limit results. The construction will be done in multiple steps, starting with simple functions.

\section{The integral of a simple function}\label{sec:integral_simple_functions}

\begin{definition}\label{def:simple_function}
A function $f: \Omega \to \bbR$ is called \emph{simple} if it takes the form
\[
f = \sum_{i=1}^N a_i \mathbf{1}_{A_i}
\]
for some positive integer $N \in \bbN$, disjoint measurable sets $A_1, \ldots, A_N \in \cF$ and constants $a_1, \dots, a_N  \in \bbR$.
\end{definition}

Sometimes, the definition of a simple function will not require the sets to be disjoint. When this is the case, it is referred to as a \emph{standard representation} of a simple functions. But this distinction does not matter.

\begin{lemma}\label{lem:simple_function}
Let $f: \Omega \to \bbR$ be a measurable function that attains only finitely many values, i.e. $f(\Omega) = \{a_1, \dots, a_m\}$ with $m < \infty$. Then there exist a finite collection of measurable and mutually disjoint sets $A_1, \dots, A_m$ such that 
\[
	f = \sum_{i = 1}^m a_i \mathbf{1}_{A_i}.
\]
In particular, $f$ is simple.
\end{lemma}

\begin{proof}
See Problem~\ref{prb:standard_representation_simple}.
\end{proof}

As a consequence of Lemma~\ref{lem:simple_function} any function of the form
\[
	f = \sum_{i=1}^N a_i \mathbf{1}_{A_i},
\]
with $a_i \in \bbR$ and $A_1, \dots, A_N$ measurable but necessarily disjoint sets, is simple. 

We can now define the Lebesgue integral of a simple function.

\begin{definition}\label{def:Lebesgue_integral_simple}
Let $f: \Omega \to \bbR_+$ be a non-negative simple function of the form
\[
	f = \sum_{i=1}^N a_i \mathbf{1}_{A_i}.
\] 
Then we define the \emph{$\mu$-integral} of $f$ by
\[
	\int_\Omega f\, \dd \mu = \int_\Omega f(\omega)\,\mu(\dd\omega) := \sum_{i = 1}^N a_i \mu(A_i).
\]
\end{definition}

Although this seems like a reasonable definition for an integral of a simple function, one has to be a bit careful. The potential problem is that the value of the integral is defined using the set $A_i$ and values $a_i$ in the representation of the simple function $f$, which might not be unique. Luckily it turns out that the value of their sum is.

\begin{lemma}\label{lem:lebesgue_simple_welldefined}
Let $f: \Omega \to \bbR_+$ be a non-negative simple function such that
\[
	\sum_{i = 1}^N a_i \mathbf{1}_{A_i} = f = \sum_{j = 1}^M b_j \mathbf{1}_{B_j},
\]
for two collections $a_i, b_j \in \bbR$ and two families of mutually disjoint measurable sets $A_i, B_j$. Then
\[
	\sum_{i = 1}^N a_i \mu(A_i)= \sum_{j = 1}^M b_j \mu(B_j).
\]
\end{lemma}

\begin{proof}
See Problem~\ref{prb:lebesgue_simple_welldefined}
\end{proof}

This lemma shows that the Lebesgue integral for simple functions given in Definition~\ref{def:Lebesgue_integral_simple} is well-defined. With that, we have made the first important step toward the definition of the Lebesgue integral for general functions. Before we can continue, we will show that any positive measurable function $f : \Omega \to \bbR$ can be approximated by simple functions.

\begin{lemma}\label{lem:linearity_lebesgue_simple}
Let $f, g: \Omega \to \bbR_+$ be two non-negative simple functions. Then $f+g$ is again a non-negative simple function and moreover,
\[
	\int_\Omega f + g \, \dd \mu = \int_\Omega f\, \dd \mu + \int_\Omega g\, \dd \mu
\]
\end{lemma}

\begin{proof}

\end{proof}

%\begin{remark}
%	In case $(\Omega, \cF, \bbP)$ is a probability space, and $X$ is a simple, real-valued random variable on $\Omega$ having the representation
%\[
%	X = \sum_{i=1}^N a_i \mathbf{1}_{A_i},
%\]
%with mutually disjoint $A_i \in \cF$ and $a_i \in \bbR$, the integral is usually called the \emph{expectation} value of $X$ and is written as
%\[
%	\ex{X} := \int_\Omega X(\omega)\, \bbP(\dd\omega) = \sum_{i=1}^N a_i \prob{A_i}.
%\]
%
%\end{remark}

\section{Approximation by simple functions}
\label{sec:simple-approximation}

\begin{proposition}[Approximation by simple functions]\label{prop:approximation_simple}
Let $(\Omega, \cF)$ be a measurable space and $f : \Omega \to [0,\infty]$ a measurable function. Then there exist a sequence $(f_n)_{n \ge 1}$ of positive simple functions such that for any $\omega \in \Omega$
\[
	\lim_{n \to \infty} f_n(\omega) = f(\omega).
\]
Moreover, the functions $f_n$ form a point-wise non-decreasing sequence and hence $f(\omega) = \sup_{n \ge 1} f_n(\omega)$.
\end{proposition}

\begin{proof}
See Problem~\ref{prb:approximation_simple}
%Fix $n \ge 1$, set $N_n = n 2^n$ and define the sets
%\[
%	A_k^n = \begin{cases}
%		\{ k 2^{-n} \le f < (k+1) 2^{-n}\} &\text{for } k = 0,1, \dots, N_n - 1,\\
%		\{ f \ge n\} &\text{for } k = N_n,
%	\end{cases}
%\]
%which are measurable due to Lemma~\ref{lem:measurable_set_real_line} and mutually disjoint.
%
%We then define the function
%\[
%	f_n = 2^n\mathbf{1}_{\{f=+\infty\}} + \sum_{k=0}^{N_n} k \, 2^{-n} \mathbf{1}_{A_k^n}.
%\]
%From this representation, we easily deduce that $f_n$ is measurable and simple for every $n \ge 1$. Also note that $f_n(\omega) = k 2^{-n} \iff \omega \in A_k^n \text{ and } f(\omega) < +\infty$. 
%
%The first observation we make is that $f_n(\omega) \le f(\omega)$ holds for all $n \ge 1$. Moreover, if $f(\omega) < n$ then $f(w) - f_n(\omega) \le 2^{-n}$. 
%
%We will now show that $f_n(\omega) \to f(\omega)$ holds for any $\omega \in \Omega$. Let us fix a $\omega \in \Omega$. Then if $f(\omega) = +\infty$ we get that $f_n(\omega) = 2^n$ holds for all $n \ge 1$ and hence $\lim_{n \to \infty} f_n(\omega) = +\infty = f(\omega)$. So assume that $f(\omega) < +\infty$. Then there exists an $M \in \bbN$ such that $f(\omega) < M$. Hence, for all $n \ge M$ we have that 
%\[
%	\|f_n(\omega) - f(\omega)\| = f(\omega) - f_n(\omega) \le 2^{-n},
%\]
%which implies that $\lim_{n \to \infty} f_n(\omega) = f(\omega)$.
%
%For the final part we need to show that for any $\omega \in \Omega$, $f_n(\omega) \le f_{n+1}(\omega)$ holds for all $n \ge 1$. So fix $n \ge 1$ and $\omega \in \Omega$. Clearly, if $f(\omega) = +\infty$ there is nothing to prove. So let's assume that $f(\omega) < +\infty$. Then $\omega \in A_k^n$ for some $0 \le k \le N_n = n 2^n$. 
%
%We will first consider the case that $k < n 2^n$, so that $k 2^{-n} \le f(\omega) < (k+1) 2^{-n}$ holds. Note that this  interval can be split into two intervals as follows:
%\[
%	[k 2^{-n}, (k+1) 2^{-n}) = [(2k) 2^{-(n+1)}, (2k +1)2^{-(n+1)}) \cup [(2k +1)2^{-(n+1)}, (2k + 2)2^{-(n+1)}).
%\] 
%Hence, we conclude that either $\omega \in A_{2k}^{n+1}$ or $\omega \in A_{2k+1}^{n+1}$. In both case we get that 
%\[
%	f_n(\omega) = k2^{-n} = 2k n^{-(n+1)} \le f_{n+1}(\omega).
%\]
%
%Now let us consider the case that $k = n 2^n$, so that $f(\omega) \ge n$. Then, if $f(\omega) \ge n + 1$ it follows that $f_n(\omega) = n < n + 1 = f_{n+1}(\omega)$. If, on the other hand, $n \le f(\omega) < n + 1$ there exists an $2n \, 2^n \le \ell \le (2n+2) \, 2^n$ such that $\omega \in A_\ell^{n+1}$, which then implies that 
%\[
%	f_n(\omega) = n = (2n 2^{n}) \, 2^{-(n+1)} \le f_{n+1}(\omega).
%\]
\end{proof}

We will often use the notation $[f]_n$ to denote the simple functions that approximate a given function $f$.

\section{The Lebesgue integral of nonnegative functions}

We now extend the $\mu$-integral from non-negative simple functions to arbitrary non-negative $\cF$-measurable functions.

\begin{definition}\label{def:lebesgue_non_negative}
	The \emph{$\mu$-integral} of a $(\cF,\cB_{[0,+\infty]})$-measurable function $f:\Omega\to[0,+\infty]$ is defined by
\[
	\int_\Omega f\, \dd \mu = \int_\Omega f(\omega)\,\mu(\dd\omega) := \sup\left\{ \int_\Omega g\, \dd \mu : \ g \text{ simple},\; 0 \leq g  \leq f \right\}.
\]
The function $f$ is said to be \emph{$\mu$-integrable} if its $\mu$-integral is finite.
\end{definition}

For a measurable set $A \in \cF$, we use the following notation and definition for integration of $f$ over the set $A$
\[
	\int_A f\, \dd \mu := \int_\Omega \mathbf{1}_A\, f\, \dd \mu.
\]
If we denote by $f_A$ the restriction of $f$ to $A$, and by $\mu_A$ the restriction of $\mu$ to $\cF_A$, then 
\[
\int_A f_A\,\dd \mu_A = \int_A f\, \dd \mu.
\]
Similarly, if $f_A:(A, \cF_A) \to ([0,+\infty],\cB_{[0,+\infty]})$ is measurable, and $f$ is a measurable extension of $f_A$ to the whole of $\Omega$, then
\[
	\int_A f\,\dd \mu = \int_A f_A\, d \mu_A.
\]

The following lemma summarize some basic properties of the Lebesgue integral for non-negative functions.

\begin{lemma}[Properties of the Lebesgue integral of non-negative functions]
	\label{prop:properties-integral-nonneg}
	Let $(\Omega, \cF, \mu)$ be a measure space, $f, g : \Omega \to \bbR_+$ two non-negative, measurable functions and $\alpha \geq 0$ be a constant. The the following holds:
	\begin{enumerate}
		\item (absolute continuity) If $B \in \cF$ satisfies $\mu(B) = 0$, then
		\[
		\int_{B} f\, \dd \mu = 0. 
		\]
		\item (monotonicity) If $f \leq g$, then
		\[
		\int_\Omega f\, \dd \mu \leq \int_\Omega g\, \dd \mu.
		\]
		\item (homogeneity)
		\[
			\alpha \int_\Omega f\, \dd \mu = \int_\Omega (\alpha f )\,\dd \mu.
		\]
	\end{enumerate}
\end{lemma}

\begin{proof}
See Problem~\ref{prb:properties-integral-nonneg}
\end{proof}

It should be noted that one key property seems to be missing from the above lemma, namely
\[
	\int_\Omega f + g \, \dd \mu = \int_\Omega f\, \dd \mu + \int_\Omega g\, \dd \mu.
\]
We know that the Riemann integral is linear and hence, if the aim of the Lebesgue integral is to extend it, it should at the very least also be linear. The main issue is that it is difficult to proof this directly using Definition~\ref{def:lebesgue_non_negative}, due to the supremum. However, recall that we did have linearity for the Lebesgue integral of simple functions, see Lemma~\ref{lem:linearity_lebesgue_simple}. So how do we use this to show that also the integral of non-negative functions is linear? This is where we first make use of the approximation of non-negative functions by simple functions. Here is how we envision this to go.
\begin{align*}
	\int_\Omega (f+g) \, \dd \mu &= \int_\Omega \lim_{n \to \infty} ([f]_n + [g]_n) \, \dd \mu \\
	&= \lim_{n \to \infty} \int_\Omega [f]_n + [g]_n \, \dd \mu \\
	&= \lim_{n \to \infty} \int_\Omega [f]_n \, \dd \mu + \lim_{n \to \infty} \int_\Omega [g]_n \, \dd \mu \\
	&= \int_\Omega \lim_{n \to \infty} [f]_n \, \dd \mu + \int_\Omega \lim_{n \to \infty} [g]_n \, \dd \mu \\
	&= \int_\Omega f \, \dd \mu + \int_\Omega g \, \dd \mu.
\end{align*}
The problem with this argument is that in both the second and fourth line, we exchanged the limit and the integral. This is something that is not generally possible, at least for Riemann integration, and needs a proof. For this we will use one of the classical limit theorems for Lebesgue integrals: the Monotone Convergence Theorem.

\section{The monotone convergence theorem}
\label{sec:monotone_convergence}

This first convergence result tells us essentially that the point-wise limit of monotone sequences of $\mu$-integrable functions is again $\mu$-integrable, highlighting the difference with Riemann integration.

\begin{theorem}[Monotone convergence theorem I]
	\label{th:monotone-convergence-I}
	Let $(\Omega, \cF, \mu)$ be a measure space. Let $f_n\colon\Omega \to [0,+\infty]$, $n \in \bbN$, be a sequence of nonnegative $(\cF,\cB_{[0,+\infty]})$-measurable functions, such that $f_n(\omega) \leq f_{n+1}(\omega)$ for all $\omega \in \Omega$ and $n \in \bbN$. Define the function 
	\[
		f(\omega) := \lim_{n \to \infty} f_n(\omega),\qquad \omega\in\Omega.
	\]
	Then $f$ is $(\cF,\cB_{[0,+\infty]})$-measurable and
	\[
	\lim_{n \to \infty} \int_\Omega f_n\, \dd \mu = \int_\Omega f\, \dd \mu.
	\]
\end{theorem}

\begin{proof}
	From the monotonicity of the integral, we immediately conclude that
	\[
	\limsup_{n \to \infty} \int_\Omega f_n\, \dd \mu \leq \int_\Omega f\, \dd \mu.
	\]
	Hence, we are left to show that
	\[
	\liminf_{n \to \infty } \int_\Omega f_n\, \dd \mu \geq \int_\Omega f\, \dd \mu.
	\]
	This is obvious if $\int_\Omega f\, \dd \mu = 0$, so we assume that $\int_\Omega f\, \dd \mu >0$.
	
	By the definition of the integral, for every $0<\varepsilon<L$, there exists a nonnegative simple function $g: \Omega \to \bbR$ such that $0\le g \leq f$ on $\Omega$ and
	\[
		\int_\Omega g\, \dd \mu > \int_\Omega f\, \dd \mu - \varepsilon.
	\]
	Because $g$ is simple, there exist an $N \in \bbN$, nonnegative constants $a_i \in (0,\infty)$ and disjoint, measurable sets $A_i \in \cF$ such that
	\[
		g = \sum_{i=1}^N a_i \mathbf{1}_{A_i}.
	\]
	Moreover, we find some $\delta>0$, such that
	\[
		g_\delta:= \sum_{i=1}^N (a_i-\delta)\mathbf{1}_{A_i}, 
	\]
	satisfies
	\[
		\int_\Omega g_\delta\,\dd \mu = \sum_{i=1}^N(a_i-\delta)\,\mu(A_i) \ge \int_\Omega f\,\dd\mu - \varepsilon. 
	\]
	
	Now define for $i \in \{ 1, \dots, N\}$ and $n \in \bbN$ the measurable set
	\[
	G_n^i := \Bigl\{ x \in A_i : \ f_n(x) \geq a_i - \delta \Bigr\}.
	\]
	Then, because $f_n \leq f_{n+1}$, we have $G_n^i \subset G_{n+1}^i$ for all $n \in \bbN$ and by the pointwise convergence of $f_n$ to $f$, we have
	\[
	\bigcup_{n=1}^\infty G_n^i = A_i,\qquad i=1,\ldots,N.
	\]
	Hence, by the continuity from below of measures
	\[
		\lim_{n \to \infty}\mu(G_n^i) = \mu(A_i).
	\]
	Since for every $n\in\bbN$,
		\[
		\int_\Omega f_n\,\dd\mu \ge \sum_{i=1}^N \int_{A_i} f_n\,\dd \mu \ge \sum_{i=1}^N \int_{A_i} f_n\,\dd \mu \ge \sum_{i=1}^N \int_{G_n^i} f_n\,\dd \mu \ge \sum_{i=1}^N (a_i-\delta)\,\mu(G_n^i),
	\]
	we find that
	\[
		\liminf_{n \to \infty} \int_\Omega f_n\, \dd \mu \ge  \liminf_{n \to \infty } \sum_{i=1}^N (a_i - \delta) \mu(G_n^i) = \int_\Omega g_\delta\,\dd\mu \ge \int_\Omega f\,\dd\mu - \varepsilon. 
	\]
	Because $\varepsilon>0$ was arbitrary, it follows that
	\[
	\liminf_{n \to \infty} \int_\Omega f_n \dd \mu \geq \int_\Omega f \dd \mu.\qedhere
	\]	
\end{proof}




\section{Additivity of the Lebesgue integral of nonnegative functions}

Armed with monotone convergence, we can now proof that the Lebesgue integral of non-negative functions is linear.

\begin{lemma}[Additivity of the Lebesgue integral of nonnegative functions]
	\label{pr:additivity-integral-nonneg}
	Let $f$, $g$ be two non-negative $(\cF,\cB_{[0,+\infty]})$-measurable functions. Then,
	\[
		\int_\Omega (f + g)\, \dd \mu = \int_\Omega f\, \dd \mu + \int_\Omega g\, \dd \mu.
	\]	
\end{lemma}
\begin{proof}
	For simple functions, the additivity of the integral follows from Lemma~\ref{lem:linearity_lebesgue_simple}. Therefore, 
	\[
	\int_\Omega ([f]_n + [g]_n)\, \dd \mu = \int_\Omega [f]_n\, \dd \mu + \int_\Omega [g]_n\, \dd \mu\qquad\text{for every $n \in \bbN$.}
	\]
	We now take the limit on both sides of the equation. On one hand, the functions $[f]_n + [g]_n$ are increasing in $n$, and converge point-wise to $(f + g)$. Hence, by the monotone convergence theorem (Theorem~\ref{th:monotone-convergence-I}),
	\[
	\lim_{n \to \infty} \int_\Omega ([f]_n + [g]_n)\, \dd \mu = 
	\int_\Omega ( f + g)\, \dd \mu.
	\]
	On the other hand, by linearity of limits we know that 
	\[
	\lim_{n \to \infty} \left( \int_\Omega [f]_n\, \dd \mu + \int_\Omega [g]_n\, \dd \mu \right) 
		= \lim_{n \to \infty} \int_\Omega [f]_n\, \dd \mu + \lim_{n \to \infty} \int_\Omega [g]_n\, \dd \mu.
	\]
	Since both $[f]_n$ and $[g]_n$ are increasing and converge point-wise to $f$ and $g$, respectively, monotone convergence implies that
	\[
		\lim_{n \to \infty} \int_\Omega [f]_n\, \dd \mu = \int_\Omega f\, \dd \mu, \text{ and }
		\lim_{n \to \infty} \int_\Omega [g]_n\, \dd \mu = \int_\Omega g\, \dd \mu.
	\]
	Therefore, we conclude that
	\[
	\int_\Omega (f + g)\, \dd \mu = \int_\Omega f\, \dd \mu + \int_\Omega g\, \dd \mu.\qedhere
	\]
\end{proof}


\section{Integrable functions}\label{sec:integrable}

The final step we need to take is to define the integral of functions $f$ that are not necessarily non-negative. We can only do this if the integral of $|f|$ is finite.

\begin{definition}\label{def:lebesgue_integral_general}
	A $(\cF,\cB_\bbR)$-measurable function $f: \Omega \to \bbR$ is \emph{$\mu$-integrable} if 
	\[
	\int_\Omega |f|\, \dd \mu < +\infty.	
	\]
	For any function $f: \Omega \to \overline{\bbR}$, we define its \emph{positive part} $f^+$ and \emph{negative part} $f^-$ as
\[
	f^+(\omega) := \max( f(\omega), 0 ),\qquad 
	f^-(\omega) := - \min( f(\omega), 0 ) 
\]
It follows that $f = f^+ - f^-$ and $|f| = f^+ + f^-$.

	The \emph{Lebesgue integral} of a $\mu$-integrable function $f: \Omega \to \bbR$ is then defined as
	\[
		\int_\Omega f\,\dd \mu := \int_\Omega f^+\, \dd \mu - \int_\Omega f^-\, \dd \mu.
	\]
\end{definition}

We say that a measurable function $f\colon(\Omega, \cF) \to (\bbR, \cB_\bbR)$ is integrable on a set $A \in \cF$ if the function $\mathbf{1}_A f$ is integrable on $\Omega$. Equivalently, we say that $f$ is integrable on $A$ if the restriction $f|_A$ is integrable on the measure space $(A, \cF_A, \mu|_A)$. 


As in the case for non-negative measurable functions, we have the following properties for $\mu$-integrable functions.

\begin{proposition}\label{prop:properties-integral}
	Let $f$, $g$ be two $\mu$-integrable functions and $\alpha \in \bbR$ be a constant.
	\begin{enumerate}
		\item (absolute continuity) If $B \in \cF$ satisfies $\mu(B) = 0$, then
		\[
		\int_{B} f\, \dd \mu = 0. 
		\]
		\item (monotonicity) If $f \leq g$ $\mu$-a.e., then
		\[
		\int_\Omega f \,\dd \mu \leq \int_\Omega g \,\dd \mu.
		\]
		\item (homogeneity) 
		\[
		\alpha \int_\Omega f \,\dd \mu = \int_\Omega (\alpha f )\,\dd \mu.
		\]
		\item (additivity)
		\[
		\int_\Omega (f + g)\, \dd \mu = \int_\Omega f 
		\,\dd \mu + \int_\Omega g \,\dd \mu.
		\]	
	\end{enumerate}
\end{proposition}

\begin{proof}
See Problem~\ref{prb:lebesgue_integral_general}
\end{proof}

\section{Riemann vs Lebesgue integration}

A fundamental fact about the Lebesgue integral is its relationship with the Riemann integral, which allows us to make use of the integration techniques we know from Calculus and Analysis to compute the Lebesgue integral of a Lebesgue integrable function.

We state an important result, which we will not prove, but will be essential for computing integrals (cf.\ Appendix~\ref{chapter:appendix-2}). The first part of the result provides a full characterization of Riemann-integrable functions, while the second provides the means to compute Lebesgue integrals.

\begin{theorem}[Riemann vs Lebesgue]\label{thm:riem-leb}
	A bounded function $f\colon [a,b] \to \bbR$ on a compact set $[a,b]\subset\bbR$ is Riemann integrable if and only if it is continuous $\lambda$-almost everywhere.
	
	If a bounded function $f\colon [a,b] \to \bbR$ is Riemann integrable, then $f$ is $\cL$-measurable and $\lambda$-integrable. Moreover,
	\[
		\int_a^b f(x) \,\dd x = \int_{[a,b]} f \,\dd \lambda,
	\]
	where the left-hand side denotes the Riemann integral of $f$.
\end{theorem}

\begin{example}\label{ex:computation_lebesgue_integral}
	Let us determine the value $\displaystyle \int_\bbR \frac{1}{x^2+1}\,\lambda(\dd x)$.
	
	\noindent To do so, we set $g(x):= \frac{1}{x^2+1}\ge 0$ and let $g_n:= g\mathbf{1}_{[-n,n]}$. Then clearly, $g_n$ is monotone and $g_n\to g$ point-wise. Thus, by monotone convergence we have that
\[
	\lim_{n\to\infty} \int_\bbR g_n\,\dd\lambda = \int_\bbR g\,\dd\lambda.
\]
On the other hand, for every $n\ge 1$,
\[
	\int_\bbR g_n\,\dd\lambda = \int_{[-n,n]} g\,\dd\lambda = \int_{-n}^n g\,\dd x = \int_{-n}^n \frac{1}{1+x^2}\,\dd x = \arctan(n) - \arctan(-n),
\]
where the second equality follows from the fact that $g$ is continuous on the compact set $[-n,n]$ and from Theorem~\ref{thm:riem-leb}. Hence,
\[
	\int_\bbR g\,\dd\lambda=\lim_{n\to\infty} \int_\bbR g_n\,\dd\lambda = \frac{\pi}{2} + \frac{\pi}{2} = \pi,
\]
thus implying that $g$ is $\lambda$-integrable.
\end{example}

\begin{remark}
The main take-away is that when considering a measurable functions $f : \mathbb{R}^d \to \mathbb{R}$ that is Riemann integrable, the Lebesgue integral is simply the same as the Riemann integral. The difference is that the Lebesgue integral is applicable to any measurable space and a much larger class of functions.
\end{remark}

\section{Change of variables formula}
\label{sec:change-of-variables}

A key tool for computing Riemann integrals was the so-called change of variables formula 
\[
	\int_a^b f(g(x)) g^\prime(x) \, \dd x = \int_{g(a)}^{g(b)} f(y) \, \dd y.
\]
This allowed one to pick a different parametrization of the function in terms of its variables to compute the integral. As you might expect by now, we also have this for the Lebesgue integral. Here we encounter the push-forward measure, see Proposition~\ref{prop:push_forward_measure}.


\begin{proposition}\label{prop:change_of_variables}
Let $(\Omega, \cF, \mu)$ be a measure space and $(E,\cG)$ be a measurable space. Further, let $f\colon \Omega \to E$ and $h\colon E \to [0,+\infty]$ be $(\cF,\cG)$- and $(\cG,\cB_{[0,+\infty]})$-measurable maps respectively. Then,
\[
\int_\Omega h \circ f\,\dd \mu = \int_E h \, \dd (f_\# \mu).
\]
In particular, $h \circ f$ is integrable with respect to $\mu$ if and only if $h$ is integrable with respect to $f_\# \mu$.
\end{proposition}

\begin{proof}
The idea of the proof is to first prove the statement for the case where $h$ is simple and nonnegative. Then we use the approximation result, Proposition~\ref{prop:approximation_simple}, and monotone convergence to prove the statement for general non-negative functions $h$. Finally, we use linearity of the integral to prove it for general functions.

Following this strategy, we first prove the statement when $h$ is simple and nonnegative, i.e.,
\[
h = \sum_{i=1}^N a_i \mathbf{1}_{A_i}
\]
for some $N \in \bbN$, $a_i \in (0,\infty)$, and $A_i \in \cF$ mutually disjoint. Then 
\[
	h \circ f = \sum_{i=1}^N a_i  \mathbf{1}_{f^{-1}(A_i)}.
\]
It follows that
\[
\begin{split}
\int_\Omega h \circ f\, \dd \mu 
= \sum_{i=1}^N a_i \, \mu(f^{-1}(A_i)) 
= \sum_{i=1}^N a_i \, (f_\# \mu)(A_i) 
= \int_E h \, \dd (f_\# \mu),
\end{split}
\]
which shows the proposition in the case when $h$ is simple and nonnegative. 

We now turn to the case in which $h$ is a general, nonnegative measurable function. Note that $[h]_n \circ f$ is a nondecreasing sequence of functions, which converges pointwise to $h \circ f$. Thus, by the monotone convergence theorem,
\[
\int_\Omega h \circ f\, \dd \mu 
= \lim_{n \to \infty} \int_\Omega [h]_n \circ f\, \dd \mu 
= \lim_{n \to \infty} \int_E [h]_n \, \dd (f_\# \mu) 
= \int_E h \, \dd (f_\# \mu).
\]

Finally, let $h: \Omega \to \bbR$ be a general measurable function and denote by $h^+$ and $h^-$ its positive and negative part, respectively. Then, since each of these is non-negative and $h \circ f = h^+ \circ f - h^- \circ f$, we get
\begin{align*}
	\int_\Omega h \circ f\, \dd \mu &= \int_\Omega h^+ \circ f\, \dd \mu - \int_\Omega h^- \circ f\, \dd \mu\\
	&= \int_E h^+ \, \dd (f_\# \mu) - \int_E h^- \, \dd (f_\# \mu) = \int_E h \, \dd (f_\# \mu).\qedhere
\end{align*}
\end{proof}

\begin{remark}[General proof strategy]
The architecture of the above proof is prototypical for many statements concerning Lebesgue integration. That is, we first proof it for simple non-negative functions. Then we extend this to general non-negative functions using monotone convergence and finally to general functions using linearity. Within this scheme, proving the required result for simple functions will be the main step, and hence the one that requires the most work.
\end{remark}

%As a direct consequence, we have the following proposition.
%\begin{proposition}
%Let $(\Omega, \cF, \mu)$ be a measure space and $(E,\cG)$ be a measurable space. Further, let $f\colon \Omega \to E$ and $h\colon E \to \bbR$ be $(\cF,\cG)$- and $(\cG,\cB_\bbR)$-measurable maps respectively. Then $h \circ f$ is integrable with respect to $\mu$ if and only if $h$ is integrable with respect to $f_\# \mu$, in which case,
%\[
%\int_\Omega h \circ f\, \dd \mu = \int_E h \, \dd (f_\# \mu).
%\]
%\end{proposition}




\section{Problems}

\begin{problem}\label{prb:standard_representation_simple}
Here you will prove Lemma~\ref{lem:simple_function}.
\begin{enumerate}[label=(\alph*)]
\item Construct a collection of sets $A_1,\dots, A_m$ such that $\omega \in A_i \iff f(\omega) = a_i$.
\item Prove that these are measurable and mutually disjoint sets.
\item Conclude that
\[
	f = \sum_{i = 1}^m a_i \mathbf{1}_{A_i}.
\]
\end{enumerate}
\end{problem}

\begin{problem}\label{prb:lebesgue_simple_welldefined}
Prove Lemma~\ref{lem:lebesgue_simple_welldefined}.
\end{problem}

\begin{problem}\label{prb:approximation_simple}
In this exercise we will prove Proposition~\ref{prop:approximation_simple}

\begin{enumerate}[label=(\alph*)]
\item Fix $n \ge 1$, set $N_n = n 2^n$ and define the sets
\[
	A_k^n = \begin{cases}
		\{ k 2^{-n} \le f < (k+1) 2^{-n}\} &\text{for } k = 0,1, \dots, N_n - 1,\\
		\{ f \ge n\} &\text{for } k = N_n,
	\end{cases}
\]

Prove that these are measurable and mutually disjoint.
\end{enumerate}

We now define the following function, as our candidate for the simple function that approximates $f$.
\[
	f_n = 2^n\mathbf{1}_{\{f=+\infty\}} + \sum_{k=0}^{N_n} k \, 2^{-n} \mathbf{1}_{A_k^n}.
\]

\begin{enumerate}[label=(\alph*)]
\setcounter{enumi}{1}
\item Show that $f_n$ is measurable and simple for every $n \ge 1$. Also note that $f_n(\omega) = k 2^{-n} \iff \omega \in A_k^n \text{ and } f(\omega) < +\infty$.
\item Show that $f_n \le f$ (point-wise) and that $f(w) - f_n(\omega) \le 2^{-n}$ holds for all $n \ge 1$.
\item Prove that for any $\omega \in \Omega$, $\lim_{n \to \infty} f_n(\omega) = f(\omega)$. [Hint: consider the case $f(\omega) = \infty$ and $f(\omega) < \infty$ separately.]
\end{enumerate}

The final thing to do is to show that $f_n$ is a point-wise monotone sequence, i.e. for any $\omega \in \Omega$, $f_n(\omega) \le f_{n+1}(\omega)$ holds for all $n \ge 1$.

\begin{enumerate}[label=(\alph*)]
\setcounter{enumi}{4}
\item Prove that $f_n(\omega) \le f_{n+1}(\omega)$ holds for all $\omega$ such that $f(\omega) = +\infty$.
\item Assume now that $f(\omega) < +\infty$ and that $\omega \in A_k^n$ for some $k < n 2^n$. Prove that $f_n(\omega) \le f_{n+1}(\omega)$. [Hint: split the interval $A_k^n$ into two parts of equal length and consider the case that $\omega$ is in one of these separately.]
\item Now consider the case that $\omega \in A_k^n$ with $k = n 2^n$. Show that $f_n(\omega) \le f_{n+1}(\omega)$.
\end{enumerate}

%Fix $n \ge 1$, set $N_n = n 2^n$ and define the sets
%\[
%	A_k^n = \begin{cases}
%		\{ k 2^{-n} \le f < (k+1) 2^{-n}\} &\text{for } k = 0,1, \dots, N_n - 1,\\
%		\{ f \ge n\} &\text{for } k = N_n,
%	\end{cases}
%\]
%which are measurable due to Lemma~\ref{lem:measurable_set_real_line} and mutually disjoint.
%
%
%From this representation, we easily deduce that $f_n$ is measurable and simple for every $n \ge 1$. Also note that $f_n(\omega) = k 2^{-n} \iff \omega \in A_k^n \text{ and } f(\omega) < +\infty$. 
%
%The first observation we make is that $f_n(\omega) \le f(\omega)$ holds for all $n \ge 1$. Moreover, if $f(\omega) < n$ then $f(w) - f_n(\omega) \le 2^{-n}$. 

%We will now show that $f_n(\omega) \to f(\omega)$ holds for any $\omega \in \Omega$. Let us fix a $\omega \in \Omega$. Then if $f(\omega) = +\infty$ we get that $f_n(\omega) = 2^n$ holds for all $n \ge 1$ and hence $\lim_{n \to \infty} f_n(\omega) = +\infty = f(\omega)$. So assume that $f(\omega) < +\infty$. Then there exists an $M \in \bbN$ such that $f(\omega) < M$. Hence, for all $n \ge M$ we have that 
%\[
%	\|f_n(\omega) - f(\omega)\| = f(\omega) - f_n(\omega) \le 2^{-n},
%\]
%which implies that $\lim_{n \to \infty} f_n(\omega) = f(\omega)$.
%
%For the final part we need to show that for any $\omega \in \Omega$, $f_n(\omega) \le f_{n+1}(\omega)$ holds for all $n \ge 1$. So fix $n \ge 1$ and $\omega \in \Omega$. Clearly, if $f(\omega) = +\infty$ there is nothing to prove. So let's assume that $f(\omega) < +\infty$. Then $\omega \in A_k^n$ for some $0 \le k \le N_n = n 2^n$. 
%
%We will first consider the case that $k < n 2^n$, so that $k 2^{-n} \le f(\omega) < (k+1) 2^{-n}$ holds. Note that this  interval can be split into two intervals as follows:
%\[
%	[k 2^{-n}, (k+1) 2^{-n}) = [(2k) 2^{-(n+1)}, (2k +1)2^{-(n+1)}) \cup [(2k +1)2^{-(n+1)}, (2k + 2)2^{-(n+1)}).
%\] 
%Hence, we conclude that either $\omega \in A_{2k}^{n+1}$ or $\omega \in A_{2k+1}^{n+1}$. In both case we get that 
%\[
%	f_n(\omega) = k2^{-n} = 2k n^{-(n+1)} \le f_{n+1}(\omega).
%\]
%
%Now let us consider the case that $k = n 2^n$, so that $f(\omega) \ge n$. Then, if $f(\omega) \ge n + 1$ it follows that $f_n(\omega) = n < n + 1 = f_{n+1}(\omega)$. If, on the other hand, $n \le f(\omega) < n + 1$ there exists an $2n \, 2^n \le \ell \le (2n+2) \, 2^n$ such that $\omega \in A_\ell^{n+1}$, which then implies that 
%\[
%	f_n(\omega) = n = (2n 2^{n}) \, 2^{-(n+1)} \le f_{n+1}(\omega).
%\]
\end{problem}

\begin{problem}
	Consider the measure space $(\bbN,2^{\bbN},\mu)$, where $\mu$ is the counting measure on $\bbN$. Show that for any function $f:\bbN\to[0,+\infty]$,
	\[
		\int_{\bbN} f\,d\mu = \sum_{n\ge 1} f(n).
	\]
\end{problem}

\begin{problem}\label{prb:properties-integral-nonneg}
Prove Proposition~\ref{prop:properties-integral-nonneg}. [Hint: first prove it for simple functions and then use the definition of the integral.]
\end{problem}

\begin{problem}
	\label{prb:simple-approx-integral}
 Let $(\Omega, \cF, \mu)$ be a measure space and let $f\colon (\Omega,\cF) \to ([0,+\infty), \cB_{[0,+\infty)})$ be a nonnegative measurable function. Show that
\[
\int_\Omega f\, \dd \mu = \lim_{n \to \infty} \int_\Omega [f]_n\, \dd \mu.
\]	
\end{problem}

\begin{problem}\label{prb:lebesgue_integral_general}
Proof Proposition~\ref{prop:properties-integral}.
\end{problem}

\begin{problem}\label{prb:measure}
	Let $(\Omega,\cF,\mu)$ be a measure space and suppose that $f$ is a non-negative $(\cF,\cB)$-measurable function such that $\int_\Omega f\,d\mu=1$. Define the set function $\nu_f\colon\cF\to[0,+\infty]$ by
	\[
		\nu_f(A):=\int_A f\,d\mu,\qquad \forall A\in\cF.
	\]
	\begin{enumerate}[label=(\alph*)]
		\item Show that $\nu_f$ is a probability measure on $(\Omega,\cF)$.
		\item Show that for all nonnegative $(\cF,\cB_{[0,+\infty]})$-measurable functions $g\colon\Omega\to [0,+\infty]$,
		\[
			\int_\Omega g\, d\nu_f = \int_\Omega g f\,d\mu.
		\]
		\textbf{Hint:} Start with simple functions and then approximate.
		\item Show that $g$ is $\nu_f$-integrable if and only if $g f$ is $\mu$-integrable, in which case
		\[
			\int_\Omega g\,d\nu_f = \int_\Omega g f\,d\mu.
		\]
	\end{enumerate}
\end{problem}


\begin{problem}
	Let $(\Omega,\cF,\mu)$ be a measure space and $\mu$ be a finite measure. Show that an $(\cF,\cB_\bbR)$-measurable function $f\colon\Omega\to \bbR$ is integrable if and only if
	\[
		\lim_{n\to\infty}\int_\Omega |f|\,\mathbbm{1}_{\{|f|\ge n\}}\,d\mu = 0.
	\]
\end{problem}


\begin{problem}[Continuity property of the integral]
	Let $(\Omega,\cF,\mu)$ be a measure space and $f$ be $\mu$-integrable. Show that for all $\varepsilon>0$ there exists $\delta>0$ such that
	\[
		\int_A |f|\,d\mu\le \varepsilon\quad \text{for all}\quad A\in\cF\quad \text{with}\quad \mu(A)<\delta.
	\]
	
\smallskip
	
	\noindent\textbf{Hint:} If $f$ is bounded, things are easy, so consider the set where $|f|$ is larger than some value and where $|f|$ is smaller than such value.
\end{problem}



