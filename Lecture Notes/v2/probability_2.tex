
Consider a sequence $(X_n)_{n \ge 1}$ of random variables. Similar to the setting of real numbers, we would like to have a notion of convergence of this sequence. In other words, we would like to say that $X_n \to X$ where $X$ is a different random variable. It turns out that there are different ways to define the concept of convergence of random variables. In this section, we will discuss three of them: convergence in distribution, convergence in probability, and almost sure convergence. While the last one has a more straightforward definition (see Definition~\ref{def:almost_sure_convergence}) the other two rely on the notion of weak convergence of finite measures, which discussed in Section~\ref{sec:weak_convergence_finite_measures}.

\section{Convergence in distribution}

Convergence in distribution of a sequence $(X_n)_{n \ge 1}$ is defined as weak convergence of the corresponding probability measures.

\begin{definition}[Convergence in distribution]\label{def:convergence_distribution}
Let $(X_n)_{n \ge 1}$ and $X$ be random variables, possibly defined on different probability spaces with probability measures $\bbP_n$ and $\bbP$, respectively. We say that $X_n$ \emph{converges in distribution} to $X$ if
\[
	(X_n)_\# \bbP_n \Rightarrow X_\# \bbP.
\]
If this is the case write we $X_n \stackrel{d}{\rightarrow} X$.
\end{definition}

Note that convergence in distribution of random variables is simply defined as weak convergence of their push-forward measures on $(\bbR, \cB_\bbR)$. This might seem strange to those who have encountered the \emph{more standard} definition used in courses on Probability Theory. There $X_n \stackrel{d}{\rightarrow} X$ if and only if
\[
	\lim_{n \to \infty} F_n(t) = F(t)
\]
holds for all continuity points $t$ of $F$, with $F_n$ and $F$ denoting the cdfs of $X_n$ and $X$ respectively.

But fear not; this definition is simply an equivalent statement for Definition~\ref{def:convergence_distribution}. 


\begin{lemma}\label{lem:convergence_distribution_cdfs}
Let $(X_n)_{n \ge 1}$ and $X$ be random variables and denote by, respectively, $F_n$ and $F$ their associated cdfs. Then
$X_n \stackrel{d}{\rightarrow} X$ if and only if
\[
	\lim_{n \to \infty} F_n(t) = F(t)
\]
holds for all continuity points $t$ of $F$.
\end{lemma}

\begin{proof}
See Problem~\ref{prb:convergence_distribution}.
\end{proof}



\section{Convergence in probability}

While convergence in distribution is can be understood as point-wise convergence of the cdfs (see Lemma~\ref{lem:convergence_distribution_cdfs}), convergence in probability looks at the difference $|X_n - X|$ as a random variable and requires this to converge to the constant zero random variable.

\begin{definition}[Convergence in probability]\label{def:convergence_probability}
Let $(X_n)_{n \ge 1}$ and $X$ be random variables that are defined on the \emph{same} probability space $(\Omega, \cF, \bbP)$ and define the random variable $Y_n := |X_n - X|$. We say that $X_n$ \emph{converges in probability} to $X$ if
\[
	(Y_n)_\# \bbP \Rightarrow 0_\# \bbP,
\] 
where $0$ denotes the constant function $\omega \mapsto 0$. If this is the case, we write $X_n \stackrel{\bbP}{\rightarrow} X$.
\end{definition}

\begin{remark}
Note that, unlike convergence in distribution, the concept of convergence in probability requires all random variables to be defined on the same probability space. This requirement can be relaxed a bit by having each $X_n$ be defined on a different probability space $(\Omega_n, \cF_n, \bbP_n)$ but have $X$ be defined on each of these spaces. From this perspective, we see that if we talk about convergence in probability to a constant $X_n \plim a \in \bbR$, then this is always true as constant random variables can be defined on any probability space.
\end{remark}

Recall that for a random variable $X$ on a probability space $(\Omega, \cF, \bbP)$ we wrote $\bbP(X \le t)$ as a short hand notation for $X_\# \bbP ((-\infty,t])$, i.e., the cdf of $X$ at $t$, and $\bbP(X > t)$ for $X_\# \bbP ((t,\infty))$, i.e., $1$ minus the cdf of $X$ at $t$.

The following result relates the definition of convergence in probability to a version that is presented in most probability courses.

\begin{lemma}\label{lem:convergence_probability_classical}
Let $(X_n)_{n \ge 1}$ and $X$ be random variables defined on the same probability space. Then $X_n \stackrel{\bbP}{\rightarrow} X$ if and only if
\[
	\lim_{n \to \infty} \bbP(\|X_n - X\| > \varepsilon) = 0 \quad \text{for every } \varepsilon > 0.
\]
\end{lemma}

\begin{proof}
See Problem~\ref{prb:convergence_probability_classic}
\end{proof}


The notion of convergence in probability is a stronger condition than convergence in distribution. In particular, the first statement implies the second. 

\begin{lemma}
Let $(X_n)_{n \ge 1}$ and $X$ be random variables such that $X_n \plim X$. Then $X_n \dlim X$.
\end{lemma}

\begin{proof}
We will use the equivalent definition given by Lemma~\ref{lem:convergence_probability_classical}. Denote by $F_n$ and $F$ the cdfs of $X_n$ and $X$, respectively. We will also write $\bbP(X > t)$. Let $t$ be a continuity point of $F$ and fix some $\varepsilon > 0$. First we note that if $X_n \le t$ and $|X-X_n|\le \varepsilon$ then $X \le t + \varepsilon$. We thus obtain
\begin{align*}
	\bbP(X_n \le t)
	&= \bbP(\{X_n \le t\} \cap \{|X_n - X| \le \varepsilon\}) + \bbP(\{X_n \le t\} \cap \{|X_n - X|>\varepsilon\})\\
	&\le \bbP(X \le t + \varepsilon) + \bbP(|X_n - X| > \varepsilon).
\end{align*}
Taking the limsup on both sides yields
\[
	\limsup_{n \to \infty} \bbP(X_n \le t)
	\le \bbP(X \le t + \varepsilon) + \limsup_{n \to \infty} \bbP(\|X_n - X| > \varepsilon)
	= \bbP(X \le t + \varepsilon),
\]
since $X_n \plim X$ implies that
\[
	\limsup_{n \to \infty} \bbP(|X_n - X| > \varepsilon)
	= \lim_{n \to \infty} \bbP(|X_n - X| > \varepsilon) = 0.
\]
Since $t$ is a continuity point of $F$ it follows that 
\[
	\lim_{\varepsilon \downarrow 0} \bbP(X \le t + \varepsilon) = \bbP(X \le t),
\]
which implies that $\limsup_{n \to \infty} \bbP(X_n \le t) \le \bbP(X \le t)$.

To prove the result, it now suffices to show that $\bbP(X \le t) \le \liminf_{n \to \infty} \bbP(X_n \le t)$.
We shall do this in a way that is similar to the case with the limsup. First observe that $X \le t - \varepsilon$ and $|X_n - X| \le \varepsilon$ implies that $X_n \le t$. This way we get
\begin{align*}
	\bbP(X \le t - \varepsilon)
	&= \bbP(\{X \le t - \varepsilon\} \cap \{|X_n -X|\le \varepsilon\}) + \bbP(\{X \le t - \varepsilon\} \cap \{|X_n - X|>\varepsilon\})\\
	&\le \bbP(X_n \le t) + \bbP(|X_n - X| > \varepsilon).
\end{align*}
Taking the liminf on both sides gives
\[
	\bbP(X \le t - \varepsilon)
	\le \liminf_{n \to \infty} \bbP(X_n \le t),
\]
and using that $t$ is a continuity point of $F_X$ we conclude that 
\[
	\bbP(X \le t) \le \liminf_{n \to \infty} \bbP(X_n \le t).\qedhere
\]
\end{proof}

While convergence in probability implies convergence in distribution, the other implication is not true in general (see Problem~\ref{prb:dlim_not_plim}). However, if $X$ is constant (deterministic instead of random) then both notions of convergence are equivalent.

\begin{lemma}{}\label{lem:dlim_constant_plim}
Let $(X_n)_{n \ge 1}$ be a sequence of random variables such that $X_n \dlim a$ for some constant $a \in \mathbb{R}$. Then we also have that $X_n \plim a$.
\end{lemma}

\begin{proof}
We again use the equivalent definition given by Lemma~\ref{lem:convergence_probability_classical}. Fix some $\varepsilon > 0$. We have to show that
\[
	\lim_{n \to \infty} \bbP(|X_n-a|>\varepsilon) = 0.
\]
Let $B_\varepsilon(a)$ denote the open ball of radius $\varepsilon$ around $a$ and consider the compliment $B_\varepsilon^c(a) := \mathbb{R} \backslash B_\varepsilon(a)$, which is a closed set. We then have
\[
	\bbP(|X_n-a|>\varepsilon) \le \bbP(|X_n-a|\ge\varepsilon) = \bbP(X_n \in B_\varepsilon^c(a)). 
\]
In particular we have
\[
	\lim_{n \to \infty} \bbP(|X_n-a|>\varepsilon) \le \limsup_{n \to \infty} \bbP(|X_n-a|>\varepsilon)
	\le \limsup_{n \to \infty} \bbP(X_n \in B_\varepsilon^c(a)).
\]
Since $X_n \dlim a$, statement 3 in Theorem~\ref{thm:portmanteau} implies that 
\[
	\limsup_{n \to \infty} \bbP(X_n \in B_\varepsilon^c(a))
	\le \limsup_{n \to \infty} \bbP(a \in B_\varepsilon^c(a)) = 0,
\]
because obviously $a \notin B_\varepsilon(a)^c$. Therefore we conclude that
\[
	\lim_{n \to \infty} \bbP(|X_n-a|>\varepsilon)
	\le \limsup_{n \to \infty} \bbP(a \in B_\varepsilon^c(a)) = 0
\]
which implies that $\lim_{n \to \infty} \bbP(|X_n-a|>\varepsilon) = 0$.
\end{proof}


\section{Almost-sure convergence}\label{ssec:almost_sure_convergence}

The final notion of convergence we will discuss is \emph{almost-sure convergence}, which looks much more natural than the previous two notions and requires less notation.

\begin{definition}[Almost-sure convergence]\label{def:almost_sure_convergence}
Let $(X_n)_{n \ge 1}$ and $X$ be random variables defined on the same probability space $(\Omega, \cF, \bbP)$. We say that $X_n$ \emph{converges almost-surely} to $X$ if
\[
	\bbP(\{\omega \in \Omega \, : \, \lim_{n \to \infty} X_n(\omega) = X(\omega)\}) = 1.
\] 
In this case we write $X_n \aslim X$.
\end{definition} 

The definition of almost-sure convergence says that the set for which $X$ is \emph{not} the limit of $X_n$ must have probability zero. This is why it is also often referred to as \emph{convergence with probability $1$}.

Note that similar to convergence in probability, almost-sure convergence by its definition only makes sense if all random variables are defined on the same probability space. 

There is a different way to characterize \emph{almost-sure} convergence which is often useful. This requires the concept of \emph{infinitely often}.

\begin{definition}[Infinitely often]\label{def:infinitely_often}
Let $(\Omega, \cF, \bbP)$ be a probability space and consider a sequence $(A_n)_{n \ge 1}$ of measurable sets. We then define the event $\{A_n \text{ i.o.}\}$ ($A_n$ happens infinitely often) as
\[
	\{A_n \text{ i.o.}\} := \bigcap_{k = 1}^\infty \bigcup_{n \ge k} A_n.
\]
\end{definition}

The use of the term \emph{infinitely often} for the event defined in Definition~\ref{def:infinitely_often} can be understood as follows: If $\{A_n \text{ i.o.}\} \ne \emptyset$ then term must exist $\omega \in \Omega$ such that $\omega \in \bigcup_{n \ge k} A_n$ for all $k \ge 1$. This implies that no mater how large you pick $k$, there will always be and $A_n$ with $n \ge k$ such that $\omega \in A_n$, i.e. there are infinitely many $A_n$ that share the same element $\omega$ and thus happen for this element.

The alternative definition of almost-sure convergence $X_n \aslim X$ is is that the probability that $|X_n - X|$ is arbitrarily small infinitely often is zero.

\begin{lemma}\label{lem:almost_sure_alternative}
Let $(X_n)_{n \ge 1}$ and $X$ be random variables defined on the same probability space $(\Omega, \cF, \bbP)$. Then
\[
	X_n \aslim X \iff \bbP(\|X_n - X\| > \varepsilon \text{ i.o.}) = 0 \quad \text{for all } \varepsilon > 0.
\]
\end{lemma}

\begin{proof}
Write $A_n(\varepsilon) :=  \{\|X_n - X\| > \varepsilon\}$ and $A = \{\omega \in \Omega \, : \, \lim_{n \to \infty} X_n(\omega) = X(\omega)\}$. We first observe that
\[
	\Omega\backslash A = \bigcup_{m \in \bbN} \{A_n(1/m) \text{ i.o.}\}.
\]

Now suppose that $X_n \aslim X$ and let $\varepsilon > 0$. Then $\bbP(A) = 1$ and there exist a $m \in \bbN$ such that $\{A_n(\varepsilon) \text{ i.o.}\} \subset \{A_n(1/m) \text{ i.o}\}$. Thus
\[
	\bbP(\{A_n(\varepsilon) \text{ i.o.}\}) \le \bbP(\{A_n(1/m) \text{ i.o.}\}) \le \bbP(\bigcup_{m \in \bbN} \{A_n(1/m) \text{ i.o.}\}) = \bbP(\Omega \backslash A) = 0.
\]

For the other implication suppose that $\bbP(\{A_n(\varepsilon) \text{ i.o.}\}) = 0$ for all $\varepsilon > 0$. Then clearly the same holds for all $\varepsilon = 1/m$ with $m \in \bbN$. Hence
\[
	\bbP(\Omega\backslash A) = \bbP(\bigcup_{m \in \bbN} \{A_n(1/m) \text{ i.o.}\}) \le \sum_{m \in \bbN} \bbP(\{A_n(1/m) \text{ i.o.}\}) = 0,
\]
which implies that $\bbP(A) = 1$.
\end{proof}

While this notion of convergence looks very natural, it turn out it is the strongest of the three notions. In practice proving almost-sure convergence is often much harder than proving convergence in probability or distribution.

\begin{lemma}\label{lem:almost_sure_implies_probability}
Let $(X_n)_{n \ge 1}$ and $X$ be random variables such that $X_n \aslim X$. Then $X_n \plim X$.
\end{lemma}

\begin{proof}
See Problem~\ref{prb:almost_sure_implies_probability}.
\end{proof}

We end this chapter with a classical result that provides sufficient conditions for proving some thing happens infinitely often.

\begin{lemma}[Borel-Cantelli]\label{lem:borel_cantelli}
Let $(\Omega, \mathcal{F}, \mu)$ be a measure space, and let $(A_n)_{n\in\bbN}$ be a family of measurable sets.
\begin{enumerate}[label={(\alph*)}]
\item If
\[
	\sum_{n=1}^\infty \mu(A_n) < \infty.
\]
then $\bbP(A_n \text{ i.o.}) = 0$.
\item If, $(A_n)_{n \ge 1}$ are mutually exclusive and,
\[
	\sum_{i=1}^\infty \mathbb{P}(A_i) = +\infty.
\]
then $\bbP(A_n \text{ i.o.}) = 1$.
\end{enumerate}
\end{lemma}

\begin{proof}
See Problem~\ref{prb:borel_cantelli}
\end{proof}


%\section{Law of Large Numbers}
%
%We end this section with proven a powerful result that pops up everywhere: The Strong Law of Large Numbers.
%
%\begin{lemma}[Borel-Cantelli]\label{lem:borel_cantelli}
%Let $(\Omega, \mathcal{F}, \mu)$ be a measure space, and let $(A_n)_{n\in\bbN}$ be a family of measurable sets.
%\begin{enumerate}[label={(\alph*)}]
%\item If
%\[
%	\sum_{n=1}^\infty \mu(A_n) < \infty.
%\]
%then $\bbP(A_n \text{ i.o.}) = 0$.
%\item If, $(A_n)_{n \ge 1}$ are mutually exclusive and,
%\[
%	\sum_{i=1}^\infty \mathbb{P}(A_i) = +\infty.
%\]
%then $\bbP(A_n \text{ i.o.}) = 1$.
%\end{enumerate}
%\end{lemma}
%
%\begin{proof}
%See Problem~\ref{prb:borel_cantelli}
%\end{proof}
%
%\begin{theorem}[Strong Law of Large Numbers]\label{thm:slln}
%Let $(X_n)_{n \ge 1}$ be a sequence of i.i.d. copies of a random variable $X$, such that $\bbE[X] < \infty$. Then
%\[
%	\frac{1}{n} \sum_{i = 1}^n X_n \aslim \bbE[X].
%\]
%\end{theorem}
%
%\begin{proof}
%The proof makes use of the Borel-Cantelli result as well as Lemma~\ref{lem:almost_sure_alternative}.
%
%We will show that the result holds for any sequence of non-negative $X_n$, i.e. $X_n \ge 0$ a.e. For a general $X_n$ we decompose it into its positive and negative part $X_n = X_n^+ - X_n^-$, each of which are non-negative. Then the result will hold due to linearity of the limit.
%
%So assume from now one that $X_n \ge 0$ and $X \ge 0$. We start with defining $Y_m := X_m \mathbf{1}_{X_m \le m}$, so that $(Y_m)_{m \ge 1}$ is a sequence of independent random variables (they are not identically distributed in general). In addition, take $\alpha > 1$, let $k_n = [\alpha^n]$ denote the integer part of $\alpha^n$ and define $S_t := \sum_{m = 1}^{t} Y_m$.
%
%Here are the keys steps of the proof:
%\begin{enumerate}
%\item We first show that $\bbP(Y_m \ne X_m \text{ i.o.}) = 0$.
%\item Then we prove that
%\[
%	\left|\frac{S_{k_n}}{k_n} - \frac{\bbE[S_{k_n}]}{k_n}\right| \aslim 0.
%\]
%\item Next we show that $\bbE[S_{k_n}]/k_n \to \bbE[X]$.
%\item Finally we conclude that 2) and 3) imply that 
%\[
%	\left|\frac{S_{n}}{n} - \bbE[X]\right| \aslim 0.
%\]
%\end{enumerate}
%
%We will first show how to finish the proof with these statements, before proving each of them. Observe that by Lemma~\ref{lem:almost_sure_alternative} we need to show that for all $\varepsilon > 0$
%\[
%	\bbP\left(\|\frac{1}{n} \sum_{i =1}^n X_i - \bbE[X]\| > \varepsilon \text{ i.o.}\right) = 0.
%\]
%Using the triangle inequality and Problem~\ref{prb:properties_io} we get
%\begin{align*}
%	\bbP\left(\|\frac{1}{n} \sum_{i =1}^n X_i - \bbE[X]\| > \varepsilon \text{ i.o.}\right)
%	&\le \bbP\left(\|\frac{1}{n} \sum_{i =1}^n Y_i - \bbE[X]\| > \varepsilon/2 \text{ i.o.}\right) \\
%	&\hspace{10pt}+ \bbP\left(\|\frac{1}{n} \sum_{i =1}^n (X_i - Y_i)\| > \varepsilon/2 \text{ i.o.}\right).
%\end{align*}
%The first probability is zero due to Lemma~\ref{lem:almost_sure_alternative} and 4), while the second probability is bounded by $\bbP(Y_m \ne X_m \text{ i.o.}) = 0$ by 1). 
%
%We are thus left to prove 1)-4). For this define $a_n := \bbE[S_{k_n}]/k_n$.
%\begin{enumerate}
%\item We will use that $\bbE[X] < \infty$ to show that 
%\[
%	\sum_{m \ge 1} \bbP(Y_m \ne X_m) < \infty,
%\]
%from which the result follows by Lemma~\ref{lem:borel_cantelli}.
%
%Recall the definition $Y_m = X_m \mathbbm{1}_{X_m \le m}$ and note that $\bbP(X>t) = \int_{(t,\infty)} \, \dd X_\# \bbP$. We now have that
%\begin{align*}
%	\sum_{m \ge 1} \bbP(Y_m \ne X_m) &= \sum_{m \ge 1} \bbP(X_m > m)\\
%	&= \sum_{m \ge 1} \int_{(m,\infty)} \, \dd X_\#\bbP \\
%	&= \sum_{m \ge 1} \sum_{t = m}^\infty \int_{(t,t+1]} \, \dd X_\#\bbP \\
%	&= \sum_{t \ge 1} t \int_{(t,t+1]} \, \dd X_\# \bbP \\
%	&\le \sum_{t \ge 1} \int_{(t,t+1]} x \, \dd X_\# \bbP \\
%	&\le \int_{(0,\infty)} x \, \dd X_\# \bbP = \bbE[X] < \infty.
%\end{align*}
%To see why the fourth equality is true, simply count for each $t$ how many times you will see the entry $\int_{(t,t+1]} \, \dd X_\#\bbP$ appear in the full summation. This will be $t$ times.
%
%\item For this step we will use Borel-Cantelli
%Let $\varepsilon > 0$. Note that for every $m \ge 1$, $\mathrm{Var}(Y_m) < \infty$. Then, using Chebyshev's inequality (see Problem ??] we get
%\begin{align*}
%	\sum_{n \ge 1} \bbP\left(\left|\frac{S_{k_n} - \bbE[S_{k_n}]}{k_n}\right| > \varepsilon\right)
%	&=  \sum_{n \ge 1} \bbP\left(\left|S_{k_n} - \bbE[S_{k_n}]\right| > k_n \varepsilon\right)\\
%	&\le \sum_{n \ge 1} \frac{\mathrm{Var}(S_{k_n})}{k_n^2 \varepsilon^2}\\
%	&\le \sum_{n \ge 1} \frac{1}{k_n^2 \varepsilon^2} \sum_{m =1}^{k_n} \mathrm{Var}(Y_m)\\
%	&\le \sum_{n \ge 1} \frac{1}{k_n^2 \varepsilon^2} \sum_{m =1}^{k_n} \bbE[Y_m^2]\\
%	&= \varepsilon^{-2} \sum_{n \ge 1} \sum_{m = 1}^{k_n} k_n^{-2} \bbE[Y_m^2]\\
%	&= \varepsilon^{-2} \sum_{m \ge 1} \bbE[Y_m^2] \sum_{n: k(n) \ge m} k_n^{-2}.
%\end{align*}
%
%To proceed we will show that
%\begin{equation}\label{eq:slln_geometric_sum_bound}
%	\sum_{n: k(n) \ge m} k_n^{-2} \le 4(1-\alpha^{-2})^{-1} m^{-2}.
%\end{equation}
%This requires us to recall some properties of the geometric sequence $\sum_{t = 0}^m r^{t}$ for $r \in (0,1)$. 
%\begin{equation}\label{eq:geometric_series_partialsum}
%	\sum_{t = 0}^m r^{t} = \frac{1 - r^{m + 1}}{1 - r.}
%\end{equation}
%
%Using~\eqref{eq:slln_geometric_sum_bound} and writing $C := 4\varepsilon^{-2}(1-\alpha^{-2})^{-1}$ we have
%\begin{align*}
%	\sum_{n \ge 1} \bbP\left(\left|\frac{S_{k_n} - \bbE[S_{k_n}]}{k_n}\right| > \varepsilon\right)
%	&\le C \sum_{m \ge 1} \frac{\bbE[Y_m^2]}{m^2} \\
%	&\le C \sum_{m \ge 1} m^{-2} \int_{(0,\infty)} 2y^2 \, \dd (Y_m)_\# \bbP \\
%	&\le C \sum_{m \ge 1} m^{-2} \int_{(0,\infty)} 2y^2 \mathbbm{1}_{y \le m} \, X_\# \bbP (\dd y) \\
%	&= C \int_{(0, \infty)} \left(\sum_{m \ge 1} m^{-2} \mathbbm{1}_{y \le m}\right) 2 y^2 \, X_\# \bbP (\dd y).
%\end{align*}
%Here the last equality is due to Fubini (note that all terms are positive).
%
%Now we note that 
%\[
%	\sum_{m \ge 1} m^{-2} \mathbbm{1}_{y \le m} = \sum_{m \ge y} m^{-2} \le 2/y.
%\]
%Hence
%\[
%	\sum_{n \ge 1} \bbP\left(\left|\frac{S_{k_n} - \bbE[S_{k_n}]}{k_n}\right| > \varepsilon\right)
%	\le 4C \int_{(0,\infty)} \, y  X_\# \bbP(\dd y) = 4C \bbE[X] < \infty,
%\]
%which is what we needed to show for 2).
%
%\item 
%
%\end{enumerate}
%
%
%\end{proof}
%
%A useful tool for proving almost everywhere convergence is the following result.
%
%\begin{lemma}[Borel-Cantelli]\label{lem:Borel-Cantelli}
%	Let $(\Omega, \mathcal{F}, \mu)$ be a measure space, and let $(A_n)_{n\in\bbN}$ be a family of measurable sets. If
%	\[
%	\sum_{n=1}^\infty \mu(A_n) < \infty,
%	\]
%	then for $\mu$-almost every $\omega \in \Omega$, there are only finitely many $n \in \mathbb{N}$ such that $\omega \in A_n$.
%\end{lemma}
%\begin{proof}
%	Define the sets
%	\[
%		B_j := \bigcup_{i \ge j} A_i,\qquad j\in\bbN.
%	\]
%	Clearly the sequence $(B_j)_{j\in\bbN}$ is decreasing and $\{A_n \text{ i.o.}\}\subset B_j$ for every $j \in \mathbb{N}$. 
%	
%	By assumption, and the $\sigma$-subadditivity of $\mu$,
%	\[
%	\mu(B_1) = \mu\left(\bigcup_{i=1}^\infty A_i \right) \leq \sum_{i=1}^\infty \mu(A_i) < +\infty.
%	\]
%	Moreover, the summability also gives
%	\[
%		\lim_{j\to\infty}\mu(B_j) \leq \limsup_{j\to\infty}\sum_{i=j}^\infty \mu(A_i) = 0.
%	\]
%	Hence, by the continuity from above of $\mu$, we obtain
%	\[
%		\mu(\{A_n \text{ i.o.}\}) \le \mu\left(\bigcup_{j=1}^\infty B_j\right) = \lim_{j\to \infty} \mu(B_j) = 0,
%	\]
%	i.e., $\{A_n \text{ i.o.}\}$ is a null set. In other words, $\mu$-almost every $\omega$ is in only finitely many $A_n$.
%\end{proof}
%
%\begin{theorem}[Borel-Cantelli Lemma II]
%	Let $(\Omega, \cF, \mathbb{P})$ be a probability space, and let $A_i \in \cF$, $i \in \mathbb{N}$ be a sequence of independent events such that
%	\[
%		\sum_{i=1}^\infty \mathbb{P}(A_i) = +\infty.
%	\]
%	Then
%	\[
%	\mathbb{P} \left( \bigcap_{m = 1}^\infty \bigcup_{i=m}^\infty A_i \right) = 1,
%	\]
%	i.e., infinitely many of the events occur almost surely.
%\end{theorem}
%\begin{proof}
%	Let $m \in \mathbb{N}$. Note that
%	\begin{align*}
%	\mathbb{P}\left(\bigcap_{i=m}^{\infty} (\Omega \backslash A_i) \right) &= 
%	\prod_{i=m}^\infty (1 - \mathbb{P}(A_i)) \\
%	&\leq \prod_{i=m}^\infty \exp(- \mathbb{P}(A_i)) 
%	= \exp\left(- \sum_{i=m}^\infty \mathbb{P}(A_i)\right) = 0.\qedhere
%	\end{align*}
%\end{proof}   

\section{Problems}

\begin{problem}\label{prb:dlim_not_plim}
Give an example (with proof) of a sequence of random variables that converge in distribution but not in probability.
\end{problem}



\begin{problem}\label{prb:convergence_distribution}
The goal of this problem is to prove Lemma~\ref{lem:convergence_distribution_cdfs}. That is
\[
	X_n \stackrel{d}{\rightarrow} X \iff F_n(t) \to F(t) \quad \text{for all } t \in \cC_F,
\]
where $F_n$ and $F$ denote the cdfs of the random variables $X_n$ and $X$, respectively.

Write $\mu_n = (X_n)_\# \bbP_n$ and $\mu = X_\# \bbP$ and note that for any measurable function $f$, $\bbE[f(X_n)] = \int f \, \dd \mu_n$ and $\bbE[f(X)] = \int f \, \dd \mu$.

We will first prove the $\Rightarrow$ implication. 
\begin{enumerate}[label={(\alph*)}]
\item Let $t \in \bbR$. Find a measurable function $h_t$, such that $F_n(t) = \int_\bbR h_t \, \dd \mu_n$ and $F(t) = \int_\bbR h_t \, \dd \mu$.
\item Show that $\mu(\cC_{h_t}) = 1$.
\item Prove the $\Rightarrow$ implication. 
\end{enumerate}

For the other implication $\Leftarrow$ let $g$ be a continuous function with compact support. Then there is some $K > 0$ such that $g$ is zero outside $[-K,K]$. You may use the fact that any continuous function with compact support is uniformly continuous, i.e. for every $\varepsilon > 0$ there exist a $\delta > 0$ such that $\|x-y\| < \delta$ implies that $\|g(x) - g(y)\| < \varepsilon$. 

For now let us fix some $\varepsilon > 0$ and let $\delta > 0$ be such that $\|x-y\| < \delta$ implies that $\|g(x) - g(y)\| < \varepsilon$. We then consider the partition of $[-K,K]$ into $L = L(\delta) := \lceil (4K/\delta) \rceil$ intervals $I_\ell = (a_\ell, b_\ell$ of equal length. This then implies that $b_\ell - a_\ell \le \delta/2 < \delta$.

We will now define an approximate function
\[
	\hat{g}(x) = \sum_{\ell = 1}^L g(b_\ell) \mathbf{1}_{I_\ell}(x).
\]
 
\begin{enumerate}[label={(\alph*)}]
\setcounter{enumi}{3}
\item Show that $|g(x) - \hat{g}(x)| \le \varepsilon$ holds for all $x \in [-K , K]$.
\item Show that there exists an $M$ and sequences $(\beta_m)_{1 \le m \le M}$ and $(t_m)_{1 \le m \le M}$ such that
\[
	\hat{g}(x) = \sum_{m = 1}^M \beta_m \mathbf{1}_{(-\infty, t_m]}.
\]
\item Prove that $\lim_{n \to \infty} \bbE[\hat{g}(X_n)] = \bbE[\hat{g}(X)]$. [Hint: Use the assumption $F_n(t) \to F(t)$ for all $t \in \cC_F$.]
\item Prove that $\bbE[g(X_n)] \to \bbE[g(X)]$. [Hint: First use the previous result to show that $\| \int g \, \dd \mu_n - \int g \, \dd \mu\| \to 2\varepsilon$ by adding and subtracting $\int \hat{g} \, \dd \mu_n$ and $\int \hat{g} \, \dd \mu$.]
\item Conclude that $X_n \dlim X$.
\end{enumerate}
\end{problem}

\begin{problem}\label{prb:convergence_probability_classic}
Prove Lemma~\ref{lem:convergence_probability_classical}.
\end{problem}

\begin{problem}\label{prb:properties_io}
Let $(A_n)_{n \ge 1}$, $(B_n)_{n \ge 1}$ be two families of measurable sets. 
\begin{enumerate}[label={(\alph*)}]
\item Suppose that $A_n \subseteq B_n$ holds for all $n \ge 1$. Prove that $\{A_n \text{ i.o.}\} \subseteq \{B_n \text{ i.o.}\}$.
\item Prove that $\{A_n \cup B_n \text{ i.o.}\} = \{A_n \text{ i.o.}\} \cup \{B_n \text{ i.o.}\}$.
\end{enumerate}
\end{problem}

\begin{problem}\label{prb:almost_sure_implies_probability}
Prove Lemma~\ref{lem:almost_sure_implies_probability}. [Hint: use the alternative definition of Lemma~\ref{lem:almost_sure_alternative} and reverse Fatou from Problem~\ref{prb:reverse_fatou}.]
\end{problem}

\begin{problem}[Borel-Cantelli]\label{prb:borel_cantelli}
In this exercise you will prove Lemma~\ref{lem:borel_cantelli}.

\begin{enumerate}[label={(\alph*)}]
\item Suppose that
\[
	\sum_{n=1}^\infty \mu(A_n) < \infty.
\]
Prove that $\bbP(A_n \text{ i.o.}) = 0$.
\item Suppose that $A_n \cap A_m = \emptyset$ for all distinct $n,m \ge 1$,
\[
	\sum_{i=1}^\infty \mathbb{P}(A_i) = +\infty.
\]
Prove that $\bbP(A_n \text{ i.o.}) = 1$.
\end{enumerate}
\end{problem}

\begin{problem}
Let $(X_n)_{n \ge 1}$ and $X$ be random variables defined on the same probability space. Prove that $X_n \aslim X$ if for every $\varepsilon > 0$ it holds that
\[
	\sum_{n \ge 1} \bbP(|X_n - X| > \varepsilon) < \infty.
\]
[Hint: Use the alternative definition involving infinitely often and Borel-Cantelli.
\end{problem}
