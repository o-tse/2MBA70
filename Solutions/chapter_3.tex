\textbf{Problem 3.2}
``$\subset$" By definition, the product $\sigma$-algebra $\cF_1\otimes\cF_2$ is defined as the $\sigma$-algebra generated by the collection
\[
	\cA := \Bigl\{ A\times B\subset \Omega_1\times\Omega_2 : A\in\cF_1,\, B\in\cF_2\Bigr\}.
\]
Since $A\times B = (A\times\Omega_2) \cap (\Omega_1\times B)$, we have that
\[
	A\times B = \pi_1^{-1}(A)\cap\pi_2^{-1}(B) \in \sigma(\pi_1,\pi_2).
\]

``$\supset$" Let $C\in \{ \pi_i^{-1}(A) : i=1,2,\, A\in\cF_1\}$. Then there exist sets $A\in \cF_1$ or $B\in \cF_2$ such that $C = \pi_1^{-1}(A) = A\times\Omega_2$ or $C= \pi_2^{-1}(B) = \Omega_1\times B$. Either way, since $\Omega_1\in\cF_1$ and $\Omega_2\in\cF_2$, we have that $C\in\cA$.

\bigskip


\textbf{Problem 3.3}
It is clear that $f_{\#}\mu(\emptyset) = \mu(f^{-1}(\emptyset)) = \mu(\emptyset) = 0$. 
	Suppose a sequence of mutually disjoint sets $B_i \in \mathcal{G}$, $i \in \mathbb{N}$, is given. Then,
	\[
	f_{\#} \mu\left( \bigcup_{i=1}^\infty B_i \right) = \mu\left( f^{-1} \left(\bigcup_{i=1}^\infty B_i \right) \right) = \mu \left( \bigcup_{i=1}^\infty  f^{-1}(B_i) \right) = \sum_{i=1}^\infty f_{\#}\mu (B_i). \qedhere
	\]
	
\bigskip

\textbf{Problem 3.5}
\begin{enumerate}[label=(\alph*)]
	\item Some meaningful explanation would suffice.
	
	\item By Proposition~2.1.8 and Problem~2.9, we know that $\mathcal{B}_\mathbb{R}$ is generated by intervals of the form $(-\infty,a]$ with $a \in \mathbb{Q}$. 
As a consequence, $\mathcal{B}_\mathbb{R}$ is also generated by intervals of the form $(a, +\infty)$ with $a \in \mathbb{Q}$.
Therefore, by Lemma~3.1.4, it suffices to show that the set
\[
	\bigl\{ \omega \in \Omega : f(\omega) + g(\omega) \in (a,+\infty) \bigr\}
\]
is measurable for every $a \in \mathbb{Q}$. For brevity, we write $\{ f + g > a\}$. The trick is to express this set as a countable union of sets of which we already know are measurable.

In fact, we will show that
\[
	\{ f + g > a\} = \bigcup_{t \in \mathbb{Q}} \Big(\{ f > t \} \cap \{ g > a - t  \} \Big).
\]
We first show the inclusion `$\subset$'. If $\omega \in \Omega$ is such that
\[
	f(\omega) + g(\omega) > a,
\]
then 
\[
	f(\omega) > a - g(\omega),
\]
so there exists some $t \in \mathbb{Q}$ such that
\[
	f(\omega) > t > a - g(\omega),
\]
and thus $f(\omega) > t$ and $g(\omega) > a - t$.
So in that case
\[
	\omega \in \bigcup_{t \in \mathbb{Q}}\Big(\{ f > t \} \cap \{ g > a - t  \}\Big).
\]
Now we will show the inclusion `$\supset$'. 
Let $\omega \in \Omega$ be such that $f(\omega) > t$ and $g(\omega) > a - t$. 
Then, by adding the inequalities, we know that $f(\omega) + g(\omega) > a$. 

	\item The constant function $f(\omega)=a$ is measurable since 
	\[
		f^{-1}(B) = f^{-1}(B\cap\{a\})\cup f^{-1}(B\setminus \{a\}) = \Omega\cup \emptyset = \Omega\in \cF\qquad\forall\,B\in\cB_\bbR.
	\]
	
	\item Similar to the proof of Point (2) of Proposition~3.2.12.
	\item Let $g(\omega)\ne 0$ for all $\omega\in\Omega$. Then, since $g$ is measurable, we have that
	\begin{align*}
		\{ 1/g > a\} &= \{ g < 1/a,\;g>0\}\cup \{ g > 1/a,\; g<0\} \\
		&= \Bigl(\{ g < 1/a\}\cap \{g>0\}\Bigr) \cup \Bigl(\{ g > 1/a\}\cap\{g<0\}\Bigr) \in \cF,
	\end{align*}
	thus implying that $1/g$ is measurable.
	
	\item Point (e) and Point (4) of Proposition~3.2.12 yields Point (5) of Proposition~3.2.12.
\end{enumerate}

\bigskip

\textbf{Problem 3.6} 
From (3.6), we have for any $a\in\bbR$,
\[
	\left\{ \sup_{n\ge 1} f_n>a\right\} = \bigcup_{n\ge 1} \bigl\{ f_n>a\bigr\}\in\cF,
\]
Since $\cF$ is a $\sigma$-algebra and $f_n$ is measurable for all $n\ge 1$, i.e., $\{f_n>a\}\in\cF$ for all $n\ge 1$.


\bigskip

\textbf{Problem 3.7}

\begin{enumerate}[label=(\alph*)]
	\item Note that
	\[
		f_M = M\mathbf{1}_{\{f\ge M\}} + f\mathbf{1}_{\{|f|< M\}} - M\mathbf{1}_{\{f\le -M\}}.
	\]
	Since the sets
	\[
		\{f\ge M\},\quad \{f\le -M\},\quad \{|f|<M\}\qquad\text{are $\cF$-measurable},
	\]
	their corresponding indicator functions are $(\cF,\cB_{\bbR})$-measurable. Since $f_M$ is the sum of products of $(\cF,\cB_{\bbR})$-measurable functions, we conclude that $f_M$ is also $(\cF,\cB_{\bbR})$-measurable.
	
	\item It is easy to see that $f_M$ converges pointwise to $f$ as $M\to\infty$, i.e.,
	\[
		\lim_{M\to\infty} f_M(\omega) = f(\omega)\qquad\forall\, \omega\in\Omega.
	\]
	Indeed, if $\omega\Omega$ is such that $f(\omega)=+\infty$, then
	\[
		\lim_{M\to\infty} f_M(\omega) = \lim_{M\to\infty} M = +\infty = f(\omega),
	\]
	and similarly for $\omega\in\Omega$ for which $f(\omega)=-\infty$. On the other hand, for any $\omega\in\Omega$ with $f(\omega)\in\bbR$, there is some $N_0(\omega)\in\bbN$ such that $f_N(\omega)=f(\omega)$ for all $N\ge N_0(\omega)$, and hence,
	\[
		\lim_{M\to\infty} f_M(\omega) = f(\omega).
	\]
	Since $f$ is the limit of a sequence of $(\cF,\cB_{\bbR})$-measurable functions, we conclude from Lemma~3.2.13 that $f$ is $(\cF,\cB_{\bbR})$-measurable.
\end{enumerate}

\bigskip
\textbf{Problem 3.9}

\begin{enumerate}[label=(\alph*)]
\item For the probability space, take $\Omega = [0,1]$, $\cF = \cB_{[0,1]}$ and $\bbP = \lambda$ the Lebesgue measure restricted to $[0,1]$. 

Observe that the function $H_\gamma(z)$ is continuous and hence has an inverse $g_\gamma(y) = \gamma \tan(\pi(y - 1/2))$ on $[0,1]$.

So the function $Y [0,1] \to \bbR$ defined by $Y(x) = g_\gamma(x)$ has the correct distribution as
\[
	\bbP(Y^{-1}((-\infty,t])) = \bbP(g_\gamma^{-1}((-\infty, t])) = \lambda(H_\gamma((-\infty,t])) = H_\gamma(t).
\]
\item Note that $g_\gamma$ is continuous on $[0,1]$ and hence measurable.
\item For any $t \ge 0$, the cdf of the Poisson random variable is given by
\[
	F_\lambda(t) = \sum_{n = 0}^{\lceil t \rceil} f_\lambda(n),
\]
where $\lceil t \rceil$ is the ceiling of $t$, i.e. the smallest integer $k \ge t$.
\item For the probability space, we again take $\Omega = [0,1]$, $\cF = \cB_{[0,1]}$ and $\bbP = \lambda$ the Lebesgue measure restricted to $[0,1]$. 

Now for any $y \in [0,1]$ let $k := k(y)$ be such that
\[
	\sum_{n = 1}^k f_\lambda(n) \ge y \quad \text{and} \quad \sum_{n = 1}^{k-1} f_\lambda(n) < y,
\]
where the last sum is interpreted as $-1$ if $k=0$.

Now define $X(y) = k(y) : [0,1] \to \bbR$. Then $k(y) \le t$ if and only if $y \le F_\lambda(t)$ and hence
\[
	X^{-1}((-\infty,t]) = \{y \in [0,1] \, : \, k(y) \in (0,t]\}
	= \{y \in [0,1] \, : \, y \in (0, F_\lambda(t)]\},
\]
from which it follows that
\[
	\bbP(X^{-1}((-\infty,t])) = \lambda((0, F_\lambda(t)]) = F_\lambda(t).
\]
\item It follows from the above computation that $X^{-1}((-\infty,t]) = \{y \in [0,1] \, : \, y \in (0, F_\lambda(t)]\}$. Since the latter is a measurable set we conclude that $X^{-1}((-\infty,t])$ is measurable for all $t$ and since these generate the Borel \sigalg/ $X$ is measurable.
\item for any $\ell \in \bbN$ define the sets $A_\ell = (n-1-1/\ell), n-1 + 1/\ell]$. Then $A_\ell$ is a decreasing set with $\lim_{\ell \to \infty} A_\ell = \{n\}$. Moreover, $A_\ell = (-\infty,n-1+1/\ell] \setminus (-\infty, n-1-1/\ell]$ and $\bbP(A_1) < \infty$. It now follows from continuity from above and (d) that
\begin{align*}
	X_\#\bbP(\{n\}) &= \lim_{\ell \to \infty} X_\# \bbP(A_\ell) \\
	&= \lim_{\ell \to \infty} X_\# \bbP((-\infty,n-1+1/\ell])
	- X_\# \bbP((-\infty,n-1-1/\ell]) \\
	&= F_\lambda(n-1+1/\ell) - F_\lambda(n-1-1/\ell) \\
	&= \sum_{k = 0}^{n} f_\lambda(k) - \sum_{k = 0}^{n-1} f_\lambda(k) = f_\lambda(n).
\end{align*}
\end{enumerate}

