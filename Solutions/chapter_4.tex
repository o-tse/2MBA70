
\textbf{Problem 4.2}

The idea is to apply the monotone convergence theorem (Theorem 4.3.4). To this end we first note that
\[
	\| f_n(\omega) - f(\omega) \| \le 2^{-n} \quad \text{for all } n \in \bbN, \, \omega \in \Omega.
\] 

From this it follows that $f_n(\omega) \le 2^{-n} + f(\omega)$ and hence
\begin{align*}
	\|[f_n](\omega) - f(\omega)\| 
	&= \|2^n - f(\omega)\| \mathbf{1}_{2^n \le f_n} + \|f_n(\omega) - f(\omega)\|\mathbf{1}_{f_n < 2^n}\\
	&\le 2^{-n} + 2^{-n}
\end{align*}
form which we conclude that $[f_n] \to f$.

The final part is to show that $[f_n] \le [f_{n+1}]$ which follows if we can show that $f_n \le f_{n+1}$. For this we first note that for all $k \ge 1$ $(k+1) 2^{-(n+1)} \le k 2^{-n}$. We also note that $2^{n} \le 2{n+1}$. Now suppose that there exist an $n \ge 1$ and $\omega$ such that $f_n(\omega) > f_{n + 1}(\omega)$. Then it must hold that $f_n(\omega) > 0$ and hence $f_n(\omega) = k 2^{-n}$ for some $k \ge 1$. This then implies that $f_{n+1}(\omega) = \ell 2^{-n}$ for some $\ell \ge k+1$. But this cannot be the case as $[\ell 2^{-n}, (\ell+1)2^{-n}) \cap [k 2^{-n}, (k+1)2^{-n}) = \emptyset$ while $f(\omega)$ should be in both sets.


\bigskip
\textbf{Problem 4.3}

\begin{enumerate}[label={(\alph*)}]
	\item By definition, we have that $\nu_f(\Omega) = \int_\Omega f\,\dd\mu = 1$. Now let $(A_n)_{n\in\bbN}$ be a family of mutually disjoint measurable sets. Then we have that the sequence 
	\[
		g_n:= \sum_{i=1}^n f\,\mathbf{1}_{A_i} = f\,\mathbf{1}_{\bigcup_{i=1}^n\! A_i}\;\longrightarrow\; g:= f\,\mathbf{1}_{\bigcup_{i\in\bbN}\! A_i}\qquad\text{pointwise monotonically}.
	\]
	By MCT, we then have that
	\[
		\nu_f\left(\bigcup_{i\in\bbN} A_i\right) = \int_{\bigcup_{i\in\bbN}\! A_i} f\,\dd\mu = \lim_{n\to\infty} \int_{\bigcup_{i=1}^n\! A_i} f\,\dd\mu = \lim_{n\to\infty} \sum_{i=1}^n \int_{A_i} f\,\dd\mu = \sum_{i\in\bbN}\nu_f(A_i),
	\]
	thus showing that $\nu_f$ is a probability measure on $(\Omega,\cF)$.
	
	\item Following the hint, we start by considering nonnegative simple functions $g$. Suppose $g=\sum_{i=1}^n a_i \mathbf{1}_{A_i}$ for $a_i\in\bbR$ and $A_i\in\cF$ mutually disjoint. Then,
	\[
		\int_\Omega g\,d\nu_f = \sum_{i=1}^n a_i =\,\nu_f(A_i) = \sum_{i=1}^n a_i\int_{A_i} f\,\dd\mu = \int_\Omega gf\,\dd\mu.
	\]
	Now let $g$ be a nonnegative measurable function and $[g]_n$ be a sequence of nonnegative simple functions that converge pointwise monotonically to $g$. Then MCT yields
	\[
		\int_\Omega g\,\dd\nu_f = \lim_{n\to\infty}\int_\Omega [g]_n\,\dd\nu_f = \lim_{n\to\infty} \int_\Omega [g]_n f\,\dd\mu = \int_\Omega gf\,\dd\mu,
	\]
	where we used the fact that $[g]_n f$ converges pointwise monotonically to $gf$.
	
	\item Let $g$ be measurable. Then $g=g^+-g^-$, where $g^\pm$ are nonnegative measurable functions. Since $f$ is nonnegative, we have that $(fg)^\pm = f g^\pm$. Due to (b), we deduce
	\[
		\int_\Omega g^\pm\,\dd\nu_f = \int_\Omega g^\pm f\,\dd\mu = \int_\Omega (gf)^\pm\,\dd\mu.
	\]
	Hence, $g^\pm$ is $\nu_f$-integrable if and only if $(gf)^\pm$ is $\mu$-integrable. Consequently, $g$ is $\nu_f$-integrable if and only if $gf$ is $\mu$-integrable, since
	\[
		\int_\Omega |g|\,\dd\nu_f = \int_\Omega g^+\,\dd\nu_f + \int_\Omega g^-\,\dd\nu_f = \int_\Omega g^+f\,\dd\mu + \int_\Omega g^-f\,\dd\mu = \int_\Omega |gf|\,\dd\mu.
	\]
\end{enumerate}

\bigskip
\textbf{Problem 4.4}

\paragraph{($\Rightarrow$)} Let $f$ be $\mu$-integrable. Then both $|f|\mathbf{1}_{\{|f|<n\}}$ and $|f|\mathbf{1}_{\{|f|\ge n\}}$ are integrable, due to the monotonicity of the integral. By linearity of the integral,
\[
	\int_\Omega |f|\mathbf{1}_{\{|f|\ge n\}}\,\dd\mu = \int_\Omega |f|\,\dd\mu - \int_\Omega |f|\mathbf{1}_{\{|f|< n\}}\,\dd\mu.
\]
Since the sequence $g_n:= |f|\mathbf{1}_{\{|f|< n\}}\ge 0$ converges pointwise monotonically to $|f|$, we can apply MCT to obtain
\[
	\lim_{n\to\infty} \int_\Omega |f|\mathbf{1}_{\{|f|< n\}}\,dd\mu = \int_\Omega |f|\,\dd\mu.
\]
Hence,
\[	
	\lim_{n\to\infty}\int_\Omega |f|\mathbf{1}_{\{|f|\ge n\}}\,\dd\mu = \int_\Omega |f|\,\dd\mu - \lim_{n\to\infty}\int_\Omega |f|\mathbf{1}_{\{|f|< n\}}\,\dd\mu = 0.
\]

\paragraph{($\Leftarrow$)} By assumption, there is some $N\ge 1$ such that
\[
	\int_\Omega |f|\mathbf{1}_{\{|f|\ge N\}}\,\dd\mu \le 1.
\]
By linearity of the integral,
\[
	\int_\Omega |f|\,\dd\mu = \int_\Omega |f|\mathbf{1}_{\{|f|< N\}}\,\dd\mu +\int_\Omega |f|\mathbf{1}_{\{|f|\ge N\}}\,\dd\mu \le N \mu\bigl(\{|f|< N\}\bigr) + 1.
\]
Since $\mu$ is a finite measure, the right-hand side is finite, implying that $f$ is $\mu$-integrable.


\bigskip
>>>>>>> 6734bbef461f3cf1f37b2ba1c0f732f410c6fb0d
\textbf{Problem 4.6}

\begin{enumerate}[label=(\alph*)]
\item Let $t \in \bbR$ and consider the set $A_t = (-\infty, t]$. Then by definition of the probability density function
\[
	\nu(A_t) = \int_{-\infty}^t \rho \, \dd \lambda = (X_\# \bbP)((-\infty, t]).
\]
We thus conclude that $\nu$ and $X_\# \bbP$ coincide on the family of set $A_t$ and since these generate $\cB$ Theorem 2.2.17 implies that $\nu = X_\# \bbP$.
\item Since $g$ is a simple function, there exist an $N \in \bbN$, constants $(a_n)_{1 \le n \le N}$ and measurable sets $(A_n)_{1 \le n \le N}$ such that
\[
	g = \sum_{n = 1}^N a_n \mathbf{1}_{A_n}.
\]

Now, by first applying Proposition 4.8.11 and then part (a), we get that
\begin{align*}
	\bbE[g(X)] &= \int_{\Omega} g(X) \, \dd \bbP
		= \int_{\Omega} g \, \dd X_\# \bbP 
		= \int_\Omega g \, \dd \nu \\
	&= \int_\Omega \sum_{n = 1}^N a_n \mathbf{1}_{A_n} \, \dd \nu 
		= \sum_{n = 1}^N a_n \nu(A_n) 
		= \sum_{n = 1}^N a_n \int_{A_n} \rho \, \dd \lambda \\
	&= \int_\bbR \sum_{n = 1}^N a_n \mathbf{1}_{A_n} \rho \, \dd \lambda = \int_\bbR g \rho \, \dd \lambda
\end{align*}

\item First note that by part (b) we have that
\[
	\int_\Omega [h]_n(X) \, \dd \bbP = \int_\bbR [h_n] \rho \, \dd \lambda.
\]
Now we split the function $[h_n] \rho$ into its positive and negative part and note that
\[
	([h_n] \rho)^+ = [h]_n^+ \rho^+ + [h]_n^- \rho^- \quad \text{and} \quad
	([h_n] \rho)^- = [h]_n^+ \rho^- + [h]_n^- \rho^+,
\] 
where $[h]_n^\pm$ and $\rho^\pm$ denote the positive and negative parts of $[h]_n$ and $\rho$.

We will show that 
\[
	\int_\Omega h^+(X) \, \dd \bbP = \int_\bbR h^+ \rho \, \dd \lambda. 
\]
The proof for the negative part is similar.

\begin{align*}
	\int_\bbR h^+ \, \dd \nu &= \lim_{n \to \infty} \int_\bbR [h]_n^+ \, \dd \nu &&\text{by Theorem 4.3.4}\\
	&= \lim_{n \to \infty} \int_\bbR [h]_n^+ \rho \, \dd \lambda &&\text{by part (b)} \\
	&= \lim_{n \to \infty} \int_\bbR [h]_n^+ \rho^+ \, \dd \lambda
		- \lim_{n \to \infty} \int_\bbR [h]_n^+ \rho^- \, \dd \lambda &&\text{by linearity of integration} \\
	&= \int_\bbR h+ \rho^+ \, \dd \lambda - \int_\bbR h+ \rho^- \, \dd \lambda &&\text{by Theorem 4.3.4} \\
	&= \int_\bbR h^+ \rho \, \dd \lambda &&\text{by linearity of integration}
\end{align*}
\item 
\begin{align*}
	\bbE[h(X)] &= \int_\Omega h(X) \, \dd \bbP \\
	&= \int_\bbR h \, \dd X_\# \bbP &&\text{by Proposition 4.8.11}\\
	&= \int_\bbR h \, \dd \nu &&\text{by part (a)}\\
	&= \int_\bbR h \rho \, \dd \lambda &&\text{by part (c)}.
\end{align*}
\end{enumerate}   
