\documentclass{lecturenotes}

\usepackage{lecture_notes}

% Reset mathcal back to previous style (not curly)
\DeclareMathAlphabet{\mathcal}{OMS}{cmsy}{m}{n}



%%%%%%%%%%%%%%%%%%%%%%%%%%%%%%%%%%%%%%%%%%%%%%%%%%%%%%%%%%%%%%%%%%%%%%%%%%%%%%%%%%%%%%%%%%%%%%%%%%%
%										For leaving comments									  %
%%%%%%%%%%%%%%%%%%%%%%%%%%%%%%%%%%%%%%%%%%%%%%%%%%%%%%%%%%%%%%%%%%%%%%%%%%%%%%%%%%%%%%%%%%%%%%%%%%%

%Name:			XXX
%Description:	Adds a piece of text in blue, surrounded by [] brackets. 
%				#1: the person the comment is addressed to
%				#2: the person the comment is from
%				#3: the comment
%Usage:			Write \XXX{collaborator}{myName}{myComment} to write [myComment] addressed to [collaborator]
\newcommand{\XXX}[3]{{\color{blue} \textbf{ [#1:  #3 \textit{ -#2-} ]}}}

%%%%%%%%%%%%%%%%%%%%%%%%%%%%%%%%%%%%%%%%%%%%%%%%%%%%%%%%%%%%%%%%%%%%%%%%%%%%%%%%%%%%%%%%%%%%%%%%%%%
%									     BibTeX commands 									      %
%%%%%%%%%%%%%%%%%%%%%%%%%%%%%%%%%%%%%%%%%%%%%%%%%%%%%%%%%%%%%%%%%%%%%%%%%%%%%%%%%%%%%%%%%%%%%%%%%%%

%Name:			Swap
%Description:	Command for properly typesetting "van der" and related expressions in Dutch and German names in the
%				bibliography. It swaps [van der] and [Lastname] in [van der Lastname] so that ordering will be performed on 
%				[Lastname] instead of [van der].
%Usage:			Write \swap{Lastname}{~van~der~}, Firstname instead of [van der Lastname, Firstname] in the author header
%				of the bibtex entry.

\newcommand*{\swap}[2]{\hspace{-0.5ex}#2#1}


%%%%%%%%%%%%%%%%%%%%%%%%%%%%%%%%%%%%%%%%%%%%%%%%%%%%%%%%%%%%%%%%%%%%%%%%%%%%%%%%%%%%%%%%%%%%%%%%%%%
%											Document											  %
%%%%%%%%%%%%%%%%%%%%%%%%%%%%%%%%%%%%%%%%%%%%%%%%%%%%%%%%%%%%%%%%%%%%%%%%%%%%%%%%%%%%%%%%%%%%%%%%%%%

\setlength\parindent{0pt}

\begin{document}

\textbf{Problem 10.1}

Let $\nu\ll \mu$. We show the required statement by contradiction. Therefore, let us suppose otherwise, i.e., there exists $\varepsilon>0$ such that for any $\delta>0$, there is some $A\in\mathcal{F}$ for which
\[
	\mu(A)<\delta,\quad\text{but}\;\;\nu(A)\ge \varepsilon.
\]
We can choose a sequence of decreasing sets $A_n$, i.e., $A_{n+1}\subset A_n\subset A_1$, such that $\mu(A_n)<1/n$ and $\nu(A_n)\ge \varepsilon$ for all $n\ge 1$. By continuity from above, we have that
\[
	\mu\Bigl(\bigcap_{n\ge 1} A_n\Bigr) = \lim_{n\to\infty} \mu(A_n) = 0,\qquad \nu\Bigl(\bigcap_{n\ge 1} A_n\Bigr) = \lim_{n\to\infty} \nu(A_n) \ge \varepsilon
\]
i.e., $\bigcap_{n\ge 1} A_n$ is a $\mu$-null set. Since $\nu$ is absolutely continuous w.r.t.\ $\mu$, then $\bigcap_{n\ge 1} A_n$ is also a $\nu$-null set, which contradicts with the deduced lower bound.

\end{document}