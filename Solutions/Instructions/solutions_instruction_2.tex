\documentclass{lecturenotes}

\usepackage{lecture_notes}

% Reset mathcal back to previous style (not curly)
\DeclareMathAlphabet{\mathcal}{OMS}{cmsy}{m}{n}



%%%%%%%%%%%%%%%%%%%%%%%%%%%%%%%%%%%%%%%%%%%%%%%%%%%%%%%%%%%%%%%%%%%%%%%%%%%%%%%%%%%%%%%%%%%%%%%%%%%
%										For leaving comments									  %
%%%%%%%%%%%%%%%%%%%%%%%%%%%%%%%%%%%%%%%%%%%%%%%%%%%%%%%%%%%%%%%%%%%%%%%%%%%%%%%%%%%%%%%%%%%%%%%%%%%

%Name:			XXX
%Description:	Adds a piece of text in blue, surrounded by [] brackets. 
%				#1: the person the comment is addressed to
%				#2: the person the comment is from
%				#3: the comment
%Usage:			Write \XXX{collaborator}{myName}{myComment} to write [myComment] addressed to [collaborator]
\newcommand{\XXX}[3]{{\color{blue} \textbf{ [#1:  #3 \textit{ -#2-} ]}}}

%%%%%%%%%%%%%%%%%%%%%%%%%%%%%%%%%%%%%%%%%%%%%%%%%%%%%%%%%%%%%%%%%%%%%%%%%%%%%%%%%%%%%%%%%%%%%%%%%%%
%									     BibTeX commands 									      %
%%%%%%%%%%%%%%%%%%%%%%%%%%%%%%%%%%%%%%%%%%%%%%%%%%%%%%%%%%%%%%%%%%%%%%%%%%%%%%%%%%%%%%%%%%%%%%%%%%%

%Name:			Swap
%Description:	Command for properly typesetting "van der" and related expressions in Dutch and German names in the
%				bibliography. It swaps [van der] and [Lastname] in [van der Lastname] so that ordering will be performed on 
%				[Lastname] instead of [van der].
%Usage:			Write \swap{Lastname}{~van~der~}, Firstname instead of [van der Lastname, Firstname] in the author header
%				of the bibtex entry.

\newcommand*{\swap}[2]{\hspace{-0.5ex}#2#1}


%%%%%%%%%%%%%%%%%%%%%%%%%%%%%%%%%%%%%%%%%%%%%%%%%%%%%%%%%%%%%%%%%%%%%%%%%%%%%%%%%%%%%%%%%%%%%%%%%%%
%											Document											  %
%%%%%%%%%%%%%%%%%%%%%%%%%%%%%%%%%%%%%%%%%%%%%%%%%%%%%%%%%%%%%%%%%%%%%%%%%%%%%%%%%%%%%%%%%%%%%%%%%%%

\setlength\parindent{0pt}

\begin{document}

\textbf{Problem 3.2}

\begin{enumerate}[label=(\alph*)]
\item First we note that $f^{-1}(\emptyset) = \emptyset \in \cF$ and $f^{-1}(E) = \Omega \in \cF$. So $\emptyset, E \in \cH$. 

Next, let $B \in \cH$. Then
\[
	f^{-1}(E\setminus B) = \Omega \setminus f^{-1}(B) \in \cF,
\]
since by definition $f^{-1}(B) \in \cF$. So $E\setminus B \in \cH$.

Finally, if $(B_i)_{i \in \bbN}$ is a sequence of sets in $\cH$, then
\[
	f^{-1}\left(\bigcup_{i = 1}^\infty B_i\right) = \bigcup_{i = 1}^\infty f^{-1}(B_i) \in \cF,
\]
which shows that $\bigcup_{i = 1}^\infty B_i \in \cH$, completing the proof that $\cH$ is a \sigalg/.
\item By construction $\cA \subseteq \cH$. It therefore follows from Lemma 2.5 that $\cG = \sigma(\cA) \subseteq \cH$. But this then implies that $f^{-1}(B) \in \cF$ for each $B \in \cG$ which means that $f$ is $(\cF, \cG)$-measurable.
\end{enumerate}

\bigskip

\textbf{Problem 3.6}
\begin{enumerate}[label=(\alph*)]
	\item By Proposition~2.8, we know that $\mathcal{B}_\mathbb{R}$ is generated by intervals of the form $(-\infty,a]$ with $a \in \mathbb{Q}$. 
As a consequence, $\mathcal{B}_\mathbb{R}$ is also generated by intervals of the form $(a, +\infty)$ with $a \in \mathbb{Q}$.
Therefore, by Lemma~3.1.4, it suffices to show that the set
\[
	\bigl\{ \omega \in \Omega : f(\omega) + g(\omega) \in (a,+\infty) \bigr\}
\]
is measurable for every $a \in \mathbb{Q}$. For brevity, we write $\{ f + g > a\}$. The trick is to express this set as a countable union of sets of which we already know are measurable.

In fact, we will show that
\[
	\{ f + g > a\} = \bigcup_{t \in \mathbb{Q}} \Big(\{ f > t \} \cap \{ g > a - t  \} \Big).
\]
We first show the inclusion `$\subset$'. If $\omega \in \Omega$ is such that
\[
	f(\omega) + g(\omega) > a,
\]
then 
\[
	f(\omega) > a - g(\omega),
\]
so there exists some $t \in \mathbb{Q}$ such that
\[
	f(\omega) > t > a - g(\omega),
\]
and thus $f(\omega) > t$ and $g(\omega) > a - t$.
So in that case
\[
	\omega \in \bigcup_{t \in \mathbb{Q}}\Big(\{ f > t \} \cap \{ g > a - t  \}\Big).
\]
Now we will show the inclusion `$\supset$'. 
Let $\omega \in \Omega$ be such that $f(\omega) > t$ and $g(\omega) > a - t$. 
Then, by adding the inequalities, we know that $f(\omega) + g(\omega) > a$. 

	\item The constant function $f(\omega)=a$ is measurable since 
	\[
		f^{-1}(B) = f^{-1}(B\cap\{a\})\cup f^{-1}(B\setminus \{a\}) = \Omega\cup \emptyset = \Omega\in \cF\qquad\forall\,B\in\cB_\bbR.
	\]
	
	\item Similar to the proof of Point (2) of Proposition~3.2.12.
	\item Let $g(\omega)\ne 0$ for all $\omega\in\Omega$. Then, since $g$ is measurable, we have that
	\begin{align*}
		\{ 1/g > a\} &= \{ g < 1/a,\;g>0\}\cup \{ g > 1/a,\; g<0\} \\
		&= \Bigl(\{ g < 1/a\}\cap \{g>0\}\Bigr) \cup \Bigl(\{ g > 1/a\}\cap\{g<0\}\Bigr) \in \cF,
	\end{align*}
	thus implying that $1/g$ is measurable.
	
	\item The previous part of this exercise together with point (4) of Proposition~3.12 yields Point (5) of Proposition~3.12.
\end{enumerate}

\end{document}