\documentclass{lecturenotes}

\usepackage{lecture_notes}

% Reset mathcal back to previous style (not curly)
\DeclareMathAlphabet{\mathcal}{OMS}{cmsy}{m}{n}



%%%%%%%%%%%%%%%%%%%%%%%%%%%%%%%%%%%%%%%%%%%%%%%%%%%%%%%%%%%%%%%%%%%%%%%%%%%%%%%%%%%%%%%%%%%%%%%%%%%
%										For leaving comments									  %
%%%%%%%%%%%%%%%%%%%%%%%%%%%%%%%%%%%%%%%%%%%%%%%%%%%%%%%%%%%%%%%%%%%%%%%%%%%%%%%%%%%%%%%%%%%%%%%%%%%

%Name:			XXX
%Description:	Adds a piece of text in blue, surrounded by [] brackets. 
%				#1: the person the comment is addressed to
%				#2: the person the comment is from
%				#3: the comment
%Usage:			Write \XXX{collaborator}{myName}{myComment} to write [myComment] addressed to [collaborator]
\newcommand{\XXX}[3]{{\color{blue} \textbf{ [#1:  #3 \textit{ -#2-} ]}}}

%%%%%%%%%%%%%%%%%%%%%%%%%%%%%%%%%%%%%%%%%%%%%%%%%%%%%%%%%%%%%%%%%%%%%%%%%%%%%%%%%%%%%%%%%%%%%%%%%%%
%									     BibTeX commands 									      %
%%%%%%%%%%%%%%%%%%%%%%%%%%%%%%%%%%%%%%%%%%%%%%%%%%%%%%%%%%%%%%%%%%%%%%%%%%%%%%%%%%%%%%%%%%%%%%%%%%%

%Name:			Swap
%Description:	Command for properly typesetting "van der" and related expressions in Dutch and German names in the
%				bibliography. It swaps [van der] and [Lastname] in [van der Lastname] so that ordering will be performed on 
%				[Lastname] instead of [van der].
%Usage:			Write \swap{Lastname}{~van~der~}, Firstname instead of [van der Lastname, Firstname] in the author header
%				of the bibtex entry.

\newcommand*{\swap}[2]{\hspace{-0.5ex}#2#1}


%%%%%%%%%%%%%%%%%%%%%%%%%%%%%%%%%%%%%%%%%%%%%%%%%%%%%%%%%%%%%%%%%%%%%%%%%%%%%%%%%%%%%%%%%%%%%%%%%%%
%											Document											  %
%%%%%%%%%%%%%%%%%%%%%%%%%%%%%%%%%%%%%%%%%%%%%%%%%%%%%%%%%%%%%%%%%%%%%%%%%%%%%%%%%%%%%%%%%%%%%%%%%%%

\setlength\parindent{0pt}

\begin{document}

\textbf{Problem 7.6}

\begin{enumerate}[label={(\alph*)}]
\item 
\item By definition
\[
	\lim_{n\to\infty} \left|\int_\bbR f \, \dd \mu_n - \int_\bbR f \, \dd \mu\right| \le \varepsilon,
\]
implies that for any $\delta > 0$
\[
	\left|\int_\bbR f \, \dd \mu_n - \int_\bbR f \, \dd \mu\right| < \varepsilon + \delta,
\]
holds for large enough $n$. Note that this holds for any $\varepsilon, \delta > 0$.

Now pick $\eta > 0$ and set $\varepsilon = \eta/2 = \delta$, then the above inequality implies that
\[
	\lim_{n\to\infty} \left|\int_\bbR f \, \dd \mu_n - \int_\bbR f \, \dd \mu\right| = 0.
\]

\item Consider the sequence of sets $A_n = \bbR\setminus [-n,n]$. Then $A_n \supset A_{n + 1}$ and $A_n \downarrow \emptyset$. Hence, it follows from Proposition 2.12 2) that $\lim_{n \to \infty} \mu(A_n) = 0$. Thus, there exists a $N$ such that $\mu(A_n) < \varepsilon/(2M)$ holds for all $n \ge N$. We can then take any $\alpha > N$.
\item The function
\[
	g(x) = \mathbbm{1}_{[-\alpha,\alpha]}(x) + \mathbbm{1}_{(-(\alpha+1), -\alpha)}(x)\left(x + (\alpha + 1)\right) + \mathbbm{1}_{(\alpha,\alpha+1)}(x)\left(-x + \alpha + 1\right)
\]
does the trick. This is simply a linear increase from zero to one from $-(\alpha + 1)$ to $-\alpha$ and from $\alpha + 1$ to $\alpha$.
\item Observe that $g$ is a non-negative continuous bounded function that is zero outside the interval $[-(\alpha+1), \alpha+1]$, and thus we can apply (3).
Using linearity of the integral, the fact that $|f| \le M$ and the definition of $g$, we get
\begin{align*}
	\left|\int_\bbR f \, \dd \mu - \int_\bbR fg \, \dd \mu\right|
	&= \left|\int_\bbR f (1-g) \, \dd \mu\right| \le M \int_\bbR (1-g) \, \dd \mu \\
	&\le M \int_\bbR (1-g) \, \dd \mu \\
	&= M\left(1-\int_\bbR g \, \dd \mu\right) \\
	&\le  M \mu(\bbR \backslash [-\alpha,\alpha]) < \frac{\varepsilon}{2}.
\end{align*}
\item Again, using linearity of the integral and the fact that $|f| \le M$ we get
\begin{align*}
	\left|\int_\bbR f \, \dd \mu_n - \int_\bbR fg \, \dd \mu_n\right|
	&= \left|\int_\bbR f (1-g) \, \dd \mu_n\right| \le M \int_\bbR (1-g) \, \dd \mu_n \\
	&\le M \int_\bbR (1-g) \, \dd \mu_n = M\left(1-\int_\bbR g \, \dd \mu_n\right)
\end{align*}

Now observe that the integral in the last term converges to $\int_\bbR g \, \dd \mu$ by (3). Thus, we obtain
\begin{align*}
	\limsup_{n \to \infty} \left|\int_\bbR f \, \dd \mu_n - \int_\bbR fg \, \dd \mu_n\right|
	&\le M \int_\bbR (1-g) \, \dd \mu \le M \mu(\bbR \backslash [-\alpha,\alpha]) < \frac{\varepsilon}{2}.
\end{align*}
\item 
Recall that
\begin{align*}
	\left|\int_\bbR f \, \dd \mu_n - \int_\bbR f \, \dd \mu\right| &\le \left|\int_\bbR f \, \dd \mu_n - \int_\bbR fg \, \dd \mu_n\right| + \left|\int_\bbR f \, \dd \mu - \int_\bbR fg \, \dd \mu\right|\\ &\hspace{10pt}+ \left|\int_\bbR fg \, \dd \mu_n - \int_\bbR fg \, \dd \mu\right|.
\end{align*}
For the first two terms, the (e) and (f) imply that the $\limsup_{n \to \infty}$ is bounded by $\varepsilon/2$. For the third term we not that $fg$ is a continuous bounded function and hence this term converges to zero by our assumption that (3) holds.

Together we then have that
\[
	\limsup_{n \to \infty} \left|\int_\bbR f \, \dd \mu_n - \int_\bbR f \, \dd \mu\right| < \varepsilon,
\]
which implies the result.
\end{enumerate}

\end{document}