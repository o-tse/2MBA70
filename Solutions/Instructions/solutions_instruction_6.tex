\documentclass{lecturenotes}

\usepackage{lecture_notes}

% Reset mathcal back to previous style (not curly)
\DeclareMathAlphabet{\mathcal}{OMS}{cmsy}{m}{n}



%%%%%%%%%%%%%%%%%%%%%%%%%%%%%%%%%%%%%%%%%%%%%%%%%%%%%%%%%%%%%%%%%%%%%%%%%%%%%%%%%%%%%%%%%%%%%%%%%%%
%										For leaving comments									  %
%%%%%%%%%%%%%%%%%%%%%%%%%%%%%%%%%%%%%%%%%%%%%%%%%%%%%%%%%%%%%%%%%%%%%%%%%%%%%%%%%%%%%%%%%%%%%%%%%%%

%Name:			XXX
%Description:	Adds a piece of text in blue, surrounded by [] brackets. 
%				#1: the person the comment is addressed to
%				#2: the person the comment is from
%				#3: the comment
%Usage:			Write \XXX{collaborator}{myName}{myComment} to write [myComment] addressed to [collaborator]
\newcommand{\XXX}[3]{{\color{blue} \textbf{ [#1:  #3 \textit{ -#2-} ]}}}

%%%%%%%%%%%%%%%%%%%%%%%%%%%%%%%%%%%%%%%%%%%%%%%%%%%%%%%%%%%%%%%%%%%%%%%%%%%%%%%%%%%%%%%%%%%%%%%%%%%
%									     BibTeX commands 									      %
%%%%%%%%%%%%%%%%%%%%%%%%%%%%%%%%%%%%%%%%%%%%%%%%%%%%%%%%%%%%%%%%%%%%%%%%%%%%%%%%%%%%%%%%%%%%%%%%%%%

%Name:			Swap
%Description:	Command for properly typesetting "van der" and related expressions in Dutch and German names in the
%				bibliography. It swaps [van der] and [Lastname] in [van der Lastname] so that ordering will be performed on 
%				[Lastname] instead of [van der].
%Usage:			Write \swap{Lastname}{~van~der~}, Firstname instead of [van der Lastname, Firstname] in the author header
%				of the bibtex entry.

\newcommand*{\swap}[2]{\hspace{-0.5ex}#2#1}


%%%%%%%%%%%%%%%%%%%%%%%%%%%%%%%%%%%%%%%%%%%%%%%%%%%%%%%%%%%%%%%%%%%%%%%%%%%%%%%%%%%%%%%%%%%%%%%%%%%
%											Document											  %
%%%%%%%%%%%%%%%%%%%%%%%%%%%%%%%%%%%%%%%%%%%%%%%%%%%%%%%%%%%%%%%%%%%%%%%%%%%%%%%%%%%%%%%%%%%%%%%%%%%

\setlength\parindent{0pt}

\begin{document}

\textbf{Problem 6.5}

Define for any $j \in \mathbb{Z}$, $p_j := \bbP(X^{-1}(\{j\}))$. Then, since $(X^{-1}(j))_{j \in \mathbb{Z}}$ is a family of disjoint sets and $\bbP$ is a probability measure we get that
\[
	1 = \bbP(\Omega) = \bbP(\bigcup_{j \in \mathbb{Z}} X^{-1}(j)) = \sum_{j \in \mathbb{Z}} p_j.
\]

Now let $A \subset \bbR$ be a measurable set and note that 
\[
	X^{-1}(A) = \bigcup_{j \in \mathbb{Z} \cap A} X^{-1}(j).
\]
Then it follows that
\[
	\bbP(X \in A) = \bbP(X^{-1}(A)) = \bbP\left(\bigcup_{j \in \mathbb{Z} \cap A} X^{-1}(j)\right)
	= \sum_{j \in \mathbb{Z} \cap A} p_j = \sum_{j \in \mathbb{Z}} \delta_j(A) p_j. 
\]


\bigskip

\textbf{Problem 6.8}

\begin{enumerate}[label=(\alph*)]
\item This follows from the following computation
\[
	\int_\Omega |f|^p \, \dd \mu \ge \int_{\Omega} |f|^p \mathbbm{1}_{|f| \ge t} \, \dd \mu
	\ge t^p \int_\Omega \mathbbm{1}_{|f| \ge t} \, \dd \mu
	= t^p \mu(\{\omega \in \Omega \, : \, |f| \ge t\}).
\]
\item Using the result for $p = 1$ we get
\[
	\bbP(|X| \ge t) = \mu(\omega \in \Omega \, : \, |X(\omega) \ge t \})
	\le \frac{1}{t} \int_{\Omega} X \, \dd \bbP = \frac{1}{t} \bbE[X].
\]
\item Take $f(\omega) = X(\omega) - \bbE[X]$, which is measurable. Then using the first result with $p = 2$ gives
\begin{align*}
	\bbP(|X - \bbE[X]| \ge t) &= \bbP(|X-\bbE[X]|^2 \ge t^2) \\
	&\le \frac{1}{t^2} \bbE[(X - \bbE[X])^2] \\
	&= \frac{1}{t^2} (\bbE[X^2] - \bbE[X]^2) = \frac{\mathrm{Var}(X)}{t^2}.
\end{align*}
\end{enumerate}


\end{document}