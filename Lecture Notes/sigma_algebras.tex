
\section{Recalling basic probability theory}\label{sec:recalling_probability_theory}

During the first course on probability theory, Probability and Modeling (2MBS10), the concept of probabilities were introduced. The idea here (in its simplest version) is that you have space $\Omega$ of possible outcomes of an experiment, and you want to assign a value in $[0,1]$ to each set $A$ of possible outcomes that represents the \emph{probability} that the experiment will yield an outcome in this set $A$. This value was then denoted by $\prob{A}$. 

It turned out that in order to properly define these concepts, we needed to impose structure on both the space of events as well as on the probability measure. For example, if we had two sets $A, B$ of possible outcomes, would like to say something about the probability that the outcome is in either $A$ or $B$. This means we not only do we need to be able compute $\prob{A \cup B}$, we actually want that $A \cup B$ is also an event in our space $\Omega$. Another example concerned the probability of the outcome not being in $A$, which means compute the probability of the event $\Omega\setminus A$, requiring that this set should also be in $\Omega$. In the end this prompted the definition of an \emph{event space} which was a collection $\cF$ of subsets of $\Omega$ such that
\begin{enumerate}
\item $\cF$ is non-empty;
\item If $A \in \cF$, then $A^c := \Omega \setminus A \in \cF$;
\item If $A_1, A_2, \dots \in \cF$, then $\bigcup_{i = 1}^\infty A_i \in \cF$.
\end{enumerate}

In addition, the probability assignment $\bbP$ was defined as a map $\bbP : \cF \to [0,1]$ such that
\begin{enumerate}
\item $\prob{\Omega} = 1$ and $\prob{\emptyset} = 0$, and
\item for any collection $A_1, A_2, \dots$ of disjoint events in $\cF$ it holds that 
\[
	\prob{\bigcup_{i = 1}^\infty A_i} = \sum_{i = 1}^\infty \prob{A_i}.
\]
\end{enumerate}

With this setup it was possible to formally define what a \emph{random variable} is. Here a random variable $X$ was defined on a triple $(\Omega, \cF, \bbP)$, consisting of a space of outcomes, an event space and a probability on that space. Formally it is a mapping $X : \Omega \to \bbR$ such that for each $x \in \bbR$ the set $X^{-1}(-\infty,x):=\{\omega \in \Omega \, : \, X(\omega) \in (-\infty,x)\}$ is in $\cF$. This then allowed us to define the \emph{cumulative distribution function} as $F_X(x) := \prob{X^{-1}(-\infty,x)}$.

It is important to note here that already it was needed to make a distinction of how to define a discrete and a continuous random variable. In addition, a separate definition was required to defined multivariate distribution functions. This limits the extend to which this theory can be applied. For example let $U$ have the uniform distribution on $[0,1]$ and $Y$ have uniform distribution on the set $\{1,2, \dots, 10\}$ and define the random variable $X$ to be equal to $U$ with probability $1/2$ and equal to $Y$ with probability $1/2$. How would you deal with this random variable, which is both discrete and continuous? However, the setting would becomes even more complex if we are not talking about random numbers in $\bbR$ but, say, random vectors of infinite length or random functions. Do these even exist?

The solutions to all these issues comes from a generalization of event spaces and probability measures introduced above. These go by the names sigma-algebra and measure, respectively. With this we can then define when any mapping between spaces is \emph{measurable} and use such mappings to define random objects in that space such a function maps to. The remainder of this chapter is dedicated to introduced all these concepts.


\section{Sigma-algebras}

% Definition sigma algebra

\subsection{Definition and examples}

We begin this section with introducing the general structure needed on a collection of sets to be able to assign a notion of measurement to them. Such a collection is called a sigma-algebra, often written as $\sigma$-algebra.

\begin{definition}[Sigma Algebra]\label{def:sigma_algebra}
A \emph{$\sigma$-algebra} $\cF$ on a set $\Omega$ is a collection of subsets of $\Omega$ with the following properties:
\begin{enumerate}
\item $\emptyset \in \cF$ and $\Omega \in \cF$;
\item For every $A \in \cF$, it holds that $A^c := \Omega \setminus A \in \cF$;
\item For every sequence $A_1, A_2, \dots \in \cF$, it holds that $\bigcup_{i = 1}^\infty A_i \in \cF$.
\end{enumerate}
\end{definition}

This definition looks very similar to that of an event space. It turns out that they are actually the same, see Problem~\ref{prb:event_space_are_sigma_algebras}. 

Before we give some examples, we first show that any \sigalg/ is also closed under countable intersections. The proof is left as an exercise to the reader (see Problem~\ref{prb:sigma_algebra_closed_intersections}).

\begin{lemma}\label{lem:sigma_algebra_closed_intersections}
Let $\cF$ be a \sigalg/ on $\Omega$ and let $A_1, A_2, \dots \in \cF$. Then it holds that $\bigcap_{i = 1}^\infty A_i \in \cF$
\end{lemma}

We now give examples and non-examples of \sigalgs/.

\begin{example}[(Non-)Examples of $\sigma$-algebras]\label{example:sigma_algebras}
\hfil
\begin{enumerate}
\item The collection $\cF = \{\emptyset, \Omega\}$ is a $\sigma$-algebra. This is called the \emph{trivial $\sigma$-algebra} or the \emph{minimal} $\sigma$-algebra on $\Omega$.
\item For any subset $A \subset \Omega$ we have that $\cF := \{\emptyset, A, \Omega\setminus A, \Omega\}$ is a $\sigma$-algebra.
\item The \emph{power set} $\cP(\Omega)$ (the collection of all possible subsets of $\Omega$) is a $\sigma$-algebra. This is sometimes called the \emph{maximal $\sigma$-algebra} on $\Omega$.
\item For any subset $A \subset \Omega$ such that $A \ne \emptyset, \Omega$, we have that $\cF := \{\emptyset, A, \Omega\}$ is \textbf{not} a $\sigma$-algebra.
\item Let $\Omega = [0,1]$ and $\cF$ be the collections of finite unions of intervals of the form $[a,b]$, $[a,b)$, $(a,b]$ and $(a,b)$ for $0 \le a < b \le 1$. Then $\cF$ is \textbf{not} a $\sigma$-algebra.
\item Let $f : \Omega \to \Omega^\prime$ and let $cF^\prime$ be a \sigalg/ on $\Omega^\prime$. Then the collection
\[
	\cF := f^{-1}(\cF^\prime) = \{f^{-1}(A^\prime) \, : \, A^\prime \in \cF^\prime\},
\]
is a \sigalg/. The converse to this is not always true, see Problem~\ref{prb:converse_preimage_sigma_algebra}.
\end{enumerate}
\end{example}

The idea of measure theory is that one can assign a notion of measure to each set in a $\sigma$-algebra. In line with this, a pair $(\Omega, \cF)$ where $\Omega$ is a set and $\cF$ a \sigalg/ on $\Omega$ is called a \emph{measurable space}. 

\subsection{Constructing \sigalgs/}

We now know what a \sigalg/ is and have seen some example and some non-examples. But the examples we have seen are still quite uninspiring. We will actually discuss a very important \sigalg/ in the next section. But for now, we will describe several ways to construct \sigalgs/. The first is restricting an existing \sigalg/ to a given set.

\begin{lemma}[Restriction of a \sigalg/]\label{lem:restriction_sigma_algebra}
Let $(\Omega, \cF)$ be a measurable space and $A \subset \Omega$. Then the collection define by 
\[
	\cF_A := \{A \cap B \, : \, B \in \cF\},
\]
is a \sigalg/ on $A$, called the \emph{restriction of $\cF$ to $A$}.
\end{lemma}

\begin{proof}
We need to check all three properties.
\begin{enumerate}
\item Since $A \cap \Omega = A$ and $A \cap \emptyset = \emptyset$, it follows that $A, \emptyset \in \cF_A$.
\item Consider a set $C \in \cF_A$. Then by definition $C = A \cap B$ for some $B \in \cF$. Next, we note
\[
	A \setminus C = A \setminus (A \cap B) = A \cap (\Omega \setminus B).
\]
Since $\cF$ is a \sigalg/, it follows that $\Omega \setminus B \in \cF$ and so $A \setminus C \in \cF_A$.
\item Let $C_1, C_2, \dots$ be sets in $\cF_A$. Then there are $B_1, B_2, \dots \in \cF$ such that $C_i = A \cap B_i$. Hence
\[
	\bigcup_{i=1}^\infty C_i = \bigcup_{i=1}^\infty A \cap B_i = A \cap \bigcup_{i=1}^\infty B_i \in \cF_A,
\]
since $\bigcup_{i=1}^\infty B_i \in \cF$ because this is a \sigalg/.
\end{enumerate}
\end{proof}

While it is nice to be able to take a given \sigalg/ and create a possibly smaller one by restricting it to a given set, we might also want to start with a given collection of sets $\cA$ and then create a \sigalg/ that contains this collection. Of course, the powerset $\cP(\Omega)$ will always work. However, it is not always desirable to take this maximal \sigalg/. It would be much better if we could create the smallest \sigalg/ that contains $\cA$. It turns out that this can be done and the resulting \sigalg/ is said to be \emph{generated by $\cA$}. 

\begin{proposition}[Generated \sigalg/]
Let $\cA$ be a collection of subsets of $\Omega$ and denote by $\Sigma_\cA$ the collection of all \sigalgs/ on $\Omega$ that contain $\cA$. Then the collection defined by
\[
	\sigma(\cA) := \bigcap_{\cF \in \Sigma_\cA} \cF,
\]
is a \sigalg/. It is called the \emph{\sigalg/ generated by $\cA$}.
\end{proposition}

\begin{proof}
Similar to Lemma~\ref{lem:restriction_sigma_algebra} we need to check all the requirements.
\begin{enumerate}
\item Since $\emptyset, \Omega \in \cF$ holds for every $\cF \in \Sigma_\cA$ it follows that $\emptyset, \Omega \in \sigma(\cA)$. In particular, we note that $\sigma(\cA)$ is not empty.
\item Take $A \in \sigma(\cA)$. Then $A \in \cF$ for each $\cF \in \Sigma_\cA$. Since $\cF$ is a \sigalg/ it holds that $\Omega \setminus A \in \cF$ for each $\cF \in \Sigma_\cA$. This then implies that $\Omega \setminus A \in \sigma(\cA)$.
\item Let $(A_i)_{i \in \bbN}$ be a sequence of sets in $\sigma(\cA)$. Then by definition $A_i \in \cF$ for each $\cF \in \Sigma_\cA$. Hence
\[
	\bigcup_{i \in \bbN} A_i \in \cF,
\] 
holds for each $\cF \in \Sigma_\cA$ and thus it follows that $\bigcup_{i \in \bbN} A_i \in \sigma(\cA)$.
\end{enumerate}
\end{proof}

If $\cF$ is a \sigalg/ on $\Omega$ and $\cA$ is a collection of subsets such that $\cF = \sigma(\cA)$, we call $\cA$ the \emph{generator of $\cF$}.

With this this powerful construction tool, we can now construct products of measurable spaces.

\begin{definition}[Product \sigalg/]
Let $(\Omega, \cF)$ and $(\Omega^\prime, \cF^\prime)$ be two measurable spaces. Then we define $\cF \otimes \cF^\prime$ to be the \sigalg/ on $\Omega \times \Omega^\prime$ generated by sets of the form $A \times B$, with $A \in \cF$ and $B \in \cF^\prime$.
\end{definition}

\subsection{Borel \sigalg/}

The Euclidean space $\bbR^d$ is omnipresent in mathematics and hence pops up in many bachelor courses as well. In particular, in the introduction we noticed that the concept of random variables, as given in the course Probability and Modeling, is mainly concerned with $\bbR$. The need to impose a measurable structure on this space, by means of a \sigalg/, should therefore not come as a surprise. This canonical \sigalg/ is called the \emph{Borel \sigalg/} and is named after the French mathematician \'{E}mile Borel, who was one of the pioneers of measure theory.

In order to define the Borel \sigalg/ we need the notion of an open set in $\bbR^d$. For any $x \in \bbR$ and $r >0$, we denote by $B_x(r) := \{y \in \bbR^d \, : \, \|x-y\|<r\}$ the open ball of radius $r$ around $x$. A set $U \subset \bbR^d$ is called \emph{open} if and only if for every $x \in U$, there exists an $r > 0$ such that $B_x(r) \subset U$.

\begin{definition}[Borel \sigalg/]
The \emph{Borel \sigalg/} on $\bbR^d$, denoted by $\cB_{\bbR^d}$, is the \sigalg/ generated by all open sets in $\bbR^d$. Elements of $\cB_{\bbR^d}$ are called \emph{Borel sets}.
\end{definition}

\begin{remark}
From the definition, it should be clear that one can actually define a \emph{Borel \sigalg/} on any topological space. However, this requires the notion of a topology which is beyond the scope of this course.
\end{remark}

While this is a perfectly fine definition, it is often a cumbersome to work with. It is therefore convenient that $\cB_{\bbR^d}$ is generated by other, more compact, collections of sets.

\begin{proposition}
The Borel \sigalg/ on $\bbR^d$ is the \sigalg/ generated by the sets
\[
	(-\infty,a_1] \times \dots \times (-\infty, a_d] \quad \text{with } a_i \in \bbQ, i = 1, \dots, d.
\]
\end{proposition}

\section{Measures}

% Definition of measure

% Properties of measures

% Probability measures

% Measurable spaces and probability spaces

\section{Measurable functions}

% Definition of measurable functions

% Properties of measurable functions

% How to check measurability

% Sigma-algebras generated by measurable functions

% Random variables

\section{Problems}

\begin{problem}\label{prb:event_space_are_sigma_algebras}
Show that the definition of an \emph{event space} as given in Section~\ref{sec:recalling_probability_theory} is equivalent to that of a $\sigma$-algebra as given in Definition~\ref{def:sigma_algebra}.
\end{problem}

\begin{problem}\label{prb:sigma_algebra_closed_intersections}
Prove Lemma~\ref{lem:sigma_algebra_closed_intersections}.
\end{problem}

\begin{problem}\label{prb:converse_preimage_sigma_algebra}
Counter example to the statement: if $(\Omega, \cF)$ is a measurable space and $f : \Omega \to \Omega^\prime$. Then $f(\cF)$ is a \sigalg/ on $\Omega^\prime$.
\end{problem}
