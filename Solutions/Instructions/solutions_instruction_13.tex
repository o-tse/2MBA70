\documentclass{lecturenotes}

\usepackage{lecture_notes}

% Reset mathcal back to previous style (not curly)
\DeclareMathAlphabet{\mathcal}{OMS}{cmsy}{m}{n}



%%%%%%%%%%%%%%%%%%%%%%%%%%%%%%%%%%%%%%%%%%%%%%%%%%%%%%%%%%%%%%%%%%%%%%%%%%%%%%%%%%%%%%%%%%%%%%%%%%%
%										For leaving comments									  %
%%%%%%%%%%%%%%%%%%%%%%%%%%%%%%%%%%%%%%%%%%%%%%%%%%%%%%%%%%%%%%%%%%%%%%%%%%%%%%%%%%%%%%%%%%%%%%%%%%%

%Name:			XXX
%Description:	Adds a piece of text in blue, surrounded by [] brackets. 
%				#1: the person the comment is addressed to
%				#2: the person the comment is from
%				#3: the comment
%Usage:			Write \XXX{collaborator}{myName}{myComment} to write [myComment] addressed to [collaborator]
\newcommand{\XXX}[3]{{\color{blue} \textbf{ [#1:  #3 \textit{ -#2-} ]}}}

%%%%%%%%%%%%%%%%%%%%%%%%%%%%%%%%%%%%%%%%%%%%%%%%%%%%%%%%%%%%%%%%%%%%%%%%%%%%%%%%%%%%%%%%%%%%%%%%%%%
%									     BibTeX commands 									      %
%%%%%%%%%%%%%%%%%%%%%%%%%%%%%%%%%%%%%%%%%%%%%%%%%%%%%%%%%%%%%%%%%%%%%%%%%%%%%%%%%%%%%%%%%%%%%%%%%%%

%Name:			Swap
%Description:	Command for properly typesetting "van der" and related expressions in Dutch and German names in the
%				bibliography. It swaps [van der] and [Lastname] in [van der Lastname] so that ordering will be performed on 
%				[Lastname] instead of [van der].
%Usage:			Write \swap{Lastname}{~van~der~}, Firstname instead of [van der Lastname, Firstname] in the author header
%				of the bibtex entry.

\newcommand*{\swap}[2]{\hspace{-0.5ex}#2#1}


%%%%%%%%%%%%%%%%%%%%%%%%%%%%%%%%%%%%%%%%%%%%%%%%%%%%%%%%%%%%%%%%%%%%%%%%%%%%%%%%%%%%%%%%%%%%%%%%%%%
%											Document											  %
%%%%%%%%%%%%%%%%%%%%%%%%%%%%%%%%%%%%%%%%%%%%%%%%%%%%%%%%%%%%%%%%%%%%%%%%%%%%%%%%%%%%%%%%%%%%%%%%%%%

\setlength\parindent{0pt}

\begin{document}




\textbf{Problem 11.2}

\begin{enumerate}[label={(\alph*)}]
\item By definition we have that
\[
	\int_B \bbE[X | \cH] \, \dd \bbP = \int_B X \, \dd \bbP,
\]
holds for all $B \in \cH$. Since by assumption both $\bbE[X | \cH]$ and $X$ are $\cH$-measurable, the result follows from problem 8.2.
\item Note that $a \bbE[X | \cH]$ is $\cH$-measurable. Moreover,
\[
	\int_B a \bbE[X | \cH] \, \dd \bbP = a \int_B \bbE[X | \cH] \, \dd \bbP = a \int_B X \, \dd \bbP = \int_B aX \, \dd \bbP.
\]
This proves the claim.
\item Similarly to the previous point, we first note that since $\bbE[X | \cH]$ and $\bbE[Y | \cH]$ are $\cH$-measurable so is $\bbE[X | \cH] + \bbE[Y | \cH]$. The result then follows because
\begin{align*}
	\int_B \bbE[X | \cH] + \bbE[Y | \cH] \, \dd \bbP 
	&= \int_B \bbE[X | \cH] \, \dd \bbP + \int_B \bbE[Y | \cH] \, \dd \bbP \\
	&= \int_B X \, \dd \bbP + \int_B Y \, \dd \bbP = \int_B X + Y \, \dd \bbP.
\end{align*}
\item First we observe that for any $B \in \cH$
\[
	\int_B \bbE[X | \cH] \, \dd \bbP = \int_B X \, \dd \bbP \le \int_B Y \, \dd \bbP = \int_B \bbE[Y | \cH] \, \dd \bbP.
\]
Now consider the event $A := \{ \bbE[X | \cH] > \bbE[Y | \cH]\} \in \cH$. If this event has non-zero measure then it would follow that
\[
	\int_A \bbE[X | \cH] \, \dd \bbP > \int_A \bbE[Y | \cH] \, \dd \bbP,
\]
which is a contradiction. Hence we conclude that $\bbE[X | \cH] \le \bbE[Y | \cH]$ holds $\bbP$-almost everywhere.
\end{enumerate}

\textbf{Problem 11.3}

\begin{enumerate}[label={(\alph*)}]
\item This follows by repeating the step for the solution to Problem 4.8 a). 
%
%By definition, we have that $\nu_X(\Omega) = \int_\Omega X\,\dd\mu = 1$. Now let $(A_n)_{n\in\bbN}$ be a family of mutually disjoint measurable sets in $\sigma(X)$. Then we have that the sequence 
%	\[
%		g_n:= \sum_{i=1}^n X\,\mathbf{1}_{A_i} = X\,\mathbf{1}_{\bigcup_{i=1}^n\! A_i}\;\longrightarrow\; g:= X\,\mathbf{1}_{\bigcup_{i\in\bbN}\! A_i}\qquad\text{pointwise monotonically}.
%	\]
%	By MCT, we then have that
%	\[
%		\nu_X\left(\bigcup_{i\in\bbN} A_i\right) = \int_{\bigcup_{i\in\bbN}\! A_i} X\,\dd\mu = \lim_{n\to\infty} \int_{\bigcup_{i=1}^n\! A_i} X\,\dd\mu = \lim_{n\to\infty} \sum_{i=1}^n \int_{A_i} X\,\dd\mu = \sum_{i\in\bbN}\nu_X(A_i),
%	\]
%	thus showing that $\nu_X$ is a measure on $(\Omega,\sigma(X))$.

\item We start by observing that the result will directly follow from Theorem 2.15 if we can show that the \sigalg/ $\sigma(X)$ satisfies the two properties.

The first one is immediate, from the fact that $X^{-1}(A) \cap X^{-1}(B) = X^{-1}(A\ cap B)$. For the second one consider the intervals $I_n = (-n,n)$ and define $A_n := X^{-1}(I_n) \in \sigma(X)$. Since $\bigcup_{n \in \bbN} I_n = \bbR$ it follows that $\bigcup_{n \in \bbN} A_n = \Omega$. Moreover since $|X| \le n$ on the set $A_n$ it holds that
\[
	\mu(A_n) = \nu_X(A_n) = \int_{A_n} X \, \dd \bbP \le n \int_{A_n} \, \dd \bbP = n \bbP(A_n) \le n < \infty.
\]
Thus the second condition of Theorem 2.15 is also satisfied and the result now follows.

	
\end{enumerate}

\bigskip

\end{document}