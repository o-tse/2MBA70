
\section{Measurable functions}

Now that we have defined measure spaces $(\Omega, \cF, \mu)$, through \sigalgs/ and measures and studied properties of both these objects, it is time to look at functions between such spaces. We will focus on functions that preserve the measurable structure of the spaces.

The main object in analysis were \emph{continuous} function $f : \bbR^d \to \bbR^m$. This property was important, as it allowed us to differentiate the function and perform integration. 

\subsection{Definition and properties}

We want to consider functions $f : \Omega \to E$ between measurable spaces $(\Omega, \cF)$ and $(E, \cG)$ that preserve the measurable structure, as imposed by the \sigalgs/. It turns out that it the best way to do this it to look at the preimage of measurable sets in $E$.

\begin{definition}[Measurable function]\label{def:measurable_function}
Let $(\Omega, \cF)$ and $(E, \cG)$ be two measurable spaces. A function $f: \Omega \to E$ is said to be \emph{$(\cF, \cG)$-measurable} is $f^{-1}(B) \in \cF$ for every $B \in \cG$.
\end{definition}

It is important to note that whether a function is measurable or not depends on the \sigalgs/ we consider in each of the measurable spaces. This means that a function $f : \Omega \to E$ might be $(\cF, \cG)$-measurable but not $(\cF^\prime, \cG)$-measurable for a different sigma algebra $\cF^\prime$ on $\Omega$. This is different from the notion of continuity of functions on $\bbR^d$. 

We will often omit the explicit reference to the \sigalgs/ in the definition of a measurable function if it is clear which \sigalgs/ are considered. That is, we will simply say that the function $f$ between the two measurable spaces $(\Omega, \cF)$ and $(E, \cG)$ is \emph{measurable}. We will sometimes make the choice of \sigalgs/ explicit by saying that $f: (\Omega, \cF) \to (E, \cG)$ is measurable.

The fact that measurability of $f$ depends on the \sigalgs/ involved mean we need to take a bit of care when considering operations on functions, as these might destroy the measurability. The most natural operation we should check first is composition, as we would like to be able to compose measurable functions into measurable functions. Luckily this is possible.

\begin{proposition}[Composition of measurable functions]
Let $(\Omega_i, \cF_i)$, for $i = 1,2,3$ be three measurable spaces and $f : \Omega_1 \to \Omega_2$, $g : \Omega_2 \to \Omega_3$ be two measurable functions. Then the composition $h := g \circ f : \Omega_1 \to \Omega_3$ is measurable.
\end{proposition}

\begin{proof}
By definition, we need to show that for every $A \in \cF_3$ the preimage $h^{-1}(A) \in \cF_1$. First note that
\begin{align*}
	h^{-1}(A) = (g\circ f)^{-1}(A) &= \{x \in \Omega \, : \, g(f(x)) \in A\} \\
	&= \{x \in \Omega \, : \, f(x) \in g^{-1}(A) \} = f^{-1}(g^{-1}(A)).
\end{align*}
Since $g$ is $(\cF_2, \cF_3)$-measurable, $g^{-1}(A) \in \cF_2$. Then, using that $f$ is $(\cF_1, \cF_2)$-measurable, we conclude that $h^{-1}(A) = f^{-1}(g^{-1}(A)) \in \cF_1$ as was required to show.
\end{proof}

The next result shows that we can also restrict a measurable function $f : \Omega \to E$ to a measurable subset $A \subset \Omega$, as long as we consider the appropriate (and natural) \sigalg/. The same holds for extensions. 

\begin{proposition}[Restriction and extension of measurable functions]
Let $f: (\Omega, \cF) \to (E, \cG)$ be a measurable function and let $A \in \cF$ be non-empty. Then the restriction map $f|_A : \Omega \to E$ is $(\cF_A, \cG)$-measurable.

Moreover, if $g_A : A \to E$ is $(\cF_A, \cG)$-measurable, and $p \in E$, then the extension
\[
	g(\omega) := \begin{cases}
		g_A(\omega) &\text{if } \omega \in A,\\
		p &\text{if } \omega \notin A,
	\end{cases}
\]
is $(\cF, \cG)$-measurable.
\end{proposition}

\begin{proof}
TODO
\end{proof}

At this stage these are the only general properties of measurable function we can consider. However, if the measurable space a function maps to has more structure we can see if this structure also respect the measurability. For example, we will see later in [REF] that for measurable functions $f, g : \Omega \to \bbR$ their product and sum are also measurable.

\subsection{Checking for measurability}

Given any function $f : \Omega \to E$ between two measurable spaces $(\Omega, \cF)$ and $(E, \cG)$, when is this measurable? Definition~\ref{def:measurable_function} tells us that to answer this question we need to check that the preimage of any measurable set is again measurable. But this can be a cumbersome exercise. Or even impossible when we do not have an explicit description of the sigma algebra. This can happen, for example, when $\cG$ is generated by some collection of sets $\cA$, which is the case for the important Borel \sigalg/. 

Fortunately, the definition of measurability works very well with generated \sigalgs/. In particular, to show that a function is measurable, it suffices to only consider sets from the generator set $\cA$, instead of the entire \sigalg/ $\sigma(\cA)$.

\begin{lemma}\label{lem:measurable_condition_generator}
Let $(\Omega, \cF)$ and $(E, \cG)$ be two measurable spaces such that $\cG = \sigma(\cA)$. Let $f: \Omega \to E$ be a function such that $f^{-1}(A) \in \cF$ for all $A \in \cA$. Then $f$ is $(\cF, \cG)$-measurable.
\end{lemma}

\begin{proof}
Consider the following collection of subsets:
\[
	\cH := \{B \subset \cG \, : \, f^{-1}(B) \in \cF\}.
\]
We claim that $\cH$ is a \sigalg/ on $E$. Suppose this is indeed true. Then, since by construction $\cA \subseteq \cH$, it follows from Lemma~\ref{lem:inclusion_sigalgs} that $\cG = \sigma(\cA) \subseteq \cH$. But this then implies that $f^{-1}(B) \in \cF$ for each $B \in \cG$ which means that $f$ is $(\cF, \cG)$-measurable.

So let's prove that $\cH$ is a \sigalg/. First we note that $f^{-1}(\emptyset) = \emptyset \in \cF$ and $f^{-1}(E) = \Omega \in \cF$. So $\emptyset, E \in \cH$. 

Next, let $B \in \cH$. Then
\[
	f^{-1}(E\setminus B) = \Omega \setminus f^{-1}(B) \in \cF,
\]
since by definition $f^{-1}(B) \in \cF$. So $E\setminus B \in \cH$.

Finally, if $(B_i)_{i \in \bbN}$ is a sequence of sets in $\cH$, then
\[
	f^{-1}\left(\bigcup_{i = 1}^\infty B_i\right) = \bigcup_{i = 1}^\infty f^{-1}(B_i) \in \cF,
\]
which shows that $\bigcup_{i = 1}^\infty B_i \in \cH$, completing the proof that $\cH$ is a \sigalg/.
\end{proof}

We thus see that at least. But that still requires us to check if any given function is measurable. For example, is the function $f : \bbR \to \bbR$ given by $f(x) = e^x$, measurable? It would be be much better if we have a more familiar criteria that would imply measurability. Continuity does exactly this. 

\begin{proposition}
Every continuous map $f : \bbR^d \to \bbR^m$ is $(\cB_{\bbR^d}, \cB_{\bbR^m})$-measurable.
\end{proposition}

\begin{proof}
Recall from analysis that a map $f : \bbR^d \to \bbR^m$ is continuous if for every $x \in \bbR^d$ and $\varepsilon > 0$, there exists an $r = r(x,\varepsilon)$ such that 
\[
	\|f(x) - f(y)\| < \varepsilon \quad \text{for every } y \in B_x(r).
\]
The key step for this proof is to show that this is equivalent to the following condition\footnote{Actually, the definition we state here using open sets is the general definition for continuous functions in the mathematical field of topology.}:
\[
	\text{for every open set } O \subset \bbR^m \quad f^{-1}(O) \text{ is open}.
\]
If this is true then, since the Borel \sigalg/ is generated by the open sets, it follows that $f^{-1}(O) \in \cB_{\bbR^d}$ for each open set $O \subset \bbR^m$. Lemma~\ref{lem:measurable_condition_generator} then implies that $f$ is measurable.

So we are left to show the equivalence of the two conditions for continuity. First assume that $f$ is continuous and take an arbitrary open set $O \subset \bbR^m$. We need to show that $f^{-1}(O)$ is open, which means that for every $x \in f^{-1}(O)$ we should find an $r$ such that $B_x(r) \subset f^{-1}(O)$. Since $O$ is open, there exists a $\varepsilon > 0$ such that $B_{f(x)}(\varepsilon) \subset O$. Continuity of $f$ now implies the existence of an $r$ such that $\|f(x) - f(y)\| < \varepsilon$ for all $y \in B_x(r)$. But this simply means that $f(y) \in B_{f(x)}(\varepsilon) \subset O$ for every $y \in B_x(r)$, which implies that $B_x(r) \in f^{-1}(O)$.

Now assume that $f^{-1}(O)$ is open in $\bbR^d$, for every open set $O \in \bbR^m$ and take $x \in \bbR^d$ and $\varepsilon > 0$. Then the ball $B_{f(x)}(\varepsilon)$ is open in $\bbR^m$, so that by assumption $f^{-1}(B_{f(x)}(\varepsilon))$ is open in $\bbR^d$. Since $x \in f^{-1}(B_{f(x)}(\varepsilon))$ there now must exist an $r > 0$ such that $B_x(r) \subset f^{-1}(B_{f(x)}(\varepsilon))$. But this then implies that for every $y \in B_x(r)$, $f(y) \in B_{f(x)}(\varepsilon)$, which is equivalent to $\|f(x) - f(y)\| < \varepsilon$.
\end{proof}

With this result we have a vast world of measurable functions $f : \bbR^d \to \bbR^m$ at our disposal. However, when dealing with functions that map to measurable spaces that are not the Borel space $(\bbR^d, \cB_{\bbR^d})$, we still need to carefully check if it is measurable. But what if we can simply construct a \sigalg/ such that it makes a function measurable?

\subsection{\sigalgs/ generated by measurable functions}

Suppose we have a function $f : \Omega \to E$ from a set $\Omega$ to some measurable space $E, \cG)$. If we want to study the function $f$ in the framework of measure theory, we need to turn $\Omega$ into a measurable space $(\Omega, \cF)$ and have $f$ be $(\cF, \cG)$-measurable. The good news is that we can construct a minimal \sigalg/ that does the job for us. It can even be done for multiple functions at the same time.

\begin{proposition}\label{prop:sigalg_generated_functions}
Let $(\Omega_i, \cF_i)$, for $i \in I$ be measurable spaces and $(f_i)_{i \in I}$ be a family of functions $f_i : \Omega \to \Omega_i$. Then the smallest \sigalg/ on $\Omega$ that makes all $f_i$ simultaneously measurable is
\[
	\sigma(f_i \, : \, i \in I):=\sigma\left(\bigcup_{i \in I} f_i^{-1}(\cF_i)\right).
\] 
\end{proposition}

\begin{proof}
First note that by Proposition~\ref{prop:generated_sigalg}, $\sigma(f_i \, : \, i \in I)$ is a \sigalg/. We will show that any \sigalg/ that makes each $f_i$ measurable much contain $\sigma(f_i \, : \, i \in I)$. So let $\cF$ be such a \sigalg/. Then in particular, for any $i\in I$ and $B \in \cF_i$ we have that $f_i^{-1}(B) \in \cF$. This implies that
\[
	\bigcup_{i \in I} f_i^{-1}(\cF_i) \subseteq \cF.
\]
Now since $\sigma(f_i \, : \, i \in I)$ is generated by the collection on the left hand side, Lemma~\ref{lem:inclusion_sigalgs} implies that 
\[
	\sigma(f_i \, : \, i \in I):=\sigma\left(\bigcup_{i \in I} f_i^{-1}(\cF_i)\right) \subset \sigma(\cF) = \cF.
\]
\end{proof}

Similar to Lemma~\ref{lem:measurable_condition_generator}, when $\cF_i = \sigma(\cA_i)$ it turns out that to construct $\sigma(f_i \, : \, i \in I)$ it suffices to consider only preimages of the generator sets $\cA_i$.

\begin{proposition}\label{prop:extension_measurable_function}
Let $(\Omega, \cF)$ and $(\Omega_i, \cF_i)$, for $i \in I$ be measurable spaces such that $\cF_i = \sigma(\cA_i)$. Let $(f_i)_{i \in I}$ be a family of functions $f_i : \Omega \to \Omega_i$. Then 
\[
	\sigma(f_i \, : \, i \in I) = \sigma\left(\bigcup_{i \in I} f_i^{-1}(\cA_i)\right).
\] 
\end{proposition}

\begin{proof}
Let us write $\cG_1 = \sigma(f_i \, : \, i \in I)$ and $\cG_2 = \sigma\left(\bigcup_{i \in I} f_i^{-1}(\cA_i)\right)$. From the definition it is clear that $\cG_2 \subseteq \cG_1$. Moreover, each $f_i$ is $(\cG_2, \cF_i)$-measurable by Lemma~\ref{lem:measurable_condition_generator}. But by Proposition~\ref{prop:sigalg_generated_functions} $\cG_1$ is the smallest \sigalg/ that makes all $f_i$ $(\cG_1, \cF_i)$-measurable and hence $\cG_1 \subseteq \cG_2$, which implies the result.
\end{proof}

We end this section by going back to the product \sigalg/ given in Definition~\ref{def:product_sigalg}. There is an alternative way to construct it using functions. Let $(\Omega_1, \cF_1)$ and $(\Omega_2, \cF_2)$ be two measurable spaces and consider the functions $\pi_i : \Omega_1 \times \Omega_2 \to \Omega_i$, defined by 
\[
	\pi_1(x,y) = x \quad \pi_2(x,y) = y.
\]
These are called the \emph{canonical projections}. Following Proposition~\ref{prop:sigalg_generated_functions} we can construct the \sigalg/ $\sigma(\pi_1, \pi_2)$ on $\Omega_1 \times \Omega_2$, which makes both canonical projections measurable. Then it follows that, see Problem~\ref{prb:product_sigalg_equivalence},
\begin{equation}\label{eq:product_sigalg_equivalence}
	\cF_1 \otimes \cF_2 = \sigma(\pi_1, \pi_2).
\end{equation}

\subsection{Push forward measure}



\section{Measurable functions on the real line}

When studying properties of measurable function we could only do a few things for general measurable spaces. So in this section we will focus on a specific measurable space: the real line $(\bbR, \cB_\bbR)$. We will see that most of the natural operations we can apply to function in a point-wise manner, such as addition and multiplication, preserve their measurability. But we will do even better. We will show that taking point-wise limit operations, such as taking a supremum of a family of measurable functions, preserves measurability as well. This makes the class of measurable functions much more powerful then that of continuous functions, as point-wise limits of continuous functions are not guaranteed to be continuous again. All thes properties will be useful when we introduce the concept of integration of measurable functions in Chapter [REF] and develop limit theorems for integrals in Chapter [REF].

To properly study limit operations on measurable functions, that could diverge, we need to have $\infty$ be a part of the real line (which it is not). So we first extend the real line to include both $\infty$ and $-\infty$.

\subsection{Extended real line}

We define $\bar{\bbR} := [-\infty, \infty]$ as the \emph{extended real line}. We impose the natural ordering on $\bar{\bbR}$, inherited from $\bbR$, with the addition that $-\infty < x$ and $x < \infty$ for all $x \in \bbR$. The extended real line also has the same operations of addition and multiplications, with are extend to include the two new elements $\pm \infty$:
\begin{enumerate}
\item for every $x\in \bbR$, $x + \infty = \infty + x = \infty$ and $x + (-\infty) = (-\infty) + x = -\infty$,
\item $\infty + \infty = \infty$ and $(-\infty) + (-\infty) = -\infty$,
\item for every $x \in (0,\infty]$, $\pm x (\infty) = (\infty) \pm x = \pm \infty$, $\pm x (-\infty) = (-\infty) \pm x = \mp \infty$,
\item $0 (\pm \infty) = (\pm \infty) 0 = 0$ and $1/\pm \infty = 0$.
\end{enumerate}

To turn $\bar{\bbR}$ into a measurable space we extend the Borel \sigalg/ to include the new elements $\pm \infty$.

\begin{definition}[Extended real line]
The Borel \sigalg/ $\bar{\cB}$ of the extended real line $\bar{\bbR}$ is defined by
\[
	\bar{\cB} := \{A \cup S \, :\, A \in \cB_\bbR \text{ and } S \in \{\emptyset, \{-\infty\}, \{\infty\}, \{-\infty, \infty\}\}
\]
\end{definition}

The following results, whose proof is left as an exercise, relates $\bar{\cB}$ to the original Borel \sigalg/.

\begin{lemma}\label{lem:characterization_extended_borel}
The extended Borel \sigalg/ $\bar{\cB}$ satisfies
\[
	\cB_\bbR = \bar{\cB} \cap \bbR.
\]
Moreover, it is generated by sets of the form $[a,\infty]$, with $a \in \bbQ$.
\end{lemma}

\subsection{Basic operations}

\begin{lemma}
Let $f, g : (\Omega, \cF) \to \bar{\bbR}$ be measurable. Then the following functions (where operations are always taken point-wise) are measurable as well:
\begin{enumerate}
\item $f + g$,
\item $f \vee g := \max\{f,g\}$,
\item $f \wedge g := \min\{f,g\}$,
\item $f g$,
\item $a f$, for any $a \in \bbR$, and
\item $f/g$ if $g \ne 0$ on $\Omega$.
\end{enumerate} 
\end{lemma}

\begin{proof}
We will prove 2 and 4. The other parts are left as an exercise, see Problem [REF].

For 2, we first note that by Lemma~\ref{lem:characterization_extended_borel} $\bar{\cB}$ is generated by the sets $[a,\infty]$, for $a \in \bbQ$. Hence, by Lemma~\ref{lem:measurable_condition_generator} it suffices to show that 
\[
	(fg)^{-1}([a,\infty]) = \{\omega \in \Omega \, : \, f(\omega) g(\omega) \in [a, \infty]\} \in \cF.
\]

\end{proof}

\subsection{Limit operations}



\section{Random variables and general stochastic objects}

% Random variables   

\subsection{Definition and examples}

\begin{definition}[Random variable]
A \emph{random variable} is a measurable function from some probability space $(\Omega,\cF, \bbP)$ to the (extended) real line.
\end{definition}

It is important to observe that the definition of a random variable does not make any specific claims on what the probability space should be. 

Let $X$ be a random variable and recall that its \emph{cumulative distribution function} $F_X : \bbR \to [0,1]$ is defined such that $F_X(t)$ denotes the "probability" that $X \in (-\infty ,t]$. In the language of measure theory this means that we have to look at the preimage of $(-\infty, a]$ under the measurable function $X$, which is an element of the \sigalg/ $\cF$, and assign a value to it. For this the only thing we have at our disposal is the probability measure $\bbP$. So in other words, the cummulative distribution is defined as
\[
	F_X(t) := \bbP(X^{-1}((-\infty,t])).
\]

Observe that this is nothing more than the \emph{push forward} of $\bbP$ under $X$. In fact we can actually define, at a much more general level, random elements in any measurable space and put an associated probability measure on this space by a push-forward.

\begin{definition}[Random elements]
Let $(\Omega,\cF, \bbP)$ be a probability space and $(E,\cG)$ some measurable space. A \emph{random element} in $(E,\cG)$ is a measurable map $X : \Omega \to E$. It associated \emph{probability measure} is defined as the push forward of $\bbP$ under $X$, i.e.
\[
	\bbP(X \in A) := \bbP(X^{-1}(A)) \quad \text{for every } A \in \cG.
\]
\end{definition}

Sometimes we use the term \emph{random} instead of \emph{stochastic}. 

With this general definition we can now easily define random vectors, random matrices, random functions and so one. The only thing we need is to start with the appropriate space (vectors, matrices, functions) and turn it into a measurable space by endowing it with a suitable \sigalg/. 

\begin{example}[Random elements]
\hfill
\begin{enumerate}
\item A random vector in $\bbR^d$ is a random element in $(\bbR^d, \cB_{\bbR^d})$.
\item A random $n \times m$ matrix is a random element in $(\bbR^n \times \bbR^m, \cB_{\bbR^n} \otimes \cB_{\bbR^m})$.
\end{enumerate}
\end{example}

\subsection{Constructing random variables}

In any course on probability theory, one of the first random variables you encounter is the \emph{standard uniform random variable}. This is a random variable $U$ that takes values in $[0,1]$ such that its cdf satisfies $F(t) = t$ for all $0\le t \le 1$. In the course Probability and Modeling this description would be enough to work with. But now that we know what a random variable actually is, we need a bit more. More precisely, we have to construct a probability space $(\Omega,\cF, \bbP)$ and a measurable function $U : \Omega \to \bbR$ such that 
\begin{equation}\label{eq:cdf_uniform_rv}
	\bbP\left(U^{-1}((-\infty,t])\right) = \begin{cases}
		0 &\text{if } t < 0,\\
		t &\text{if } 0 \le t\le 1,\\
		1 &\text{if } t > 1.
	\end{cases}
\end{equation}

As you might expect, this is possible. And it is here that we see the first nice usage of the Lebesgue measure.

\begin{proposition}[Uniform random variable]
There exist a probability space $(\Omega,\cF, \bbP)$ and random variable $U$, such that $\bbP\left(U^{-1}((-\infty,t])\right)$ satisfies~\eqref{eq:cdf_uniform_rv}.
\end{proposition}

\begin{proof}
Consider the space $\Omega = [0,1]$ together with the restricted Borel \sigalg/ $\cF = \cB_\bbR|_{[0,1]}$ and as probability measure the restricted Lebesgue measure $\bbP := \lambda|_{[0,1]}$. Now consider the function $U(t) = \mathbf{1}_{[0,1]} \, t$. Then, it follows that
\[
	U^{-1}((-\infty,t]) = \begin{cases}
		\emptyset &\text{if } t \le 0,\\
		(0,t] &\text{if } 0 < t \le 1, \\
		[0,1] &\text{if } t > 1.
	\end{cases}
\]
Since by Theorem~\ref{thm:lebesgue_measure}
\[
	\lambda|_{[0,1]}((0,t]) = \lambda((0,t]) = t,
\]
for any $0 < t \le 1$ we have
\[
	\bbP\left(U^{-1}((-\infty,t])\right) := \lambda|_{[0,1]}\left(U^{-1}((-\infty,t])\right)
	= \begin{cases}
		0 &\text{if } t < 0,\\
		t &\text{if } 0 \le t\le 1,\\
		1 &\text{if } t > 1.
	\end{cases}
\]
\end{proof}

The standard uniform random variable is extremely important, as it is the base from which we can construct any other random variable. To illustrate this let us first consider the case of an \emph{exponential random variable} with rate $\lambda > 0$. This is a random variable $X$ with cdf
\[
	F_X(t) = \begin{cases}
		0 &\text{if } t \le 0,\\
		1-e^{-\lambda t} &\text{if } t > 0.
	\end{cases}
\]

For $u \in (0,1)$, write $H(u) := F_X^{-1}(u)$ and note that
\[
	H(u) = \frac{1}{\lambda} \log\left(\frac{1}{1-u}\right).
\]
Now let $U$ be the standard normal random variable and consider the composition $H \circ U : [0,1] \to \bbR$. First we note that since cdf $F_X(x)$ is strictly monotonic increase, so is $H$. In particular it follows that for any $t > 0$,
\[
	H^{-1}((-\infty,t]) = (-\infty, H^{-1}(t)] = (-\infty, F_X(t)].
\]
While $H^{-1}((-\infty,t]) = \emptyset$ if $t \le 0$.

Hence we get
\begin{align*}
	(H \circ U)^{-1}((-\infty, t]) = U^{-1}(H^{-1}((-\infty, t]))
	&= \begin{cases}
		U^{-1}(\emptyset) &\text{if } t \le 0,\\
		U^{-1}((-\infty, F_X(t)]) &\text{if } t > 0.
	\end{cases}
\end{align*}
From this it follows that 
\[
	\bbP\left((H \circ U)^{-1}((-\infty, t])\right) = \begin{cases}
			0 &\text{if } t \le 0,\\
			1-e^{-\lambda t} &\text{if } t > 0,
		\end{cases}
\]
from which we conclude that $H \circ U$ is a way to construct an exponential random variable with rate $\lambda$.

The main point of the construction above is to consider the inverse of the cdf $F^{-1}$ and evaluate this on a standard uniform random variable. However, when extending this to the more general case we have to deal with the fact that not every cdf has an inverse. [EXAMPLE]

Nevertheless, if does hold than any cdf $F$ is monotonic increasing and right continuous. For these type of functions there exists the notion of a \emph{generalized inverse}, defined as
\begin{equation}
	\overleftarrow{F}(u) := \inf\{ x \in \bbR \, : \, F(x) \ge y\}. 
\end{equation}
The construction we used for the exponential random variable can now be generalize by using $\overleftarrow{F}$ instead of $F^{-1}$. This results in the following theorem on the existence of random variables with a given cdf.

\begin{theorem}[Constructing random variables]\label{thm:construction_random_variable}
Let $F : \bbR \to [0,1]$ be a right continuous, monotonic increasing function with 
\[
	\lim_{x \to -\infty} F(x) = 0 \quad \text{and} \quad \lim_{x \to \infty} F(x) = 1.
\]
Then there exists a probability space  $(\Omega,\cF, \bbP)$ and random variable $X$, such that 
\[
	\bbP\left(X \in (-\infty,t]\right) := \bbP\left(X^{-1}((-\infty,t])\right) = F(t).
\]
In other words, $X$ is a random variable with cdf $F$.

Moreover, $(\Omega,\cF, \bbP)$ can be chosen as $([0,1], \cB_{[0,1]}, \lambda|_{[0,1]})$ and $X = \overleftarrow{F}\circ U$, where $U$ is the standard uniform random variable.
\end{theorem}

\begin{proof}
TODO
\end{proof}

\section{Problems}

\begin{problem}[Equivalence of product \sigalg/]\label{prb:product_sigalg_equivalence}
Prove equation~\eqref{eq:product_sigalg_equivalence}.
\end{problem}
