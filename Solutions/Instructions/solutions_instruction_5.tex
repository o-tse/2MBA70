\documentclass{lecturenotes}

\usepackage{lecture_notes}

% Reset mathcal back to previous style (not curly)
\DeclareMathAlphabet{\mathcal}{OMS}{cmsy}{m}{n}



%%%%%%%%%%%%%%%%%%%%%%%%%%%%%%%%%%%%%%%%%%%%%%%%%%%%%%%%%%%%%%%%%%%%%%%%%%%%%%%%%%%%%%%%%%%%%%%%%%%
%										For leaving comments									  %
%%%%%%%%%%%%%%%%%%%%%%%%%%%%%%%%%%%%%%%%%%%%%%%%%%%%%%%%%%%%%%%%%%%%%%%%%%%%%%%%%%%%%%%%%%%%%%%%%%%

%Name:			XXX
%Description:	Adds a piece of text in blue, surrounded by [] brackets. 
%				#1: the person the comment is addressed to
%				#2: the person the comment is from
%				#3: the comment
%Usage:			Write \XXX{collaborator}{myName}{myComment} to write [myComment] addressed to [collaborator]
\newcommand{\XXX}[3]{{\color{blue} \textbf{ [#1:  #3 \textit{ -#2-} ]}}}

%%%%%%%%%%%%%%%%%%%%%%%%%%%%%%%%%%%%%%%%%%%%%%%%%%%%%%%%%%%%%%%%%%%%%%%%%%%%%%%%%%%%%%%%%%%%%%%%%%%
%									     BibTeX commands 									      %
%%%%%%%%%%%%%%%%%%%%%%%%%%%%%%%%%%%%%%%%%%%%%%%%%%%%%%%%%%%%%%%%%%%%%%%%%%%%%%%%%%%%%%%%%%%%%%%%%%%

%Name:			Swap
%Description:	Command for properly typesetting "van der" and related expressions in Dutch and German names in the
%				bibliography. It swaps [van der] and [Lastname] in [van der Lastname] so that ordering will be performed on 
%				[Lastname] instead of [van der].
%Usage:			Write \swap{Lastname}{~van~der~}, Firstname instead of [van der Lastname, Firstname] in the author header
%				of the bibtex entry.

\newcommand*{\swap}[2]{\hspace{-0.5ex}#2#1}


%%%%%%%%%%%%%%%%%%%%%%%%%%%%%%%%%%%%%%%%%%%%%%%%%%%%%%%%%%%%%%%%%%%%%%%%%%%%%%%%%%%%%%%%%%%%%%%%%%%
%											Document											  %
%%%%%%%%%%%%%%%%%%%%%%%%%%%%%%%%%%%%%%%%%%%%%%%%%%%%%%%%%%%%%%%%%%%%%%%%%%%%%%%%%%%%%%%%%%%%%%%%%%%

\setlength\parindent{0pt}

\begin{document}


\textbf{Problem 6.2}
\begin{enumerate}[label=(\alph*)]
\item The implication from right to left is by definition of $\overleftarrow{F}$ and the fact that $F$ is non-decreasing. The implication from left to right is because $F$ is right continuous.
\item Consider the preimage of $(-\infty, t]$ under $X$. Then, using the above observation, we have
\begin{align*}
	X^{-1}((-\infty,t]) &= \{\omega \in \Omega \, : \, \overleftarrow{F}(U(\omega)) \in (-\infty,t]\}\\
	&= \{\omega \in \Omega \, :\ , U(\omega) \in (-\infty, F(t)]\} = U^{-1}((-\infty, F(t)]) \in \cB_{[0,1]}.
\end{align*}
Hence, $X$ is measurable. Finally, the above computation, together with Lemma 6.5, also implies that
\[
	\bbP\left(X^{-1}((-\infty,t])\right) = \bbP\left(U^{-1}((-\infty, F(t)])\right) = F(t).
\]
\end{enumerate}


Now let $(\Omega,\cF, \bbP)$ be a probability space and $U$ a standard normal random variable. We will show that $X = \overleftarrow{F} \circ U$ is a random variable with the right probability measure. Since we can construct a standard uniform random variable on the probability $([0,1], \cB_{[0,1]}, \lambda|_{[0,1]})$ this also implies the last part. 


which finished the proof.

\bigskip

\textbf{Problem 6.3}

\begin{enumerate}[label=(\alph*)]
\item For the probability space, take $\Omega = [0,1]$, $\cF = \cB_{[0,1]}$ and $\bbP = \lambda$ the Lebesgue measure restricted to $[0,1]$. 

Observe that the function $H_\gamma(z)$ is continuous and hence has an inverse $g_\gamma(y) = \gamma \tan(\pi(y - 1/2))$ on $[0,1]$.

So the function $Y [0,1] \to \bbR$ defined by $Y(x) = g_\gamma(x)$ has the correct distribution as
\[
	\bbP(Y^{-1}((-\infty,t])) = \bbP(g_\gamma^{-1}((-\infty, t])) = \lambda(H_\gamma((-\infty,t])) = H_\gamma(t).
\]
\item Note that $g_\gamma$ is continuous on $[0,1]$ and hence measurable.
\item For any $t \ge 0$, the cdf of the Poisson random variable is given by
\[
	F_\lambda(t) = \sum_{n = 0}^{\lceil t \rceil} f_\lambda(n),
\]
where $\lceil t \rceil$ is the ceiling of $t$, i.e. the smallest integer $k \ge t$.
\item For the probability space, we again take $\Omega = [0,1]$, $\cF = \cB_{[0,1]}$ and $\bbP = \lambda$ the Lebesgue measure restricted to $[0,1]$. 

Now for any $y \in [0,1]$ let $k := k(y)$ be such that
\[
	\sum_{n = 1}^k f_\lambda(n) \ge y \quad \text{and} \quad \sum_{n = 1}^{k-1} f_\lambda(n) < y,
\]
where the last sum is interpreted as $-1$ if $k=0$.

Now define $X(y) = k(y) : [0,1] \to \bbR$. Then $k(y) \le t$ if and only if $y \le F_\lambda(t)$ and hence
\[
	X^{-1}((-\infty,t]) = \{y \in [0,1] \, : \, k(y) \in (0,t]\}
	= \{y \in [0,1] \, : \, y \in (0, F_\lambda(t)]\},
\]
from which it follows that
\[
	\bbP(X^{-1}((-\infty,t])) = \lambda((0, F_\lambda(t)]) = F_\lambda(t).
\]
\item It follows from the above computation that $X^{-1}((-\infty,t]) = \{y \in [0,1] \, : \, y \in (0, F_\lambda(t)]\}$. Since the latter is a measurable set we conclude that $X^{-1}((-\infty,t])$ is measurable for all $t$ and since these generate the Borel \sigalg/ $X$ is measurable.
\item for any $\ell \in \bbN$ define the sets $A_\ell = (n-1-1/\ell), n-1 + 1/\ell]$. Then $A_\ell$ is a decreasing set with $\lim_{\ell \to \infty} A_\ell = \{n\}$. Moreover, $A_\ell = (-\infty,n-1+1/\ell] \setminus (-\infty, n-1-1/\ell]$ and $\bbP(A_1) < \infty$. It now follows from continuity from above and (d) that
\begin{align*}
	X_\#\bbP(\{n\}) &= \lim_{\ell \to \infty} X_\# \bbP(A_\ell) \\
	&= \lim_{\ell \to \infty} X_\# \bbP((-\infty,n-1+1/\ell])
	- X_\# \bbP((-\infty,n-1-1/\ell]) \\
	&= F_\lambda(n-1+1/\ell) - F_\lambda(n-1-1/\ell) \\
	&= \sum_{k = 0}^{n} f_\lambda(k) - \sum_{k = 0}^{n-1} f_\lambda(k) = f_\lambda(n).
\end{align*}
\end{enumerate}

\end{document}